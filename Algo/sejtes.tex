\documentclass[12pt]{article}
\usepackage{tikz}
\usepackage{setspace}
\usepackage{hyperref}
\usepackage{amsfonts, amssymb, amsmath}
\usepackage{titlesec}
\usepackage{pgfplots}
\pgfplotsset{compat=newest}
\usepackage{graphicx}
\usepackage{wrapfig}
\usepackage{caption}
\usepackage{enumitem}
\usetikzlibrary{shadows.blur}
\usepackage{lmodern}
\usepackage{mathrsfs}
\setlength{\parskip}{0pt}
\setlength{\parindent}{0pt}

\title{\textcolor{purple}{\Huge\textbf{Asymptotics}}}
\author{From the notes of Dr. Ásványi Tibor}

\renewcommand{\contentsname}{Contents}
\newcommand{\R}{\mathbb{R}}
\newcommand{\N}{\mathbb{N}}
\newcommand{\E}{\exists}
\newcommand{\mm}{\mathbf{m}}
\newcommand{\MM}{\mathbf{M}}
\newcommand{\K}{\mathbb{K}}
\newcommand{\D}{\mathcal{D}_f}

\definecolor{modernyellow}{HTML}{F4E4BC}
\definecolor{moderngreen}{HTML}{BDDCBD}

% Adjusting the TikZ settings
\tikzset
{
        definition/.style={
                draw,
                fill=white,
                line width=1pt,
                rounded corners,
                drop shadow={shadow blur steps=5,shadow xshift=1ex,shadow yshift=-1ex},
                text width=1\textwidth,
                inner sep=8pt,
                align=justify
        },
        theorem/.style={
            draw,
            fill=moderngreen,
            line width=1pt,
            rounded corners,
            drop shadow={shadow blur steps=5,shadow xshift=1ex,shadow yshift=-1ex},
            text width=1\textwidth,
            inner sep=8pt,
            align=justify
        },
        proof/.style={
                fill=white,
                rectangle,
                drop shadow={shadow blur steps=5,shadow xshift=1ex,shadow yshift=-1ex, moderngreen},
                text width=1\textwidth,
                inner sep=8pt,
                align=justify
        },
        proof1/.style={
                fill=white,
                rectangle,
                drop shadow={shadow blur steps=5,shadow xshift=1ex,shadow yshift=0, moderngreen},
                text width=1\textwidth,
                inner sep=8pt,
                align=justify
        }
}

\begin{document}
    Legyen
    \[
        \mathrm{P} := \{ f : \N \to \R : \E N \in \N, \, \forall n \in \N, \, n > N \text{ esetén } f(n) > 0\}.
    \]
    Más szavakkal $f \in \mathrm{P}$ pontosan akkor, ha $f$ egy $\N \to \R$ típusú függvény (sorozat) és $f$ \textit{majdnem minden (m.m.) $n$-re} pozitív értékeket vesz fel. $\mathrm{P}$ az \textit{aszimptotikusan pozitív} $f : \N \to \R$ függvények (sorozatok) halmaza.\\

    \tikz \node[definition]
    {
        \textbf{Definíció.} Legyen $g \in \mathrm{P}$, és az $O(g), \, \Omega(g)$ szimbólumok jelöljék a alábbi függvényhalmazokat.
        \[
            O(g) := \{ f \in \mathrm{P} : \E d > 0, \text{ hogy } d \cdot g(n) \geq f(n) \text{ m.m. $n$-re}\}.
        \]
        Azaz ha $f \in O(g)$, akkor létezik egy $d > 0$ pozitív szám, hogy majdnem minden $n$-re $f$ alulról becsüli $d \cdot g$-t.
        \[
        \Omega(g) := \{ f \in \mathrm{P} : \E c > 0, \text{ hogy } c \cdot g(n) \leq f(n) \text{ m.m. $n$-re}\}.
        \]
        Azaz ha $f \in \Omega(g)$, akkor létezik egy $c > 0$ pozitív szám, hogy majdnem minden $n$-re $f$ felülről becsüli $c \cdot g$-t. 
    };\\

    Legyen adott egy $g \in \mathrm{P}$ aszimptotikusan pozitív függvény és legyen
    \[
        \mathscr{U} := O(g) \, \backslash \, \Omega(g).
    \]
    Azaz $\mathscr{U}$ egy olyan halmaz, amiben olyan $f \in \mathrm{P}$ függvények vannak, amik benne vannak $O(g)$-ben, azaz
    \[
        \E d > 0, \text{ hogy } d \cdot g(n) \geq f(n) \quad (\text{m.m. } \N \ni n \text{-re}),
    \]
    de nincsenek benne $\Omega(g)$-ben, azaz
    \[
        \forall c > 0 \text{ esetén } c \cdot g(n) > f(n) \quad (\text{m.m. } \N \ni n \text{-re}).
    \]
    
    Minden az jelenti, hogy
    \[
        \boxed{\mathscr{U} = \{ f \in \mathrm{P} : \forall c > 0 \text{ esetén } c \cdot g(n) > f(n) \text{ m.m. $n$-re}\}}.
    \]

    \tikz \node[definition]
    {
        \textbf{Definíció.} Adott $g \in \mathrm{P}$ függvény esetén legyen
        \[
            o(g) := \left\{ f \in \mathrm{P} : \lim_{n \to + \infty} \frac{f(n)}{g(n)} = 0 \right\}.
        \]
    };\\

    Az előző definícióban, mivel $g \in \mathrm{P}$ aszimptotikusan pozitív függvény, ezért fel lehet tenni, hogy elég nagy $n$-ek esetén az $\frac{f}{g}$ hányados értelmes. Azt akarjuk belátni, hogy akármilyen $g \in \mathrm{P}$ függvényt veszünk, akkor
    \[
        o(g) = \mathscr{U}.
    \]
    Legyen adott egy $f \in o(g)$, ekkor
    \[
        \lim_{n \to + \infty} \frac{f(n)}{g(n)} = 0,
    \]
    azaz
    \[
        \forall \varepsilon > 0 \text{-hoz } \E N \in \N, \text{ hogy } \left| \frac{f(n)}{g(n)} \right| < \varepsilon \quad (n \in \N, \, n > N).
    \]
    Mivel $f, \, g \in \mathrm{P}$, ezért $N$-ről az is feltehető, hogy $f, \, g$ értékei pozitívak
    \[
        f(n) < \varepsilon \cdot g(n) \quad (n \in \N, \, n > N).
    \]
    Azaz
    \[
        \forall c > 0 \text{ esetén } c \cdot g(n) > f(n) \quad (\text{m.m. $n$-re}).
    \]
    
    Tehát valóban, ha $f \in o(g)$, akkor $f \in \mathscr{U}$ is igaz. \\

    Most azt tegyük fel, hogy $f \in \mathscr{U}$, azaz
    \[
        \forall c > 0 \text{ esetén } c \cdot g(n) > f(n) \quad (\text{m.m. $n$-re}).
    \]
    Mivel $g \in \mathrm{P}$ ezért m.m. $n$-re $g$ pozitív, tehát 
    \[
        \forall c > 0 \text{ esetén } c > \frac{f(n)}{g(n)} \quad (\text{m.m. $n$-re}).
    \]
    Ez pedig azt jelenti, hogy $\displaystyle \lim_{n \to + \infty} \frac{f(n)}{g(n)} = 0$, azaz $f \in o(g)$.
    


\end{document}
