\documentclass[12pt]{article}
\usepackage{tikz}
\usepackage{setspace}
\usepackage{hyperref}
\usepackage{amsfonts, amssymb, amsmath}
\usepackage{titlesec}
\usepackage{pgfplots}
\pgfplotsset{compat=newest}
\usepackage{graphicx}
\usepackage{wrapfig}
\usepackage{caption}
\usepackage{enumitem}
\usetikzlibrary{shadows.blur}
\usepackage{lmodern}
\usepackage{mathrsfs}
\setlength{\parskip}{0pt}
\setlength{\parindent}{0pt}

\title{\textcolor{purple}{\Huge\textbf{Aszimptotika}}}
\author{Dr. Ásványi Tibor jegyzetéből}

\renewcommand{\contentsname}{Tartalom}
\newcommand{\R}{\mathbb{R}}
\newcommand{\N}{\mathbb{N}}
\newcommand{\E}{\exists}
\newcommand{\mm}{\mathbf{m}}
\newcommand{\MM}{\mathbf{M}}
\newcommand{\K}{\mathbb{K}}
\newcommand{\D}{\mathcal{D}_f}

\definecolor{modernyellow}{HTML}{F4E4BC}
\definecolor{moderngreen}{HTML}{BDDCBD}

% Adjusting the TikZ settings
\tikzset
{
        definition/.style={
                draw,
                fill=modernyellow,
                line width=1pt,
                rounded corners,
                drop shadow={shadow blur steps=5,shadow xshift=1ex,shadow yshift=-1ex},
                text width=1\textwidth,  % Reduced text width for better spacing
                inner sep=7pt,  % Reduced padding inside the node
                align=justify  % Improved text alignment and spacing
        },
        theorem/.style={
            draw,
            fill=moderngreen,
            line width=1pt,
            rounded corners,
            drop shadow={shadow blur steps=5,shadow xshift=1ex,shadow yshift=-1ex},
            text width=1\textwidth,  % Reduced text width for better spacing
            inner sep=7pt,  % Reduced padding inside the node
            align=justify  % Improved text alignment and spacing
        },
        proof/.style={
                fill=white,
                rectangle,
                drop shadow={shadow blur steps=5,shadow xshift=1ex,shadow yshift=-1ex, moderngreen},
                text width=0.8\textwidth,
                inner sep=6pt,
                align=justify
        },
        proof1/.style={
                fill=white,
                rectangle,
                drop shadow={shadow blur steps=5,shadow xshift=1ex,shadow yshift=0, moderngreen},
                text width=0.8\textwidth,
                inner sep=6pt,
                align=justify
        }
}

\begin{document}
    \maketitle
    \tableofcontents
    \newpage

    \section{Függvények aszimptotikus viselkedése}

    E fejezet célja, hogy tisztázza a programok hatékonyságának nagyságrendjeivel kapcsolatok alapvető fogalmakat, és az ezekhez kepcsolódó függvényosztályok legfontosabb tulajdonságait.\\

    \tikz \node[definition]
    {
        \textbf{1. Definíció.} \textit{Valamely $P(n)$ tulajdonság elég nagy $n$-ekre pontosan akkor teljesül, ha}
        \[
            \E N \in \N, \text{ \textit{hogy} } \forall n \in \N, \, n \geq N: P(n) \text{ \textit{igaz}}.
        \]
    };\\

    \tikz \node[definition]
    {
        \textbf{2. Definíció.} \textit{Az} $f \in \R \to \R$ \textit{függvény AP (aszimptotikusan pozitív), ha elég nagy $n$-ekre $f(n) > 0$}.
    };\\

    Azaz egy $f \in \R \to \R$ függvény AP pontosan akkor, ha
    \[
        \E N \in \N, \, \forall n \in \N, \, n \geq N : f(n) > 0.
    \]

    Egy tetszőleges helyes program futási ideje és tárigénye is nyilvánvalóan, tetszőleges megfelelő mértékegységben (másodperc, perc Mbyte stb.) mérve pozitív számérték. Amikor (alsó és / vagy felső) becsléseket végzünk a futási időre vagy a tárigényre, legtöbbször az input adatszerkezetek méretének függvényében végezzük a becsléseket. Így a becsléseket leíró függvények természetesen $\N \to \R$ típusúak. Megkövetelhetnénk, hogy $\N \to \mathbb{P}$ típusúak legyenek, de annak érdekében, hogy képleteink minél egyszerűbbek legyenek, általában megelégszünk azzal, hogy a becsléseket leíró függvények aszimptotikusan pozitívak (AP) legyenek.

    Legyen
    \[
        \mathrm{P} := \{ f : \N \to \R : f \text{ aszimptotikusan pozitív függvény} \}.
    \]

    \tikz \node[definition]
    {
        \textbf{3. Definíció.} Legyen $g \in \mathrm{P}$. Ekkor legyen $O(g)$ egy függvényhalmaz ami olyan $f \in \mathrm{P}$ függvényekből áll, amiket elég nagy $n$ helyettesítési értékekre, megfelelő $d \in \R^+$ szorzóval felülről becsül a $g$ függvény, azaz
        \[
             O(g) := \{ f \in \mathrm{P} : \E d \in \R^+, \text{ hogy elég nagy $n$-ekre } d \cdot g(n) \geq f(n) \}.
        \]
    };\\

    \tikz \node[definition]
    {
        \textbf{4. Definíció.} Legyen $g \in \mathrm{P}$. Ekkor legyen $\Omega(g)$ egy függvényhalmaz ami olyan $f \in \mathrm{P}$ függvényekből áll, amiket elég nagy $n$ helyettesítési értékekre, megfelelő $d \in \R^+$ szorzóval alulról becsül a $g$ függvény, azaz
        \[
             \Omega(g) := \{ f \in \mathrm{P} : \E d \in \R^+, \text{ hogy elég nagy $n$-ekre } d \cdot g(n) \leq f(n) \}.
        \]
    };\\

    \tikz \node[definition]
    {
        \textbf{5. Definíció.} Legyen $g \in \mathrm{P}$. Ekkor legyen
        \[
            \Theta(g) := O(g) \cap \Omega(g).
        \]
    };

    \tikz \node[theorem]
    {
        \textbf{Következmény.} Egy $g \in P$ esetén a $\Theta(g)$ függvényhalmaz olyan $f \in P$ függvényekből áll, amiket elég nagy $n$ helyettesítési értékekre, megfelelő pozitív konstans szorzókkal alulról és felülről is becsül a $g$ függvény
        \[
            \Theta(g) = \{ f \in P : \E c, \, d > 0, \text{ hogy elég nagy } n \text{-ekre} c \cdot g(n) \leq f(n) \leq d \cdot g(n).\}
        \]
    };
    
    
\end{document}