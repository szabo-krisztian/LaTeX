\documentclass[12pt]{article}
\usepackage{tikz}
\usepackage{setspace}
\usepackage{hyperref}
\usepackage{amsfonts, amssymb, amsmath}
\usepackage{titlesec}
\usepackage{pgfplots}
\pgfplotsset{compat=newest}
\usepackage{graphicx}
\usepackage{wrapfig}
\usepackage{caption}
\usepackage{enumitem}
\usetikzlibrary{shadows.blur}
\usepackage{lmodern}
\usepackage{mathrsfs}
\setlength{\parskip}{0pt}
\setlength{\parindent}{0pt}

\title{\textcolor{purple}{\Huge\textbf{Asymptotics}}}
\author{From the notes of Dr. Ásványi Tibor}

\renewcommand{\contentsname}{Contents}
\newcommand{\R}{\mathbb{R}}
\newcommand{\N}{\mathbb{N}}
\newcommand{\E}{\exists}
\newcommand{\mm}{\mathbf{m}}
\newcommand{\MM}{\mathbf{M}}
\newcommand{\K}{\mathbb{K}}
\newcommand{\D}{\mathcal{D}_f}

\definecolor{modernyellow}{HTML}{F4E4BC}
\definecolor{moderngreen}{HTML}{BDDCBD}

% Adjusting the TikZ settings
\tikzset
{
        definition/.style={
                draw,
                fill=modernyellow,
                line width=1pt,
                rounded corners,
                drop shadow={shadow blur steps=5,shadow xshift=1ex,shadow yshift=-1ex},
                text width=1\textwidth,
                inner sep=8pt,
                align=justify
        },
        theorem/.style={
            draw,
            fill=moderngreen,
            line width=1pt,
            rounded corners,
            drop shadow={shadow blur steps=5,shadow xshift=1ex,shadow yshift=-1ex},
            text width=1\textwidth,
            inner sep=8pt,
            align=justify
        },
        proof/.style={
                fill=white,
                rectangle,
                drop shadow={shadow blur steps=5,shadow xshift=1ex,shadow yshift=-1ex, moderngreen},
                text width=1\textwidth,
                inner sep=8pt,
                align=justify
        },
        proof1/.style={
                fill=white,
                rectangle,
                drop shadow={shadow blur steps=5,shadow xshift=1ex,shadow yshift=0, moderngreen},
                text width=1\textwidth,
                inner sep=8pt,
                align=justify
        }
}

\begin{document}
    \maketitle
    \tableofcontents
    \newpage

    \section{Asymptotic Behavior of Functions}

    The goal of this chapter is to clarify basic concepts related to the orders of magnitude of program efficiency and the most important properties of function classes associated with them.\\

    \tikz \node[definition]
    {
        \textbf{1. Definition.} \textit{A property $P(n)$ holds for sufficiently large $n$ if and only if}
        \[
            \E N \in \N, \text{ \textit{such that} } \forall n \in \N, \, n \geq N: P(n) \text{ \textit{is true}}.
        \]
    };\\

    \tikz \node[definition]
    {
        \textbf{2. Definition.} \textit{The function} $f \in \R \to \R$ \textit{is AP (asymptotically positive) if for sufficiently large $n$, $f(n) > 0$}.
    };\\

    That is, a function $f \in \R \to \R$ is AP if and only if
    \[
        \E N \in \N, \, \forall n \in \N, \, n \geq N : f(n) > 0.
    \]

    The runtime and memory requirements of any correct program, measured in appropriate units (seconds, minutes, Mbytes, etc.), are obviously positive values. When we estimate runtime or memory requirements (from below and/or above), we often do so as a function of the size of the input data structures. Therefore, the functions that describe these estimates are naturally of type $\N \to \R$. We could require them to be of type $\N \to \mathbb{P}$, but for simplicity's sake, we usually assume that the functions describing the estimates are asymptotically positive (AP).

    Let
    \[
        \mathrm{P} := \{ f : \N \to \R : f \text{ is an asymptotically positive function} \}.
    \]

    \tikz \node[definition]
    {
        \textbf{3. Definition.} \textit{Let $g \in \mathrm{P}$. Then let $O(g)$ be a set of functions that consists of such functions $f \in \mathrm{P}$ that are bounded above by the function $g$ for sufficiently large values of $n$, multiplied by an appropriate constant $d \in \R^+$, that is}
        \[
             O(g) := \{ f \in \mathrm{P} : \E d \in \R^+, \textit{ such that for sufficiently large $n$, } d \cdot g(n) \geq f(n) \}.
        \]
    };\\

    \tikz \node[definition]
    {
        \textbf{4. Definition.} \textit{Let $g \in \mathrm{P}$. Then let $\Omega(g)$ be a set of functions that consists of such functions $f \in \mathrm{P}$ that are bounded below by the function $g$ for sufficiently large values of $n$, multiplied by an appropriate constant $d \in \R^+$, that is}
        \[
             \Omega(g) := \{ f \in \mathrm{P} : \E d \in \R^+, \textit{ such that for sufficiently large $n$, } d \cdot g(n) \leq f(n) \}.
        \]
    };\\

    \tikz \node[definition]
    {
        \textbf{5. Definition.} \textit{Let $g \in \mathrm{P}$. Then let}
        \[
            \Theta(g) := O(g) \cap \Omega(g).
        \]
    };\\

    \tikz \node[theorem]
    {
        \textbf{Corollary.} For a function $g \in P$, the set $\Theta(g)$ consists of such functions $f \in P$ that are bounded from below and above by $g$ for sufficiently large values of $n$, multiplied by appropriate positive constants, that is
        \[
            \Theta(g) =
            \left\{ 
            \begin{aligned}
                & \, f \in P : \E c, \, d > 0, \text{ such that for sufficiently large } n, \, \\
                & \, c \cdot g(n) \leq f(n) \leq d \cdot g(n).
            \end{aligned}
            \right\}.
        \]

        \textit{In the case where \( f \in \Theta(g) \), we can say that \( g \) is the asymptotic lower and upper bound of \( f \); intuitively: \( f \) is roughly proportional to \( g \).}
    };\\

    Arra, hogy egy függvény egy másikhoz képest nagy $n$ értékekre elhanyagolható, bevezetjuk az aszimptotikusan kisebb fogalmát.\\

    \tikz \node[definition]
    {
        \textbf{Definíció.} Legyen $\varphi : \N \to \R$, $g \in \mathrm{P}$. Ekkor legyen a jelölés
        \[
            \varphi \prec g
        \]
        ekvivalens azzal, hogy
        \[
            \lim_{n \to + \infty} \frac{\varphi(n)}{g(n)} = 0.
        \]
        Ilyenkor azt mondhatjuk, hogy $\varphi$ aszimptotikusan kisebb, mint $g$. (Vegyük észre, hogy $\varphi$ nem okvetlenül AP.) Legyen
        \[
            o(g) := \{ f \in \mathrm{P} : f \prec g \}.
        \]
    };\\

            
    \tikz \node[definition]
    {
        \textbf{Definíció.} Legyen $\varphi : \N \to \R$, $g \in \mathrm{P}$. Ekkor legyen a jelölés
        \[
            \varphi \succ g
        \]
        ekvivalens azzal, hogy
        \[
            g \prec \varphi.
        \]
        Ilyenkor azt mondhatjuk, hogy $\varphi$ aszimptotikusan nagyobb, mint $g$. (Vegyük észre, hogy $\varphi$ nem okvetlenül AP.) Legyen
        \[
            \omega(g) := \{ f \in \mathrm{P} : f \succ g \}.
        \]
    };\\

    \tikz \node[theorem]
    {
        \textbf{Tulajdonság.} Legyen $g \in \mathrm{P}$. Ekkor
        \[
            \Theta(g) = O(g) \cap \Omega(g),
        \]
        \[
            o(g) \subsetneq O(g) \backslash \Omega(g),
        \]
        \[
            \omega(g) \subsetneq \Omega(g) \backslash O(g).
        \]
    };\\

    \tikz \node[proof1]
    {
        \textbf{Bizonyítás.} Az első az definíció szerint nyilván igaz. Legyen $g \in \mathrm{P}$. Balra irányt fogjukbizonyítani először. $f \in o(g)$, azaz
        \[
            \lim_{n \to + \infty} \frac{f(n)}{g(n)} = 0.
        \]
        Ez pontosan azt jelenti, hogy minden $\varepsilon > 0$ hibakorláthoz létezik olyan $n_0 \in \N$ küszönindex, hogy
        \[
            \left| \frac{f(n)}{g(n)} \right| < \varepsilon \quad (n \in \N, \, n \geq n_0).
        \]
        Mivel $g, \, f \in \mathrm{P}$, létezik olyan $n_1 \in \N$, hogy a hányados pozitív lesz
        \[
            \frac{f(n)}{g(n)} > 0 \quad (n \in \N, \, n \geq n_1).
        \]
    };

    \tikz \node[proof]
    {
        ha $N := \max\{n_0, \, n_2\}$, akkor
        \[
            f(n) > g(n) \cdot \varepsilon \quad (n \in \N, \, n \geq N).
        \]
        azaz, nemhogy létezik egy $d > 0$ szám, hogy az előző egyenlőtlenség fennál, hanem akármilyen pozitív konstanssal igaz lesz.\\

        Most azt bizonyítjuk be, hogy létezik olyan $f \in O(g) \backslash \Omega(g)$ ami nincs benne $o(g)$-ben. Erre elég egy ellenpéldát hozni. $g := n^2$, ekkor nyilván $f \in \mathrm{P}$.
        \[
            n^2 - n \in O(g) \backslash \Omega(g).
        \]
        Azaz megfelelő $c>0$-vel $f$ alulról becsüli $g$-t, de akármelyik $d>0$ konstanssal nem tujda elég nagy $n$-ekre felülbecsülni $g$-t. Nyilván $c:=1$ lehetséges. Tegyük fel indirekt, hogy létezik olyan $d>0$, $N \in \N$, hogy
        \[
            d \cdot (n^2-n) \geq n^2 \quad (n \in \N, \, n \geq N).
        \]
        \[
            d n^2- dn -n^2 \geq 0 \quad (n \in \N, \, n \geq N).
        \]
        \[
            n^2\left(d-\frac{d}{n}-1\right) \geq 0 \quad (n \in \N, \, n \geq N).
        \]
    };
    
    
    
    
\end{document}
