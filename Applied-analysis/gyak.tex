\documentclass[12pt]{article}
\usepackage[left=0.9in, right=0.9in, top=1in, bottom=1in]{geometry}
\usepackage{tikz}
\usepackage{setspace}
\usepackage{hyperref}
\usepackage{amsfonts, amssymb, amsmath} 
\usepackage{titlesec}
\usepackage{pgfplots}
\pgfplotsset{compat=1.18}
\usepackage{graphicx}
\usepackage{wrapfig}
\usepackage{caption}
\usepackage{enumitem}
\usetikzlibrary{shadows.blur}
\usepackage{lmodern}
\setlength{\parskip}{0pt}
\setlength{\parindent}{0pt}

\title{\textcolor{purple}{\Huge\textbf{Analízis III}}}
\author{1. gyakorlat}
\date{Szabó Krisztián}

\renewcommand{\contentsname}{Tartalom}
\newcommand{\R}{\mathbb{R}}
\newcommand{\N}{\mathbb{N}}
\newcommand{\E}{\exists}
\newcommand{\mm}{\mathbf{m}}
\newcommand{\MM}{\mathbf{M}}
\newcommand{\K}{\mathbb{K}}
\newcommand{\D}{\mathcal{D}_f}

\definecolor{modernyellow}{HTML}{F4E4BC}
\definecolor{moderngreen}{HTML}{BDDCBD}

\tikzset
{
	definition/.style={
		draw,
		fill=modernyellow,
		line width=1pt,
		rounded corners,
		drop shadow={shadow blur steps=5,shadow xshift=1ex,shadow yshift=-1ex},
		text width=0.9\textwidth,
		inner sep=10pt
	},
	theorem/.style={
		draw,
		fill=moderngreen,
		line width=1pt,
		rounded corners,
		drop shadow={shadow blur steps=5,shadow xshift=1ex,shadow yshift=-1ex},
		text width=0.9\textwidth,
		inner sep=10pt
	},
	proof/.style={
		fill=white,
		rectangle,
		drop shadow={shadow blur steps=5,shadow xshift=1ex,shadow yshift=-1ex, moderngreen},
		text width=0.9\textwidth,
		inner sep=6pt,
	},
	proof1/.style={
		fill=white,
		rectangle,
		drop shadow={shadow blur steps=5,shadow xshift=1ex,shadow yshift=0, moderngreen},
		text width=0.9\textwidth,
		inner sep=6pt,
	}
}

\begin{document}
    \maketitle
    \tableofcontents
    \newpage
    \section{Gyakorlat}
    \subsection{Integráltranszformáció}
    \tikz \node[theorem]
    {
        \textbf{Tétel.} Tekintsük a nyílt halmazon értelmezett és folytonosan differenciálható
        \[
            g \in \R^n \to \R^n
        \]
        függvényt. Tegyük fel, hogy az $I \subset \mathcal{D}_g$ halmaz kompakt intervallum, továbbá az $I$ belsejére való $g_{|\text{int} \, I}$ leszűkítés injektív függvény. Ekkor az
        \[
            f : g[I] \to \R
        \]
        korlátos függvény akkor és csak akkor integrálható, ha az
        \[
            I \ni x \to f\big( g(x) \big) \cdot |\text{det} \, g'(x)|
        \]
        függvény is integrálható. Az utóbbi esetben
        \[
            \int_I f\big( g(x) \big) \cdot |\text{det} \, g'(x)| \, dx = \int_{g[I]} f. 
        \]
    };\newline
    
    Ennek speciális esetei az alább részletezett transzformációk.

    \subsection{}


\end{document}