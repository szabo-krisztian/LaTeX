\section{Relations}

Using ordered pairs, we can formulate the mathematical theory of relations in set-theoretic language. By a relation we mean here something like marriage (between men and women) belonging (between elements and sets). More explicitly, what we shall call a relation is sometimes called a \textit{binary} relation. An example of a ternary relation is parenthood for people (Adam and Eve are the parents of Cain). In this book we shall have no occasion to treat the theory of relations that are ternary, quaternary, or worse.

Looking at any specific relation, such as marriage for instance, we might be tempted to consider certain ordered pairs $(x, \, y)$, namely just those for which $x$ is man, $y$ is a woman, and $x$ is married to $y$. We have not yet seen the definition of the general concept of a relation, but it seems plausible that, just as in this marriage example, every relation should uniquely determine the set of all those ordered pairs for which the first coordinate does stand in that relation to the second. If we know the relation, we know the set, and, better yet, if we know the set, we know the relation. If, for instance, we were presented with the set of ordered pairs of people that corresponds to marriage, then, even if we forgot the definition of marriage, we could always tell when a man $x$ is married to a woman $y$ and when not; we would just have to see whether the ordered pair $(x, \, y)$ does or does not belong to the set

We may not know what a relation is, but we do know what a set is, and the preceding considerations establish a close connection between relations and sets. The precise set-theoretic treatment of relations takes advantage of that heuristic connection; the simplest to do is to define a relation to be the corresponding set. This is what we do; we hereby define a \textit{relation} as a set of ordered pairs. Explicitly: a set $R$ is a relation if each element of $R$ is an ordered pair; this means, of course, that if $z \in R$, then there exists $x$ and $y$ so that $z = (x, \, y)$. If $R$ is a relation, it is sometimes convenient to express the fact that $(x, \, y) \in R$ by writing
\[
	xRy
\]
and saying, as in everyday language, that $x$ stands in the relation $R$ to $y$.

The least exciting relation is the empty one. (To prove that $\emptyset$ is a set of ordered pairs, look for an element of $\emptyset$ that is not an ordered pair.) Another dull example is the Cartesian product of any two sets $X$ and $Y$. Here is a slightly more interesting example: let $X$ be any set, and let $R$ be the set of all those pairs $(x, \, y)$ in $X \times X$ for which $x = y$. The relation $R$ is just the relation of equality between elements of $X$; if $x$ and $y$ are in $X$; then $xRy$ mean the same as $x = y$. One more example will suffice for now: let $X$ be any set, and let $R$ be the set of all those pairs $(x, \, A)$ in $X \times \mathcal{P}(X)$ for which $x \in A$. This relation $R$ is just the relation of belonging between elements of $X$ and subsets of $X$; if $x \in X$ and $A \in \mathcal{P}(X)$, then $xRA$ means the same as $x \in A$.

In the preceding section we saw that associated with every set $R$ of ordered pairs there are two sets called the projections of $R$ onto the first and second coordinates. In the theory of relations these sets are known as the \textit{domain} and the \textit{range} of $R$ (abbreviated dom $R$ and ran $R$); we recall that they are defined by
\[
	\text{dom} \, R = \{ x : \text{ for some } y \, (xRy)\}
\]
and
\[
	\text{ran} \, R = \{ y : \text{ for some } x \, (xRy)\}.
\]
If $R$ is the relation of marriage, so that $xRy$ means that $x$ is a man, $y$ is a woman, and $x$ and $y$ are married to one another; then $\text{dom} \, R$ is the set of married men and $\text{ran} \, R$ is the set of married women. Both the domain and the range of $\emptyset$ are equal to $\emptyset$. If $R = X \times Y$, then $\text{dom} \, R = X$ and $\text{ran} \, R = Y$. If $R$ is equality in $X$, then $\text{dom} \, R = \text{ran} \, R = X$. If $R$ is belonging, between $X$ and $\mathcal{P}(x)$, then $\text{dom} \, R = X$ and $\text{ran} \, R = \mathcal{P}(X) - \{\emptyset\}$.

If $R$ is a relation included in a Cartesian product $X \times Y$ (so that $\text{dom} \, R \subset X$ and $\text{ran} \, R \subset Y$), it is sometimes convenient to say that $R$ is a relation \textit{from} $X$ \textit{to} $Y$; instead of a relation $X$ to $X$ we may speak of a relation \textit{in} $X$. A relation $R$ in $X$ is \textit{reflexive} if $xRx$ for every $x$ in $X$; it is \textit{symmetric} if $xRy$ implies that $yRx$; and it is \textit{transitive} if $xRy$ and $yRx$ imply that $xRz$.

\begin{quote}
	\textbf{Exercise.} For each of these three possible properties, find a relation that does not have that property but does have the other two.
\end{quote}

If we have a set $X = \{ a, \, b, \, c\}$, then
\begin{itemize}
	\item $R_1$ is not reflexive, but is symmetric and transitive if
	\[
		R_1 = \emptyset;
	\]
	\item $R_2$ is not symmetric, but is reflexive and transitive if
	\[
		R_2 = \{ (a, \, a), \, (b, \, b), \, (c, \, c), \, (a, \, b)\};
	\]
	\item $R_3$ is not transitive, but is reflexive and symmetric if
	\[
		R_3 = \{ (a, \, a), \, (b, \, b), \, (c, \, c), \, (a, \, b), \, (b, \, a), \, (b, \, c), \, (c, \, b)\}.
	\]
\end{itemize}