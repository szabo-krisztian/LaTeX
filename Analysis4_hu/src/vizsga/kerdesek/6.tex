\newpage
\section{Vizsgakérdés}
\begin{quote}
	\textit{A lineáris differenciálegyenlet-rendszer vizsgálata: homogén, inhomogén rendszerek. A megoldáshalmaz szerkezete.}
\end{quote}

\subsection{Lineáris differenciálegyenlet-rendszer}
Valamilyen $1 \leq n \in \N$ és egy nyílt $I \subset \R$ intervallum esetén adottak a folytonos
\[
a_{ik} : I \to \R \quad (i, \, k = 1, \, \dots, \, n), \, b = (b_1, \, \dots, \, b_n) : I \to \R^n
\]
függvények, és tekintsük az
\[
I \ni x \mapsto A(x) := \big( a_{ik}(x) \big)_{i, \, k =1}^n \in \R^{n \times n}
\]
\textit{mátrixfüggvényt}. Ha
\[
f(x, \, y) := A(x) \cdot y + b(x) \quad ((x, \, y) \in I \times \K^n),
\]
akkor az $f$ függvény, mint jobb oldal által meghatározott
\[
\varphi'(x) = A(x) \cdot \varphi(x) + b(x) \quad (x \in \Dp)
\]
differenciálegyenletet \textit{lineáris differenciálegyenletnek} ($n > 1$ esetén \textit{lineáris} \textit{differenciálegyenlet-rendszernek}) nevezzük.\\

Legyenek a fentieken túl adottak még a $\tau \in I, \, \xi \in \K^n$ értékek, és vizsgáljuk a $\varphi(\tau) = \xi$ k.é.p.-t. Ha $I_* \subset I$, $\tau \in \text{int} \, I_*$, kompakt intervallum, akkor
\[
\sup \{ |a_{ik}(x)| : x \in I_* \} \in \R \quad (i, \, k = 1, \, \dots, \, n),
\]
ezért
\[
q := \sup \{ \|A(x)\|_{(\infty)} : x \in I_* \} \in \R.
\]
Következésképpen
\[
\| f(x, \, y) - f(x, \, z) \|_\infty = \| A(x) \cdot (y- z) \|_\infty \leq
\]
\[
\|A(x)\|_{(\infty)} \cdot \|y-z\|_\infty \leq q \cdot \|y-z\|_\infty \quad (x \in I_*, \, y, \, z \in \K^n).
\]
Továbbá a
\[
\beta := \sup\{\|b(x)\|_\infty : x \in I_*\}  \quad (\in \R)
\]
jelöléssel
\[
\|f(x, \, y)\|_\infty = \| A(x) \cdot y + b(x) \|_\infty \leq \|A(x) \cdot y\|_\infty + \|b(x)\|_\infty \leq
\]
\[
\|A(x)\|_{(\infty)} \cdot \|y\|_\infty + \|b(x)\|_\infty \leq q \cdot \|y\|_\infty + \beta \quad (x \in I_*, \, y \in \K^n),
\]
ezért minden k.é.p. teljes megoldása az $I$ intervallumon van értelmezve. Azt mondjuk, hogy a szóban forgó d.e. \textit{homogén}, ha $b \equiv 0$, \textit{inhomogén}, ha létezik $x \in I$, hogy $b(x) \neq 0$. Legyenek
\[
\mathcal{M}_h := \{ \psi : I \to \K^n : \psi \in D, \, \psi' = A \cdot \psi \},
\]
\[
\mathcal{M} := \{ \psi : I \to \K^n : \psi \in D, \, \psi' = A \cdot \psi + b\}.
\]

\subsection{Lineáris differenciálegyenlet-rendszerek alaptétele}
\label{subsec:diff_systems_core}

\tikz \node[theorem]
{
	\textbf{Tétel.} A bevezetésben mondott feltételek mellett
	\begin{enumerate}
		\item az $\mathcal{M}_h$ halmaz $n$ dimenziós lineáris tér a $\K$-ra vonatkozóan;
		\item tetszőleges $\psi \in \mathcal{M}$ esetén
		\[
		\mathcal{M} = \psi + \mathcal{M}_h := \{\psi + \chi : \chi \in \mathcal{M}_h \};
		\]
		\item ha a $\phi_k = (\phi_{k1}, \, \dots, \, \phi_{kn})$ $(k = 1, \, \dots, \, n)$ függvények bázist alkotnak az $\mathcal{M}_h$-ban, akkor léteznek olyan $g_k : I \to \K$ $(k = 1, \, \dots, \, n)$ differenciálható függvények, amelyekkel
		\[
		\psi := \sum_{k=1}^n g_k \cdot \phi_k \in \mathcal{M}.
		\]
	\end{enumerate}
};\\

\textbf{Bizonyítás.} Az 1. állítás bizonyításához mutassuk meg először is azt, hogy bármilyen $\psi, \, \p \in \mathcal{M}_h$ és $c \in \K$ esetén $\psi + c \cdot \p \in \mathcal{M}_h$:
\[
	(\psi + c \cdot \p)' = \psi' + c \cdot \p' = A \cdot \psi + c \cdot A \cdot \p = A(\psi + c \cdot \p),
\]
amiből a mondott állítás az $\mathcal{M}_h$ definíciója alapján nyilvánvaló. Tehát az $\mathcal{M}_h$ lineáris tér a $\K$ felett.\\

Most megmutatjuk, hogy ha $m \in \N$, és $\chi_1, \, \dots, \, \chi_m \in \mathcal{M}_h$, tetszőleges függvények, akkor az alábbi ekvivalencia igaz:

\begin{quote}
	a $\chi_1, \, \dots, \, \chi_m$ függvények akkor és csak akkor alkotnak lineárisan független rendszert az $\mathcal{M}_h$ vektortérben, ha bármilyen $\tau \in I$ esetén a $\chi_1(\tau), \, \dots, \, \chi_m(\tau)$ vektorok lineárisan függetlenek a $\K^n$-ben.
\end{quote}

Az ekvivalencia egyik fele nyilvánvaló: ha a $\chi_1, \, \dots, \, \chi_m$-ek lineárisan összefüggnek, akkor alkalmas $c_1, \, \dots, \, c_m \in \K$, $|c_1| + \dots + |c_n| > 0$ együtthatókkal
\[
	\sum_{k=1}^m c_k \cdot \chi_k \equiv 0.
\]
Speciálisan minden $\tau \in I$ helyen is
\[
	\sum_{k=1}^m c_k \cdot \chi_k(\tau) = 0 \quad (\in \K^n).
\]
Így a $\chi_1(\tau), \, \dots, \, \chi_m(\tau)$ vektorok összefüggő rendszert alkotnak a $\K^n$-ben.\\

Fordítva, legyen $\tau \in I$, és tegyük fel, hogy a $\chi_1(\tau), \, \dots, \, \chi_m(\tau)$ vektorok összefüggnek. Ekkor az előbbi (nem csupa nulla) $c_1, \, \dots, \, c_m \in \K$ együtthatókkal
\[
	\sum_{k=1}^m c_k \cdot \chi_k(\tau) = 0.
\]
Már tudjuk, hogy
\[
	\phi := \sum_{k=1}^m c_k \cdot \chi_k \in \mathcal{M}_h,
\]
ezért az így definiált $\phi : I \to \K^n$ függvény megoldása a
\[
	\p' = A \cdot \p, \, \p(\tau) = 0
\]
homogén lineáris differenciálegyenletre vonatkozó kezdetiérték-problémának. Világos ugyanakkor, hogy a $\Psi \equiv 0$ is a most mondott k.é.p. megoldása az $I$-n. Azt is tudjuk azonban, hogy (ld. fent) ez a k.é.p. (is) egyértelműen oldható meg, ezért $\phi \equiv \Psi \equiv 0$. Tehát a $\chi_1, \, \dots, \, \chi_m$ függvények is összefüggnek.\\

Ezzel egyúttal azt is beláttuk, hogy az $\mathcal{M}_h$ vektortér véges dimenziós és a $\dim \mathcal{M}_h$ dimenziója legfeljebb $n$.\\

Tekintsük most a
\[
	\p' = A \cdot \p, \, \p(\tau) = e_i  \quad (i = 1, \, \dots, \, n)
\]
kezdetiérték-problémákat, ahol az $e_i \in \K^n \quad (i = 1, \, \dots, \, n)$ vektorok a $\K^n$ tér "szokásos" (kanonikus) bázisvektorait jelölik. Ha
\[
	\chi_i : I \to \K^n
\]
jelöli az említett k.é.p. teljes megoldását, akkor a
\[
	\chi_i(\tau) = e_i \quad (i = 1, \, \dots, \, n)
\]
vektorok lineárisan függetlenek. Így az előbbiek alapján a $\chi_1, \, \dots, \, \chi_n$ függvények is azok. Tehát az $\mathcal{M}_h$ dimenziója legalább $n$, azaz a fentiekre tekintettel $\dim \mathcal{M}_h = n$.\\

A 2. állítás igazolásához legyen $\chi \in \mathcal{M}_h$. Ekkor $\psi + \chi \in D$, és
\[
	(\psi + \chi)' = \psi' + \chi' = A \cdot \psi + b + A \cdot \chi = A \cdot (\psi + \chi) + b,
\]
amiből $\psi + \chi \in \mathcal{M}$ következik. Ha most egy $\p \in \mathcal{M}$ függvényből indulunk ki és $\chi := \p - \psi$, akkor $\chi \in D$, és 
\[
	\chi' = \p' - \psi' = A \cdot \p + b - (A \cdot \psi + b) = A \cdot (\p - \psi) = A \cdot \chi,
\]
amiből $\chi \in \mathcal{M}_h$ adódik. Tehát $\p = \psi + \chi$ a 2.-ben mondott előállítása a $\p$ függvénynek.\\

A tétel 3. részének a bizonyítása érdekében vezessük be az alábbi jelöléseket, ill. fogalmakat. A
\[
	\phi_k = (\phi_{k1}, \, \dots, \, \phi_{kn}) \quad (k = 1, \, \dots, \, n)
\]
bázisfüggvények mint oszlopvektor-függvények segítségével tekintsük a
\[
	\Phi : I \to \K^{n \times n}
\]
mátrixfüggvényt:
\[
	\Phi := \begin{bmatrix}
		\phi_1 & \cdots & \phi_n
	\end{bmatrix} = \begin{bmatrix}
	\phi_{11} & \phi_{21} & \cdots & \phi_{n1} \\
	\phi_{12} & \phi_{22} & \cdots & \phi_{n2} \\
	\vdots & \vdots & \cdots & \vdots \\
	\phi_{1n} & \phi_{2n} & \cdots & \phi_{nn} \\
	\end{bmatrix}.
\]
Legyen
\[
	\Phi' := \begin{bmatrix}
		\phi'_1 & \cdots & \phi'_n
	\end{bmatrix} = \begin{bmatrix}
		\phi'_{11} & \phi'_{21} & \cdots & \phi'_{n1} \\
		\phi'_{12} & \phi'_{22} & \cdots & \phi'_{n2} \\
		\vdots & \vdots & \cdots & \vdots \\
		\phi'_{1n} & \phi'_{2n} & \cdots & \phi'_{nn} \\
	\end{bmatrix}
\]
a $\Phi$ deriváltja. Ekkor könnyen belátható, hogy
\[
	\Phi' = A \cdot \Phi.
\]
Továbbá tetszőleges $g_1, \, \dots, \, g_n : I \to \K$ differenciálgató függvényekkel a
\[
	g := (g_1, \, \dots, \, g_n) : I \to \K^n
\]
vektorfüggvény differenciálható,
\[
	\psi := \sum_{k=1}^n g_k \cdot \phi_k = \Phi \cdot g,
\]
és
\[
	\psi' = \Phi' \cdot g + \Phi \cdot g' = (A \cdot \Phi) \cdot g + \Phi \cdot g'.
\]
A $\psi \in \mathcal{M}$ tartalmazás nyilván azzal ekvivalens, hogy
\[
	\psi' = (A \cdot \Phi) \cdot g + \Phi \cdot g' = A \cdot \psi + b = A \cdot (\Phi \cdot g) + b = (A \cdot \Phi) \cdot g + b,
\] 
következésképpen azzal, hogy
\[
	\Phi \cdot g' = b.
\]

A 2. pont alapján tetszőleges $x \in I$ helyen a $\phi_1(x), \, \dots, \, \phi_n(x)$ vektorok lineárisan függetlenek, azaz a $\Phi(x)$ mátrix nem szinguláris. A mátrixok inverzének a kiszámítása alapján egyszerűen adódik, hogy a
\[
	\Phi^{-1}(x) := \big(\Phi(x)\big)^{-1} \quad (x \in I)
\]
definícióval értelmezett
\[
	\Phi^{-1} : I \to \K^{n \times n}
\]
mátrixfüggvény komponens-függvényei is folytonosak. Ezért a
\[
	(h_1, \, \dots, \, h_n) := \Phi^{-1} \cdot b : I \to \K^n
\]
függvény is folytonos. Olyan folytonosan differenciálható
\[
	g : I \to \K^n
\]
függvényt keresünk tehát amelyikre $g' = \Phi^{-1} \cdot b$, azaz
\[
	g_i' = h_i \quad (i = 1, \, \dots, \, n).
\]
Ilyen $g_i$ létezik, nevezetesen a (folytonos) $h_i \quad (i = 1, \, \dots, \, n)$ függvények bármelyik primitív függvénye ilyen. $\hfill \blacksquare$