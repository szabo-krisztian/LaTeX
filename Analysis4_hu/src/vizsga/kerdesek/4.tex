\newpage
\section{Vizsgakérdés}
\begin{quote}
	\textit{Lineáris differenciálegyenlet. Az állandók variálásának módszere. A radioaktív bomlás felezési idejének meghatározása.}
\end{quote}

\subsection{Lineáris differenciálegyenlet}
Legyen most $n := 1$ és az $I \subset \R$ egy nyílt intervallum, valamint a
\[
g, \, h : I \to \R
\]
folytonos függvények segítségével
\[
f(x, \, y) := g(x) \cdot y + h(x) \quad \B{(x, \, y) \in I \times \R}.
\]
Ekkor
\[
\varphi'(t) = g(t) \cdot \varphi(t) + h(t) \quad (t \in \Dp).
\]
Ezt a feladatot \textit{lineáris differenciálegyenletnek} nevezzük. \\

Ha valamilyen $\tau \in I, \, \xi \in \R$ mellett
\[
\tau \in \Dp, \, \varphi(\tau) = \xi,
\]
akkor az illető lineáris differenciálegyenletre vonatkozó kezdetiérték-problémáról beszélünk.\\

Tegyük fel, hogy a $\theta$ függvény is és a $\psi$ függvény is megoldása a lineáris d.e.-nek és $\mathcal{D}_\theta \cap \mathcal{D}_\psi \neq \emptyset$. Ekkor
\[
(\theta - \psi)'(t) = g(t) \cdot \big( \theta(t) - \psi(t) \big) \quad (t \in \mathcal{D}_\theta \cap \mathcal{D}_\psi).
\]
Így a $\theta - \psi$ függvény megoldása annak a lineáris d.e.-nek, amelyben $h \equiv 0$:
\[
\varphi'(t) = g(t) \cdot \varphi(t) \quad (t \in \Dp).
\]
Ez utóbbi feladatot \textit{homogén lineáris differenciálegyenletnek} fogjuk nevezni. (Ennek megfelelően a szóban forgó lineáris differenciálegyenlet \textit{inhomogén}, ha a benne szereplő $h$ függvény vesz fel 0-tól különböző értéket is.)\\

\tikz \node[theorem]
{
	\textbf{Tétel.} Minden lineáris differenciálegyenletre vonatkozó kezdetiérték-probléma megoldható, és tetszőleges $\varphi, \, \psi$ megoldásaira
	\[
	\varphi(t) = \psi(t) \quad (t \in \mathcal{D}_\varphi \cap \mathcal{D}_\psi).
	\]
};\\

\textbf{Bizonyítás.} Legyen a
\[
G : I \to \R
\]
olyan függvény, amelyik differenciálható és $G' = g$ (a $g$-re tett  feltételek miatt ilyen $G$ primitív függvény van). Ekkor a
\[
\varphi_0(t) := e^{G(t)} \quad (t \in I)
\]
(csak pozitív értékeket felvevő) függvény megoldása az előbb említett homogén lineáris differenciálegyenletnek:
\[
\varphi_0'(t) = G'(t) \cdot e^{G(t)} = g(t) \cdot \varphi_0(t) \quad (t \in I).
\]
Tegyük fel most azt, hogy a
\[
\chi \in I \to \R
\]
függvény is megoldása a szóban forgó homogén lineáris differenciálegyenletnek:
\[
\chi'(t) = g(t) \cdot \chi(t) \quad (t \in \mathcal{D}_\chi).
\]
Ekkor a differenciálható
\[
\frac{\chi}{\varphi_0} : \mathcal{D}_\chi \to \R
\]
függvényre azt kapjuk, hogy bármelyik $t \in \mathcal{D}_\chi$ helyen
\[
\left( \frac{\chi}{\varphi_0} \right)'(t) = \frac{\chi'(t) \cdot \varphi_0(t) - \chi(t) \cdot \varphi_0'(t)}{\varphi_0^2(t)} =
\]
\[
\frac{g(t) \cdot \chi(t) \cdot \varphi_0(t) - \chi(t) \cdot g(t) \cdot \varphi_0(t)}{\varphi_0^2(t)} = 0,
\]
azaz (lévén a $\mathcal{D}_\chi$ nyílt intervallum) egy alkalmas $c \in \R$ számmal
\[
\frac{\chi(t)}{\varphi_0(t)} = c \quad (t \in \mathcal{D}_\chi).
\]
Más szóval, az illető homogén lineáris differenciálegyenlet bármelyik
\[
\chi \in I \to \R
\]
megoldása a következő alakú:
\[
\chi(t) = c \cdot \varphi_0(t) \quad (t \in \mathcal{D}_\chi),
\]
ahol $c \in \R$. Nyilván minden ilyen $\chi$ függvény -- könnyen ellenőrizhető módon -- megoldása a mondott homogén lineáris differenciálegyenletnek.\\

Ha tehát a fenti (inhomogén) lineáris differenciálegyenletnek a $\theta$ függvény is és a $\psi$ függvény is megoldása és $\mathcal{D}_\theta \cap \mathcal{D}_\psi \neq \emptyset$, akkor egy alkalmas $c \in \R$ együtthatóval
\[
\theta(t) - \psi(t) = c \cdot \varphi_0(t) \quad (t \in \mathcal{D}_\theta \cap \mathcal{D}_\psi).
\]
Mutassuk meg, hogy van olyan differenciálható
\[
m : I \to \R
\]
függvény, hogy az $m \cdot \varphi_0$ függvény megoldása a most vizsgált (inhomogén) lineáris differenciálegyenletnek (\textit{az állandók variálásának módszere}). Ehhez azt kell "biztosítani", hogy
\[
(m \cdot \varphi_0)' = g \cdot m \cdot \varphi_0 + h,
\]
azaz
\[
m' \cdot \varphi_0 + m \cdot \varphi_0' = m' \cdot \varphi_0 + m \cdot g \cdot \varphi_0 = g \cdot m \cdot \varphi_0 + h.
\]
Innen szükséges feltételként az adódik az $m$-re, hogy
\[
m' = \frac{h}{\varphi_0}.
\]
Ilyen $m$ függvény valóban létezik, mivel a
\[
\frac{h}{\varphi_0} : I \to \R
\]
folytonos leképezésnek van primitív függvénye. Továbbá -- az előbbi rövid számolás "megfordításából" -- azt is beláthatjuk, hogy a $h / \varphi_0$ függvény bármelyik $m$ primitív függvényét is véve, az $m \cdot \varphi_0$ függvény megoldása a lineáris differenciálegyenletünknek.\\

Összefoglalva az eddigieket azt mondhatjuk, hogy a fenti lineáris differenciálegyenletnek van megoldása, és tetszőleges $\varphi \in I \to \R$ megoldása
\[
\varphi(t) = m(t) \cdot \varphi_0(t) + c \cdot \varphi_0(t) \quad (t \in \Dp)
\]
alakú, ahol $m$ egy tetszőleges primitív függvénye a $h / \varphi_0$ függvénynek. Sőt, az is kiderül, hogy akármilyen $c \in \R$ és $J \subset I$ nyílt intervallum esetén a
\[
\varphi(t) := m(t) \cdot \varphi_0(t) + c \cdot \varphi_0(t) \quad (t \in J)
\]
függvény megoldás. Ez  megint csak egyszerű behelyettesítéssel ellenőrizhető:
\[
\varphi'(t) = m'(t) \cdot \varphi_0(t) + (c + m(t)) \cdot \varphi_0'(t) =
\]
\[
\frac{h(t)}{\varphi_0(t)} \cdot \varphi_0(t) + (c + m(t)) \cdot g(t) \cdot \varphi_0(t) = g(t) \cdot \varphi(t) + h(t) \quad (t \in J).
\]
Speciálisan az "egész" $I$ intervallumon értelmezett
\[
\psi_c(t) := m(t) \cdot \varphi_0(t) + c \cdot \varphi_0(t) \quad (c \in \R, \, t \in I)
\]
megoldások olyanok, hogy bármelyik $\varphi$ megoldásra egy alkalmas $c \in \R$ mellett
\[
\varphi(t) = \psi_c(t) \quad (t \in \Dp),
\]
azaz a $J := \Dp$ jelöléssel $\varphi = \psi_{c_{|_J}}$.\\

Ha $\tau \in I, \, \xi \in \R$, és a $\varphi(\tau) = \xi$ kezdetiérték-feladatot kell megoldanunk, akkor a
\[
c := \frac{\xi - m(\tau) \cdot \varphi_0(\tau)}{\varphi_0(\tau)}
\]
választással a szóban forgó kezdetiérték-probléma
\[
\psi_c : I \to \R
\]
megoldását kapjuk. Mivel a fentiek alapján a szóban forgó k.é.p. minden $\varphi, \, \psi$ megoldására $\varphi = \psi_{c_{|_{\Dp}}}$ és $\psi = \psi_{c_{|_{\mathcal{D}_\psi}}}$, ezért egyúttal az is teljesül, hogy
\[
\varphi(t) = \psi(t) \quad (t \in \Dp \cap \mathcal{D}_\psi).
\]
$\hfill \blacksquare$

A tétel bizonyításából a következők is kiderültek: legyen
\[
	\mathcal{M} := \{ \p : I \to \R : \p \in D, \, \p'(t) = g(t) \cdot \p(t) + h(t) \quad (t \in I) \},
\]
\[
	\mathcal{M}_h := \{ \p : I \to \R : \p \in D, \, \p'(t) = g(t) \cdot \p(t) \quad (t \in I) \}.
\]
Ekkor
\[
	\mathcal{M}_h = \{ c \cdot \p_0 : c \in \R \}
\]
(azaz algebrai nyelven mondva az $\mathcal{M}_h$ egy 1 dimenziós vektortér), és
\[
	\mathcal{M} = m \cdot \p_0 + \mathcal{M}_h := \{ \p + m \cdot \p_0 : \p \in \mathcal{M}_h \}.
\]
Itt $m \cdot \p_0$ helyébe bármelyik $\psi \in \mathcal{M}$ (ún. \textit{partikuláris megoldás}) írható, így
\[
	\mathcal{M} = \psi + \mathcal{M}_h = \{ \p + \psi : \p \in \mathcal{M}_h \}.
\]

\subsection{Radioaktív bomlás}
\textit{Radioaktív anyag bomlik, a bomlási sebesség egyenesen arányos a még fel nem bomlott anyag mennyiségével. A bomlás kezdetétől számítva mennyi idő alatt bomlik el az anyag fele?}\\

Legyen $m_0$ az anyag eredeti, $\p(t)$ pedig a $t \quad (t \in \R)$ időpontban még el nem bomlott anyag mennyisége. A feladatban szereplő arányossági tényező $0 < \alpha \in \R$. Ekkor
\[
	\p'(t) = -\alpha \p(t) \quad (t \in \R),
\]
ahol $\p(0) = m_0.$ A $T$ \textit{(felezési időt)} keressük, amikor is $\p(T) = m_0 / 2$.\\

Ez egy homogén lineáris differenciálegyenlet, ahol $g \equiv -\alpha$. Ezért (pl.)
\[
	G(t) = -\alpha t \quad (t \in I).
\]
valamint
\[
	\p_0(t) = e^{- \alpha t} \quad (t \in I),
\]
ill.
\[
	\p(t)  = c e ^{- \alpha t } \quad (t \in I, \, c \in \R).
\]
Mivel
\[
	m_0 = \p(0) = c,
\]
ezért
\[
	\p(t) = m_0 e^{- \alpha t} \quad (t \in I).
\]
A $T$ definíciója alapján
\[
	\p(T) = m_0 e^{- \alpha T} = \frac{m_0}{2},
\]
azaz $e^{- \alpha T} = 1 /2$. Innen
\[
	T = \frac{\ln 2}{\alpha}.
\]