\newpage
\section{Vizsgakérdés}
\begin{quote}
	\textit{Alaprendszer, alapmátrix. Az állandók variálásának a módszere. Alapmátrix előállítása állandó együtthatós diagonalizálható mátrix esetén. Az $n=2$ eset vizsgálata tetszőleges, állandó együtthatós mátrixra.}
\end{quote}

\subsection{Alaprendszer, alapmátrix}

Az $\mathcal{M}_h$ vektortérben minden bázist az illető egyenlet \textit{alaprendszerének} nevezünk. Ha $\phi_1, \, \dots, \, \phi_n \in \mathcal{M}_h$ egy alaprendszer, akkor a
\[
	\Phi := \begin{bmatrix}
		\phi_1 & \cdots & \phi_n
	\end{bmatrix} : I \to \K^{n \times n}
\]
mátrixfüggvény a szóban forgó lineáris differenciálegyenlet (egy) ún. \textit{alapmátrixa}. Tehát
\[
	\mathcal{M}_h = \left\{ \sum_{k=1}^n c_k \cdot \phi_k : c_1, \, \dots, \, c_n \in \K \right\} = \{ \Phi \cdot c : c \in \K^n \}.
\]
Az $\mathcal{M} = \psi + \mathcal{M}_h$ előállításban minden $\psi \in \mathcal{M}$ függvényt \textit{partikuláris megoldásként} említünk.

\subsection{Állandók variálásának módszere}
Ld. ~\ref{subsec:diff_systems_core} alcímben tárgyalt tétel 3. pontjának bizonyítása. A partikuláris megoldás
\[
	\psi = \Phi \cdot g
\]
alakban való előállítása (alkalmas $g : I \to \K^n$ differenciálható függvénnyel) az ~\ref{subsec:diff_systems_core} alcímben tárgyalt tétel bizonyításában bemutatott módszer az \textit{állandók variálása}. Tetszőleges $\phi_1, \, \dots, \, \phi_n$ alaprendszerrel és $\psi \in \mathcal{M}$ partikuláris megoldással
\[
	\mathcal{M} = \Big\{ \psi + \sum_{k=1}^n c_k \cdot \phi_k : c_1, \, \dots, \, c_n \in \K  \Big\}.
\]
Ha $\Phi$ egy alapmátrix, akkor ugyanez a következőképpen írható:
\[
	\mathcal{M} = \{ \psi + \Phi \cdot c : c \in \K^n \} = \{ \Phi \cdot (g + c) : c \in \K^n \}.
\]

\subsection{Állandó együtthatós diagonalizálható eset}

Legyen most
\[
f(x, \, y) := A \cdot y + b(x) \quad \big( (x, \, y) \in I \times \K^n \big),
\]
ahol $1 \leq n \in \N, \, I \subset \R$ nyílt intervallum mellett
\[
A \in \R^{n \times n}, \, b : I \to \R^n, \, b \in C.
\]
Tegyük fel, hogy $A$ diagonalizálható, azaz létezik $T \in \K^{n \times n}, \, \det T \neq 0$, hogy $T^{-1}AT$ mátrix diagonális: alkalmas $\lambda_1, \, \dots, \, \lambda_n \in \K$ számokkal
\[
\Lambda := T^{-1}AT = \begin{bmatrix}
	\lambda_1 & 0 & \cdots & 0 \\
	0 & \lambda_2 & \cdots & 0 \\
	\vdots & \vdots & \cdots & \vdots \\
	0 & 0 & \cdots & \lambda_n
\end{bmatrix}.
\]
$T$ invertálhatósága miatt a
\[
T = [t_1 \cdots t_n]
\]
$t_i$ $(i=1, \, \dots, \, n)$ oszlopvektorok lineárisan függetlenek, azaz
\[
AT = [At_1 \cdots At_n] = T\Lambda = [\lambda_1 \cdot t_1 \cdots \lambda_n \cdots t_n]
\]
miatt
\[
A \cdot t_i = \lambda_i \cdot t_i \quad (i = 1, \, \dots, \, n).
\]
Mivel
\[
t_i \neq 0 \quad (i = 1, \, \dots, \, n),
\]
ezért mindez röviden azt jelenti, hogy a $\lambda_1, \, \dots, \, \lambda_n$ számok az $A$ mátrix sajátértékei, a $t_1, \, \dots, \, t_n$ vektorok pedig rendre a megfelelő sajátvektorok. Lévén, a $t_i$-k lineárisan függetlenek, az $A$-ra vonatkozó feltételünk úgy fogalmazható, hogy van a $\K^n$-ben (az $A$ sajátvektoraiból álló) sajátvektorbázis.\\

A homogén egyenlet tehát a következőképpen írható fel:
\[
\varphi' = A \cdot \varphi = T \Lambda T^{-1} \cdot \varphi,
\]
amiből
\[
(T^{-1}\varphi)' = \Lambda \cdot (T^{-1} \varphi)
\]
következik. Vegyük észre, hogy ha $\varphi \in \mathcal{M}_h$, akkor a $\psi := T^{-1}\varphi$ függvény megoldása a $\Lambda$ diagonális mátrix által meghatározott állandó együtthatós homogén lineáris egyenletnek. Ez utóbbit ez előző tétel alapján nem nehéz megoldani. Legyenek ui. a
\[
\psi_i : I \to \K^n \quad (i = 1, \, \dots, \, n)
\]
függvények a következők:
\[ 
\psi_i(x) := e^{\lambda_i \cdot x} \cdot e_i \quad (x \in I, \, i = 1, \, \dots, \, n).
\]
Világos, hogy $\psi_i \in D$ és
\[
\psi_i'(x) = \lambda_i \cdot e^{\lambda_i \cdot x} \cdot e_i = e^{\lambda_i \cdot x} \cdot (\Lambda \cdot e_i) =
\]
\[
\Lambda \cdot (e^{\lambda_i \cdot x} \cdot e_i) = \Lambda \cdot \psi_i(x) \quad (x \in I, \, i = 1, \, \dots, \, n).
\]
Más szóval a $\psi_i$-k valóban megoldásai a $\Lambda$ által meghatározott homogén lineáris differenciálegyenletnek. Mivel bármely $\tau \in I$ esetén a
\[
\psi_i(\tau) = e^{\lambda_i \cdot \tau} \cdot e_i \quad (i = 1, \, \dots, \, n)
\]
vektorok nyilván lineárisan függetlenek, ezért az előző tétel bizonyításában mondottak szerint a $\psi_i$ $(i = 1, \, \dots, \, n)$ függvények lineárisan függetlenek. Ha
\[
\phi_i := T \cdot \psi_i \quad (i = 1, \, \dots, \, n),
\]
akkor nyilván a $\phi_i$-k is lineárisan függetlenek,
\[
\phi_i(x) = e^{\lambda_i \cdot x} \cdot t_i \quad (x \in I, \, i = 1, \, \dots, \, n),
\]
és minden $i = 1, \, \dots, \, n$ indexre
\[
\phi_i' = A \cdot \phi_i.
\]
Tehát $\phi_i \in \mathcal{M}_h$ $(i = 1, \, \dots, \, n)$ egy bázis. Ezzel beláttuk az alábbi tételt:\\

\tikz \node[theorem]
{
	\textbf{Tétel.} Tegyük fel, hogy az $A \in \R^{n \times n}$ mátrix diagonalizálható. Legyenek a sajátértékei $\lambda_1, \, \dots, \, \lambda_n \in \K$, egy-egy megfelelő sajátvektora pedig $t_1, \, \dots, \, t_n \in \K^n$. Ekkor a
	\[
	\varphi' = A \cdot \varphi
	\]
	homogén lineáris differenciálegyenletnek a
	\[
	\phi_i(x) := e^{\lambda_i \cdot x} \cdot t_i \quad (x \in \R, \, i = 1, \, \dots, \, n)
	\]
	függvények lineárisan független megoldásai.
};

\subsection{Tetszőleges állandó együtthatós mátrix}

Tekintsük az $n=2$ esetet, amikor is valamilyen $a, \, b, \, c, \, d \in \R$ számokkal
\[
	A = \begin{bmatrix}
		a & b \\
		c & d
	\end{bmatrix} \in \R^{2 \times 2}.
\]
Könnyű meggyőződni arról, hogy ez a mátrix pontosan akkor nem diagonalizálható, ha
\[
	(a-d)^2 + 4bc = 0 \text{ és } |b| + |c| > 0.
\]
Ekkor egyetlen sajátértéke van az $A$-nak nevezetesen
\[
	\lambda := \frac{a+d}{2},
\]

legyen a $t_1$ egy hozzá tartozó sajátvektor:
\[
	0 \neq t_1 \in \R^2, \, At_1 = \lambda t_1.
\]
Egyszerű számolással igazolható olyan $t_2 \in \R^2$ vektor létezése, amelyik lineárisan független a $t_1$-től és
\[
	At_2 = t_1 + \lambda t_2.
\]
Ha mármost a $T \in \R^{2 \times 2}$ mátrix oszlopvektorai rendre a $t_1, \, t_2$ vektorok: $T := [t_1 \, t_2]$, akkor
\[
	T^{-1}AT = \begin{bmatrix}
		\lambda & 1 \\
		0 & \lambda
	\end{bmatrix}.
\]
Ezt felhasználva könnyen belátható, hogy a
\[
	\phi_1(x) := e^{\lambda x} \cdot t_1, \, \phi_2(x) := e^{\lambda x} \cdot (t_2 + xt_1) \quad (x \in \R)
\]
függvénypár egy alaprendszer. Valóban, $\phi_i \in \mathcal{M}_h \quad (i = 1, \, 2)$, mert egyrészt
\[
	\phi_1'(x) = \lambda e^{\lambda x} \cdot t_1 = e^{\lambda x} A t_1 = A(e^{\lambda x} \cdot t_1) = A\phi_1(x),
\]
másrészt
\[
	\phi_2'(x) = \lambda e^{\lambda x} \cdot t_2 + e^{\lambda x} \cdot t_1 + \lambda e^{\lambda x}x \cdot t_1 = e^{\lambda x} \big( (t_1 + \lambda \cdot t_2) + \lambda x \cdot t_1 \big) =
\]
\[
	e^{\lambda x} (At_2 + xAt_1) = A\big( e^{\lambda x}(t_2 + x \cdot t_1) \big) = A \phi_2(x) \quad (x \in \R).
\]
Mivel a
\[
	\phi_1(0) = t_1, \, \phi_2(0) = t_2
\]
vektorok lineárisan függetlenek, ezért a $\phi_1, \, \phi_2$ függvények is lineárisan függetlenek, azaz a $\phi_1, \, \phi_2$ egy alaprendszer.

\subsection{Valós értékű megoldások}
Tegyük fel, hogy a $\lambda \in \K$ szám az $A \in \R^{n \times n}$ (diagonalizálható) együtthatómátrixnak egy sajátértéke, a $t_\lambda \in \K^n$ vektor pedig egy $\lambda$-hoz tartozó sajátvektor. Ha a $\lambda$ valós, akkor nyilván a $t_\lambda$ sajátvektor is választható "valósnak", azaz feltehető, hogy $t_\lambda \in \R^n$. Ebben az esetben az $\mathcal{M}_h$-beli
\[
	\phi_\lambda(x) := e^{\lambda x} \cdot t_\lambda \quad (x \in \R)
\]
bázisfüggvény is "valós" tehát $\phi_\lambda : \R \to \R^n$.\\

Ha viszont a $\lambda$ (nem valós) komplex szám, azaz
\[
	\lambda = u + \imath v \in \mathbf{C} \, \backslash \, \R
\]
(alkalmas $u \in \R$ és $0 \neq v \in \R$ számokkal), akkor -- lévén az $A$ karakterisztikus polinomja valós együtthatós -- az $A$-nak egyúttal a
\[
	\overline{\lambda} = u + \imath v
\]
(komplex konjugált) is (ugyanannyiszoros) sajátértéke. Hasonlóan, ha a
\[
	t_\lambda = S_\lambda + \imath Y_\lambda \in \K^n
\]
vektor (alkalmas $S_\lambda, \, Y_\lambda \in \R^n$ vektorokkal) az $A$-nak a $\lambda$-hoz tartozó sajátvektora, akkor a
\[
	\overline{t}_\lambda = S_\lambda - \imath Y_\lambda
\]
vektor a $\overline{\lambda}$-hoz tartozó sajátvektor. Továbbá a megfelelő bázisfüggvények a következők:
\[
	\phi_\lambda(x) := e^{\lambda x} \cdot t_\lambda, \, \phi_{\overline{\lambda}}(x) := e^{\overline{\lambda}x} \cdot \overline{t}_\lambda \quad (x \in \R).
\]
Rövid számolással ellenőrizhető, hogy
\[
	\phi_{\overline{\lambda}}(x) = \overline{\phi_\lambda(x)} \quad (x \in \R).
\]
Mivel a homogén egyenlet (teljes) megoldásainak az $\mathcal{M}_h$ halmaza a $\K$ felett vektortér, ezért a
\[
	\frac{\phi_\lambda + \overline{\phi_\lambda}}{2} = \text{Re}\, \phi_\lambda, \, \frac{\phi_\lambda - \overline{\phi_\lambda}}{2} = \text{Im} \, \phi_\lambda
\]
függvények is $\mathcal{M}_h$-beliek. Világos, hogy
\[
	e^{\lambda x} = e^{u x} \cdot \B{ \cos(vx) + \imath \sin(vx) } \quad (x \in \R)
\]
miatt
\[
	\phi_{\lambda, \, r}(x) := \text{Re} \, \phi_\lambda(x) = e^{ux} \cdot \B{ \cos(vx) \cdot S_\lambda - \sin(vx) \cdot Y_\lambda } \quad (x \in \R),
\]
\[
	\phi_{\lambda, \, i}(x) := \text{Im} \, \phi_\lambda(x) = e^{ux} \cdot \B{ \sin(vx) \cdot S_\lambda + \cos(vx) \cdot Y_\lambda } \quad (x \in \R).
\]
Továbbá
\[
	\phi_{\lambda, \, r}(0) = S_\lambda = \frac{t_\lambda + \overline{t_\lambda}}{2},
\]
\[
	\phi_{\lambda, \, i}(0) = Y_\lambda = \frac{t_\lambda - \overline{t_\lambda}}{2 \imath}.
\]
A $t_\lambda, \, \overline{t}_\lambda$ sajátvektorok lineárisan függetlenek, ezért az $S_\lambda, \, Y_\lambda$ vektorok is azok, következésképpen a $\phi_{\lambda, \, r}, \, \phi_{\lambda, \, i}$ függvények is lineárisan függetlenek. Így $\phi_\lambda, \, \phi_{\overline{\lambda}}$ függvényeket kicserélve az előző függvényekre, továbbra is alaprendszert kapunk.
