\documentclass[12pt]{article}
\usepackage[left=0.9in, right=0.9in, top=1in, bottom=1in]{geometry}
\usepackage{tikz}
\usepackage{setspace}
\usepackage{hyperref}
\usepackage{amsfonts, amssymb, amsmath} 
\usepackage{titlesec}
\usepackage{pgfplots}
\pgfplotsset{compat=1.18}
\usepackage{graphicx}
\usepackage{wrapfig}
\usepackage{caption}
\usepackage{enumitem}
\usetikzlibrary{shadows.blur}
\usepackage{lmodern}
\setlength{\parskip}{0pt}
\setlength{\parindent}{0pt}

\title{\textcolor{purple}{\Huge\textbf{Analízis III}}}
\author{Vizsga jegyzet}
\date{Szabó Krisztián}

\renewcommand{\contentsname}{Tartalom}
\newcommand{\R}{\mathbb{R}}
\newcommand{\N}{\mathbb{N}}
\newcommand{\E}{\exists}
\newcommand{\mm}{\mathbf{m}}
\newcommand{\MM}{\mathbf{M}}
\newcommand{\K}{\mathbb{K}}
\newcommand{\D}{\mathcal{D}_f}


\begin{document}
    \maketitle
    \textit{A jegyzet egy az egyben Dr. Simon Péter analízis 3 segédanyagából lett összegyűjtve. Elsősorban magamnak írtam, hogy elősegítse a felkészültést a vizsgára.}
    \tableofcontents
    \newpage

    \section{Metrikus-, normált-, euklideszi-terek}
    \textbf{Teljes vizsgacím:} Metrikus-, normált-, euklideszi-terek. Környezet, belső pont, nyílt halmaz. Torlódási pont, zárt halmaz. Nyílt (zárt) halmazok uniója, metszete. A $(\K^n, \, \varrho_p)$, $(\K^n, \, ||.||_p)$, $(\K^n, \, \langle . \rangle)$, $(C[a, \, b], \, \varrho_p)$, $(C[a, \, b], \, ||.||_p)$, $(0 < n \in \N, \, 1 \leq p \leq + \infty)$ terek.

    \subsection{Metrikus terek}
    Konkrét példák sokasága vezet el a távolság-fogalom absztakciójához: legyen az $X \neq \emptyset$ egy nem üres halmaz, és tegyük fel, hogy a
    \[
        \varrho : X^2 \to [0, \, + \infty)
    \]
    függvény a következő tulajdonságokkal rendelkezik:
    \begin{enumerate}
        \item minden $x \in X$ esetén $\varrho(x, \, x) = 0$;
        \item ha $x, \, y \in X$ és $\varrho(x, \, y) = 0$, akkor $x = y$;
        \item bármely $x, \, y \in X$ választással $\varrho(x, \, y) = \varrho(y, \, x)$;
        \item tetszőleges $x, \, y, \, z \in X$ elemekkel $\varrho(x, \, y) \leq \varrho(x, \, z) + \varrho(y, \, z)$.
    \end{enumerate}
    Azt mondjuk, hogy ekkor a $\varrho$ egy \textit{távolságfüggvény} (vagy idegen szóval \textit{metrika}). Ha $x, \, y \in X$, akkor $\varrho(x, \, y)$ az $x, \, y$ elemek \textit{távolsága}. Az $(X, \, \varrho)$ rendezett párt \textit{metrikus térnek} nevezzük.\\

    Az $X$-beli elemek távolsága tehát nemnegatív szám. Bármely elem önmagától vett távolsága \textit{nulla} (ld. 1.), továbbá két kölönböző elem távolsága mindig \textit{pozitív} (ld. 2.). A távolság \textit{szimmetrikus}, azaz két elem távolsága független az illető elemek sorrendjétől (ld. 3.). A 4. tulajdonságot \textit{háromszög-egyenlőtlenségként} fogjuk idézni.\\

    Mutassuk meg, hogy a háromszög-egyenlőtlenségből annak az alábbi változata is következik:
    \[
        |\varrho(x, \, z) - \varrho(y, \, z)| \leq \varrho(x, \, z) \quad (x, \, y, \, z \in X).
    \]
    Ugyanis a 4. axióma miatt
    \[
        \varrho(x, \, z) \leq \varrho(x, \, y) + \varrho(y, \, z),
    \]
    tehát
    \[
        \varrho(x, \, z) - \varrho(y, \, z) \leq \varrho(x, \, y).
    \]
    Ha itt $x$-et és az $y$-t felcseréljük, akkor a
    \[
        -\big( \varrho(x, \, z) - \varrho(y, \, z) \big) = \varrho(y, \, z) - \varrho(x, \, z) \leq \varrho(y, \, x) = \varrho(x, \, y)
    \]
    egyenlőtlenséghez jutunk. Az utóbbi két becslés egybevetésével kapjuk a jelzett egyenlőtlenséget.\\

    Bármely $X \neq \emptyset$ halmaz esetén megadható
    \[
        \varrho : X^2 \to [0, \, +\infty)
    \]
    távolságfüggvény, ui., pl. a
    \[
        \varrho(x, \, y) :=
        \begin{cases}
        0 & (x = y) \\
        1 & (x \neq y)    
        \end{cases}
        \quad \big((x, \, y) \in X^2\big)
    \]
    leképezés nyilván eleget tesz a fenti, a metrikát meghatározó 1.-4. axiómáknak. Az így definiált $(X, \, \varrho)$ teret \textit{diszkrét} jelzővel illetjük.\\

    Megmutatható, hogy az 1.-3. axiómák nem függetlenek egymástól, nevezetesen: ha egy
    \[
        \varrho : X^2 \to \R
    \]
    függvény rendelkezik az 1., 2., 4., tulajdonságokkal, akkor a $\varrho$ metrika.\\

    \subsubsection{Példák}
    Soroljunk fel néhány példát amelyek nem csupán az analízisben játszanak fontos szerepet.
    \begin{enumerate}
        \item Legyen $1 \leq n \in \N, \, 0 < p < + \infty$, és
        \[
            x = (x_1, \, \dots, \, x_n), \, y = (y_1, \, \dots, \, y_n) \in \K^n
        \]
        esetén definiáljuk az $(x, \, y)$-ban a
        \[
            \varrho_p : \K^n \times \K^n \to [0, \, +\infty)
        \]
        függvény helyettesítési értékét a következőképpen:
        \[
            \varrho_p(x, \, y) :=
            \begin{cases}
                \displaystyle\sum_{i=1}^n |x_i - y_i|^p & (p \leq 1) \\
                \displaystyle\left( \sum_{i=1}^n |x_i - y_i| \right)^{1/p} & (p > 1).
            \end{cases}
        \]
        Terjesszük ki a $\varrho_p$ értelmezését $p = \infty := +\infty$-re is az alábbiak szerint:
        \[
            \varrho_\infty(x, \, y) := \text{max}\{ |x_i - y_i| : i = 1, \, \dots, \, n \}.
        \]
        Belátható, hogy $(\K^n, \, \varrho_p)$ metrikus tér. A későbbiekben a $\varrho_\infty$ metrika mellett a $\K^n$-beli vektorok távolságának a mérésére többnyire a
        \[
            \varrho_2(x, \, y) = \sqrt{\sum_{i=1}^n |x_i - y_i|^2} \quad (x, \, y \in \K^n),
        \]
        \[
            \varrho_1(x, \, y) = \sum_{i=1}^n |x_i - y_i| \quad (x, \, y \in \K^n),
        \]
        metrikákat fogjuk használni. Speciálisan az $n=1$ esetben
        \[
            \varrho_p(x, \, y) = |x - y| \quad (x, \, y \in \K, \, p \geq 1).
        \]
        \item Tekintsük egy $0 < p < + \infty$ mellett az
        \[
            \ell_p := \Big\{ (x_n) : \N \to \K : \sum_{n = 0}^{+\infty} |x_n|^p < + \infty \Big\}
        \]
        halmazokat. Legyen továbbá $x = (x_n), \, y = (y_n) \in \ell_p$ esetén
        \[
            \varrho_p(x, \, y) := 
            \begin{cases}
                \displaystyle \sum_{n=0}^{+\infty} |x_n - y_n|^p & (0 < p \leq 1) \\
                \displaystyle \left( \sum_{n=0}^{+\infty} |x_n - y_n|^p \right)^{1/p} & (p > 1). \\
            \end{cases}
        \]
        Bebizonyítható, hogy az így definiált $\varrho_p$ függvény is metrika, azaz $(\ell_p, \, \varrho_p)$ metrikus tér. A $p = \infty := + \infty$-re való "kiterjeszést" a következőképpen kapjuk:
        \[
            \ell_\infty := \Big\{ (x_n) : \N \to \K : \sup\{|x_n| : n \in \N \} < +\infty \Big\}
        \]
        (más szóval az $\ell_\infty$ szimbólum a korlátos számsorozatok halmazát jelöli), valamint az $\ell_\infty$-beli $x = (x_n), \, y = (y_n)$ elemekre
        \[
            \varrho_\infty(x, \, y) := \text{sup}\big\{ |x_n - y_n| : n \in \N \big\}.
        \]
        A $\varrho_\infty$ függvény is metrika, tehát $(\ell_\infty, \, \varrho_\infty)$ is metrikus tér.

        \item Valamilyen $[a, \, b]$ korlátos és zárt intervallum esetén $(a, \, b \in \R, \, a < b)$ esetén jelöljük $C[a, \, b]$-vel az $[a, \, b]$-n értelmezett, valós értékű és folytonos függvények halmazát. Ha $0 < p \leq + \infty$, akkor tekintsük az 1., 2. példák alábbi "folytonos" változatait: ha $f, \, g \in C[a, \, b]$, akkor
        \[
            \varrho_p(f, \, g) :=
            \begin{cases}
                \displaystyle \int\limits_a^b |f-g|^p & (0 < p \leq 1) \\
                \displaystyle \left( \int\limits_a^b |f-g|^p \right)^{1/p} & (1 < p < + \infty) \\
                \text{max}\{ |f(x) - g(x)| : x \in [a, \, b] \} & (p = \infty := + \infty).
            \end{cases}
        \]
        Az előbbi példákhoz hasonlóan látható be, hogy $(C[a, \, b], \, \varrho_p)$ is metrikus tér.
    \end{enumerate}

    Azt mondjuk, hogy valamilyen $X \neq \emptyset$ halmaz és egy $X^2$-en értelmezett
    \[
        \varrho, \, \sigma : X^2 \to [0, \, +\infty)
    \]
    metrikák esetén a $\varrho$ és a $\sigma$ \textit{ekvivalens}, ha alkalmas $c, C$ pozitív számokkal
    \[
        c \cdot \varrho(x, \, y) \leq \sigma(x, \, y) \leq C \cdot \varrho(x, \, y) \quad (x, \, y \in X).
    \]
    Könnyű belátni, hogy ha $\mathcal{M}$ jelöli az előbb említett metrikák halmazát, és a $\varrho, \, \sigma \in \mathcal{M}$ elemekre $\varrho \sim \sigma$ azt jelenti, hogy a $\varrho$ és a $\sigma$ ekvivalens, akkor az így értelmezett ($\mathcal{M}^2$-beli) $\sim$ reláció ekvivalencia.\\

    Pl. a fenti $(\K^n, \, \varrho_p)$ metrikus terekre a $\varrho_p$ metrikák közül $p \geq 1$ esetén bármelyik kettő ekvivalens. A továbbiakban az
    \[
        X := \K^n \quad (1 \leq n \in \N)
    \]
    esetben a $\varrho_2, \, \varrho_1, \, \varrho_\infty$ metrikák bármelyikét fogjuk használni.

\end{document}
