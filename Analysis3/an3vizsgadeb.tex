\documentclass[12pt]{article}
\usepackage[left=1in, right=1in, top=1in, bottom=1in]{geometry}
\usepackage{tikz}
\usepackage{setspace}
\usepackage{hyperref}
\usepackage{amsfonts, amssymb, amsmath} 
\usepackage[T1]{fontenc}
\usetikzlibrary{shadows.blur}
\usepackage{titlesec}
\usepackage{pgfplots}
\usepackage{graphicx}
\usepackage{wrapfig}

\usepackage{caption}
\usepackage{enumitem}

\title{\textcolor{purple}{\Huge\textbf{\textsc{Analízis III}}}}
\author{Vizsga jegyzet}
\date{Szabó Krisztián}

\newcommand{\nagyszoveg}[1]{\vspace{1em}\Large\textcolor{purple}{\textbf{#1}}\normalsize\vspace{1em}}
\newcommand{\kisszoveg}[1]{\vspace{0.5em}\large\textcolor{purple}{\textbf{#1}}\normalsize\vspace{0.5em}}
\newcommand{\norm}[1]{||#1||}

% Címek formázása
\titleformat{\section}{\color{purple}\normalfont\Large\bfseries}{\thesection}{1em}{}


\newcommand{\R}{\mathbb{R}}
\newcommand{\N}{\mathbb{N}}
\newcommand{\E}{\exists}
\newcommand{\D}{\mathcal{D}_f}
\newcommand{\ba}{\boldmath$a$\unboldmath}
\newcommand{\lila}[1]{{\large\textbf{\textcolor{purple}{#1}}}}

\renewcommand{\contentsname}{Tartalom}


\setlength{\parindent}{0pt}

% Definiáljuk a színeket

\definecolor{modernyellow}{HTML}{F4E4BC} % Modern sárga
\definecolor{moderngreen}{HTML}{BDDCBD}  % Modern zöld

\tikzset
{
	definition/.style={
		draw,
		fill=modernyellow, % Sárga színű keret a definíciókhoz
		line width=1pt,
		rounded corners, % Lekerekített szélek
		drop shadow={shadow blur steps=5,shadow xshift=1ex,shadow yshift=-1ex}, % Árnyék hozzáadása
		text width=0.9\textwidth, % Szöveg szélessége
		inner sep=10pt
	},
	myboxgreen/.style={
		draw,
		fill=moderngreen, % Zöld színű keret a tételekhez
		line width=1pt,
		rounded corners, % Lekerekített szélek
		drop shadow={shadow blur steps=5,shadow xshift=1ex,shadow yshift=-1ex}, % Árnyék hozzáadása
		text width=0.9\textwidth,
		inner sep=10pt
	},
	proof/.style={
		fill=white,
		rectangle,
		drop shadow={shadow blur steps=5,shadow xshift=1ex,shadow yshift=-1ex, moderngreen},
		%rounded corners=3pt,
		text width=0.9\textwidth,
		inner sep=6pt,
	},
	proof1/.style={
		fill=white,
		rectangle,
		drop shadow={shadow blur steps=5,shadow xshift=1ex,shadow yshift=0, moderngreen},
		text width=0.9\textwidth,
		inner sep=6pt,
	}
}

\begin{document}
	\maketitle
	\tableofcontents
	\newpage

    \section{A koordináta-függvények szerepe a differenciálhatóságban. A \textit{Jacobi}-mátrix kiszámítása.}
    \subsection{Koordináta-függvények és a differenciálhatóság kapcsolata}
    
    \begin{center}
        \tikz \node[myboxgreen]
        {
            \textbf{Tétel.} Legyen $1 \leq n, \, m \in \N$. Az
            \[
                f = (f_1, \, \dots, \, f_m) \in \R^n \to \R^m
            \]
            függvény akkor és csak akkor differenciálható az $a \in \text{int} \, \D$ helyen, ha minden $i = 1, \, \dots, \, m$ esetén az
            \[
                f_i \in \R^n \to \R
            \]
            koordináta-függvény differenciálható az $a$-ban. Ha $f \in D\{a\}$, akkor az $f'(a)$ Jacobi-mátrix a következő alakú:
            \[
                f'(a) =
                \begin{bmatrix}
                    \text{grad} \, f_1(a) \\
                    \text{grad} \, f_2(a)  \\
                    \vdots \\
                    \text{grad} \, f_m(a)    
                \end{bmatrix}
            \]
        };    
    \end{center}

    \begin{center}
        \tikz \node[proof1]
        {
            \textbf{Bizonyítás.} Tegyük fel először is azt, hogy $f \in D\{a\}$, és jelöljük az $f'(a) \in \R^{m \times n}$ Jacobi-mátrix sorvektorait $A_i$-vel ($i = 1, \, \dots, \, m$:)
            \[
                f'(a) =
                \begin{bmatrix}
                    A_1 \\
                    A_2 \\
                    \vdots \\
                    A_m
                \end{bmatrix}.
            \]
            Ekkor alkalmas
            \[
                \eta = (\eta_1, \, \dots, \, \eta_m) \in \R^n \to \R^m
            \]
            függvénnyel
            \[
                \eta(h) \to 0 \quad (||h|| \to 0)
            \]
            és a
            \[
                h \in \R^n \quad (a + h \in \D)
            \]
            helyeken
            \[
                f(a + h) - f(a) = \big( f_1(a + h) - f_1(a), \, \dots, \, f_m(a + h) - f_m(a) \big) =
            \]
            \[
                f'(a)\cdot h + \eta(h) \cdot ||h|| =
            \]
            \[
                \big( \langle A_1, \, h\rangle, \, \dots, \, \langle A_m, \, h \rangle \big) + \big( \eta_1(h) \cdot ||h||, \, \dots, \, \eta_m(h) \cdot ||h|| \big).
            \]
        };
    \end{center}

    \newpage
    \begin{center}
        \tikz \node[proof1]
        {
            Következésképpen minden $i = 1, \, \dots, \, m$ mellett az $\eta$ függvény
            \[
                \eta_i \in \R^n \to \R
            \]
            koordináta-függvényeivel
            \[
                \tag{$\star$} f_i(a + h) - f_i(a) = \langle A_i, \, h \rangle + \eta_i(h) \cdot ||h|| \quad (h \in \R^n, \, a + h \in \D).
            \]
            Mivel bármely $i = 1, \, \dots, \, m$ indexre
            \[
                \eta_i(h) \to 0 \quad (||h|| \to 0),
            \]
            ezért az előbbi $(\star)$ összefüggés azt jelenti, hogy $f_i \in D \{a\}$ és
            $A_i = \text{grad} \, f_i(a) \quad (i = 1, \, \dots, \, m)$.
            Most azt tegyük fel, hogy $f_i \in D\{a\} \, \, (i = 1, \, \dots, \, m)$, amikor is valamilyen
            \[
                \eta_i \in \R^n \to \R, \, \eta_i \to 0 \, \, (||h|| \to 0) \quad (i = 1, \, \dots, \, m)
            \]
            függvényekkel
            \[
                f_i(a + h) - f_i(a) = \big\langle  \text{grad} \, f_i(a), \, h \big\rangle + \eta_i(h) \cdot ||h|| \quad (h\in \R^n, \, a + h \in \D, \, i = 1, \, \dots, \, m).
            \]
            Ha tehát
            \[ A :=
                \begin{bmatrix}
                    \text{grad} \, f_1(a) \\
                    \text{grad} \, f_2(a) \\
                    \vdots \\
                    \text{grad} \, f_m(a)
                \end{bmatrix}
                \in \R^{m \times n},
            \]
            akkor az
            \[
                \eta := (\eta_1, \, \dots, \, \eta_m) \in \R^n \to \R^m
            \]
            függvénnyel
            \[
                f(a + h) - f(a) = A\cdot h + \eta(h) \cdot ||h||,
            \]
            ahol $\eta(h) \to 0 \, \, (||h|| \to 0)$. Ezért $f \in D\{a\}$ és $f'(a) = A.$
            
            $\hfill \blacksquare$
        };
    \end{center}
    
    

    \newpage
    \section{Többször differenciálható függvények. Young-tétel.}
    \subsection{$f \in \R^n \to \R$ függvények másodrendű differenciálhatósága}
    \begin{center}
        \tikz \node[definition]
        {
            \textbf{Definíció.} Legyen valamilyen $1 \leq n \in \N$ esetén $f \in \R^n \to \R$. Tegyük fel, hogy $a \in \text{int}\, \D$. Azt mondjuk, hogy az $f$ függvény \textit{kétszer differenciálható} az $a$-ban ha minden $x \in K(a) \subset \D$ esetén $f \in D\{x\}$, és
            \[
                \partial_if \in D\{a\} \quad (i = 1, \, \dots, \, n).
            \]
        };    
    \end{center}
    
    Ha a fenti feltételek teljesülnek akkor léteznek a
    \[
        \partial_j(\partial_if)(a) \quad (i, \, j = 1, \, \dots, \, n)
    \]
    parciális deriváltak. Ehhez persze nem szükséges, hogy a $\partial_if \, \, (i = 1, \, \dots, \, n)$ függvények deriválhatók legyenek az $a$ helyen.
    Ha tehát a fenti
    \[
        f \in \R^n \to \R
    \]
    függvényre $f \in D^2\{a\}$, akkor minden $i, \, j = 1, \, \dots, \, n$ mellett létezik a $\partial_{ij}f(a)$ másodrendű parciális derivált. Az
    \[
        f''(a) := \big( \partial_{ij}f(a) \big)_{i, \, j = 1}^n =
        \begin{bmatrix}
            \partial_{11}f(a) & \partial_{12}f(a) & \dots & \partial_{1n}f(a) \\
            \partial_{21}f(a) & \partial_{22}f(a) & \dots & \partial_{2n}f(a) \\
            \vdots & \vdots & \dots & \vdots \\
            \partial_{n1}f(a) & \partial_{n2}f(a) & \dots & \partial_{nn}f(a)
        \end{bmatrix}
        \in \R^{n \times n}
    \]
    mátrixot az $f$ függvény $a$-beli \textit{másodrendű deriváltmátrixnának nevezzük}. A későbbiekben tárgyalandó Young-tétel miatt ez egy szimmetrikus mátrix. 

    \subsection{$f \in \R^n \to \R$ függvények magasabb rendű differenciálhatósága}
    \begin{center}
        \tikz \node[definition]
        {
            \textbf{Definíció.} Legyen valamilyen $1 \leq n \in \N$ esetén $f \in \R^n \to \R$. Tegyük fel, hogy $a \in \text{int} \, \D, \, 1 \leq s \in \N$, továbbá egy alkalmas $K(a) \subset \D$ környezettel minden $x \in K(a)$ pontban az $f$ függvény $s$-szer differenciálható: $f \in D^s\{x\}$. Belátható, hogy ekkor a $K(a)$ pontjaiban az $f$ összes $s$-edrendű parciális deriváltja létezik. Azt mondjuk, hogy az $f$ \textit{függvény az $a$-ban $(s+1)$-szer differenciálható}, ha minden $s$-edrendű parciális deriváltfüggvénye differenciálható az $a$-ban.
        };        
    \end{center}

    \subsection{$f \in \R^n \to \R^m$ függvények magasabb rendű differenciálhatósága}
    \begin{center}
        \tikz \node[definition]
        {
            \textbf{Definíció.} Legyen $1 \leq n, \, m \in \N$ és
            \[
                f = (f_1, \, \dots, \, f_m) \in \R^n \to \R^m, \, a \in \text{int} \, \D,
            \]
            ill. $1 \leq k \in \N$. Azt mondjuk, hogy az \textit{$f$ függvény $k$-szor differenciálható az $a$-ban}, ha
            \[
                f_j \in D^k\{a\} \quad (j = 1, \, \dots, \, m).
            \]
        };
    \end{center}

    \subsection{Young-tétel}
    \begin{center}
        \tikz \node[myboxgreen]
        {
            \textbf{Tétel.} Legyen $2 \leq n \in \mathbb{N}, \, f \in \R^n \to \R, \, a \in \text{int} \, \D, \, 2 \leq s \in \mathbb{N}$ és $f \in D^s\{a\}$. Ekkor tetszőleges $k_1, \, \dots, \, k_s \in \{1, \, \dots, \, n\}$ indexek esetén ezek bármely $j_1, \, \dots, \, j_s$ permutációjára
            \[
                \partial_{k_1 \, \dots \, k_s}f(a) = \partial_{j_1 \, \dots \, j_s}f(a).
            \]
        };
    \end{center}

    \begin{center}
        \tikz \node[proof1]
        {
            \textbf{Bizonyítás.} Az $s$-szerinti teljes indukcióra gondolva elegendő az $s = 2$ esettel foglalkoznunk. Ekkor tehát azt kell belátnunk, hogy ha $f \in D^2\{a\}$, akkor
            \[
                \partial_{ij}f(a) = \partial_{ji}f(a) \quad (i, \, j = 1, \, \dots, \, n).
            \]
            Világos, hogy csak az $i \neq j$ eset az "érdekes". Ezen túl (könnyen meggondolhatóan) azt is feltehetjük, hogy $n = 2$. Más szóval az
            \[
                f \in \R^2 \to \R
            \]
            függvényekre
            \[
                a = (a_1, \, a_2) \in \text{int}\, \mathcal{D}_f, \, f \in D^2\{a\},
            \]
            és ennek alapján azt kell bebizonyítanunk, hogy
            \[
            \partial_{12}f(a) = \partial_{21}f(a).
            \]
            Legyen ehhez $r > 0$ olyan, amellyel ($\R^n$-ben a $||.|| := ||.||_\infty$ normát választva)
            \[
                K(a) = \{ x \in \R^2 : ||x-a|| < r \} \subset \mathcal{D}_f,
            \]
            és vezessük be az alábbi jelölést: az $u, \, v \in (-r, \, r)$ helyeken
            \[
                \Delta(u, \, v) := f(a_1 + u, \, a_2 + v) - f(a_1 + u, \, a_2) + f(a_1, \, a_2) - f(a_1, \, a_2 + v).
            \]
        };
    \end{center}

    \newpage
    \begin{center}
        \tikz \node[proof1]
        {
            Ha rögzítjük a $v \in (-r, \, r)$ számot, akkor a
            \[
                \varphi(u) := f(a_1 + u, \, a_2 + v) - f(a_1 + u, \, a_2) \quad \big( u \in (-r, \, r) \big)
            \]        
            függvénnyel
            \[
                \Delta(u, \, v) = \varphi(u) - \varphi(0) \quad \big( u \in (-r, \, r) \big).
            \]
            Az $f \in D^2\{a\}$ feltétel miatt az előbbi $K_r(a)$ környezettől azt is megkövetelhetjük, hogy egyrészt minden $x \in K_r(a)$ helyen $f \in D\{x\}$ (így egyúttal léteznek az $\partial_1f(x), \, \partial_2f(x)$ parciális deriváltak is), másrészt
            \[
                \partial_1f, \, \partial_2f \in D \{a\}.
            \]
            Következésképpen a most definiált
            \[
                \varphi : (-r, \, r) \to \R
            \]
            függvény differenciálható, ezért a Lagrange-középérték-tétel alapján
            \[
                \varphi(u) - \varphi(0) = \varphi'(\xi)\cdot u \quad \big( u \in (-r, \, r) \big),
            \]
            ahol $\xi \in (0, \, u)$ (vagy $\xi \in (u, \, 0)$). A parciális deriváltak definíciójára gondolva
            \[
                \varphi'(u) = \partial_1f(a_1 + u, \, a_2 + v) - \partial_1f(a_1 + u, \, a_2) \quad \big( u \in (-r, \, r) \big),
            \]
            így
            \[
                \varphi(u) - \varphi(0) = \big( \partial_1f(a_1 + \xi, \, a_2 + v) - \partial_1f(a_1 + \xi, \, a_2) \big) \cdot u \quad \big( u \in (-1, \, r) \big).
            \]
            A $\partial_1f \in D\{a\}$ differenciálhatósági feltételből
            \[
                \text{grad} \, \partial_1f(a) = \big( \partial_{11}f(a), \, \partial_{12}f(a) \big),
            \]
            és egy alkalmas
            \[
                \eta \in \R^2 \to \R, \, \eta(z) \to 0 \quad (||z|| \to 0)
            \]
            függvénnyel
            \[
                \partial_1f(a_1 + \xi, \, a_2 + v) - \partial_1f(a_1 + \xi, \, a_2) =
            \]
            \[
                \partial_1f(a_1 + \xi, \, a_2 + v) - \partial_1f(a_1, \, a_2) - \big( \partial_1f(a_1 + \xi, \, a_2) - \partial_1f(a_1, \, a_2) \big) = 
            \]
            \[
                \big\langle \text{grad} \, \partial_1f(a), \, (\xi, \, v) \big\rangle + \eta(\xi, \, v) \cdot ||(\xi, \, v)|| - \big\langle \text{grad} \, \partial_1f(a), \, (\xi, \, 0) \big\rangle - \eta(\xi, \, 0) \cdot ||(\xi, \, 0)|| =
            \]
            \[
                \partial_{12}f(a) \cdot v + \eta(\xi, \, v) \cdot ||(\xi, \, v)|| - \eta(\xi, \, 0) \cdot |\xi|.
            \]
            Speciálisan a $0 \neq u = v \in (-r, \, r)$ választással
            \[
                \Delta(u, \, u) = \varphi(u) - \varphi(0) = \partial_{12}f(a)\cdot u^2 + \eta(\xi, \, u) \cdot ||(\xi, \, u)|| \cdot u - \eta(\xi, \, 0) \cdot |\xi| \cdot u,
            \]
        };  
    \end{center}
    \newpage
    \begin{center}
        \tikz \node[proof1]
        {
            amiből
            \[
                \frac{\Delta(u, \, u)}{u^2} = \partial_{12}f(a) + \eta(\xi, \, u) \cdot \frac{||(\xi, \, u)||}{u} - \eta(\xi, \, 0) \cdot \frac{|\xi|}{u}
            \]
            következik. Ezért $|\xi| < |u|$ alapján
            \[
                \left| \frac{\Delta(u, \, u)}{u^2} - \partial_{12}f(a) \right| \leq |\eta(\xi, \, u)| + |\eta(\xi, \, 0)| \to 0 \quad (u \to 0),
            \]
            hiszen
            \[
                ||(\xi, \, u)||, \, ||(\xi, \, 0)|| \leq |u| \to 0 \quad (u \to 0).
            \]
            Azt kapjuk ezzel, hogy
            \[
                \tag{$\star$} \partial_{12}f(a) = \lim_{u \to 0} \frac{\Delta(u, \, u)}{u^2}.
            \]
            Legyen most rögzített $u \in (-r, \, r)$ mellett
            \[
                \psi(v) := f(a_1 + u, \, a_2 + v) - f(a_1, \, a_2 + v) \quad \big( v \in (-r, \, r) \big).
            \]
            Ekkor
            \[
                \Delta(u, \, v) = \psi(v) - \psi(0) \quad \big( v \in (-r, \, r) \big)
            \]
            és az előbbiekkel analóg módon az adódik, hogy
            \[
                \partial_{21}f(a) = \lim_{v \to 0} \frac{\Delta(v, \, v)}{v^2}.
            \]
            Itt a jobb oldali limesz ugyanaz, mint a $(\star)$-ban. Így $\partial_{21}f(a) = \partial_{12}f(a)$. 

            $\hfill \blacksquare$
        };
    \end{center}
\end{document}