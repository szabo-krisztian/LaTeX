\documentclass[12pt]{article}
\usepackage[left=1in, right=1in, top=1in, bottom=1in]{geometry}
\usepackage{tikz}
\usepackage{setspace}
\usepackage{hyperref}
\usepackage{amsfonts, amssymb, amsmath} 
\usepackage{titlesec}
\usepackage{pgfplots}
\usepackage{graphicx}
\usepackage{wrapfig}
\usepackage{caption}
\usepackage{enumitem}
\usetikzlibrary{shadows.blur}

\title{\textcolor{purple}{\Huge\textbf{\textsc{Analízis III}}}}
\author{Vizsga jegyzet}
\date{Szabó Krisztián}

\titleformat{\section}{\color{purple}\normalfont\Large\bfseries}{\thesection}{1em}{}
\renewcommand{\contentsname}{Tartalom}
\newcommand{\R}{\mathbb{R}}
\newcommand{\N}{\mathbb{N}}
\newcommand{\E}{\exists}
\newcommand{\K}{\mathbb{K}}
\newcommand{\D}{\mathcal{D}_f}
\setlength{\parindent}{0pt}

\definecolor{modernyellow}{HTML}{F4E4BC}
\definecolor{moderngreen}{HTML}{BDDCBD}

\tikzset
{
	definition/.style={
		draw,
		fill=modernyellow,
		line width=1pt,
		rounded corners,
		drop shadow={shadow blur steps=5,shadow xshift=1ex,shadow yshift=-1ex},
		text width=0.9\textwidth,
		inner sep=10pt
	},
	theorem/.style={
		draw,
		fill=moderngreen,
		line width=1pt,
		rounded corners,
		drop shadow={shadow blur steps=5,shadow xshift=1ex,shadow yshift=-1ex},
		text width=0.9\textwidth,
		inner sep=10pt
	},
	proof/.style={
		fill=white,
		rectangle,
		drop shadow={shadow blur steps=5,shadow xshift=1ex,shadow yshift=-1ex, moderngreen},
		text width=0.9\textwidth,
		inner sep=6pt,
	},
	proof1/.style={
		fill=white,
		rectangle,
		drop shadow={shadow blur steps=5,shadow xshift=1ex,shadow yshift=0, moderngreen},
		text width=0.9\textwidth,
		inner sep=6pt,
	}
}

\begin{document}
	\maketitle
	\tableofcontents
	\newpage

    \section{Konvergens sorozatok metrikus terekben. Halmazok zártságának jellemzése konvergens sorozatokkal. Banach- és Hilbert-tér.}
    \textbf{Eredeti vizsgacím:}
    
    Konvergens sorozatok metrikus terekben. Konvergencia $\K^n$-ben, a koordináta-sorozatok szerepe. \textit{Bolzano-Weierstrass-kiválasztási} tétel. Konvergencia a $\big( C[a, \, b], \, ||.||_\infty \big)$ térben (függvénysorozatok, az egyenletes, ill. a pontonkénti konvergencia fogalma). Halmazok zártságának a jellemzése konvergens sorozatokkal. A teljesség fogalma, \textit{Banach}-tér, \textit{Hilber}-tér. A $\big( C[a, \, b], \, ||.||_\infty \big)$ tér teljessége.

    \subsection{Konvergencia metrikus térben}
    \begin{center}
        \tikz \node[definition]
        {
            \textbf{Definíció.} Legyen $(X, \, \varrho)$ metrikus tér, és legyen az
            \[
                (x_n) : \N \to X
            \]
            egy, az $X$ elemeiből álló sorozat. Az $(x_n)$ sorozatot \textit{konvergensnek} nevezzük, ha van olyan $\alpha \in X$, amelyre bármely $\varepsilon > 0$ "hibakorlát" mellett egy alkalmas $N \in \N$ indexszel igaz a
            \[
                \varrho(x_n, \, \alpha) < \varepsilon \quad (n \in \N, \, n > N)
            \]
            becslés. Ha ilyen $\alpha$ nincs, akkor azt mondjuk, hogy az $(x_n)$ sorozat \textit{divergens}.
        };
    \end{center}

    Például a diszkrét metrikus térben valamely $(x_n)$ sorozat akkor és csak akkor konvergens, ha \textit{kvázikonstasn}, azaz létezik olyan $M \in \N$ természetes szám, hogy
    \[
        x_n = x_M \quad (n \in \N, \, n \geq M).
    \]
    Ha ui. egy sorozat ilyen, akkor a konvergencia definíciójában az $\alpha$ helyébe az $x_M$-et, tetszőleges $\varepsilon > 0$ mellett pedig $N$ helyébe $M$-et írva triviálisan fennáll a
    \[
        \varrho(x_n, \, \alpha) = \varrho(x_n, \, x_M) = \varrho(x_M, \, x_M) = 0 < \varepsilon \quad (n \in \N, \, n \geq M)
    \]
    egyenlőtlenség.

    \subsection{Határérték egyértelműsége}
    \begin{center}
        \tikz \node[theorem]
        {
            \textbf{Tétel.} Legyen valamilyen $(X, \, \varrho)$ metrikus tér esetén az
            \[
                x = (x_n) : \N \to X
            \]
            sorozat konvergens. Ekkor a konvergencia definíciójában szereplő $\alpha \in X$ elem egyértelműen létezik.
        };
    \end{center}
    \newpage

    \begin{center}
        \tikz \node[proof]
        {
            \textbf{Bizonyítás.} Tegyük fel, hogy a tételben említett $ X \ni \alpha$-n kívül egy $\beta \in X$ elemre is igaz a konvergencia definíciója: bármilyen $\varepsilon > 0$ számhoz megadható olyan $M \in \N$, hogy
            \[
                \varrho(x_n, \, \beta) < \varepsilon \quad (n \in \N, \, n > M).
            \]
            Ekkor a $\varrho$ metrikára vonatkozó háromszög-egyenlőtlenség miatt tetszőlegesen választott
            \[
                n \in \N,\, n > \text{max} \, \{N, \, M\}
            \]
            indexre
            \[
                \varrho(\alpha, \, \beta) \leq \varrho(\alpha,\, x_n) + \varrho(x_n, \, \beta) < 2\cdot \varepsilon.
            \]
            Következésképpen
            \[
                0 \leq \varrho(\alpha, \, \beta) < 2 \cdot \varepsilon.
            \]
            Mivel itt az $\varepsilon > 0$ bármilyen (pozitív) szám lehet, ezért csak $\varrho(\alpha, \, \beta) = 0$ lehetséges. A metrika axiómái szerint innen viszont $\alpha = \beta$ adódik.
            
            $\hfill \blacksquare$
        };
    \end{center}

    \subsection{Vektorsorozatok}
    Legyen most
    \[
        1 \leq s \in \N, \, 0 < p \leq + \infty, \, X := \K^s, \, \varrho := \varrho_p,
    \]
    és tekintsünk egy
    \[
        x = (x_n) : \N \to \K^s
    \]
    sorozatot. Ha
    \[
        x_n = (x_{n1}, \, \dots, \, x_{ns}) \, \, (\in \K^s) \quad (n \in \N),
    \]
    akkor minden $i = 1, \, \dots, \, s$ mellett definiálhatjuk az
    \[
        x^{(i)} := (x_{ni})
    \]
    számsororatot, az $x$ sorozat $i$-edik \textit{koordináta-sorozatát}. Ekkor az $x$ \textit{vektorsorozat} konvergenciája az alábbiak szerint "kezelhető" a koordináta-sorozatainak a konvergenciája révén.

    \subsection{Koordináta-sorozatok konvergenciája}
    \begin{center}
        \tikz \node[theorem]
        {
            \textbf{Tétel.} Az
            \[
                x = (x_n) : \N \to \K^s
            \]
            sorozat akkor és csak akkor konvergens az előbbi $(\K^s, \, \varrho_p)$ metrikus térben, ha minden $x^{(i)} \, \, (i = 1, \, \dots, \, s)$ koordináta-sorozata konvergens. Továbbá
            \[
                \K^s \ni (\alpha_1, \, \dots, \, \alpha_s) = \lim_{n \to + \infty} (x_n) \quad \Longleftrightarrow \quad \alpha_i = \lim_{n \to + \infty} \big(x^{(i)}\big) \quad (i = 1, \, \dots, \, s).
            \]
        };
    \end{center}

    \newpage
    \begin{center}
        \tikz \node[proof1]
        {
            \textbf{Bizonyítás.} Tegyük fel először, hogy a tételbeli $x$ sorozat konvergens, legyen
            \[
                \alpha = (\alpha_1, \, \dots, \, \alpha_s) \in \K^s
            \]
            a határértéke. A $\varrho_p$ metrikak definíciója szerint ez azt jelenti, hogy tetszőleges $\varepsilon > 0$ számot megadva van olyan $N \in \N$, amellyel az $n \in \N, \, n > N$ indexekre
            \[
                \varepsilon > \varrho_p(x_n, \, \alpha) =
                \begin{cases}
                    \displaystyle \sum_{i = 1}^s |x_{ni} - \alpha_i|^p & (p < 1) \\
                    \displaystyle \left( \sum_{i = 1}^s |x_{ni} - \alpha_i|^p \right)^{1/p} & (1 \leq p < + \infty) \\
                    \text{max} \, \{ |x_{ni} - \alpha_i : i = 1, \, \dots, \, s \} & (p = + \infty).
                \end{cases}
            \]
            Világos, hogy bármely $i = 1, \, \dots, \, s$ esetén
            \[
                \varrho(x_n, \, \alpha) \geq
                \begin{cases}
                    |x_{ni} - \alpha_i|^p & (0 < p < 1) \\
                    |x_{ni} - \alpha_i| & (1 \leq p \leq + \infty)
                \end{cases}
                \quad (n \in \N),
            \]
            ezért
            \[
                |x_{ni} - \alpha_i| < \varepsilon \quad (n \in N, \, n > N, \, p \geq 1)
            \]
            és
            \[
                |x_{ni} - \alpha_i| < \sqrt[p]{\varepsilon} \quad (n \in \N, \, n > N, \, 0 < p < 1).
            \]
            Mindez pontosan azt jelenti, hogy az $x^(i)$ koordináta-sorozat konvergens és
            \[
                \lim_{n \to + \infty} \big( x^{(i)} \big) = \alpha_i \quad (i = 1, \, \dots,\, s).
            \]
            Fordítva, ha minden $x^{(i)}$ koordinátat-sorozat konvergens, akkor legyen
            \[
                \alpha_i := \lim_{n \to + \infty} \big( x^{(i)} \big) \quad (i = 1, \, \dots, \, s)
            \]
            és
            \[
                \alpha := (\alpha_1, \, \dots, \, \alpha_s) \in \K^s.
            \]
            Ha $\varepsilon > 0$ tetszőleges, akkor minden $i = 1, \, \dots, \, s$ mellett létezik olyan $N_i \in \N$ küszöbindex, hogy
            \[
                |x_{ni} - \alpha_i| < \varepsilon \quad (n \in \N, \, n > N_i).
            \]
            Legyen $N := \text{max} \, \{N_1, \, \dots,\, N_s\}$, ekkor
            \[
                |x_{ni} - \alpha_i| < \varepsilon \quad (n \in \N, \, n > N, \, i = 1, \, \dots, \, s).
            \]
            Ezért
            \[
                \varrho_p(x_n, \, \alpha) = \sum_{i=1}^s |x_{ni}-\alpha_i|^p < s \cdot \varepsilon^p \quad (n \in \N, \, n > N, \, 0 < p < 1),
            \]
        };
    \end{center}

    \newpage
    \begin{center}
        \tikz \node[proof]
        {
            \[
                \varrho_p(x_n, \, \alpha) = \left( \sum_{i=1}^s |x_{ni} - \alpha_i|^p \right)^{1 / p} < s^{1/p} \cdot \varepsilon \quad (n \in \N, \, n > N, \, 1 \leq p < + \infty),
            \]
            \[
                \varrho_p(x_n, \, \alpha) = \text{max} \, \{ |x_{ni} - \alpha_i|  : i = 1, \, \dots, \, s\} < \varepsilon \quad (n \in \N, \, n > N, \, 1 \leq p = \infty).
            \]
            Így minden $0 < p \leq + \infty$ esetén az $(x_n)$ sorozat konvergens a $(\K^s, \, \varrho_p)$ metrikus térben, és $\displaystyle \lim_{n \to + \infty} (x_n) = \alpha$.

            $\hfill \blacksquare$
        };
    \end{center}

    \subsection{Függvényterek konvergenciája}
    Valamely $- \infty < a < b < + \infty$ mellett tekintsük az $X := C[a, \, b]$ halmazt és a $\varrho_\infty$ metrikát. Ha az
    \[
        f_n \in C[a, \, b] \quad (n \in \N)
    \]
    (függvény)sorozat konvergens és
    \[
        f := \lim_{n \to + \infty} (f_n) \in C[a, \, b],
    \]
    akkor tetszőleges $\varepsilon > 0$ számhoz van olyan $N \in \N$, hogy minden $n \in \N, \, n > N$ esetén
    \[
        \varrho_\infty(f_n, \, f) < \varepsilon,
    \]
    azaz
    \[
        \text{max} \, \big\{ |f_n(x) - f(x) | : a \leq x \leq b \big\} < \varepsilon.
    \]
    Azt mondjuk, hogy az $(f_n)$ függvénysorozat \textit{egyenletesen konvergens}. Az $f$ az $(f_n)$ határfüggvénye.\newline

    Nyilvánvaló, hogy ekkor az előbbi $n$ indexekre bármelyik $x \in [a, \, b]$ helyen
    \[
        |f_n(x) - f(x)| < \varepsilon
    \]
    igaz. Más szóval ez azt jelenti, hogy az $\big( f_n(x) \big)$ (szám)sorozat konvergens és a határértéke $f(x)$. Röviden: az $(f_n)$ függvénysorozat \textit{pontonként konvergens}.

    \subsection{Halmazok zártságának jellemzése konvergens sorozatokkal}
    \begin{center}
        \tikz \node[theorem]
        {
            \textbf{Tétel.} Legyen $(X, \, \varrho)$ metrikus tér. Az $\emptyset \neq A \subset X$ halmaz akkor és csak akkor zárt, ha minden konvergens
            \[
                (x_n) : \N \to A
            \] 
            sorozatra $\displaystyle \lim_{n \to + \infty} (x_n) \in A$. 
        };
    \end{center}

    \newpage
    \begin{center}
        \tikz \node[proof]
        {
            \textbf{Bizonyítás.} Tegyük fel először azt, hogy az $A$ halmaz zárt, de valamilyen
            \[
                (x_n) : \N \to A
            \]
            konvergens sorozatra
            \[
                \alpha := \lim_{n \to + \infty} (x_n) \not \in A.
            \]
            Ekkor tehát $\alpha \in X \, \backslash \, A$, ahol az $X \, \backslash \, A$ halmaz nyílt. Így van olyan $K(\alpha)$ környezet, hogy
            \[
                K(\alpha) \subset X \, \backslash \, A.
            \]
            Ugyanakkor egy $N \in \N$ indexszel
            \[
                A \ni x_k \in K(\alpha) \subset X \, \backslash \, A \quad (N < k \in \N),
            \]
            ami nyilván nem lehet.\newline

            Most tegyük fel azt, hogy tetszőleges konvergens
            \[
                (x_n) : \N \to A
            \]
            sorozat határértékére $\lim(x_n) \in A$, és lássuk be, hogy az $A$ halmaz zárt. Legyen ehhez $\alpha \in A'$, ekkor egy alkalmas
            \[
                (z_n) : \N \to A
            \]
            sorozatra $\lim(z_n) = \alpha$. A kiinduló feltételünk szerint ezért $\alpha \in A$, azaz $A' \subset A$ és (egy korábbi tételre hivatkozva) az $A$ zárt.
            $\hfill \blacksquare$
        };
    \end{center}

    \subsection{Cauchy-sorozat fogalma}
    \begin{center}
        \tikz \node[definition]
        {
            \textbf{Definíció.} Legyen $(X, \, \varrho)$ metrikus tér és
            \[
                (x_n) : \N \to X
            \]
            sorozat. Ezt a sorozat \textit{Cauchy-sorozat}, ha tetszőleges $\varepsilon > 0$ esetén létezik olyan $N \in \N$, hogy
            \[
                \varrho(x_n, \, x_m) < \varepsilon \quad (m, \, n \in \N, \, m,\, n > N).
            \] 
        };
    \end{center}

    \subsection{Banach- és Hilbert-tér fogalma}
    Legyen adott az $(X, ||.||)$ normált tér. Azt mondjuk, hogy ez a tér \textit{teljes} (vagy \textit{Banach-tér}), ha a $||.||$ norma által indukált
    \[
        \varrho(x, \, y) := ||x-y|| \quad (x, \, y \in X)
    \]
    metrikával az $(X, \, \varrho)$ metrikus tér teljes. Világos, hogy a
    \[
        (\K^s, \, ||.||_p) \quad (1 \leq s \in \N, \, 1 \leq p \leq + \infty)
    \]
    terek valamennyien Banach-terek. Hasonlóan: a $(C[a, \,b], \, ||.||_\infty)$ tér is Banach-tér.\newline

    Azt mondjuk, hogy az $(X, \, \langle . \rangle)$ euklideszi tér \textit{teljes} (vagy \textit{Hilbert-tér}), ha a $\langle . \rangle$ skaláris szorzás által meghatározott
    \[
        ||x|| := \sqrt{\langle x, \, x \rangle} \quad (x \in X)
    \]
    normával $(X, \, ||.||)$ Banach-tér. Így pl. a
    \[
        (\K^s, \, \langle . \rangle) \quad (1 \leq s \in \N)
    \]
    tér Hilbert-tér, ahol
    \[
        \langle x, \, y \rangle = \sum_{i =1}^s x_i\overline{y_i} \quad \big( x = (x_1, \, \dots, \, x_s), \, y = (y_1, \, \dots, \, y_s) \in \K^s \big).
    \]


    \subsection{Bolzano-Weierstrass-kiválasztási tétel}
    \begin{center}
        \tikz \node[theorem]
        {
            \textbf{Tétel.} A $(\K^s, \, \varrho_p) \, \, (1 \leq s \in \N, \, 0 < p \leq + \infty)$ metrikus térben minden korlátos sorozatnak van konvergens részsorozata.
        };
    \end{center}

    \begin{center}
        \tikz \node[proof1]
        {
            \textbf{Bizonyítás.} Emlékeztetünk egy korábbi tételre, miszerint az
            \[
                x = (x_n) : \N \to \K^s
            \]
            sorozat akkor és csak akkor konvergens, ha minden $x^{(i)} \, \, i = 1, \, \dots, \, s$ koordináta-sorozata konvergens. \newline

            A feltételezésünk szerint most az $(x_n)$ sorozat korlátos. Van tehát olyan $r > 0$ szám, amellyel
            \[
                \varrho(x_n, \, 0) < r \quad (n \in \N).
            \] 
            A $\varrho_p$ metrika definícióját figyelembe véve innen az is rögtön adódik, hogy $1 \leq p \leq + \infty$ esetén
            \[
                |x_{ni}| < r \quad (n \in \N, \, i = 1, \, \dots, \, s),
            \]
            ill. $0 < p < 1$ mellett
            \[
                |x_{ni}| < r^{1/p} \quad (n \in \N, \, i = 1, \, \dots, \, s),
            \]
            azaz, hogy minden $x^{(i)} \, \, (i = 1, \, \dots, \, s)$ koordináta-sorozat (mint számsorozat) is korlátos. A számsorozatokra ismert Bolzano-Weierstrass-kiválasztási tétel alapján ezért létezik olyan $\nu^{(1)}$ indexsorozat, hogy az
            \[
                x^{(1)} \circ \nu^{(1)}
            \]
        };
    \end{center}

    \newpage
    \begin{center}
        \tikz \node[proof]
        {
            részsorozat konvergens. Világos, hogy az $x^{(2)} \circ \nu^{(1)}$ részsorozat is korlátos, ezért van olyan $\nu^{(2)}$ indexsorozat is amelyre az
            \[
                \big( x^{(2)} \circ \nu^{(1)} \big) \circ \nu^{(2)} = x^{(2)} \circ \big( \nu^{(1)} \circ \nu^{(2)} \big)
            \]
            részsorozat is konvergens. A konstrukciót folytatva végül olyan
            \[
                \nu^{(i)} \quad (i = 1, \, \dots, \, s)
            \]
            indexsorozatokat kapunk, hogy az
            \[
                x^{(i)} \circ \big( \nu^{(1)} \circ \dots \circ \nu^{(i)} \big) \quad (i = 1, \, \dots, \, s)
            \]
            részsorozatok konvergensek. Legyen
            \[
                \nu := \nu^{(1)} \circ \dots \circ \nu^{(s)},
            \]
            ekkor a $\nu$ sorozat indexsorozat, és minden
            \[
                x^{(i)} \circ \nu \quad (i = 1, \, \dots, \, s)
            \]
            sorozat részsorozata a konvergens $x^{(i)} \circ \big( \nu^{(1)} \circ \dots \circ \nu{(i)} \big)$ sorozatnak. Így az
            \[
                x^{(i)} \circ \nu \quad (i = 1, \, \dots, \, s)
            \]
            számsorozatok mindegyike konvergens. Ez azt jelenti, hogy az $x \circ \nu$ részsorozat is konvergens.
            
            $\hfill \blacksquare$
        };
    \end{center}

    \subsection{Függvénytér teljessége}
    \begin{center}
        \tikz \node[theorem]
        {
            \textbf{Tétel.} A $(C[a, \, b], \, \varrho_\infty)$ metrikus tér teljes.
        };
    \end{center}

    \begin{center}
        \tikz \node[proof1]
        {
            \textbf{Bizonyítás.} Tegyük fel, hogy az
            \[
                (f_n) : \N \to C[a, \, b]
            \]
            (függvény)sorozat (a $\varrho_\infty$ metrika értelmében) Cauchy-sorozat. Ez most azt jelenti, hogy bármilyen $\varepsilon > 0$ számot is adunk meg, ehhez találunk olyan $N \in \N$ indexet, hogy
            \[
                \varrho_\infty(f_n, \, f_m) =
            \]
            \[
                \text{max} \, \big\{ |f_n(x) - f_m(x)| : a \leq x \leq b  \big\} < \varepsilon \quad (n, \, m \in \N, \, n, \, m > N).
            \]
            Világos, hogy tetszőleges $x \in [a, \, b]$ esetén egyúttal
        };
    \end{center}

    \newpage
    \begin{center}
        \tikz \node[proof]
        {
            \[
                \tag{$\star$} |f_n(x) - f_m(x)| \varepsilon \quad (n, \, m \in \N, \, n, \, m > N)
            \]
            is teljesül, más szóval az $\big( f_n(x) \big)$ számsorozat Cauchy-soroozat. Létezik tehát az
            \[
                f(x) := \lim_{n \to + \infty} f_n(x) \quad (x \in [a, \, b])
            \]
            ("pontonkénti") határérték. Továbbá $(\star)$ miatt
            \[
                |f_n(x) - f(x)| =
            \]
            \[
                \tag{$\star \star$} \lim_{m \to + \infty} |f_n(x) - f_m(x)| \leq \varepsilon \quad (x \in [a, \, b], \, n \in \N, \, n > N).
            \]
            Mutassuk meg, hogy az így definiált
            \[
                f : [a, \, b] \to \R
            \]
            függvény folytonos, azaz $f \in C[a, \, b]$. Legyen ehhez valamilyen $\xi \in [a, \, b]$ mellett $\varepsilon > 0$ tetszőleges, ekkor az előbbiek szerint bármilyen (rögzített) $n \in \N, \, n > N$ esetén
            \[
                |f(x) - f(\xi)| \leq |f(x) - f_n(x)| + |f_n(x) - f_n(\xi)| + |f_n(\xi) - f(\xi)| \leq 
            \]
            \[
                2 \cdot \xi + |f_n(x) - f_n(\xi)| \quad (x \in [a, \, b]).
            \]
            Mivel $f_n \in C[a, \, b]$, így $f_n \in C\{\xi\}$ is igaz. Következésképpen létezik olyan $\delta > 0$ szám, amellyel
            \[
                |f_n(x) - f_n(\xi)| < \varepsilon \quad (x \in [a, \, b], \, |x-\xi| < \delta).
            \]
            Mindezeket figyelembe véve azt mondhatjuk, hogy
            \[
                |f_n(x) - f_n(\xi)| < 2 \cdot \varepsilon + \varepsilon \quad (x \in [a, \, b], \, |x - \xi| < \delta).
            \]
            Ez nem jelent mást, mint azt, hogy $f \in C\{\xi\}$. Az itt szereplő $\xi$ tetszőleges eleme volt az $[a, \, b]$ intervallumnak, ezért $f \in C[a, \, b]$.
            Végül, a $(\star \star)$ becslés szerint (az ottani szereplőkkel)
            \[
                \varrho_\infty(f_n, \, f) = \text{max} \, \big\{ |f_n(x) - f(x)| : a \leq x \leq b \big\} \leq \varepsilon \quad (n \in \N, \, n > N),
            \]
            azaz
            \[
                \varrho_\infty(f_n, \, f) \to 0 \quad (n \to + \infty).
            \]
            Ez azzal ekvivalens, hogy a $(C[a, \, b], \varrho_\infty)$ metrikus térben az $(f_n)$ sorozat konvergál az $f$ függvényhez. Ezzel beláttuk, hogy a szóban forgó térben minden Cauchy-sorozat konvergens, azaz a $(C[a,\, b], \varrho_\infty)$ teljes metrikus tér.
            
            $\hfill \blacksquare$
        };

    \end{center}

    \newpage
    \section{A koordináta-függvények szerepe a differenciálhatóságban. A \textit{Jacobi}-mátrix kiszámítása.}
    \subsection{Koordináta-függvények és a differenciálhatóság kapcsolata}
    
    \begin{center}
        \tikz \node[theorem]
        {
            \textbf{Tétel.} Legyen $1 \leq n, \, m \in \N$. Az
            \[
                f = (f_1, \, \dots, \, f_m) \in \R^n \to \R^m
            \]
            függvény akkor és csak akkor differenciálható az $a \in \text{int} \, \D$ helyen, ha minden $i = 1, \, \dots, \, m$ esetén az
            \[
                f_i \in \R^n \to \R
            \]
            koordináta-függvény differenciálható az $a$-ban. Ha $f \in D\{a\}$, akkor az $f'(a)$ Jacobi-mátrix a következő alakú:
            \[
                f'(a) =
                \begin{bmatrix}
                    \text{grad} \, f_1(a) \\
                    \text{grad} \, f_2(a)  \\
                    \vdots \\
                    \text{grad} \, f_m(a)    
                \end{bmatrix}
            \]
        };    
    \end{center}

    \begin{center}
        \tikz \node[proof1]
        {
            \textbf{Bizonyítás.} Tegyük fel először is azt, hogy $f \in D\{a\}$, és jelöljük az $f'(a) \in \R^{m \times n}$ Jacobi-mátrix sorvektorait $A_i$-vel ($i = 1, \, \dots, \, m$:)
            \[
                f'(a) =
                \begin{bmatrix}
                    A_1 \\
                    A_2 \\
                    \vdots \\
                    A_m
                \end{bmatrix}.
            \]
            Ekkor alkalmas
            \[
                \eta = (\eta_1, \, \dots, \, \eta_m) \in \R^n \to \R^m
            \]
            függvénnyel
            \[
                \eta(h) \to 0 \quad (||h|| \to 0)
            \]
            és a
            \[
                h \in \R^n \quad (a + h \in \D)
            \]
            helyeken
            \[
                f(a + h) - f(a) = \big( f_1(a + h) - f_1(a), \, \dots, \, f_m(a + h) - f_m(a) \big) =
            \]
            \[
                f'(a)\cdot h + \eta(h) \cdot ||h|| =
            \]
            \[
                \big( \langle A_1, \, h\rangle, \, \dots, \, \langle A_m, \, h \rangle \big) + \big( \eta_1(h) \cdot ||h||, \, \dots, \, \eta_m(h) \cdot ||h|| \big).
            \]
        };
    \end{center}

    \newpage
    \begin{center}
        \tikz \node[proof1]
        {
            Következésképpen minden $i = 1, \, \dots, \, m$ mellett az $\eta$ függvény
            \[
                \eta_i \in \R^n \to \R
            \]
            koordináta-függvényeivel
            \[
                \tag{$\star$} f_i(a + h) - f_i(a) = \langle A_i, \, h \rangle + \eta_i(h) \cdot ||h|| \quad (h \in \R^n, \, a + h \in \D).
            \]
            Mivel bármely $i = 1, \, \dots, \, m$ indexre
            \[
                \eta_i(h) \to 0 \quad (||h|| \to 0),
            \]
            ezért az előbbi $(\star)$ összefüggés azt jelenti, hogy $f_i \in D \{a\}$ és
            $A_i = \text{grad} \, f_i(a) \quad (i = 1, \, \dots, \, m)$.
            Most azt tegyük fel, hogy $f_i \in D\{a\} \, \, (i = 1, \, \dots, \, m)$, amikor is valamilyen
            \[
                \eta_i \in \R^n \to \R, \, \eta_i \to 0 \, \, (||h|| \to 0) \quad (i = 1, \, \dots, \, m)
            \]
            függvényekkel
            \[
                f_i(a + h) - f_i(a) = \big\langle  \text{grad} \, f_i(a), \, h \big\rangle + \eta_i(h) \cdot ||h|| \quad (h\in \R^n, \, a + h \in \D, \, i = 1, \, \dots, \, m).
            \]
            Ha tehát
            \[ A :=
                \begin{bmatrix}
                    \text{grad} \, f_1(a) \\
                    \text{grad} \, f_2(a) \\
                    \vdots \\
                    \text{grad} \, f_m(a)
                \end{bmatrix}
                \in \R^{m \times n},
            \]
            akkor az
            \[
                \eta := (\eta_1, \, \dots, \, \eta_m) \in \R^n \to \R^m
            \]
            függvénnyel
            \[
                f(a + h) - f(a) = A\cdot h + \eta(h) \cdot ||h||,
            \]
            ahol $\eta(h) \to 0 \, \, (||h|| \to 0)$. Ezért $f \in D\{a\}$ és $f'(a) = A.$
            
            $\hfill \blacksquare$
        };
    \end{center}
    
    

    \newpage
    \section{Többször differenciálható függvények. Young-tétel.}
    \subsection{Többváltozós-valós függvények másodrendű differenciálhatósága}
    \begin{center}
        \tikz \node[definition]
        {
            \textbf{Definíció.} Legyen valamilyen $1 \leq n \in \N$ esetén $f \in \R^n \to \R$. Tegyük fel, hogy $a \in \text{int}\, \D$. Azt mondjuk, hogy az $f$ függvény \textit{kétszer differenciálható} az $a$-ban ha minden $x \in K(a) \subset \D$ esetén $f \in D\{x\}$, és
            \[
                \partial_if \in D\{a\} \quad (i = 1, \, \dots, \, n).
            \]
        };    
    \end{center}
    
    Ha a fenti feltételek teljesülnek akkor léteznek a
    \[
        \partial_j(\partial_if)(a) \quad (i, \, j = 1, \, \dots, \, n)
    \]
    parciális deriváltak. Ehhez persze nem szükséges, hogy a $\partial_if \, \, (i = 1, \, \dots, \, n)$ függvények deriválhatók legyenek az $a$ helyen.
    Ha tehát a fenti
    \[
        f \in \R^n \to \R
    \]
    függvényre $f \in D^2\{a\}$, akkor minden $i, \, j = 1, \, \dots, \, n$ mellett létezik a $\partial_{ij}f(a)$ másodrendű parciális derivált. Az
    \[
        f''(a) := \big( \partial_{ij}f(a) \big)_{i, \, j = 1}^n =
        \begin{bmatrix}
            \partial_{11}f(a) & \partial_{12}f(a) & \dots & \partial_{1n}f(a) \\
            \partial_{21}f(a) & \partial_{22}f(a) & \dots & \partial_{2n}f(a) \\
            \vdots & \vdots & \dots & \vdots \\
            \partial_{n1}f(a) & \partial_{n2}f(a) & \dots & \partial_{nn}f(a)
        \end{bmatrix}
        \in \R^{n \times n}
    \]
    mátrixot az $f$ függvény $a$-beli \textit{másodrendű deriváltmátrixnának nevezzük}. A későbbiekben tárgyalandó Young-tétel miatt ez egy szimmetrikus mátrix. 

    \subsection{Többváltozós-valós függvények magasabb rendű differenciálhatósága}
    \begin{center}
        \tikz \node[definition]
        {
            \textbf{Definíció.} Legyen valamilyen $1 \leq n \in \N$ esetén $f \in \R^n \to \R$. Tegyük fel, hogy $a \in \text{int} \, \D, \, 1 \leq s \in \N$, továbbá egy alkalmas $K(a) \subset \D$ környezettel minden $x \in K(a)$ pontban az $f$ függvény $s$-szer differenciálható: $f \in D^s\{x\}$. Belátható, hogy ekkor a $K(a)$ pontjaiban az $f$ összes $s$-edrendű parciális deriváltja létezik. Azt mondjuk, hogy az $f$ \textit{függvény az $a$-ban $(s+1)$-szer differenciálható}, ha minden $s$-edrendű parciális deriváltfüggvénye differenciálható az $a$-ban.
        };        
    \end{center}

    \subsection{Többváltozós-vektorfüggvények függvények magasabb rendű differenciálhatósága}
    \begin{center}
        \tikz \node[definition]
        {
            \textbf{Definíció.} Legyen $1 \leq n, \, m \in \N$ és
            \[
                f = (f_1, \, \dots, \, f_m) \in \R^n \to \R^m, \, a \in \text{int} \, \D,
            \]
            ill. $1 \leq k \in \N$. Azt mondjuk, hogy az \textit{$f$ függvény $k$-szor differenciálható az $a$-ban}, ha
            \[
                f_j \in D^k\{a\} \quad (j = 1, \, \dots, \, m).
            \]
        };
    \end{center}

    \subsection{Young-tétel}
    \begin{center}
        \tikz \node[theorem]
        {
            \textbf{Tétel.} Legyen $2 \leq n \in \mathbb{N}, \, f \in \R^n \to \R, \, a \in \text{int} \, \D, \, 2 \leq s \in \mathbb{N}$ és $f \in D^s\{a\}$. Ekkor tetszőleges $k_1, \, \dots, \, k_s \in \{1, \, \dots, \, n\}$ indexek esetén ezek bármely $j_1, \, \dots, \, j_s$ permutációjára
            \[
                \partial_{k_1 \, \dots \, k_s}f(a) = \partial_{j_1 \, \dots \, j_s}f(a).
            \]
        };
    \end{center}

    \begin{center}
        \tikz \node[proof1]
        {
            \textbf{Bizonyítás.} Az $s$-szerinti teljes indukcióra gondolva elegendő az $s = 2$ esettel foglalkoznunk. Ekkor tehát azt kell belátnunk, hogy ha $f \in D^2\{a\}$, akkor
            \[
                \partial_{ij}f(a) = \partial_{ji}f(a) \quad (i, \, j = 1, \, \dots, \, n).
            \]
            Világos, hogy csak az $i \neq j$ eset az "érdekes". Ezen túl (könnyen meggondolhatóan) azt is feltehetjük, hogy $n = 2$. Más szóval az
            \[
                f \in \R^2 \to \R
            \]
            függvényekre
            \[
                a = (a_1, \, a_2) \in \text{int}\, \mathcal{D}_f, \, f \in D^2\{a\},
            \]
            és ennek alapján azt kell bebizonyítanunk, hogy
            \[
            \partial_{12}f(a) = \partial_{21}f(a).
            \]
            Legyen ehhez $r > 0$ olyan, amellyel ($\R^n$-ben a $||.|| := ||.||_\infty$ normát választva)
            \[
                K(a) = \{ x \in \R^2 : ||x-a|| < r \} \subset \mathcal{D}_f,
            \]
            és vezessük be az alábbi jelölést: az $u, \, v \in (-r, \, r)$ helyeken
            \[
                \Delta(u, \, v) := f(a_1 + u, \, a_2 + v) - f(a_1 + u, \, a_2) + f(a_1, \, a_2) - f(a_1, \, a_2 + v).
            \]
        };
    \end{center}

    \newpage
    \begin{center}
        \tikz \node[proof1]
        {
            Ha rögzítjük a $v \in (-r, \, r)$ számot, akkor a
            \[
                \varphi(u) := f(a_1 + u, \, a_2 + v) - f(a_1 + u, \, a_2) \quad \big( u \in (-r, \, r) \big)
            \]        
            függvénnyel
            \[
                \Delta(u, \, v) = \varphi(u) - \varphi(0) \quad \big( u \in (-r, \, r) \big).
            \]
            Az $f \in D^2\{a\}$ feltétel miatt az előbbi $K_r(a)$ környezettől azt is megkövetelhetjük, hogy egyrészt minden $x \in K_r(a)$ helyen $f \in D\{x\}$ (így egyúttal léteznek az $\partial_1f(x), \, \partial_2f(x)$ parciális deriváltak is), másrészt
            \[
                \partial_1f, \, \partial_2f \in D \{a\}.
            \]
            Következésképpen a most definiált
            \[
                \varphi : (-r, \, r) \to \R
            \]
            függvény differenciálható, ezért a Lagrange-középérték-tétel alapján
            \[
                \varphi(u) - \varphi(0) = \varphi'(\xi)\cdot u \quad \big( u \in (-r, \, r) \big),
            \]
            ahol $\xi \in (0, \, u)$ (vagy $\xi \in (u, \, 0)$). A parciális deriváltak definíciójára gondolva
            \[
                \varphi'(u) = \partial_1f(a_1 + u, \, a_2 + v) - \partial_1f(a_1 + u, \, a_2) \quad \big( u \in (-r, \, r) \big),
            \]
            így
            \[
                \varphi(u) - \varphi(0) = \big( \partial_1f(a_1 + \xi, \, a_2 + v) - \partial_1f(a_1 + \xi, \, a_2) \big) \cdot u \quad \big( u \in (-1, \, r) \big).
            \]
            A $\partial_1f \in D\{a\}$ differenciálhatósági feltételből
            \[
                \text{grad} \, \partial_1f(a) = \big( \partial_{11}f(a), \, \partial_{12}f(a) \big),
            \]
            és egy alkalmas
            \[
                \eta \in \R^2 \to \R, \, \eta(z) \to 0 \quad (||z|| \to 0)
            \]
            függvénnyel
            \[
                \partial_1f(a_1 + \xi, \, a_2 + v) - \partial_1f(a_1 + \xi, \, a_2) =
            \]
            \[
                \partial_1f(a_1 + \xi, \, a_2 + v) - \partial_1f(a_1, \, a_2) - \big( \partial_1f(a_1 + \xi, \, a_2) - \partial_1f(a_1, \, a_2) \big) = 
            \]
            \[
                \big\langle \text{grad} \, \partial_1f(a), \, (\xi, \, v) \big\rangle + \eta(\xi, \, v) \cdot ||(\xi, \, v)|| - \big\langle \text{grad} \, \partial_1f(a), \, (\xi, \, 0) \big\rangle - \eta(\xi, \, 0) \cdot ||(\xi, \, 0)|| =
            \]
            \[
                \partial_{12}f(a) \cdot v + \eta(\xi, \, v) \cdot ||(\xi, \, v)|| - \eta(\xi, \, 0) \cdot |\xi|.
            \]
            Speciálisan a $0 \neq u = v \in (-r, \, r)$ választással
            \[
                \Delta(u, \, u) = \varphi(u) - \varphi(0) = \partial_{12}f(a)\cdot u^2 + \eta(\xi, \, u) \cdot ||(\xi, \, u)|| \cdot u - \eta(\xi, \, 0) \cdot |\xi| \cdot u,
            \]
        };  
    \end{center}
    \newpage
    \begin{center}
        \tikz \node[proof1]
        {
            amiből
            \[
                \frac{\Delta(u, \, u)}{u^2} = \partial_{12}f(a) + \eta(\xi, \, u) \cdot \frac{||(\xi, \, u)||}{u} - \eta(\xi, \, 0) \cdot \frac{|\xi|}{u}
            \]
            következik. Ezért $|\xi| < |u|$ alapján
            \[
                \left| \frac{\Delta(u, \, u)}{u^2} - \partial_{12}f(a) \right| \leq |\eta(\xi, \, u)| + |\eta(\xi, \, 0)| \to 0 \quad (u \to 0),
            \]
            hiszen
            \[
                ||(\xi, \, u)||, \, ||(\xi, \, 0)|| \leq |u| \to 0 \quad (u \to 0).
            \]
            Azt kapjuk ezzel, hogy
            \[
                \tag{$\star$} \partial_{12}f(a) = \lim_{u \to 0} \frac{\Delta(u, \, u)}{u^2}.
            \]
            Legyen most rögzített $u \in (-r, \, r)$ mellett
            \[
                \psi(v) := f(a_1 + u, \, a_2 + v) - f(a_1, \, a_2 + v) \quad \big( v \in (-r, \, r) \big).
            \]
            Ekkor
            \[
                \Delta(u, \, v) = \psi(v) - \psi(0) \quad \big( v \in (-r, \, r) \big)
            \]
            és az előbbiekkel analóg módon az adódik, hogy
            \[
                \partial_{21}f(a) = \lim_{v \to 0} \frac{\Delta(v, \, v)}{v^2}.
            \]
            Itt a jobb oldali limesz ugyanaz, mint a $(\star)$-ban. Így $\partial_{21}f(a) = \partial_{12}f(a)$. 

            $\hfill \blacksquare$
        };
    \end{center}


\end{document}