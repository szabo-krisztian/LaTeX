\documentclass[12pt]{article}
\usepackage{tikz}
\usepackage{setspace}
\usepackage{hyperref}
\usepackage{amsfonts, amssymb, amsmath}
\usepackage{titlesec}
\usepackage{pgfplots}
\pgfplotsset{compat=newest}
\usepackage{graphicx}
\usepackage{wrapfig}
\usepackage{caption}
\usepackage{enumitem}
\usetikzlibrary{shadows.blur}
\usepackage{lmodern}
\usepackage{mathrsfs}
\setlength{\parskip}{0pt}
\setlength{\parindent}{0pt}

\title{\textcolor{purple}{\Huge\textbf{Analízis 4 vizsgatematika}}}
\author{Dr. Simon Péter jegyzetéből}
\date{}

\renewcommand{\contentsname}{Tartalom}
\newcommand{\R}{\mathbb{R}}
\newcommand{\N}{\mathbb{N}}
\newcommand{\E}{\exists}
\newcommand{\mm}{\mathbf{m}}
\newcommand{\MM}{\mathbf{M}}
\newcommand{\K}{\mathbb{K}}
\newcommand{\D}{\mathcal{D}_f}
\newcommand{\Dp}{\mathcal{D}_\varphi}

\definecolor{modernyellow}{HTML}{F4E4BC}
\definecolor{moderngreen}{HTML}{BDDCBD}

\titleformat{\section}[block]{\large\bfseries}{\thesection}{1em}{}
\titlespacing*{\section}{0pt}{1.5ex plus 1ex minus .2ex}{1.3ex plus .2ex}

\tikzset
{
    definition/.style={
        draw,
        fill=white,
        line width=1pt,
        rounded corners,
        drop shadow={shadow blur steps=5,shadow xshift=1ex,shadow yshift=-1ex},
        text width=1\textwidth,
        inner sep=8pt,
        align=justify
    },
    theorem/.style={
        draw,
        fill=white,
        line width=1pt,
        rounded corners,
        drop shadow={shadow blur steps=5,shadow xshift=1ex,shadow yshift=-1ex},
        text width=1\textwidth,
        inner sep=8pt,
        align=justify
    },
    proof/.style={
            fill=white,
            rectangle,
            drop shadow={shadow blur steps=5,shadow xshift=1ex,shadow yshift=-1ex, moderngreen},
            text width=1\textwidth,
            inner sep=8pt,
            align=justify
    },
    proof1/.style={
            fill=white,
            rectangle,
            drop shadow={shadow blur steps=5,shadow xshift=1ex,shadow yshift=0, moderngreen},
            text width=1\textwidth,
            inner sep=8pt,
            align=justify
    }
}

\begin{document}
    \maketitle
    \tableofcontents
    \newpage
    
    \section{Az implicitfüggvény fogalma, kapcsolata a feltételes szélsőérték problémával és az inverzfüggvénnyel. Implicitfüggvény-tétel, inverzfüggvény-tétel. (a bizonyítás vázlata)}
    \subsection{Az implicitfüggvény fogalma}
    Legyenek $n, \, m \in \N$ természetes számok, ahol $2 \leq n$ és $1 \leq m \leq n$. Ha
    \[
        \xi = (\xi_1, \, \dots, \, \xi_n) \in \R^n,
    \]
    akkor legyen
    \[
        x := (\xi_1, \, \dots, \, \xi_{n-m}) \in \R^{n-m}, \, y := (\xi_{n-m+1}, \, \dots, \, \xi_n) \in \R^m,
    \]
    és ezt következőképpen fogjuk jelölni:
    \[
        \xi = (x, \, y).
    \]
    Röviden:
    \[
        \R^n \equiv \R^{n-m} \times \R^{m}.
    \]
    Ha tehát
    \[
        f = (f_1, \, \dots, \, f_m) \in \R^n \to \R^m,
    \]
    azaz
    \[
        f \in \R^{n-m} \times \R^m \to \R^m,
    \]
    akkor az $f$-et olyan kétváltozós vektorfüggvénynek tekintjük, ahol az $f(x, \, y)$ helyettesítési értékben az argumentum első változójára $x \in \R^{n-m}$, a második változójára pedig $y \in \R^m$ teljesül.\\

    Tegyük fel, hogy ebben az értelemben valamilyen $(a, \, b) \in \D$ zérushelye az $f$-nek:
    \[
        f(a, \, b) = 0.
    \]
    Tételezzük fel továbbá, hogy van az $a$-nak egy olyan $K(a) \subset \R^{n-m}$ környezete, a $b$-nek pedig olyan $K(b) \subset \R^m$ környezete, hogy tetszőleges $x \in K(a)$ esetén egyértelműen létezik olyan $y \in K(b)$, amellyel
    \[
        f(x, \, y) = 0.
    \]
    Definiáljuk ekkor a $\varphi(x) := y$ hozzárendeléssel a
    \[
        \varphi : K(a) \to K(b)
    \]
    függvényt, amikor is
    \[
        f(x, \, \varphi(x)) = 0 \quad (x \in K(a)).
    \]
    Itt minden $x \in K(a)$ mellett az $y = \varphi(x)$ az egyetlen olyan $y \in K(b)$ hely amelyre
    \[
        f(x, \, y) = 0.
    \]
    Az előbbi $\varphi$ függvényt az $f$ által (az $(a, \, b)$ körül) meghatározott \textit{implicitfüggvénynek} nevezzük. Tehát az
    
    \[
    \left\{
    \begin{array}{ccc}
        f_1(x_1, \, \dots, \, x_{n-m}, \, y_1, \, \dots, \, y_m) & = & 0 \\
        \vdots & & \vdots \\
        f_m(x_1, \, \dots, \, x_{n-m}, \, y_1, \, \dots, \, y_m) & = & 0
    \end{array}
    \right.
    \]

    egyenletrendszernek minden $x = (x_1, \, \dots, \, x_{n-m}) \in K(a)$ mellett egyértelműen létezik
    \[
        y = (y_1, \, \dots, \, y_m) = \varphi(x) \in K(b)
    \]
    megoldása. Nyilván $\varphi(a) = b$.

    \subsection{Implicitfüggvény-tétel}
    \tikz \node[theorem]
    {
        \textbf{Tétel (implicitfüggvény-tétel).} Adott $n, \, m \in \N, \, 2 \leq n$, valamint $1 \leq m < n$ mellett az
        \[
            f \in \R^{n-m} \times \R^m \to \R^m
        \]
        függvényről tételezzük fel az alábbiakat: $f \in C^1$, és az $(a, \, b) \in \text{int} \, \D$ helyen
        \[
            f(a, \, b) = 0, \, \det \partial_2f(a, \, b) \neq 0.
        \]
        Ekkor alkalmas $K(a), \, K(b)$ környezetekkel létezik az $f$ által az $(a, \, b)$ körül meghatározott
        \[
            \varphi : K(a) \to K(b)
        \]
        implicitfüggvény, ami folytonosan differenciálható, és
        \[
            \varphi'(x) = -\partial_2 f(x, \, \varphi(x))^{-1} \cdot \partial_1 f(x, \, \varphi(x)) \quad (x \in K(a)).
        \]
    };\\

    \subsection{Lokális invertálhatóság}
    Elöljáróban idézzük fel az egyváltozós valós függvényekkel kapcsolatban tanultakat. Ha pl.
    \[
        h \in \R \to \R, \, h \in C^1\{a\}
    \]
    és $h'(a) \neq 0$, akkor egy alkalmas $r>0$ mellett
    \[
        I := (a-r, \, a+r) \subset \mathcal{D}_h,
    \]
    létezik a $(h_{|_I})^{-1}$ inverzfüggvény, a $g := (h_{|_I})^{-1}$ függvény differenciálható és 
    \[
        g'(x) = \frac{1}{h'(g(x))} \quad (x \in \mathcal{D}_g).
    \]
    A továbbiakban a most megfogalmazott "egyváltozós" állítás megfelelőjét fogjuk vizsgálni többáltozós vektorfüggvényekre.\\

    Legyen ehhez valamilyen $1 \leq n \in \N$ mellett adott az
    \[
        f \in \R^n \to \R^n
    \]
    függvény és az $a \in \text{int} \, \D$ pont. Azt mondjuk, hogy az $f$ függvény \textit{lokálisan invertálható} az $a$-ban, ha létezik olyan $K(a) \subset \D$ környezet, hogy a $g := f_{|_{K(a)}}$ leszűkítés invertálható függvény. Minden ilyen esetben a $g^{-1}$ inverzfüggvényt az $f$ $a$-beli \textit{lokális inverzének} nevezzük.\\

    \tikz \node[theorem]
    {
        \textbf{Tétel.} Legyen $1 \leq n \in \N, \, f \in \R^n \to \R^n$. Tegyük fel, hogy egy $a \in \text{int} \, \D$ pontban $f \in C^1\{a\}$ és $\det f'(a) \neq 0$. Ekkor az $f$ függvény az $a$-ban lokálisan invertálható, és az $a$-beli lokális inverze folytonos.
    };\\
    
    \textbf{Bizonyítás (vázlat).} Feltehető, hogy
    \[
        a = f(a) = 0 \in \R^n, \, f'(0) = (v_{ik})_{i, \, k = 1}^n \in \R^{n \times n}
    \]
    pedig az $n \times n$-es egységmátrix:
    \[
        v_{ii} = 1, \, v_{ik} = 0 \quad (i, \, k \in \{ i, \, \dots, \, n \}, \, i \neq k).
    \]
    Legyen $y \in \R^n$ és
    \[
        \Phi_y(x) := x - f(x) + y =: g(x) + y \quad (x \in \D).
    \]
    Ekkor $g \in C^1\{0\}$ és
    \[
        g'(0) = \Theta := (0)^n_{i, \, k = 1} \in \R^{n \times n}
    \]
    (az $n \times n$-es nullmátrix). Ha tehát
    \[
        g = (g_1, \, \dots, \, g_n),
    \]
    akkor egy $K_r(0) \subset \D \quad (r > 0)$ környezet minden pontjában a $g$ függvény differenciálható, a $g_k$ koordinátafüggvények $\partial_i g_k$ $(i, \, k = 1, \, \dots, \, n)$ parciális deriváltfüggvényei valamennyien folytonosak a $0$-ban, és
    \[
        \partial_i g_k(0) = 0 \quad (i, \, k = 1, \, \dots, \, n).
    \]
    Válasszuk a $q > 0$ számot úgy, hogy $n \cdot q < 1 / 2$, ekkor az előbbiek szerint a $K_r(0)$ környezetről az is feltehető, hogy
    \[
        G_r(0) := \overline{K_r(0)} = \{ x \in \R^n : \|x\|_\infty \leq r \} \subset \D
    \]
    és
    \[
        |\partial_i g_k(x)| < q \quad (x \in G_r(0), \, i, \, k = 1, \, \dots, \, n).
    \]
    Ezért
    \[
        \|g'(x)\|_{(\infty)} = \max \left\{ \sum_{i=1}^n |\partial_i g_k(x)| : k = 1, \, \dots, \, n \right\} < n \cdot q < \frac{1}{2} \quad (x \in G_r(0)).
    \]
    Így azt mondhatjuk, hogy
    \[
        \| g(x) - g(t) \| \leq
    \]
    \[
        \sup \left\{ \|g'(\xi)\|_{(\infty)} : \xi \in G_r(0) \right\} \cdot \|x-t\|_\infty \leq \frac{1}{2} \cdot \|x-t\|_\infty \quad (x, \, t \in G_r(0)).
    \]
    Tehát egyúttal
    \[  
        \| \Phi_y(x) - \Phi_y(t)\|_\infty = \|g(x) - g(t)\|_\infty \leq
    \]
    \[
        \frac{1}{2} \cdot \|x-t\|_\infty \quad (x \in \R^n, \, x, \, t \in G_r(0)),
    \]
    speciálisan tetszőleges $y \in K_{r/2}(0)$ mellett
    \[
        \| \Phi_y(x) \|_\infty \leq \|g(x)\|_\infty + \|y\|_\infty = \|g(x) - g(0)\|_\infty + \|y\|_\infty \leq
    \]
    \[
        \frac{1}{2} \cdot \|x\|_\infty + \|y\|_\infty < r \quad (x \in G_r(0)),
    \]
    azaz $\Phi_y(x) \in K_r(0)$. Mindez azt jelenti, hogy bármelyik $y \in K_{r/2}(0)$ esetén a
    \[
        \Phi_y : G_r(0) \to K_r(0) \quad (\subset G_r(0))
    \]
    leképezés kontrakció. A fixponttétel alkalmazásával ezért $y \in K_{r/2}(0)$ mellett egyértelműen létezik olyan $x_y \in G_r(0)$, ami fixpontja a $\Phi_y$-nak:
    \[
        \Phi_y(x_y) = x_y.
    \]
    Definiálhatunk tehát egy
    \[
        \varphi : K_{r/2}(0) \to G_r(0)
    \]
    leképezést, amelyre
    \[
        \varphi(y) := x_y \quad (y \in K_{r/2}(0)),
    \]
    azaz
    \[
        \Phi_y(\varphi(y)) = \varphi(y) \quad (y \in K_{r/2}(0)).
    \]
    Más szóval
    \[
        \Phi_y(\varphi(y)) = \varphi(y) - f(\varphi(y)) + y = g(\varphi(y)) + y = \varphi(y) \quad (y \in K_{r/2}(0)),
    \]
    amiből
    \[
        f(\varphi(y)) = y \quad (y \in K_{r/2}(0))
    \]
    is rögtön következik. Az eddigieket figyelembe véve
    \[
        \|\varphi(y)\|_\infty = \|\Phi_y(\varphi(y))\|_\infty < r \quad (y \in K_{r/2}(0))
    \]
    miatt
    \[
        \varphi : K_{r/2}(0) \to K_r(0).
    \]
    Világos, hogy $\varphi(0) = 0$, hiszen $\Phi_0(0) = -f(0) + 0 = 0$, azaz $y = 0$-ra a $0 \in G_r(0)$ fixpontja a $\Phi_0$-nak.\\

    Mutassuk meg, hogy a $\varphi$ függvény folytonos. Ha ui. $y, \, z \in K_{r/2}(0)$, akkor
    \[
        \varphi(y) = \Phi_y(\varphi(y)) + y, \, \varphi(z) = \Phi_z(\varphi(z)) + z
    \]
    amiből
    \[
        \|\varphi(y) - \varphi(z)\|_\infty \leq \|g(\varphi(y)) - g(\varphi(z))\|_\infty + \|y-z\|_\infty \leq
    \]
    \[
        \frac{1}{2} \cdot \| \varphi(y) - \varphi(z) \|_\infty + \|y-z\|_\infty,
    \]
    ezért
    \[
        \| \varphi(y) - \varphi(z)\|_\infty \leq 2 \cdot \| y- z \|_\infty.
    \]
    Innen már világos, hogy a $\varphi$ (egyenletesen) folytonos. $\blacksquare$\\
    
    \tikz \node[theorem]
    {
        \textbf{Tétel (inverzfüggvény-tétel).} Legyen $1 \leq n \in \N$, és $f \in \R^n \to \R^n$. Tegyük fel, hogy egy $a \in \text{int} \, \D$ pontban $f \in C^1\{a\}$, $\det f'(a) \neq 0$. Ekkor alkalmas $K(a) \subset \D$ környezettel az $f_{|_{K(a)}}$ leszűkítés invertálható, a $g := (f_{|_{K(a)}})^{-1}$ lokális inverzfüggvény folytonosan differenciálható, és
		\[
			h'(x) = \big( f'(h(x)) \big)^{-1} \quad (x \in \mathcal{D}_h).
		\]
    };\\

    \newpage
    \section{A differenciálegyenlet (rendszer) fogalma. Kezdetiérték-probléma (Cauchy-feladat). Egzakt egyenlet, szeparábilis egyenlet, a rakéta emelkedési idejének a kiszámítása.}
    \subsection{Közönséges differenciálegyenletek}
    Legyen $0 < n \in \N, \, I \subset \R, \, \Omega \subset \R^n$ egy-egy nyílt intervallum. Tegyük fel, hogy az
    \[
        f : I \times \Omega \to \R^n
    \]    
    függvény folytonos, és tűzzük ki az alábbi feladat megoldását:\\

    határozzunk meg olyan $\varphi \in I \to \Omega$ függvényt, amelyre igazak a következő állítások:
    \begin{enumerate}
        \item $\Dp$ nyílt intervallum;
        \item $\varphi \in \D$;
        \item $\varphi'(x) = f(x, \, \varphi(x)) \quad (x \in \Dp)$.
    \end{enumerate}
    A most megfogalmazott feladatot \textit{explicit elsőrendű közönséges differenciálegyenletnek} (röviden \textit{differenciálegyenletnek}) fogjuk nevezni, és a \textit{d.e.} rövidítéssel idézni.\\

    Ha adottak a $\tau \in I, \, \xi \in \Omega$ elemek, akkor a fenti $\varphi$ függvény $1. \, 2.$ és 3. mellett tegyen eleget a
    \begin{enumerate}[start=4]
        \item $\tau \in \Dp$ és $\varphi(\tau) = \xi$
    \end{enumerate}
    kikötésnek is. Az így "kibővített" feladatot \textit{kezdetiérték-problémának} (vagy röviden \textit{Cauchy-feladatnak}) nevezzük, és a továbbiakban minderre a \textit{k.é.p.} rövidítést fogjuk használni. Az 1., 2., 3. feltételeknek (ill. az 1., 2., 3., 4. feltételeknek) eleget tevő bármelyik $\varphi$ függvényt a \textit{d.e.} (ill. a \textit{k.é.p.}) \textit{megoldásának} nevezzük. A fenti definícióban szereplő $f$ függvény az illető \textit{d.e.} ún. \textit{jobb oldala}.

    \subsection{Teljes megoldás}
    Azt mondjuk, hogy a szóban forgó \textit{k.é.p. egyértelműen oldható meg}, ha tetszőleges $\varphi, \, \psi$ megoldásai esetén
    \[
        \varphi(x) = \psi(x) \quad (x \in \Dp \cap \mathcal{D}_\psi).
    \]
    (Mivel $\tau \in \Dp \cap \mathcal{D}_\phi$, ezért $\Dp \cap \mathcal{D}_\phi$ egy ($\tau$-t tartalmazó) nyílt intervallum.) Legyen ekkor $\mathcal{M}$ a szóban forgó \textit{k.é.p.} megoldásainak a halmaza és
    \[
        J := \bigcup_{\varphi \in \mathcal{M}} \Dp.
    \]
    Ez egy $\tau$-t tartalmazó nyílt intervallum és $J \subset I$. Az egyértelmű megoldhatóság értelmezése miatt definiálhatjuk a
    \[
        \Phi : J \to \Omega
    \]
    függvényt az alábbiak szerint:
    \[
        \Phi(x) := \varphi(x) \quad (\varphi \in \mathcal{M}, \, x \in \Dp).
    \]
    Nyilvánvaló, hogy $\Phi(\tau) = \xi, \, \Phi \in D$ és
    \[
        \Phi'(x) = f(x, \, \Phi(x)) \quad (x \in J).
    \]
    Ez azt jelenti, hogy $\Phi \in \mathcal{M}$, és (ld. a $\mathcal{D}_\Phi = J$ definícióját) bármelyik $\varphi \in \mathcal{M}$ esetén
    \[
        \varphi(x) = \Phi(x) \quad (x \in \Dp),
    \]
    röviden $\varphi = \Phi_{|_{\Dp}}$.\\

    A $\Phi$ függvényt a kezdetiérték-probléma \textit{teljes megoldásának} nevezzük.

    \subsection{Szeparábilis differenciálegyenlet}
    Legyen $n := 1$, továbbá az $I, \, J \subset \R$ nyílt intervallumokkal és a
    \[
        g : I \to \R, \, h : J \to \R \, \backslash \, \{0\}
    \]
    folytonos függvényekkel
    \[
        f(x, \, y) := g(x) \cdot h(y) \quad ((x, \, y) \in I \times J).
    \]
    A $\varphi \in I \to J$ megoldásra tehát
    \[
        \varphi'(t) = g(t) \cdot h(\varphi(t)) \quad (t \in \Dp).
    \]
    Legyenek még adottak a $\tau \in I, \, \xi \in J$ számok, amikor is
    \[
        \tau \in \Dp, \, \varphi(\tau) = \xi
    \]
    (kezdetiérték-probléma).\\

    \tikz \node[theorem]
    {
        \textbf{Tétel.} \textit{Tetszőleges szeparábilis differenciálegyenletre vonatkozó kezdetiérték-probléma megoldható, és bármilyen $\varphi, \, \psi$ megoldásaira}
        \[
            \varphi(t) = \psi(t) \quad (t \in \D \cap \mathcal{D}_\psi).
        \]
    };\\

    \textbf{Bizonyítás.} Mivel a $h$ függvény sehol sem nulla, ezért egy $\varphi$ megoldásra
    \[
        \frac{\varphi'(t)}{h(\varphi(t))} = g(t) \quad (t \in \Dp).
    \]
    A $g : I \to \R$ is, és az $1 / h : J \to \R$ is egy-egy nyílt intervallumon értelmezett folytonos függvény, így léteznek a
    \[
        G : I \to \R, \, H : J \to \R
    \]
    primitív függvényeik: $G' = g$ és $H' = 1/h$. Az összetett függvény deriválásával kapcsolatos tétel szerint
    \[
        \frac{\varphi'(t)}{h(\varphi(t))} = (H \circ \varphi)'(t) = g(t) = G'(t) \quad (t \in \Dp),
    \]
    azaz
    \[
        (H \circ \varphi - G)'(t) = 0 \quad (t \in \Dp).
    \]
    Tehát (mivel a $\Dp$ is egy nyílt intervallum) van olyan $c \in \R$, hogy
    \[
        H(\varphi(t)) - G(t) = c \quad (t \in \Dp).
    \]
    Az $1/h$ függvény nyilván nem vesz fel $0$-t a $J$ intervallum egyetlen pontjában sem, így ugyanez igaz a $H'$ függvényre is. A deriváltfüggvény Darboux-tulajdonsága miatt tehát a $H'$ állandó előjelű. Következésképpen a $H$ függvény szigorúan monoton függvény, amiért invertálható. A $H^{-1}$ inverz függvény segítségével ezért azt kapjuk, hogy
    \[
        \varphi(t) = H^{-1}(G(t) + c) \quad (t \in \Dp).
    \]
    Ha $\tau \in I, \, \xi \in J$, és a $\varphi$ megoldás eleget tesz a $\varphi(\tau) = \xi$ kezdeti feltételnek is, akkor
    \[
        \xi = H^{-1}(G(\tau) + c),
    \]
    azaz
    \[
        c = H(\xi) - G(\tau).
    \]
    Így
    \[
        \varphi(t) = H^{-1}\big(G(t) + H(\xi) - G(\tau)\big)  \quad (t \in \Dp).
    \]
    Ha a $G, \, H$ helyett más primitív függvényeket választunk (legyenek ezek $\tilde{G}, \, \tilde{H}$), akkor alkalmas $\alpha, \, \beta \in \R$ konstansokkal
    \[
        \tilde{G} = G + \alpha, \, \tilde{H} = H + \beta,
    \]
    és
    \[
        \tilde{H}(\varphi(t)) - \tilde{G}(t) = H(\varphi(t)) - G(t) + \beta - \alpha = \tilde{c} \quad (t \in \Dp)
    \]
    adódik valamilyen $\tilde{c} \in \R$ konstanssal. Ezért
    \[
        \varphi(t) = H^{-1}(G(t) + \tilde{c} - \beta + \alpha) \quad (t \in \Dp),
    \]
    ahol (a $t := \tau$ helyettesítés után)
    \[
        H(\xi) - G(\tau) = \tilde{c} - \beta + \alpha,
    \]
    amiből megint csak
    \[
        \varphi(t) = H^{-1}(G(t) + H(\xi) - G(\tau)) \quad (t \in \Dp)
    \]
    következik. Ez azt jelenti, hogy a fentiekben mindegy, hogy melyik $G, \, H$ primitív függvényekből indulunk ki. Más szóval, ha a $\psi$ függvény is megoldása a vizsgált kezdetiérték-problémának, akkor
    \[
        \psi(t) = H^{-1}(G(t) + H(\xi) - G(\tau)) \quad (t \in \mathcal{D}_\psi).
    \]
    Mivel a $\Dp, \, \mathcal{D}_\psi$ értelmezési tartományok mindegyike egy-egy $\tau$-t tartalmazó nyílt intervallum, ezért $\Dp \cap \mathcal{D}_\psi$ is ilyen intervallum, és
    \[
        \psi(t) = \varphi(t) \quad (t \in \Dp \cap \mathcal{D}_\psi).
    \]
    Elegendő már csak azt belátnunk, hogy van megoldás. Tekintsük ehhez azokat a $G, \, H$ primitív függvényeket, amelyekre
    \[
        H(\xi) = G(\tau) = 0,
    \]
    és legyen
    \[
        F(x, \, y) := H(y) - G(x) \quad (x \in I, \, y \in J).
    \]
    Ekkor az
    \[
        F : I \times J \to \R
    \]
    függvényre léteznek és folytonosak a
    \[
        \partial_1 F(x, \, y) = -G'(x) = -g(x) \quad (x \in I, \, y \in J),
    \]
    \[
        \partial_2 F(x, \, y) = H'(y) = \frac{1}{h(y)} \quad (x \in I, \, y \in J)
    \]
    parciális deriváltfüggvények. Ez azt jelenti, hogy az $F$ függvény folytonosan differenciálható,
    \[
        F(\tau, \, \xi) = H(\xi) - G(\tau) = 0,
    \]
    továbbá
    \[
        \partial_2 F(\tau, \, \xi) = H'(\xi) = \frac{1}{h(\xi)} \neq 0.
    \]
    Ezért az $F$-re alkalmazható az implicitfüggvény-tétel, miszerint alkalmas $K(\tau) \subset I, \, K(\xi) \subset J$ környezetekkel létezik az $F$ által a $(\tau, \, \xi)$ körül meghatározott
    \[
        \varphi : K(\tau) \to K(\xi)
    \]
    folytonosan differenciálható implicitfüggvény, amire $\varphi(\tau) = \xi$ és
    \[
        \varphi'(t) = -\frac{\partial_1 F(t, \, \varphi(t))}{\partial_2 F(t, \, \varphi(t))} = g(t) \cdot h(\varphi(t)) \quad (t \in K(\tau)).
    \]
    Röviden: a $\varphi$ implicitfüggvény megoldása a szóban forgó kezdetiérték-problémának. $\blacksquare$\\

    A tétel bizonyítása során egyúttal egy módszert ("képletet") is adtunk a
    \[
        \varphi'(t) = g(t) \cdot h(\varphi(t)) \quad (t \in \Dp)
    \]
    szeparábilis egyenletre vonatkozó $\varphi(\tau) =\xi$ kezdetiérték-probléma megoldására. Ennek a kulcsmozzanata a $G, \, H$ primitív függvények meghatározása:
    \[
        G(x) = \int_\tau^x g(t) \, dt, \, H(y) = \int_\xi^y \frac{dt}{h(t)} \quad (x \in I, \, y \in J),
    \]
    amiből aztán a $\varphi$-re vonatkozó
    \[
        H\big( \varphi(x) \big) = \int_\xi^{\varphi(x)} \frac{dt}{h(t)} = G(x) = \int_\tau^x g(t) \, dt \quad (x \in \Dp),
    \]
    azaz
    \[
        \int_\xi^{\varphi(x)} \frac{dt}{h(t)} = \int_\tau^x g(t) \, dt \quad (x \in \Dp)
    \]
    implicit egyenlet adódott.

    \newpage
    \section{Lipschitz-feltétel. A Picard-Lindelöf-féle egzisztencia-tétel. A k.é.p. megoldásának az egyértelműsége, unicitási tétel.}
    \subsection{Lipschitz-feltétel}
    Az előzőekben definiáltuk a \textit{k.é.p.} fogalmát: határozzunk meg olyan $\varphi \in I \to \Omega$ függvényt, amelyre (a korábban bevezetett jelölésekkel) igazak a következő állítások:
    \begin{enumerate}
        \item $\mathcal{D}_\varphi$ nyílt intervallum;
        \item $\varphi \in D$;
        \item $\varphi'(x) = f(x, \, \varphi(x)) \quad (x \in \Dp)$;
        \item adott $\tau \in I, \, \xi \in \Omega$ mellett $\tau \in \Dp$ és $\varphi(\tau) = \xi$.
    \end{enumerate}
    Értelmeztünk a megoldást, az egyértelműen való megoldhatóságot, a teljes megoldást. Speciális esetekben meg is oldottuk a gyakorlat számára is fontos kezdetiérték-problémákat. A továbbiakban megmutatjuk, hogy bizonyos feltételek mellett egy \textit{k.é.p.} mindig megoldható (egzisztenciatétel).\\

    Legyenek tehát $0 < n \in \N$ mellett az $I \subset \R, \, \Omega \subset \R^n$ nyílt intervallumok, az
    \[
        f : I \times \Omega \to \R^n
    \]
    függvény pedig legyen folytonos. A $\tau \in I, \, \xi \in \Omega$ esetén keressük a fenti differenciálható $\varphi \in I \to \Omega$ függvényt. Az $f$ függvényről feltesszük, hogy minden kompakt $\emptyset \neq Q \subset \Omega$ halmazhoz létezik olyan $L_Q \geq 0$ konstans, amellyel
    \[
        \| f(t, \, y) - f(t, \, z) \|_\infty \leq L_Q \cdot \| y - z \|_\infty \quad (t \in I, \, y, \, z \in Q).
    \]
    Ekkor azt mondjuk, hogy az $f$ (a \textit{d.e.} jobb oldala) eleget tesz a \textit{Lipschitz-feltételnek}.

    \subsection{Egzisztenciatétel}
    \begin{center}
        \tikz \node[theorem]
        {
            \textbf{Tétel (Picard-Lindelöf).} \textit{Tegyük fel, hogy egy differenciálegyenlet jobb oldala eleget tesz a Lipschitz-feltételnek. Ekkor a szóban forgó differenciálegyenletre vonatkozó tetszőleges kezdetiérték-probléma megoldható.}
        };    
    \end{center}
    \textbf{Bizonyítás (vázlat).} Legyenek a $\delta_1, \, \delta_2 > 0$ olyan számok, hogy
    \[
        I_* := [\tau - \delta_1, \, \tau + \delta_2] \subset I,
    \]
    és tekintsük az alábbi függvényhalmazt:
    \[
        \mathcal{F} := \{ \psi : I_* \to \Omega : \psi \in C \}.
    \]
    Az $\mathcal{F}$ halmaz a
    \[
        \rho(\phi, \, \psi) := \max \{ \| \phi(x) - \psi(x)\|_\infty : x \in I_* \} \quad (\phi, \, \psi \in \mathcal{F})
    \]
    távolságfüggvénnyel teljes metrikus tér. Ha $\mathcal{X}$ jelöli a
    \[
        g : I_* \to \R^n
    \]
    függvények összességét, akkor definiáljuk a
    \[
        T : \mathcal{F} \to \mathcal{X}
    \]
    leképezést a következőképpen:
    \[
        T\psi(x) := \xi + \int_\tau^x f(t, \, \psi(t)) \, dt \in \R^n \quad (\psi \in \mathcal{F}, \, x \in I_*).
    \]
    Tehát az $f$ függvény koordinátafüggvényeit a "szokásos" $f_1, \, \dots, \, f_n$ szimbólumokkal jelölve, a $\psi$, $f$ függvények (és egyúttal az $f_i$-k) folytonossága miatt
    \[
        I_* \ni t \mapsto f_i(t, \, \psi(t)) \in \R \quad (i = 1, \, \dots, \, n)
    \]
    függvények folytonosak. következőképpen (minden $x \in I_*$ esetén) van értelme a
    \[
        d_i := \int_\tau^x f_i(t, \, \psi(t)) \, dt \quad (i = 1, \, \dots, \, n)
    \]
    integráloknak, és így a
    \[
        \xi + \int_\tau^x f(t, \, \psi(t)) \, dt := (\xi_1 + d_1, \, \dots, \, \xi_n + d_n) \in \R^n
    \]
    "integrálvektoroknak". Továbbá az integrálfüggvények tulajdonságai miatt a $T\psi$ függvény folytonos, minden $x \in (\tau - \delta_1, \, \tau + \delta_2)$ helyen differenciálható, és
    \[
        (T\psi)'(x) = f(x, \, \psi(x)).
    \]
    Belátjuk, hogy az $I_*$ alkalmas megválasztásával minden $\psi \in \mathcal{F}$ függvényre $T\psi \in \mathcal{F}$, azaz ekkor
    \[
        T : \mathcal{F} \to \mathcal{F}.
    \]
    Ehhez azt kell biztosítani, hogy
    \[
        \xi + \int_\tau^x f(t, \, \psi(t)) \, dt \in \Omega \quad (x \in I_*)
    \]
    teljesüljön. Válasszuk ehhez először is a $\mu > 0$ számot úgy, hogy a
    \[
        K_\mu := \{ y \in \R^n : \| y - \xi\|_\infty \leq \mu \} \subset \Omega
    \]
    tartalmazás fennáljon (ilyen $\mu$ az $\Omega$ nyíltsága miatt létezik), és legyen
    \[
        M := \max\{ \| f(x, \, y)\|_\infty : x\in I_*, \, y \in K_\mu \}
    \]
    (ami meg az $f$ folytonossága és a Weierstrass-tétel miatt létezik, ti. az $I_* \times K_\mu$ halmaz kompakt). A jelzett $T\psi \in \mathcal{F}$ tartalmazás nyilván teljesül, ha
    \[
        \max \left\{  \left| \int_\tau^x f_i(t, \, \psi(t)) \, dt \right| : i = 1, \, \dots, \, n \right\} \leq \mu \quad (x \in I_*).
    \]
    Módosítsuk most már az $\mathcal{F}$ definícióját úgy, hogy
    \[
        \mathcal{F} := \{ \psi : I_* \to K_\mu : \psi \in C \}.
    \]
    Ekkor az előbbi maximum becsülhető $M \cdot \delta$-val, ahol
    \[
        \delta := \max\{\delta_1, \, \delta_2\}.
    \]
    Így $M \cdot \delta \leq \mu$ esetén a fenti $T\psi$ is $\mathcal{F}$-beli. (Ha a kiindulásul választott $\delta_1, \, \delta_2$-re $M \cdot \delta > \mu$, akkor írjunk a $\delta_1, \, \delta_2$ helyébe olyan "új" $0 < \tilde{\delta_1}, \, \tilde{\delta_2}$-t, hogy
    \[
        [\tau - \tilde{\delta_1}, \, \tau + \tilde{\delta_2}] \subset [\tau - \delta_1, \, \tau + \delta_2]
    \]
    és
    \[
        M \cdot \max \{ \tilde{\delta_1}, \, \tilde{\delta_2} \} \leq \mu
    \]
    legyen. Az $I_*$ helyett az $\tilde{I_*} := [\tau - \tilde{\delta_1}, \, \tau + \tilde{\delta_2}]$ intervallummal az "új" $M$ az előzőnél legfeljebb kisebb lesz, így az
    \[
        M \cdot \max \{ \tilde{\delta_1}, \, \tilde{\delta_2}\} \leq \mu
    \]
    becslés nem "romlik" el.)
    Ezzel értelmeztünk egy $T : \mathcal{F} \to \mathcal{F}$ leképezést, amelyre tetszőleges $\phi, \, \psi \in \mathcal{F}$ mellett
    \[
        \rho(T\psi, \, T\phi) = \max \{ \| T\psi(x) - T\phi(x)\|_\infty : x \in I_* \} =
    \]
    \[
        \max \left\{  \max \left\{  \left| \int_\tau^x (f_i(t, \, \psi(t)) - f_i(t, \, \phi(t)) \, dt) \right| : i = 1, \, \dots, \, n \right\} : x \in I_* \right\} \leq
    \]
    \[
        \max \left\{  \left| \int_\tau^x \max \{ |f_i(t, \, \psi(t)) - f_i(t, \, \phi(t))| : i = 1, \, \dots, \, n \} \, dt \right| : x \in I_* \right\} =
    \]
    \[
        \max \left\{ \left| \int_\tau^x \| f(t, \, \psi(t)) - f(t, \, \phi(t))\|_\infty \, dt \right| : x \in I_* \right\}.
    \]
    A Lipschitz-feltétel miatt a $Q := K_\mu$ (nyilván kompakt) halmazhoz van olyan $L_Q \geq 0$ konstans, amellyel
    \[
        \| f(t, \, y) - f(t, \, z)\|_\infty \leq L_Q \cdot \| y-z \|_\infty \quad (t \in I, \, y, \, z \in Q),
    \]
    speciálisan
    \[
        \| f(t, \, \psi(t)) - f(t, \, \phi(t))\|_\infty \leq
    \]
    \[
        L_Q \cdot \| \psi(t) - \phi(t) \|_\infty \leq L_Q \cdot \rho(\psi, \, \phi) \quad (t \in I_*).
    \]
    Ezért
    \[
        \rho(T\psi, \, T\phi) \leq L_Q \cdot \delta \cdot \rho(\psi, \, \phi).
    \]
    Tehát a $T$ leképezés
    \[
        L_Q \cdot \max \{\delta_1, \, \delta_2\} < 1
    \]
    esetén kontrakció. Válasszuk így a $\delta_1, \, \delta_2$-t, (ezt - az "eddigi" $I_*$-ot legfeljebb újra leszűkítve - megtehetjük), és alkalmazzuk a fixpont-tételt, miszerint van olyan $\psi \in \mathcal{F}$, amelyre
    \[
        T\phi = \phi.
    \]
    Legyen
    \[
        \varphi(x) := \psi(x) \quad \big( x \in (\tau - \delta_1, \, \tau + \delta_2) \big).
    \]
    A $T$ definíciója szerint
    \[
        \varphi(x) = \xi + \int_\tau^x f(t, \, \varphi(t)) \, dt \quad \big( x \in (\tau - \delta_1, \, \tau + \delta_2) \big).
    \]
    Ez azt jelenti, hogy a $\varphi$ függvény egy folytonos függvény integrálfüggvénye, ezért $\varphi \in D$ és
    \[
        \varphi'(x) = f(x, \, \varphi(x)) \quad \big( x \in (\tau - \delta_1, \, \tau + \delta_2) \big).
    \]
    Világos, hogy a $\varphi(\tau) = \xi$, más szóval a $\varphi$ megoldása a szóban forgó kezdetiérték-problémának. $\blacksquare$
    
    


\end{document}