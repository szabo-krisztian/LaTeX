\documentclass[12pt]{article}
\usepackage[utf8]{inputenc} % Ensures proper encoding
\usepackage[left=0.9in, right=0.9in, top=0.7in, bottom=0.7in]{geometry}
\usepackage{tikz}
\usepackage{xcolor} % Explicitly include xcolor for color commands
\usepackage{setspace}
\usepackage{hyperref}
\usepackage{amsfonts, amssymb, amsmath} 
\usepackage{mathrsfs} % **Added Package**
\usepackage{titlesec}
\usepackage{pgfplots}
\pgfplotsset{compat=newest}
\usepackage{graphicx}
\usepackage{wrapfig}
\usepackage{caption}
\usepackage{enumitem}
\usetikzlibrary{shadows.blur}
\usepackage{lmodern}
\setlength{\parskip}{0pt}
\setlength{\parindent}{0pt}


\title{\textcolor{purple}{\Huge\textbf{Matematikai statisztika}}}

\renewcommand{\contentsname}{Tartalom}
\newcommand{\R}{\mathbb{R}}
\newcommand{\p}{\mathbf{P}}
\newcommand{\N}{\mathbb{N}}
\newcommand{\E}{\exists}
\newcommand{\mm}{\mathbf{m}}
\newcommand{\MM}{\mathbf{M}}
\newcommand{\K}{\mathbb{K}}
\newcommand{\D}{\mathcal{D}_f}

\definecolor{modernyellow}{HTML}{F4E4BC}
\definecolor{moderngreen}{HTML}{BDDCBD}

\tikzset
{
	definition/.style={
		draw,
		fill=white,
		line width=1pt,
		rounded corners,
		drop shadow={shadow blur steps=5, shadow xshift=1ex, shadow yshift=-1ex},
		text width=0.9\textwidth,
		inner sep=10pt
	},
	theorem/.style={
		draw,
		fill=white,
		line width=1pt,
		rounded corners,
		drop shadow={shadow blur steps=5, shadow xshift=1ex, shadow yshift=-1ex},
		text width=0.9\textwidth,
		inner sep=10pt
	},
	proof/.style={
		fill=white,
		rectangle,
		drop shadow={shadow blur steps=5, shadow xshift=1ex, shadow yshift=-1ex, shadow color=gray},
		text width=0.9\textwidth,
		inner sep=6pt,
	},
	proof1/.style={
		fill=white,
		rectangle,
		drop shadow={shadow blur steps=5, shadow xshift=1ex, shadow yshift=0, shadow color=gray},
		text width=0.9\textwidth,
		inner sep=6pt,
	}
}

\begin{document}
	\maketitle
	\tableofcontents
    \newpage

	\section{Alapfogalmak}
	\subsection{Kolmogorov-féle axiómarendszer}
	\tikz \node[definition]
	{
		\textbf{Definíció.} $\Omega$ legyen egy halmaz, jelölje a \textit{lehetséges kimenetelek halmazát}. Ekkor legyen $\mathscr{A} \subseteq \mathcal{P}(\Omega)$ a \textit{ megfigyelhető események családja}, továbbá egy $\mathbf{P} : \mathscr{A} \to \R$ függvény. A valószínűségszámítás axiómái szerint az $(\Omega, \, \mathscr{A}, \, \mathbf{P})$ \textit{valószínűségi mező} a következő tulajdonságokkal rendelkezik:
        \begin{enumerate}
            \item Minden $A \in \mathscr{A}$ esetén $\mathbf{P}(A) \geq 0$.
            \item $\Omega \in \mathscr{A}$ és $\mathbf{P}(\Omega) = 1$.
            \item Ha $A \in \mathscr{A}$, akkor $A^c \in \mathscr{A}$ is igaz.
            \item Ha $A_1,\, A_2, \, \dots$ legfeljebb megszámlálható sok esemény, akkor
            \[
                \bigcup_{k=1}^\infty A_k \in \mathscr{A}.
            \] 
            \item Ha $A_1, \, A_2, \, \dots$ legfeljebb megszámlálható sok páronként diszjunkt esemény, akkor
            \[
                \mathbf{P}\left( \bigcup_{k=1}^\infty A_k \right) = \sum_{k=1}^\infty \mathbf{P}(A_k).
            \]
        \end{enumerate}
	};\\

    Legyen $A_1, \, A_2, \, \dots$ legfeljebb megszámlálható sok esemény, ekkor
    \[
        \bigcap_{k=1}^\infty A_k = \bigcap_{k=1}^\infty (A_k^c)^c = \left( \bigcup_{k=1}^\infty A_k^c\right)^c.
    \]
    Azaz metszetre is zártak az események.\\

    A valószínűségi mező definíciójából egy sor kézenfekvő tulajdonság vezethető le. Ezek közül tekintsük néhány igen egyszerűt:
    \begin{enumerate}
        \item Tetszőleges $A \in \mathcal{A}$ esetén $\mathbf{P}(A^c) = 1 - \mathbf{P}(A)$. Valóban
        \[
            1 = \mathbf{P}(\Omega) = \mathbf{P}(A \cup A^c) = \mathbf{P}(A) + \mathbf{P}(A^c).
        \]
        \item Ha $A, \, B \in \mathscr{A}$ és $A \subseteq B$, akkor $\mathbf{P}(A \backslash B) = \mathbf{P}(B) - \mathbf{P}(A)$, speciálisan $\mathbf{P}(B \backslash A) \geq 0$ miatt $\p(B) \geq \p(a).$ Valóban
        \[
            (B \backslash A) \cup A = (B \cap A^c) \cup A = (B \cap A^c) \cup (B \cap A) = B \cup (A \cup A^c) = B \cap \Omega = B,
        \]
        ebből
        \[
            \p(B \backslash A) + \p(A) = \p(B).
        \]
    \end{enumerate}

    \tikz \node[theorem]
    {
        Legyen $(A_n) : \N \to \Omega$ egy halmazsorozat. Azt mondjuk, hogy a sorozat határértéke a $\mathscr{H}$ halmaz, ha
        \[
            
        \]
    };

\end{document}
