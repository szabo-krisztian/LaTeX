\documentclass[12pt]{article}
\usepackage[left=0.9in, right=0.9in, top=1in, bottom=1in]{geometry}
\usepackage{tikz}
\usepackage{setspace}
\usepackage{hyperref}
\usepackage{amsfonts, amssymb, amsmath} 
\usepackage{titlesec}
\usepackage{pgfplots}
\pgfplotsset{compat=newest}
\usepackage{graphicx}
\usepackage{wrapfig}
\usepackage{caption}
\usepackage{enumitem}
\usetikzlibrary{shadows.blur}
\usepackage{lmodern}
\setlength{\parskip}{0pt}
\setlength{\parindent}{0pt}

\title{\textcolor{purple}{\Huge\textbf{Analízis IV}}}
\author{1. gyakorlat}
\date{Szabó Krisztián}

\renewcommand{\contentsname}{Tartalom}
\newcommand{\R}{\mathbb{R}}
\newcommand{\N}{\mathbb{N}}
\newcommand{\E}{\exists}
\newcommand{\mm}{\mathbf{m}}
\newcommand{\MM}{\mathbf{M}}
\newcommand{\K}{\mathbb{K}}
\newcommand{\D}{\mathcal{D}_f}

\definecolor{modernyellow}{HTML}{F4E4BC}
\definecolor{moderngreen}{HTML}{BDDCBD}

\tikzset
{
	definition/.style={
		draw,
		fill=modernyellow,
		line width=1pt,
		rounded corners,
		drop shadow={shadow blur steps=5,shadow xshift=1ex,shadow yshift=-1ex},
		text width=0.9\textwidth,
		inner sep=10pt
	},
	theorem/.style={
		draw,
		fill=moderngreen,
		line width=1pt,
		rounded corners,
		drop shadow={shadow blur steps=5,shadow xshift=1ex,shadow yshift=-1ex},
		text width=0.9\textwidth,
		inner sep=10pt
	},
	proof/.style={
		fill=white,
		rectangle,
		drop shadow={shadow blur steps=5,shadow xshift=1ex,shadow yshift=-1ex, moderngreen},
		text width=0.9\textwidth,
		inner sep=6pt,
	},
	proof1/.style={
		fill=white,
		rectangle,
		drop shadow={shadow blur steps=5,shadow xshift=1ex,shadow yshift=0, moderngreen},
		text width=0.9\textwidth,
		inner sep=6pt,
	}
}

\begin{document}
    \maketitle
    \tableofcontents
    \newpage
    \section{Feladat}
    Hengerkoordinátákra áttérve számítsuk ki az alábbi integrált:
    \[
        \int\int\int_H (x^2 + y^2) \, dx \, dy \, dz,
    \]
    ahol $H$ az alábbi egyenletű felületek által határolt korlátos és zárt térrész:
    \[
        x^2 + y^2 = 2z, \quad \wedge \quad z = 2.
    \]

    Vizsgáljuk meg a $H$ halmazt geometriailag:
    \[
        z = \frac{x^2 + y^2}{2}
    \]

    Ha a $z = 2$ sík berajzolásra kerülne, akkor az alábbi korlátos és zárt térrészt kapjuk:
    \[
        H := \left\{ \left(x, \, y, \, \frac{x^2 + y^2}{2} \right) \in \R^3 : \frac{x^2 + y^2}{2} \leq 2 \right\}.
    \]
    Ahhoz, hogy hengertranszformációt alkalmazzunk, induljunk ki a következő ötletből: a $z$ értékek fussák be a $[0, \, 2]$ intervallumot, majd minden $z$ pontnál kapjuk az alábbi körlapot:
    \[
        x^2 + y^2 \leq \left( \sqrt{2z} \right)^2.
    \]
    Ezekből az információkból rakjuk össze a következő transzformációt:
    \[
        H \ni    
        \begin{pmatrix}
            x \\
            y \\
            z
        \end{pmatrix}
        \sim
        \Phi(r, \, \varphi, \, z) :=
        \begin{pmatrix}
            r \cdot \cos(\varphi) \\
            r \cdot \sin(\varphi) \\
            z
        \end{pmatrix}
        \quad \left(z \in [0, \, 2], \, \varphi \in [0, \, 2\pi], \, r \in [0, \, \sqrt{2z}]\right).
    \]
    Az integráltranszformáció tétele alapján az eredeti integál a következő alakot ölti:
    \[
        \int\limits_0^2 \int\limits_0^{\sqrt{2z}} \int\limits_0^{2\pi} r^2 \cdot r \, d\varphi \, dr \, dz
    \].

    \newpage
    \section{Feladat}
    Határozzuk meg az alábbi $K$ kúptest \textit{tehetetlenségi nyomatékát} a $z$ illetve az $x$ tengelyre nézve, ha a kitöltő anyag sűrűsége minden pontban egyenesen arányos az origótól mért távolság négyzetével:
    \[
        K := \left\{ (x, \, y, \, z) \in \R^3 : 0 \leq \frac{1}{2} \cdot \sqrt{y^2 + z^2} \leq x \leq 1 \right\}.
    \]
    

\end{document}