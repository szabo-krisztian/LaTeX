\newpage
\section{Vizsgakérdés}
\begin{quote}
	\textit{Feltételes szélsőérték, szükséges, ill. elégséges feltétel (a szükséges feltétel bizonyítása).}
\end{quote}

Legyen $1 \leq n, \, m \in \N, \, \emptyset \neq U \subset \R^n$, és
\[
	f : U \to \R, \, g = (g_1, \, \dots, \, g_m) : U \to \R^m.
\]
Azt mondjuk, hogy az $f$ függvénynek a $g = 0$ \textit{feltételre vonatkozóan feltételes lokális maximuma (minimuma) van} a
\[
	c \in \{ g = 0 \} := \{ \xi \in U : g(\xi) = 0 \}
\]
pontban, ha az
\[
	\tilde{f}(\xi) := f(\xi) \quad (\xi \in \{g = 0\})
\]
függvénynek a $c$-ben lokális maximuma (minimuma) van. Feltesszük, hogy
\[
	\{ g = 0 \} \neq \emptyset.
\]
Használjuk az $f(c)$-re a \textit{feltételes lokális maximum (minimum)}, ill. \textit{szélsőérték}, továbbá $c$-re a \textit{feltételes lokális maximumhely (minimumhely)}, ill. \textit{szélsőértékhely} elnevezést is.\\

Ha tehát az $f$-nek a $c \in \{g=0\}$ helyen feltételes lokális szélsőértéke van a $g=0$ feltételre nézve, akkor egy alkalmas $K(c)$ környezettel
\[
	f(\xi) \leq f(c) \quad \big(\xi \in \{g=0\} \cap K(c)\big)
\]
(ha maximumról van szó), ill.
\[
	f(\xi) \geq f(c) \quad \big(\xi \in \{g=0\} \cap K(c)\big)
\]
(ha minimumról van szó) teljesül.

\subsection{Elsőrendű szükséges feltétel}

\tikz \node[theorem]
{
	\textbf{Tétel.} Tegyük fel, hogy $1 \leq n, \, m \in \N, \, m < n, \, \emptyset \neq U \subset \R^n$ nyílt halmaz, és $f : U \to \R, \, g : U \to \R^m$. Ha $f \in D, \, g \in C^1$, az $f$-nek a $c \in \{g=0\}$ helyen feltételes lokális szélsőértéke van a $g=0$ feltételre vonatkozóan, továbbá a $g'(c)$ Jacobi-mátrix rangja megegyezik $m$-mel, akkor létezik olyan $\lambda \in \R^m$ vektor, hogy
	\[
		\text{grad} \, (f + \lambda g)(c)=0.
	\]
};\\

A tételben szereplő $\lambda g$ függvényen a következőt értjük:
\[
(\lambda g)(\xi) := \langle \lambda, \, g(\xi) \rangle \quad (\xi \in U).
\]
Ez tehát ugyanolyan jellegű, mint a feltétel nélküli esetben, csak a szóban forgó $f$ függvény helyett (egy alkalmas $\lambda \in \R^m$ vektorral) az $F := f + \lambda g$ függvényre vonatkozóan.\\

Ez az analógia megmarad a másodrendű feltételeket illetően is.\\

\textbf{Bizonyítás.} A rangfeltétel szerint a $g'(c) \in \R^{m\times n}$ Jacobi-mátrixnak van olyan $A \in \R^{m \times m}$ részmátrixa, amelyre $\det A \neq 0$. Feltehető, hogy az $A$-t a $g'(c)$ mátrix utolsó $m$ oszlopa határozza meg, amikor is az $\R^n \equiv \R^{n-m} \times \R^{m}$ felbontást úgy képzeljük el, hogy a
\[
	\xi = (\xi_1, \, \dots, q, \xi_n) = (x, \, y) \in \R^n
\]
vektorokra
\[
	x := (\xi_1, \, \dots, \, \xi_{n-m}) \in \R^{n-m}, \, y := (\xi_{n-m+1}, \, \dots, \, \xi_n) \in \R^m.
\]
Legyen ennek megfelelően $c = (a, \, b)$. Ekkor tehát
\[
	\det \partial_2g(a, \, b) = \det A \neq 0.
\]
Mivel $g(a, \, b) = 0$, ezért alkalmazható az implicitfüggvény tétel: alkalmas
\[
	K(a) \subset \R^{n-m}, \, K(b) \subset \R^m
\]
környezettel létezik a $g$ függvény által az $(a, \, b)$ körül meghatározott
\[
	h : K(a) \to K(b)
\]
$h \in C^1$ implicitfüggvény:
\[
	\big( K(a) \times K(b) \big) \cap \{ g = 0\} = \{ (x, \, h(x)) \in U : x \in K(a)\},
\]
és
\[
	h'(x) = - \big( \partial_2g(x, \, h(x)) \big)^{-1} \cdot \partial_1g(x, \, h(x)) \quad \big( x \in K(a) \big).
\]
A feltételeink szerint egy alkalmas $K(c) \subset U$ környezettel (pl.)
\[
	f(\xi) \leq f(c) \quad \big(\xi = (x, \, y) \in K(c) \cap \{g=0\}\big).
\]
Nyilván feltehető, hogy
\[
	K(a) \times K(b) \subset K(c),
\]
ezért a
\[
	\Phi(x) := f(x, \, h(x)) \quad \big( x \in K(a) \big)
\]
függvényre $\Phi \in \R^{n-m} \to \R$ és
\[
	\Phi(x) \leq f(c) = \Phi(a) \quad \big( x \in K(a) \big).
\]
Más szóval a $\Phi$ függvénynek az $a$-ban lokális maximuma van. A $\Phi$ differenciálható, ezért
\[
	\Phi'(a) = \text{grad} \, \Phi(a) = 0.
\]
A
\[
	\p(x) := (x, \, h(x)) \quad \big(x \in K(a)\big)
\]
függvénnyel $\Phi = f \circ \p$ és $\p \in D$, valamint $I$-vel jelölve az $\R^{(n-m) \times (n-m)}$-beli egységmátrixot
\[
	\p'(x) = \begin{bmatrix}
		I\\
		h'(x)
	\end{bmatrix} \in \R^{n \times (n-m)} \quad \big(x \in K(a)\big).
\]
Következésképpen
\[
	0 = \Phi'(a) = f'(\p(a)) \cdot \p'(a) = f'(a, \, h(a)) \cdot \begin{bmatrix}
		I\\
		h'(a)
	\end{bmatrix} =
\]
\[
	f'(c) \cdot \begin{bmatrix}
		I\\
		h'(a)
	\end{bmatrix} = \partial_1f(c) + \partial_2f(c) \cdot h'(a) =
\]
\[
	\partial_1f(c) - \partial_2f(c) \cdot \big( \partial_2 f(c) \big)^{-1} \cdot \partial_1g(c) = \partial_1f(c) + \lambda \cdot \partial_1g(c),
\]
ahol
\[
	\lambda := -\partial_2f(c) \cdot \big( \partial_2g(c) \big)^{-1} \in \R^m.
\]
Tehát (a $\partial_1$ értelmezéséből)
\begin{equation}
	\partial_k f(c) + \sum_{l=1}^{m} \lambda_l \cdot \partial_k g_l(c) = 0 \quad (k = 1, \dots, n - m).
	\label{eq:eq1}
\end{equation}
A $\lambda$ vektor definíciójából "átszorzással" azt kapjuk, hogy
\[
	\partial_2f(c) + \lambda \cdot g(c) = 0,
\]
azaz (a $\partial_2$ definíciójából)
\begin{equation}
	\partial_jf(c) + \sum_{l=1}^m \lambda_l \cdot \partial_j g_l(c) = 0 \quad (j = n-m+1, \, \dots, \, n).
	\label{eq:eq2}
\end{equation}
A (1), (2) formulák együtt nyilván azt jelentik, hogy
\[
	\partial_kf(c) + \sum_{l=1}^m \lambda_l \cdot \partial_k g_l(c) = 0 \quad (k = 1, \, \dots, \, n),
\]
más szóval
\[
	\text{grad} \, (f + \lambda g)(c) = 0 =
\]
\[
	\Big( \partial_1f(c) + \sum_{l=1}^m \lambda_l \cdot \partial_1 g_l(c), \, \dots, \, \partial_nf(c) + \sum_{l=1}^m \lambda_l \cdot \partial_n g_l(c) \Big) = 0. 
\]
$\hfill \blacksquare$

Legyen adott a
\[
	Q : \R^n \to \R
\]
kvadratikus alak, a $B \in \R^{m \times n}$ mátrix, és tekintsük az alábbi halmazt:
\[
	\mathcal{A}_B := \{ x \in \R^n : Bx = 0 \}.
\]
Feltesszük, hogy $m < n$, és a $B$ mátrix rangja $m$. Ekkor azt mondjuk, hogy a $Q$ kvadratikus alak a $B$-re nézve
\begin{enumerate}
	\item \textit{feltételesen pozitív definit}, ha $Q(x) > 0$ $(0 \neq x \in \mathcal{A}_B)$;
	\item \textit{feltételesen negatív definit}, ha $Q(x) > 0$ $(0 \neq x \in \mathcal{A}_B)$;
	\item \textit{feltételesen pozitív szemidefinit}, ha $Q(x) \geq 0$ $(x \in \mathcal{A}_B)$;
	\item \textit{feltételesen negatív szemidefinit}, ha $Q(x) \leq 0$ $(x \in \mathcal{A}_B)$.
\end{enumerate}

\subsection{Másodrendű elégséges feltétel}

\tikz \node[theorem]
{
	\textbf{Tétel.} Az $1 \leq n, \, m \in \N, m < n$ paraméterek mellett legyen adott az $\emptyset \neq U \subset \R^n$ nyílt halmaz, és tekintsük az $f : U \to \R, \, g : U \to \R^m$ függvényeket. Feltesszük, hogy $f, \, g \in D^2, \, c \in \{g=0\}$, a $g'(c)$ Jacobi-mátrix rangja $m$, továbbá valamilyen $\lambda \in \R^m$ vektorral az $F := f + \lambda g$ függvényre
	\begin{enumerate}
		\item $\text{grad} \, F(c) = 0$;
		\item A $Q_c^F$ kvadratikus alak a $g'(c)$ mátrixra nézve feltételesen pozitív (negatív) definit.
	\end{enumerate}
	Ekkor az $f$-nek a $c$-ben a $g=0$ feltételre vonatkozóan feltételes lokális minimuma (maximuma) van.
};\\

Idézzük fel, hogy
\[
	Q_c^F(x) := \langle F''(c) \cdot x, \, x \rangle \quad (x \in \R^n).
\]