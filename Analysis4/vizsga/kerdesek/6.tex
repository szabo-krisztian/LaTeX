\newpage
\section{Vizsgakérdés}
\begin{quote}
	\textit{A lineáris differenciálegyenlet-rendszer vizsgálata: homogén, inhomogén rendszerek. A megoldáshalmaz szerkezete.}
\end{quote}

\subsection{Lineáris differenciálegyenlet-rendszer}
Valamilyen $1 \leq n \in N$ és egy nyílt $I \subset \R$ intervallum esetén adottak a folytonos
\[
a_{ik} : I \to \R \quad (i, \, k = 1, \, \dots, \, n), \, b = (b_1, \, \dots, \, b_n) : I \to \R^n
\]
függvények, és tekintsük az
\[
I \ni x \mapsto A(x) := \big( a_{ik}(x) \big)_{i, \, k =1}^n \in \R^{n \times n}
\]
\textit{mátrixfüggvényt}. Ha
\[
f(x, \, y) := A(x) \cdot y + b(x) \quad ((x, \, y) \in I \times \K^n),
\]
akkor az $f$ függvény, mint jobb oldal által meghatározott
\[
\varphi'(x) = A(x) \cdot \varphi(x) + b(x) \quad (x \in \Dp)
\]
differenciálegyenletet \textit{lineáris differenciálegyenletnek} ($n > 1$ esetén \textit{lineáris} \textit{differenciálegyenlet-rendszernek}) nevezzük.\\

Legyenek a fentieken túl adottak még a $\tau \in I, \, \xi \in \K^n$ értékek, és vizsgáljuk a $\varphi(\tau) = \xi$ k.é.p.-t. Ha $I_* \subset I$, $\tau \in \text{int} \, I_*$, kompakt intervallum, akkor
\[
\sup \{ |a_{ik}(x)| : x \in I_* \} \in \R \quad (i, \, k = 1, \, \dots, \, n),
\]
ezért
\[
q := \sup \{ \|A(x)\|_{(\infty)} : x \in I_* \} \in \R.
\]
Következésképpen
\[
\| f(x, \, y) - f(x, \, z) \|_\infty = \| A(x) \cdot (y- z) \|_\infty \leq
\]
\[
\|A(x)\|_{(\infty)} \cdot \|y-z\|_\infty \leq q \cdot \|y-z\|_\infty \quad (x \in I_*, \, y, \, z \in \K^n).
\]
Továbbá a
\[
\beta := \sup\{\|b(x)\| : x \in I_*\} (\in \R)
\]
jelöléssel
\[
\|f(x, \, y)\|_\infty = \| A(x) \cdot y + b(x) \|_\infty \leq \|A(x) \cdot y\|_\infty + \|b(x)\|_\infty \leq
\]
\[
\|A(x)\|_{(\infty)} \cdot \|y\|_\infty + \|b(x)\|_\infty \leq q \cdot \|y\|_\infty + \beta \quad (x \in I_*, \, y \in \K^n),
\]
ezért minden k.é.p. teljes megoldása az $I$ intervallumon van értelmezve. Azt mondjuk, hogy a szóban forgó d.e. \textit{homogén}, ha $b \equiv 0$, \textit{inhomogén}, ha létezik $x \in I$, hogy $b(x) \neq 0$. Legyenek
\[
\mathcal{M}_h := \{ \psi : I \to \K^n : \psi \in D, \, \psi' = A \cdot \psi \},
\]
\[
\mathcal{M} := \{ \psi : I \to \K^n : \psi \in D, \, \psi' = A \cdot \psi + b\}.
\]
A lineáris d.e.-ek "alaptétele" a következő\\

\tikz \node[theorem]
{
	\textbf{Tétel.} A bevezetésben mondott feltételek mellett
	\begin{enumerate}
		\item az $\mathcal{M}_h$ halmaz $n$ dimenziós lineáris tér a $\K$-ra vonatkozóan;
		\item tetszőleges $\psi \in \mathcal{M}$ esetén
		\[
		\mathcal{M} = \psi + \mathcal{M}_h := \{\psi + \chi : \chi \in \mathcal{M}_h \};
		\]
		\item ha a $\phi_k = (\phi_{k1}, \, \dots, \, \phi_{kn})$ $(k = 1, \, \dots, \, n)$ függvények bázist alkotnak az $\mathcal{M}_h$-ban, akkor léteznek olyan $g_k : I \to \K$ $(k = 1, \, \dots, \, n)$ differenciálható függvények, amelyekkel
		\[
		\psi := \sum_{k=1}^n g_k \cdot \phi_k \in \mathcal{M}.
		\]
	\end{enumerate}
};\\