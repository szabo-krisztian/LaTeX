\newpage
\section{Vizsgakérdés}
\begin{quote}
	\textit{A trigonometrikus rendszer ortogonalitása. Egyenletesen konvergens trigonometrikus sor összegfüggvényének a Fourier-sora. Bessel-azonosság, Bessel-egyenlőtlenség.}
\end{quote}

\subsection{Trigonometrikus rendszer}

Legyen 
\[
	c_0(x) := 1, \, c_n(x) := \cos(nx), \, s_n(x) := \sin(nx) \quad (x \in \R, \, 1 \leq n \in \N),
\]
és
\[
	\mathcal{R} := \{ c_0, \, c_n, \, s_n : 1 \leq n \in \N \}
\]
az ún. \textit{trigonometrikus rendszer.}\\

\tikz \node[theorem]
{
	\textbf{Lemma.} A trigonometrikus rendszer ortogonális a következő értelemben: bármely $2\pi$ hosszúságú korlátos és zárt $I \subset \R$ intervallumra
	\[
		\int_I \p \cdot \psi = \begin{cases}
			0 & (\p \neq \psi) \\
			2 \pi & (\p = \psi = c_0) \\
			\pi & (\p = \psi \neq c_0)
		\end{cases} \quad (\p, \, \psi \in \mathcal{R}).
	\]
};\\

\textbf{Bizonyítás.} Az $\mathcal{R}$ elemeinek a $2\pi$ szerinti periodicitása miatt nyilván feltehető, hogy (pl.) $I = [0, \, 2\pi]$. Ekkor:
\[
	\int_I c_0^2(x) \, dx = \int_I 1 \, dx = 2\pi;
\]
\[
	\int_I c_n^2(x) \, dx = \int_I \cos^2(nx) \, dx = \int_I \frac{1 + \cos(2nx)}{2} \, dx = 
\]
\[
	\frac{1}{2} \left( \int_I 1 \, dx + \int_I \cos(2nx) \, dx \right) = \pi \quad (1 \leq n \in \N);
\]
\[
	\int_I s_n^2(x) \, dx = \int_I \sin^2(nx) \, dx = \int_I \frac{1 - \cos(2nx)}{2} \, dx = 
\]
\[
	\frac{1}{2} \left( \int_I 1 \, dx - \int_I \sin(2nx) \, dx \right) = \pi \quad (1 \leq n \in \N);
\]
\[
	\int_I s_n(x) \cdot c_k(x) \, dx = \int_I \sin(nx) \cdot \cos(kx) \, dx =
\]
\[
	\int_I \frac{\sin\B{(n+k)x} + \sin\B{(n-k)x}}{2} \, dx
\]
\[
	\frac{1}{2} \left( \int_I \sin\B{(n+k)x} \, dx + \int_I \sin\B{(n-k)x} \, dx \right) = 0 \quad (1 \leq n \in \N, \, k \in \N).
\]
ui. $n = k$ esetén
\[
	\int_I \sin\B{(n-k)x} \, dx = 0
\]
és
\[
	\int_I \sin\B{(n+k)x} \, dx = \int_I \sin(2nx) \, dx = 0,
\]
míg $n \neq k$ mellett alkalmas $1 \leq m, \, r \in \N$ paraméterekkel
\[
	\int_I \sin\B{(n+k)x} \, dx = \int_I \sin(mx) \, dx = 0 =
\]
\[
	\int_I \sin(rx) \, dx = \pm \int_I \sin\B{(n-k)x} \, dx.
\]
$\hfill \blacksquare$

\subsection{Egyenletesen konvergens trigonometrikus sorok}

Lássuk be, hogy egyenletesen konvergens trigonometrikus sorok együtthatói "kiszámíthatók" a sor összegfüggvényének a segítségével.\\

\tikz \node[theorem]
{
	\textbf{Tétel.} Tegyük fel, hogy a $\sum\B{ \alpha_n \cdot \cos(nx) + \beta_n \cdot \sin(nx) }$ trigonometrikus sor egyenletesen konvergens, legyen
	\[
		f(x) := \alpha_0 + \sum_{k=1}^\infty \B{ \alpha_k \cdot \cos(kx) + \beta_k \cdot \sin(kx) } \quad (x \in \R).
	\]
	Ekkor $f$ Fourier-sora megegyezik a szóban forgó trigonometrikus sorral, azaz
	\[
		\alpha_0 = a_0(f), \, \alpha_n = a_n(f), \, \beta_n = b_n(f) \quad (1 \leq n \in \N).
	\]
};\\

\textbf{Bizonyítás.} A tagonkénti integrálhatóságból
\[
	\int\limits_0^{2\pi} f(x) \, dx =
\]
\[
	\int\limits_0^{2\pi} \alpha_0 \, dx + \sum_{k=1}^\infty \left( \alpha_k \cdot \int\limits_0^{2\pi} \cos(kx) \, dx + \beta_k \cdot \int\limits_0^{2\pi} \sin(kx) \, dx \right) =
\]
\[
	\int\limits_0^{2\pi} \alpha_0 \, dx = 2\pi \cdot \alpha_0.
\]
Mivel a $\cos, \, \sin$ függvények korlátosak, ezért $1 \leq \N$ esetén a
\[
	\sum\BB{ \cos(nx) (\alpha_k \cdot \cos(kx) + \beta_k \cdot \sin(kx)) },
\]
\[
	\sum\BB{ \sin(nx) (\alpha_k \cdot \cos(kx) + \beta_k \cdot \sin(kx)) },	
\]
sorok is egyenletesen konvergensek. Megint csak a tagonkénti integrálhatóság és a trigonometrikus rendszer ortogonalitása miatt
\[
	\int\limits_0^{2\pi} f(x) \cdot \cos(nx) \, dx = \alpha_0 \cdot \int\limits_0^{2\pi} \cos(nx) \, dx +
\]

\[
	\sum_{k=1}^\infty \left( \alpha_k \cdot \int\limits_0^{2\pi} \cos(kx) \cdot \cos(nx) \, dx + \beta_k \cdot \int\limits_0^{2\pi} \sin(kx) \cdot \cos(nx) \, dx \right) =
\]
\[
	\alpha_n \cdot \int\limits_0^{2\pi} \cos^2(nx) \, dx = \pi \alpha_n,
\]
és
\[
	\int\limits_0^{2\pi} f(x) \cdot \sin(nx) \, dx = \alpha_0 \cdot \int\limits_0^{2\pi} \sin(nx) \, dx +
\]
\[
	\sum_{k=1}^\infty \left( \alpha_k \cdot \int\limits_0^{2\pi} \cos(kx) \cdot \sin(nx) \, dx + \beta_k \cdot \int\limits_0^{2\pi} \sin(kx) \cdot \sin(nx) \, dx \right) =
\]
\[
	\beta_n \cdot \int\limits_0^{2\pi} \sin^2(nx) \, dx = \pi \cdot \beta_n,
\]
amiből az állításunk már következik. $\hfill \blacksquare$\\

Tehát: ha a
\[
	\sum\Big( \alpha_k \cdot \cos(kx) + \beta_k \sin(kx)\Big)
\]
trigonometrikus sor egyenletesen konvergens és
\[
	f(x) := \alpha_0 + 	\sum_{k=1}^\infty\Big( \alpha_k \cdot \cos(kx) + \beta_k \sin(kx)\Big) \quad (x \in \R)
\]
az összegfüggvénye, akkor $f \in C_{2\pi}$, és a szóban forgó trigonometrikus sor az $f$ függvény Fourier-sora. Azt is mondjuk ilyenkor, hogy az $f$ függvény \textit{Fourier-sorba fejthető}.

\subsection{Bessel - Parseval}

Legyenek $f \in R_{2\pi}$ esetén
\[
	a_0 := \frac{1}{2\pi} \int\limits_0^{2\pi} f(x) \, dx,
\]
\[
	a_n := \frac{1}{\pi} \int\limits_0^{2\pi} f(x) \cdot \cos(nx) \, dx \quad (1 \leq n \in \N),
\]
\[
	b_n := \frac{1}{\pi} \int\limits_0^{2\pi} f(x) \cdot \sin(nx) \, dx \quad (1 \leq n \in \N),
\]
\[
	S_nf(x) := a_0 + \sum_{k=1}^n \BB{ a_n \cdot \cos(kx) + b_n \cdot \sin(kx) } \quad (n \in \N, \, x \in \R)
\]
az $f$ függvény Fourier-együtthatói, $n$-edik Fourier-részletösszege, és
\[
	\| f\| := \sqrt{ \int\limits_0^{2\pi} |f(x)|^2 \, dx }.
\]
Továbbá adott
\[
	\alpha_k \in \R \quad (k \in \N) \text{ és } \beta_j \in \R \quad (1 \leq j \in \N)
\]
együtthatókkal
\[
	T_n(x) := \alpha_0 + \sum_{k=1}^n \BB{ \alpha_k \cdot \cos(kx) + \beta_k \cdot \sin(kx) } \quad (n \in \N, \, x \in \R).
\]
Ekkor (a trigonometrikus rendszer ortogonalitása miatt)
\[
	\|f-T_n \|^2 = \int\limits_0^{2\pi} (f - T_n)^2 = \int\limits_0^{2\pi} f^2 - 2 \cdot \int\limits_0^{2\pi} f T_n + \int\limits_0^{2\pi} T^2_n =
\]
\[
	\int\limits_0^{2\pi} f^2 - 2\left( 2 \pi a_0 \alpha_0 + \pi \cdot \sum_{k=1}^n (\alpha_k a_k + \beta_k b_k) \right) + 2 \pi \alpha_0^2 + \pi \sum_{k=1}^n (\alpha_k^2 + \beta_k^2).
\]
Speciálisan (\textit{Bessel-azonosság})
\[
	\|f - S_nf\|^2 = \int\limits_0^{2\pi} f^2 - \pi \left( 2a_0^2 + \sum_{k=1}^n (a_k^2 + b_k^2) \right) \quad (n \in \N).
\]
Innen
\[
	2a_0^2 + \sum_{k=1}^\infty (a_k^2 + b_k^2) \leq \frac{1}{\pi} \int\limits_0^{2\pi} f^2
\]
(\textit{Bessel-egyenlőtlenség}). \\

Belátható, hogy a fenti Bessel-egyenlőtlenségben valójában egyenlőség van:
\[
	2a_0^2 + \sum_{k=1}^\infty (a_k^2 + b_k^2)  = \frac{1}{\pi} \int\limits_0^{2\pi} f^2
\]
(\textit{Parseval-egyenlőtlenség}).
