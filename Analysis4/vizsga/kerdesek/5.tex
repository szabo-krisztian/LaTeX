\newpage
\section{Vizsgakérdés}
\begin{quote}
	\textit{Lipschitz-feltétel. A Picard-Lindelöf-féle egzisztencia-tétel (a fixpont-tétel alkalmazása). A k.é.p. megoldásának az egyértelműsége, unicitási tétel (bizonyítás nélkül).}
\end{quote}

\subsection{Lipschitz-feltétel}
Az előzőekben definiáltuk a \textit{k.é.p.} fogalmát: határozzunk meg olyan $\varphi \in I \to \Omega$ függvényt, amelyre (a korábban bevezetett jelölésekkel) igazak a következő állítások:
\begin{enumerate}
	\item $\mathcal{D}_\varphi$ nyílt intervallum;
	\item $\varphi \in D$;
	\item $\varphi'(x) = f(x, \, \varphi(x)) \quad (x \in \Dp)$;
	\item adott $\tau \in I, \, \xi \in \Omega$ mellett $\tau \in \Dp$ és $\varphi(\tau) = \xi$.
\end{enumerate}
Értelmeztünk a megoldást, az egyértelműen való megoldhatóságot, a teljes megoldást. Speciális esetekben meg is oldottuk a gyakorlat számára is fontos kezdetiérték-problémákat. A továbbiakban megmutatjuk, hogy bizonyos feltételek mellett egy \textit{k.é.p.} mindig megoldható (egzisztenciatétel).\\

Legyenek tehát $0 < n \in \N$ mellett az $I \subset \R, \, \Omega \subset \R^n$ nyílt intervallumok, az
\[
f : I \times \Omega \to \R^n
\]
függvény pedig legyen folytonos. A $\tau \in I, \, \xi \in \Omega$ esetén keressük a fenti differenciálható $\varphi \in I \to \Omega$ függvényt. Az $f$ függvényről feltesszük, hogy minden kompakt $\emptyset \neq Q \subset \Omega$ halmazhoz létezik olyan $L_Q \geq 0$ konstans, amellyel
\[
\| f(t, \, y) - f(t, \, z) \|_\infty \leq L_Q \cdot \| y - z \|_\infty \quad (t \in I, \, y, \, z \in Q).
\]
Ekkor azt mondjuk, hogy az $f$ (a \textit{d.e.} jobb oldala) eleget tesz a \textit{Lipschitz-feltételnek}.

\subsection{Egzisztenciatétel}
\begin{center}
	\tikz \node[theorem]
	{
		\textbf{Tétel (Picard-Lindelöf).} Tegyük fel, hogy egy differenciálegyenlet jobb oldala eleget tesz a Lipschitz-feltételnek. Ekkor a szóban forgó differenciálegyenletre vonatkozó tetszőleges kezdetiérték-probléma megoldható.
	};    
\end{center}
\textbf{Bizonyítás (vázlat).} Legyenek a $\delta_1, \, \delta_2 > 0$ olyan számok, hogy
\[
I_* := [\tau - \delta_1, \, \tau + \delta_2] \subset I,
\]
és tekintsük az alábbi függvényhalmazt:
\[
\mathcal{F} := \{ \psi : I_* \to \Omega : \psi \in C \}.
\]
Az $\mathcal{F}$ halmaz a
\[
\rho(\phi, \, \psi) := \max \big\{ \| \phi(x) - \psi(x)\|_\infty : x \in I_* \big\} \quad (\phi, \, \psi \in \mathcal{F})
\]
távolságfüggvénnyel teljes metrikus tér. Ha $\mathcal{X}$ jelöli a
\[
g : I_* \to \R^n
\]
függvények összességét, akkor definiáljuk a
\[
T : \mathcal{F} \to \mathcal{X}
\]
leképezést a következőképpen:
\[
T\psi(x) := \xi + \int_\tau^x f(t, \, \psi(t)) \, dt \in \R^n \quad (\psi \in \mathcal{F}, \, x \in I_*).
\]
Tehát az $f$ függvény koordinátafüggvényeit a "szokásos" $f_1, \, \dots, \, f_n$ szimbólumokkal jelölve, a $\psi$, $f$ függvények (és egyúttal az $f_i$-k) folytonossága miatt
\[
I_* \ni t \mapsto f_i(t, \, \psi(t)) \in \R \quad (i = 1, \, \dots, \, n)
\]
függvények folytonosak. Következőképpen (minden $x \in I_*$ esetén) van értelme a
\[
d_i := \int_\tau^x f_i(t, \, \psi(t)) \, dt \quad (i = 1, \, \dots, \, n)
\]
integráloknak, és így a
\[
\xi + \int_\tau^x f(t, \, \psi(t)) \, dt := (\xi_1 + d_1, \, \dots, \, \xi_n + d_n) \in \R^n
\]
"integrálvektoroknak". Továbbá az integrálfüggvények tulajdonságai miatt a $T\psi$ függvény folytonos, minden $x \in (\tau - \delta_1, \, \tau + \delta_2)$ helyen differenciálható, és
\[
(T\psi)'(x) = f(x, \, \psi(x)).
\]
Belátjuk, hogy az $I_*$ alkalmas megválasztásával minden $\psi \in \mathcal{F}$ függvényre $T\psi \in \mathcal{F}$, azaz ekkor
\[
T : \mathcal{F} \to \mathcal{F}.
\]
Ehhez azt kell biztosítani, hogy
\[
\xi + \int_\tau^x f(t, \, \psi(t)) \, dt \in \Omega \quad (x \in I_*)
\]
teljesüljön. Válasszuk ehhez először is a $\mu > 0$ számot úgy, hogy a
\[
K_\mu := \{ y \in \R^n : \| y - \xi\|_\infty \leq \mu \} \subset \Omega
\]
tartalmazás fennáljon (ilyen $\mu$ az $\Omega$ nyíltsága miatt létezik), és legyen
\[
M := \max\{ \| f(x, \, y)\|_\infty : x\in I_*, \, y \in K_\mu \}
\]
(ami meg az $f$ folytonossága és a Weierstrass-tétel miatt létezik, ti. az $I_* \times K_\mu$ halmaz kompakt). A jelzett $T\psi \in \mathcal{F}$ tartalmazás nyilván teljesül, ha
\[
\max \left\{  \left| \int_\tau^x f_i(t, \, \psi(t)) \, dt \right| : i = 1, \, \dots, \, n \right\} \leq \mu \quad (x \in I_*).
\]
Módosítsuk most már az $\mathcal{F}$ definícióját úgy, hogy
\[
\mathcal{F} := \{ \psi : I_* \to K_\mu : \psi \in C \}.
\]
Ekkor az előbbi maximum becsülhető $M \cdot \delta$-val, ahol
\[
\delta := \max\{\delta_1, \, \delta_2\}.
\]
Így $M \cdot \delta \leq \mu$ esetén a fenti $T\psi$ is $\mathcal{F}$-beli. (Ha a kiindulásul választott $\delta_1, \, \delta_2$-re $M \cdot \delta > \mu$, akkor írjunk a $\delta_1, \, \delta_2$ helyébe olyan "új" $0 < \tilde{\delta_1}, \, \tilde{\delta_2}$-t, hogy
\[
[\tau - \tilde{\delta_1}, \, \tau + \tilde{\delta_2}] \subset [\tau - \delta_1, \, \tau + \delta_2]
\]
és
\[
M \cdot \max \{ \tilde{\delta_1}, \, \tilde{\delta_2} \} \leq \mu
\]
legyen. Az $I_*$ helyett az $\tilde{I_*} := [\tau - \tilde{\delta_1}, \, \tau + \tilde{\delta_2}]$ intervallummal az "új" $M$ az előzőnél legfeljebb kisebb lesz, így az
\[
M \cdot \max \{ \tilde{\delta_1}, \, \tilde{\delta_2}\} \leq \mu
\]
becslés nem "romlik" el.)
Ezzel értelmeztünk egy $T : \mathcal{F} \to \mathcal{F}$ leképezést, amelyre tetszőleges $\phi, \, \psi \in \mathcal{F}$ mellett
\[
\rho(T\psi, \, T\phi) = \max \big\{ \| T\psi(x) - T\phi(x)\|_\infty : x \in I_* \big\} =
\]
\[
\max \left\{  \max \left\{  \left| \int_\tau^x (f_i(t, \, \psi(t)) - f_i(t, \, \phi(t)) \, dt) \right| : i = 1, \, \dots, \, n \right\} : x \in I_* \right\} \leq
\]
\[
\max \left\{  \left| \int_\tau^x \max \{ |f_i(t, \, \psi(t)) - f_i(t, \, \phi(t))| : i = 1, \, \dots, \, n \} \, dt \right| : x \in I_* \right\} =
\]
\[
\max \left\{ \left| \int_\tau^x \| f(t, \, \psi(t)) - f(t, \, \phi(t))\|_\infty \, dt \right| : x \in I_* \right\}.
\]
A Lipschitz-feltétel miatt a $Q := K_\mu$ (nyilván kompakt) halmazhoz van olyan $L_Q \geq 0$ konstans, amellyel
\[
\| f(t, \, y) - f(t, \, z)\|_\infty \leq L_Q \cdot \| y-z \|_\infty \quad (t \in I, \, y, \, z \in Q),
\]
speciálisan
\[
\| f(t, \, \psi(t)) - f(t, \, \phi(t))\|_\infty \leq
\]
\[
L_Q \cdot \| \psi(t) - \phi(t) \|_\infty \leq L_Q \cdot \rho(\psi, \, \phi) \quad (t \in I_*).
\]
Ezért
\[
\rho(T\psi, \, T\phi) \leq L_Q \cdot \delta \cdot \rho(\psi, \, \phi).
\]
Tehát a $T$ leképezés
\[
L_Q \cdot \max \{\delta_1, \, \delta_2\} < 1
\]
esetén kontrakció. Válasszuk így a $\delta_1, \, \delta_2$-t, (ezt - az "eddigi" $I_*$-ot legfeljebb újra leszűkítve - megtehetjük), és alkalmazzuk a fixpont-tételt, miszerint van olyan $\phi \in \mathcal{F}$, amelyre
\[
T\phi = \phi.
\]
Legyen
\[
\varphi(x) := \phi(x) \quad \big( x \in (\tau - \delta_1, \, \tau + \delta_2) \big).
\]
A $T$ definíciója szerint
\[
\varphi(x) = \xi + \int_\tau^x f(t, \, \varphi(t)) \, dt \quad \big( x \in (\tau - \delta_1, \, \tau + \delta_2) \big).
\]
Ez azt jelenti, hogy a $\varphi$ függvény egy folytonos függvény integrálfüggvénye, ezért $\varphi \in D$ és
\[
\varphi'(x) = f(x, \, \varphi(x)) \quad \big( x \in (\tau - \delta_1, \, \tau + \delta_2) \big).
\]
Világos, hogy a $\varphi(\tau) = \xi$, más szóval a $\varphi$ megoldása a szóban forgó kezdetiérték-problémának. $\blacksquare$\\

A fenti Picard-Lindelöf-egzisztenciatételben szereplő Lipschitz-feltétel nem csupán a kezdetiérték-problémák megoldhatóságát, hanem azok egyértelmű megoldhatóságát is biztosítja.\\

\tikz \node[theorem]
{
	\textbf{Tétel.} Az előző tétel feltételei mellett az abban szereplő tetszőleges kezdetiérték-probléma egyértelműen oldható meg, azaz bármely $\varphi, \, \psi$ megoldásaira
	\[
	\varphi(t) = \psi(t) \quad (t \in \Dp \cap \mathcal{D}_\psi).
	\]
};\\

Legyen az $f : I \times \Omega \to \R^n$ jobb oldal olyan, hogy $\Omega := \R^n$, és (az előző tétel feltételein kívül) valamilyen $\alpha, \, \beta$ pozitív együtthatókkal
\[
	\| f(x, \, y)\|_\infty \leq \alpha \cdot \|y\|_\infty + \beta \quad (x \in I, \, y \in \R^n).
\] 
Ekkor belátható, hogy az $f$ által meghatározott differenciálegyenletre vonatkozó bármelyik k.é.p. teljes megoldása az $I$-n van értelmezve. 
