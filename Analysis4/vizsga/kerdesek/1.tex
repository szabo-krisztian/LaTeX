\newpage
\section{Vizsgakérdés}
\begin{quote}
	\textit{Az implicitfüggvény fogalma, kapcsolata a feltételes szélsőérték problémával és az inverzfüggvénnyel. Implicitfüggvény-tétel, inverzfüggvény-tétel (a bizonyítás vázlata).}
\end{quote}

Legyenek $n, \, m \in \N$ természetes számok, ahol $2 \leq n$ és $1 \leq m \leq n$. Ha
\[
	\xi = (\xi_1, \, \dots, \, \xi_n) \in \R^n,
\]
akkor legyen
\[
	x := (\xi_1, \, \dots, \, \xi_{n-m}) \in \R^{n-m}, \, y := (\xi_{n-m+1}, \, \dots, \, \xi_n) \in \R^m,
\]
és ezt következőképpen fogjuk jelölni:
\[
	\xi = (x, \, y).
\]
Röviden:
\[
	\R^n \equiv \R^{n-m} \times \R^{m}.
\]
Ha tehát
\[
	f = (f_1, \, \dots, \, f_m) \in \R^n \to \R^m,
\]
azaz
\[
	f \in \R^{n-m} \times \R^m \to \R^m,
\]
akkor az $f$-et olyan kétváltozós vektorfüggvénynek tekintjük, ahol az $f(x, \, y)$ helyettesítési értékben az argumentum első változójára $x \in \R^{n-m}$, a második változójára pedig $y \in \R^m$ teljesül.\\

Tegyük fel, hogy ebben az értelemben valamilyen $(a, \, b) \in \D$ zérushelye az $f$-nek:
\[
	f(a, \, b) = 0.
\]
Tételezzük fel továbbá, hogy van az $a$-nak egy olyan $K(a) \subset \R^{n-m}$ környezete, a $b$-nek pedig olyan $K(b) \subset \R^m$ környezete, hogy tetszőleges $x \in K(a)$ esetén egyértelműen létezik olyan $y \in K(b)$, amellyel
\[
	f(x, \, y) = 0.
\]
Definiáljuk ekkor a $\varphi(x) := y$ hozzárendeléssel a
\[
	\varphi : K(a) \to K(b)
\]
függvényt, amikor is
\[
	f(x, \, \varphi(x)) = 0 \quad (x \in K(a)).
\]
Itt minden $x \in K(a)$ mellett az $y = \varphi(x)$ az egyetlen olyan $y \in K(b)$ hely amelyre
\[
	f(x, \, y) = 0.
\]
Az előbbi $\varphi$ függvényt az $f$ által (az $(a, \, b)$ körül) meghatározott \textit{implicitfüggvénynek} nevezzük. Tehát az

\[
	\left\{
	\begin{array}{ccc}
		f_1(x_1, \, \dots, \, x_{n-m}, \, y_1, \, \dots, \, y_m) & = & 0 \\
		\vdots & & \vdots \\
		f_m(x_1, \, \dots, \, x_{n-m}, \, y_1, \, \dots, \, y_m) & = & 0
	\end{array}
	\right.
\]

egyenletrendszernek minden $x = (x_1, \, \dots, \, x_{n-m}) \in K(a)$ mellett egyértelműen létezik
\[
	y = (y_1, \, \dots, \, y_m) = \varphi(x) \in K(b)
\]
megoldása. Nyilván $\varphi(a) = b$.\\

\subsection{Implicitfüggvény-tétel}

\tikz \node[theorem]
{
	\textbf{Tétel.} \textit{Adott $n, \, m \in \N, \, 2 \leq n$, valamint $1 \leq m < n$ mellett az}
	\[
	f \in \R^{n-m} \times \R^m \to \R^m
	\]
	\textit{függvényről tételezzük fel az alábbiakat: $f \in C^1$, és az $(a, \, b) \in \text{int} \, \D$ helyen}
	\[
	f(a, \, b) = 0, \, \det \partial_2f(a, \, b) \neq 0.
	\]
	\textit{Ekkor alkalmas $K(a), \, K(b)$ környezetekkel létezik az $f$ által az $(a, \, b)$ körül meghatározott}
	\[
	\varphi : K(a) \to K(b)
	\]
	\textit{implicitfüggvény, ami folytonosan differenciálható, és}
	\[
	\varphi'(x) = -\partial_2 f(x, \, \varphi(x))^{-1} \cdot \partial_1 f(x, \, \varphi(x)) \quad (x \in K(a)).
	\]
};\\

Elöljáróban idézzük fel az egyváltozós valós függvényekkel kapcsolatban tanultakat. Ha pl.
\[
h \in \R \to \R, \, h \in C^1\{a\}
\]
és $h'(a) \neq 0$, akkor egy alkalmas $r>0$ mellett
\[
I := (a-r, \, a+r) \subset \mathcal{D}_h,
\]
létezik a $(h_{|_I})^{-1}$ inverzfüggvény, a $g := (h_{|_I})^{-1}$ függvény differenciálható és 
\[
g'(x) = \frac{1}{h'(g(x))} \quad (x \in \mathcal{D}_g).
\]
A továbbiakban a most megfogalmazott "egyváltozós" állítás megfelelőjét fogjuk vizsgálni többáltozós vektorfüggvényekre.\\

Legyen ehhez valamilyen $1 \leq n \in \N$ mellett adott az
\[
f \in \R^n \to \R^n
\]
függvény és az $a \in \text{int} \, \D$ pont. Azt mondjuk, hogy az $f$ függvény \textit{lokálisan invertálható} az $a$-ban, ha létezik olyan $K(a) \subset \D$ környezet, hogy a $g := f_{|_{K(a)}}$ leszűkítés invertálható függvény. Minden ilyen esetben a $g^{-1}$ inverzfüggvényt az $f$ $a$-beli \textit{lokális inverzének} nevezzük.

\subsection{Inverzfüggvény-tétel}

\tikz \node[theorem]
{
	\textbf{Tétel.} \textit{Legyen $1 \leq n \in \N$, és $f \in \R^n \to \R^n$. Tegyük fel, hogy egy $a \in \text{int} \, \D$ pontban $f \in C^1\{a\}$, $\det f'(a) \neq 0$. Ekkor alkalmas $K(a) \subset \D$ környezettel az $f_{|_{K(a)}}$ leszűkítés invertálható, a $g := (f_{|_{K(a)}})^{-1}$ lokális inverzfüggvény folytonosan differenciálható, és}
	\[
	h'(x) = \big( f'(h(x)) \big)^{-1} \quad (x \in \mathcal{D}_h).
	\]
};