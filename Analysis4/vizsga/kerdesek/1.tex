\newpage
\section{Vizsgakérdés}
\begin{quote}
	\textit{Az implicitfüggvény fogalma, kapcsolata a feltételes szélsőérték problémával és az inverzfüggvénnyel. Implicitfüggvény-tétel, inverzfüggvény-tétel (a bizonyítás vázlata).}
\end{quote}

Legyenek $n, \, m \in \N$ természetes számok, $1 \leq m < n$. Ha
\[
	\xi = (\xi_1, \, \dots, \, \xi_n) \in \R^n,
\]
akkor legyen
\[
	x := (\xi_1, \, \dots, \, \xi_{n-m}) \in \R^{n-m}, \, y := (\xi_{n-m+1}, \, \dots, \, \xi_n) \in \R^m,
\]
és ezt következőképpen fogjuk jelölni:
\[
	\xi = (x, \, y).
\]
Röviden:
\[
	\R^n \equiv \R^{n-m} \times \R^{m}.
\]
Ha tehát
\[
	f = (f_1, \, \dots, \, f_m) \in \R^n \to \R^m,
\]
azaz
\[
	f \in \R^{n-m} \times \R^m \to \R^m,
\]
akkor az $f$-et olyan kétváltozós vektorfüggvénynek tekintjük, ahol az $f(x, \, y)$ helyettesítési értékben az argumentum első változójára $x \in \R^{n-m}$, a második változójára pedig $y \in \R^m$ teljesül.\\

Tegyük fel, hogy ebben az értelemben valamilyen $(a, \, b) \in \D$ zérushelye az $f$-nek:
\[
	f(a, \, b) = 0.
\]
Tételezzük fel továbbá, hogy van az $a$-nak egy olyan $K(a) \subset \R^{n-m}$ környezete, a $b$-nek pedig olyan $K(b) \subset \R^m$ környezete, hogy tetszőleges $x \in K(a)$ esetén egyértelműen létezik olyan $y \in K(b)$, amellyel
\[
	f(x, \, y) = 0.
\]
Definiáljuk ekkor a $\varphi(x) := y$ hozzárendeléssel a
\[
	\varphi : K(a) \to K(b)
\]
függvényt, amikor is
\[
	f(x, \, \varphi(x)) = 0 \quad (x \in K(a)).
\]
Itt minden $x \in K(a)$ mellett az $y = \varphi(x)$ az egyetlen olyan $y \in K(b)$ hely amelyre
\[
	f(x, \, y) = 0.
\]
Az előbbi $\varphi$ függvényt az $f$ által (az $(a, \, b)$ körül) meghatározott \textit{implicitfüggvénynek} nevezzük. Tehát az

\[
	\left\{
	\begin{array}{ccc}
		f_1(x_1, \, \dots, \, x_{n-m}, \, y_1, \, \dots, \, y_m) & = & 0 \\
		\vdots & & \vdots \\
		f_m(x_1, \, \dots, \, x_{n-m}, \, y_1, \, \dots, \, y_m) & = & 0
	\end{array}
	\right.
\]

egyenletrendszernek minden $x = (x_1, \, \dots, \, x_{n-m}) \in K(a)$ mellett egyértelműen létezik
\[
	y = (y_1, \, \dots, \, y_m) = \varphi(x) \in K(b)
\]
megoldása. Nyilván $\varphi(a) = b$.\\

A $\p : K(a) \to K(b)$ implicitfüggvényre a következő igaz:
\[
	\big(K(a) \times K(b)\big) \cap \{ f=0\} = \{ (x, \, \p(x)) \in \R^n : x \in K(a) \}.
\]
Geometriai szóhasználattal éve
\[
	\text{graf} \, \p := \{ (x, \, \p(x)) \in \R^n : x \in K(a) \}
\]
(a $\p$ függvény "grafikonja", ami a függvény definíciója miatt persze maga a $\p$ függvény), tehát az előbbi egyenlőség így néz ki:
\[
	\big( K(a) \times K(b) \big) \cap \{f=0\} = \text{graf} \, \p = \p.
\]

\subsection{Implicitfüggvény-tétel}

\tikz \node[theorem]
{
	\textbf{Tétel.} Adott $n, \, m \in \N$, valamint $1 \leq m < n$ mellett az
	\[
	f \in \R^{n-m} \times \R^m \to \R^m
	\]
	függvényről tételezzük fel az alábbiakat: $f \in C^1$, és az $(a, \, b) \in \text{int} \, \D$ helyen
	\[
	f(a, \, b) = 0, \, \det \partial_2f(a, \, b) \neq 0.
	\]
	Ekkor alkalmas $K(a), \, K(b)$ környezetekkel létezik az $f$ által az $(a, \, b)$ körül meghatározott
	\[
	\varphi : K(a) \to K(b)
	\]
	implicitfüggvény, ami folytonosan differenciálható, és
	\[
	\varphi'(x) = -\partial_2 f(x, \, \varphi(x))^{-1} \cdot \partial_1 f(x, \, \varphi(x)) \quad (x \in K(a)).
	\]
};\\

A tételben $f \in C^1, \, \partial_2f(a, \, b) \neq 0$ feltételekből következően a $K(a), \, K(b)$ környezetekről az is feltehető, hogy
\[
	\det \partial_2 f(x, \, y) \neq 0 \quad (x \in K(a), \, y \in K(b)),
\]
egyúttal
\[
	\det \partial_2f(x, \, \p(x)) \neq 0 \quad (x \in K(a)).
\]
Ezért az $x \in K(a)$ helyeken a $\partial_2f(x, \, \p(x))$ mátrix valóban invertálható.\\

\subsection{Inverzfüggvény-tétel}

Elöljáróban idézzük fel az egyváltozós valós függvényekkel kapcsolatban tanultakat. Ha pl.
\[
h \in \R \to \R, \, h \in C^1\{a\}
\]
és $h'(a) \neq 0$, akkor egy alkalmas $r>0$ mellett
\[
I := (a-r, \, a+r) \subset \mathcal{D}_h,
\]
létezik a $(h_{|_I})^{-1}$ inverzfüggvény, a $g := (h_{|_I})^{-1}$ függvény differenciálható és 
\[
g'(x) = \frac{1}{h'(g(x))} \quad (x \in \mathcal{D}_g).
\]
A továbbiakban a most megfogalmazott "egyváltozós" állítás megfelelőjét fogjuk vizsgálni többáltozós vektorfüggvényekre.\\

Legyen ehhez valamilyen $1 \leq n \in \N$ mellett adott az
\[
f \in \R^n \to \R^n
\]
függvény és az $a \in \text{int} \, \D$ pont. Azt mondjuk, hogy az $f$ függvény \textit{lokálisan invertálható} az $a$-ban, ha létezik olyan $K(a) \subset \D$ környezet, hogy a $g := f_{|_{K(a)}}$ leszűkítés invertálható függvény. Minden ilyen esetben a $g^{-1}$ inverzfüggvényt az $f$ $a$-beli \textit{lokális inverzének} nevezzük.\\

\tikz \node[theorem]
{
	\textbf{Tétel.} Legyen $1 \leq n \in \N$, és $f \in \R^n \to \R^n$. Tegyük fel, hogy egy $a \in \text{int} \, \D$ pontban $f \in C^1\{a\}$, $\det f'(a) \neq 0$. Ekkor alkalmas $K(a) \subset \D$ környezettel az $f_{|_{K(a)}}$ leszűkítés invertálható, a $h := (f_{|_{K(a)}})^{-1}$ lokális inverzfüggvény folytonosan differenciálható, és
	\[
	h'(x) = \big( f'(h(x)) \big)^{-1} \quad (x \in \mathcal{D}_h).
	\]
};

\subsection{Hiperkoordinátás parciális deriváltak}
Legyen adott $n, \, m \in \N, \, 1 \leq m < n \in \N$ mellett
\[
	f : \R^n \to \R^m.
\]
Az előbbiek alapján most adott $k \in \N, \, 1 \leq k < n$ esetén legyen $\R^n \equiv \R^{n-k} \times \R^k$. Ha $\xi \in \D$, akkor legyen $(a, \, b) = \xi$, ahogy eddig. Azaz $f$-et fel lehet fogni egy kétváltozós függvénynek. Tekintsük az alábbi definíciót:
\[
	\mathcal{D}_1^{(a, \, b)} := \{ x \in \R^{n-k} : (x, \, b) \in \D \},
\]
\[
	\mathcal{D}_2^{(a, \, b)} := \{ y \in \R^{k} : (a, \, y) \in \D \}.
\]
Ekkor analóg módon a \textit{szokásos} parciális deriváltakhoz
\[
	f_{(a, \, b), \, 1}  \in \R^{n-k} \to \R^m, 
\]
\[
	f_{(a, \, b), \, 2}  \in \R^{k} \to \R^m,
\]
ahol
\[
	f_{(a, \, b), \, 1}(x) := f(x, \, b) \quad (x \in \mathcal{D}_1^{(a, \, b)}),
\]
\[
	f_{(a, \, b), \, 2}(y) := f(a, \, y) \quad (y \in \mathcal{D}_2^{(a, \, b)}).
\]
Ebben az esetben a \textit{hiperkoordinátás} alakja a parciális deriváltaknak (amennyiben értelmes a derivált):
\[
	\partial_{\mathbf{1}}f(a, \, b) := \partial_1f(a, \, b) := f'_{(a, \, b), \, 1}(a),
\]
\[
	\partial_{\mathbf{2}}f(a, \, b) := \partial_2f(a, \, b) := f'_{(a, \, b), \, 2}(b).	
\]
Azaz egy $(a, \, b)$ helyen lerögzítjük az első vagy második változók és az így kapott függvénynek vesszük a deriváltját. Ha $f$ egy differenciálható függvény az $(a, \, b) \in \D$ helyen, akkor
\[
	f'(a, \, b) = \begin{bmatrix}
		\partial_{\mathbf{1}}f(a, \, b) & \partial_{\mathbf{2}}f(a, \, b)
	\end{bmatrix} = \begin{bmatrix}
		\partial_1f_1(a, \, b) & \partial_2f_1(a, \, b) & \cdots & \partial_nf_1(a, \, b) \\
		\partial_1f_2(a, \, b) & \partial_2f_2(a, \, b) & \cdots & \partial_nf_2(a, \, b) \\
		\vdots & \vdots &\cdots & \vdots \\
		\partial_1f_n(a, \, b) & \partial_2f_n(a, \, b) & \cdots & \partial_nf_n(a, \, b)
	\end{bmatrix},
\]
ahol a $\partial_{\mathbf{1}}f(a, \, b) \in \R^{m \times (n-k)}, \, \partial_{\mathbf{2}}f(a, \, b) \in \R^{m \times k}$ mátrixok rendre az $f'(a, \, b) \in \R^{m \times n}$ mátrix első $n-k$-adik és utolsó $k$-adik oszlopvektorai. Pl. legyen $f \in \R^3 \to \R^4$ $(a, \, b) \in \text{int} \, \D$-ben differenciálható függvény, $k := 2$. Ekkor
\[
	f \in \R \times \R^2 \to \R^4
\]
és 
\[
	f'(a, \, b) = \begin{bmatrix}
		\partial_1f_1(a, \, b) & \partial_2f_1(a, \, b) & \partial_3f_1(a, \, b) \\
		\partial_1f_2(a, \, b) & \partial_2f_2(a, \, b) & \partial_3f_2(a, \, b) \\
		\partial_1f_3(a, \, b) & \partial_2f_3(a, \, b) & \partial_3f_3(a, \, b) \\
		\partial_1f_4(a, \, b) & \partial_2f_4(a, \, b) & 	\partial_3f_4(a, \, b) \\
	\end{bmatrix},
\]
\[
	\partial_{\mathbf{1}}f(a, \, b) = \begin{bmatrix}
		\partial_1f_1(a, \, b) \\
		\partial_1f_2(a, \, b) \\
		\partial_1f_3(a, \, b) \\
		\partial_1f_4(a, \, b) \\
	\end{bmatrix},
	\partial_{\mathbf{2}}f(a, \, b) = \begin{bmatrix}
		\partial_2f_1(a, \, b) & \partial_3f_1(a, \, b) \\
		\partial_2f_2(a, \, b) & \partial_3f_2(a, \, b) \\
		\partial_2f_3(a, \, b) & \partial_3f_3(a, \, b) \\
		\partial_2f_4(a, \, b) & \partial_3f_4(a, \, b) \\
	\end{bmatrix}.
\]