\newpage
\section{Vizsgakérdés}
\begin{quote}
	\textit{A függvénysorozat, függvénysor fogalma. Hatványsorok, trigonometrikus sorok, Fourier-sorok. A Dirichlet-féle magfüggvény. Konvergencia, határfüggvény (összegfüggvény), egyenletes konvergencia. A Weierstrass-féle majoráns kritérium.}
\end{quote}

\subsection{Függvénysorozatok, függvénysorok}

A függvénysorozatok fogalmával részben találkoztunk már korábban is: az $(f_n)$ sorozatot \textit{függvénysorozatnak} nevezzük, ha minden $n \in \N$ esetén az $f_n$ függvény. A továbbiakban mindig azzal a feltételezéssel élünk, hogy valamilyen $\empty \neq X$ halmazzal
\[
	f_n \in X \to \K \quad (n \in \N),
\]
és egy $\emptyset \neq \DD \subset X$ halmazzal
\[
	\DD_{f_n} = \DD \quad (n \in \N).
\]
Pl. a
\[
	h_n(t) := t^n \quad (t \in \DD := \R, \, n \in \N)
\]
függvények egy $(h_n)$ függvénysorozatot határoznak meg.\\

A fenti $(f_n)$ függvénysorozat által meghatározott $\sum(f_n)$ \textit{függvénysor}:
\[
	\sum(f_n) := \left( \sumk f_k \right).
\]
A $\sum(f_n)$ függvénysor tehát nem más, mint az
\[
	F_n := \sumk f_k \quad (n \in \N)
\]
\textit{részletösszegfüggvények} által meghatározott $(F_n)$ függvénysorozat:
\[
	\sum(f_n) := (F_n).
\]
Így pl. az előbbi
\[
	h_n(t) := t^n \quad (t \in \R, \, n \in \N)
\]
függvények esetén $\sum(h_n) = (H_n)$, ahol az $n \in \N$ indexekre
\[
	H_n(t) = \sumk h_k(t) = \sumk t^k = \begin{cases}
		n + 1 & (t = 1) \\
		\displaystyle \frac{1-t^{n+1}}{1-t} & (t \neq 1)
	\end{cases} \quad (t \in \R).
\]

\subsection{Konvergencia, határfüggvény}

Tekintsük a fenti $(f_n)$ függvénysorozatot. Ha egy $x \in \DD$ elem esetén konvergens a helyettesítési értékeknek az $\big(f_n(x)\big)$ sorozata, akkor azt mondjuk, hogy az $(f_n)$ függvénysorozat \textit{konvergens} az $x$ helyen. A
\[
	\DD_0 := \big\{ x \in \DD : \big(f_n(x)\big) \text{ konvergens} \big\}
\]
halmaz az $(f_n)$ függvénysorozat \textit{konvergenciatartománya}. Ha $\DD_0 \neq \emptyset$, akkor az
\[
	f(x) := \limn f_n(x) \quad (x \in \DD_0)
\]
definícióval értelmezett
\[
	f : \DD_0 \to \K
\]
függvény az $(f_n)$ függvénysorozat \textit{határfüggvénye}. A $\DD_0 = \D$ esetben röviden azt mondjuk, hogy az $(f_n)$ függvénysorozat \textit{pontonként konvergens}.\\

Pl. az előbbi $(h_n)$ függvénysorozattal $\DD_0 = (-1, \, 1]$, és
\[
	h(x) := \begin{cases}
		0 & (-1 < x < 1) \\
		1 & (x = 1)
	\end{cases}
\]
a $(h_n)$ sorozat határfüggvénye.\\

A függvénysorok "nyelvén" a pontonkénti konvergencia a következőképpen fogalmazható meg: legyen $X \neq \emptyset$, és a $\emptyset \neq \DD \subset X$ halmazzal adott az
\[
	f_n : \DD \to \K \quad (n \in \N)
\]
függvénysorozat. Ekkor a $\sum(f_n)$ függvénysor $x$-beli konvergenciája azt jelenti, hogy a részletösszegek $\displaystyle \Big(\sumk f_k \Big)$ sorozata konvergens az $x$ helyen, azaz a $\displaystyle \Big(\sumk f_k(x) \Big)$ sorozat konvergens. Nem fog félreértést okozni, ha az ilyen $x \in \DD$ elemek összegét fogjuk most $\DD_0$-val jelölni. Tehát $\DD_0$ most nem más, mint a $\displaystyle \Big(\sumk f_k \Big)$ függvénysorozat konvergenciatartománya. Ha $\DD_0 \neq \emptyset$, akkor legyen
\[
	F(x) := \limsum f_k(x) = \limn \sumk f_k(x) \quad (x \in \DD_0)
\]
A szóban forgó függvénysor \textit{összegfüggvénye}. Pl. a
\[
	h_n(x) := x^n \quad (x \in \R, \, n \in \N)
\]
függvényekkel
\[
	\sumk h_k(x) = \sumk x^k = \begin{cases}
		n + 1 & (x = 1) \\
		\displaystyle \frac{1 - x^{n+1}}{1-x} & (x \neq 1)
	\end{cases} \quad (x \in \R, \, n \in \N)
\]
miatt a $\sum(h_n)$ függvénysor konvergenciatartománya a $(-1, \, 1)$ intervallum, a $H$ összegfüggvénye pedig a
\[
	H(x) := \frac{1}{1-x} \quad (-1 < x < 1)
\]
függvény.\\

Emlékeztetünk a hatványsor fogalmára: legyen valamilyen $a \in \K$ \textit{középpont} és egy
\[
	(a_n) : \N \to \K
\]
\textit{együttható-sorozat} esetén
\[
	f_n(x) := a_n(x-a)^n \quad (x \in \R, \, n \in \N).
\]
Ekkor a
\[
	\sum\Big( a_n(x-a)^n \Big) := \sum(f_n)
\]
függvénysort neveztük \textit{hatványsornak}. A Cauchy-Hadamard-tétel szerint egyértelműen létezik olyan
\[
	0 \leq r \leq + \infty
\]
(\textit{konvergenciasugár}) amellyel a hatványsor $\DD_0$ konvergenciatartományára a $0 < r < + \infty$ esetben
\[
	K_r(a) \subset \DD_0 \subset \overline{K_r(a)}.
\]
Nyilvánvaló, hogy $a \in \DD_0$ mindig igaz, és az $a$ helyen a fenti hatványsor összege $0$.

\subsection{Trigonometrikus sorok, Fourier-sorok}
A $\sum(f_n)$ függvénysort \textit{trigonometrikus sornak} nevezzük, ha
\[
	f_0(x) := \alpha_0, \, f_n(x) := \alpha_n \cdot \cos(nx) + \beta_n \cdot \sin(nx) \quad (1 \leq n \in \N, \, x \in \R),
\]
ahol adottak az $\alpha_k \in \R \quad (k \in \N)$ és a $\beta_j \quad (1 \leq j \in \N)$ \textit{együtthatók}. Használni fogjuk minderre a
\[
	\trigseries
\]
szimbólumot is. Tehát egy adott trigonometrikus sor $n$-edik részletösszege egy $x \in \R$ helyen az alábbi módon néz ki:
\[
	\alpha_0 + \alpha_1 \cdot \cos(x) + \beta_1 \cdot \sin(x) + \cdots + \alpha_n \cdot \cos(nx) + \beta_n \cdot \sin(nx).
\]
A szóban forgó $\trigseries$ trigonometrikus sor
\[
	S_n(x) := \alpha_0 + \sum_{k=1}^n \Big( \alpha_k \cdot \cos(kx) + \beta_k \cdot \sin(kx) \Big) \quad (x \in \R, \, n \in \N)
\]
részletösszegfüggvényei \textit{trigonometrikus polinomok}.\\

Legyen $R_{2\pi}$ az összes olyan $2\pi$ szerint periodikus
\[
	f : \R \to \R
\]
függvény halmaza, amelyre
\[
	f \in R[0, \, 2\pi]
\]
teljesül. A periodicitás miatt nyilvánvaló, hogy ekkor tetszőleges $2\pi$-hosszúságú kompakt $I \subset \R$ intervallumra is (az előbbi értelemben) $f \in R(I)$.\\

Legyen továbbá $C_{2\pi}$ az olyan $2\pi$ szerint periodikus
\[
	f : \R \to \R
\]
függvények halmaza, amelyekre $f \in C$. Ekkor
\[
	C_{2\pi} \subset R_{2\pi},
\]
továbbá $C_{2\pi}, \, R_{2\pi}$ lineáris terek az $\R$-re vonatkozóan, a $C_{2\pi}$ altere az $R_{2\pi}$-nek. Továbbá bármely $f \in R_{2\pi}$ függvény az $f \in R[0, \, 2\pi]$ integrálhatóság miatt korlátos, azaz
\[
	\sup\{|f(x)| : x \in \R\} = \sup\{|f(x)| : x \in [0, \, 2\pi]\} < + \infty.
\]
Vezessük be az alábbi fogalmakat: $f \in R_{2\pi}$ esetén legyen
\[
	a_0(f) := a_0 := \frac{1}{2\pi} \int\limits_0^{2\pi} f(x) \, dx,
\]
\[
	a_n(f) := a_n := \frac{1}{\pi} \int\limits_0^{2\pi} f(x) \cdot \cos(nx) \, dx \quad (1 \leq n \in \N),
\]
\[
	b_n(f) := b_n := \frac{1}{\pi} \int\limits_0^{2\pi} f(x) \cdot \sin(nx) \, dx \quad (1 \leq n \in \N),
\]
\[
	Sf := \trigserieslatin \quad (n \in \N, \, x \in \R).
\]
Ekkor az $Sf$ trigonometrikus sor az $f$ \textit{Fourier-sora}, az együtthatói az $f$ \textit{Fourier-együtthatói}, az $S_nf \quad (n \in \N)$ trigonometrikus polinom pedig az $f$ függvény $n$-edik \textit{Fourier-részletösszege}.\\

Ha $f \in R_{2\pi}, \, n \in \N$, akkor a fenti $f$ \textit{Fourier-részletösszegei} a következők:
\[
	S_0f(x) = a_0 \quad (x \in \R),
\]
ill. $1 \leq n \in \N, \, x \in \R$ esetén
\[
	S_nf(x) = \frac{1}{2\pi} \int\limits_0^{2\pi} f(t) \, dt +
\]
\[
	\sum_{k=1}^n \left( \frac{1}{\pi} \int\limits_0^{2\pi} f(t) \cdot \cos(kt) \, dt \cdot \cos(kx) + \frac{1}{\pi} \int\limits_0^{2\pi} f(t) \cdot \sin(kt) \, dt \cdot \sin(kx) \right) =
\]
\[
	\frac{1}{2\pi} \int\limits_0^{2\pi} f(t) \, dt + \frac{1}{\pi} \int\limits_0^{2\pi} f(t) \cdot \sum_{k=1}^n \Big( \cos(kt) \cdot \cos(kx) + \sin(kt) \cdot \sin(kx) \Big) \, dt =
\]
\[
	\frac{1}{\pi} \int\limits_0^{2\pi} f(t) \cdot \left( \frac{1}{2} + \sum_{k=1}^n \cos\Big(k(x-t)\Big) \right) \, dt.
\]
Ha tehát
\[
	D_0(z) := \frac{1}{2}, \, D_n(z) := \frac{1}{2} + \sum_{k=1}^n \cos(kz) \quad (1 \leq n \in \N, \, z \in \R),
\]
akkor
\[
	S_nf(x) = \frac{1}{\pi} \int\limits_0^{2\pi} f(t) \cdot D_n(x-t) \, dt \quad (n \in \N, \, x \in \R).
\]
A most definiált $D_n \quad (n \in \N)$ függvény az $n$-edik \textit{Dirichlet-magfüggvény}. Világos, hogy minden $D_n$ páros függvény, periodikus $2\pi$ szerint, és bármilyen $2\pi$ hosszúságú kompakt $I \subset \R$ intervallumra
\[
	\int_I D_n = \int_I \frac{1}{2} \, dz + \sum_{k=1}^n \int_I \cos(kz) \, dz = \int_I \frac{1}{2} \, dz = \pi \quad (n \in \N).
\]
Nem nehéz "zárt" alakra hozni a szóvan forgó magfüggvényeket. Ha ui. $0 < u < 2\pi$ és $n \in \N$, akkor
\[
	\sin(z/2) \cdot D_n(z) = \frac{\sin(z/2)}{2} + \sum_{k=1}^n \sin(z/2) \cdot \cos(kz) =
\]
\[
	\frac{\sin(z/2)}{2} + \frac{1}{2} \cdot \sum_{k=1}^n \Bigg( \sin\Big( (k+ 1/2)z \Big) - \sin\Big( (k-1/2)z \Big) \Bigg) =
\]
\[
	\frac{\sin(z/2)}{2} + \frac{\sin\Big( (n+1 / 2)z  - \sin(z/2) \Big)}{2} = \frac{\sin\Big( (n+1/2)z \Big)}{2}.
\]
Innen az következik, hogy
\[
	D_n(z) = \frac{\sin\Big( (n+1/2)z \Big)}{2 \cdot \sin(z/2)} \quad (0 < z < 2\pi).
\]
Tehát a $D_n$ definíciójából adódóan a
\[
	\frac{\sin\Big( (n+1/2)0 \Big)}{2 \cdot \sin(0/2)} := D_n(0) = \frac{1}{2} + n
\]
megállapodással tetszőleges $f \in R_{2\pi}$ függvényre az alábbi integrál-előállítást kapjuk a Fourier-részletösszegekre:
\[
	S_nf(x) = \frac{1}{\pi} \int\limits_0^{2\pi} f(t) \cdot D_n(x-t) \, dt \quad (n \in \N, \, x \in \R).
\]

\subsection{Egyenletes konvergencia}
Tekintsük az $(f_n)$ függvénysorozatot, ahol
\[
	f_n \in X \to \K \quad (n \in \N)
\]
és
\[
	\DD_{f_n} =: \DD \quad (n \in \N).
\]
Legyen
\[
	\DD_0 := \Big\{ t \in \DD : \big( f_n(x)\big) \text { konvergens} \Big\} \neq \emptyset
\]
az $(f_n)$ konvergenciatartománya, és 
\[
	f(x) := \limn f_n(x) \quad (x \in \DD_0)
\]
az $(f_n)$ függvénysorozat határfüggvénye. Tehát $f : \DD_0 \to \K$ és tetszőleges $x \in \DD_0$, valamint $\varepsilon > 0$ esetén van olyan $N_{x, \, \varepsilon} \in \N$, hogy
\[
	|f_n(x) - f(x)| < \varepsilon \quad (N_{x, \, \varepsilon} < n \in \N).
\]
Hangsúlyozni kell, hogy az itt szereplő $N_{x, \, \varepsilon}$ küszöbindex általában függ az $x$-től is, és az $\varepsilon$-tól is. Elképzelhető ugyanakkor, hogy bizonyos esetekben bármilyen $\varepsilon > 0$ mellett olyan (csak az $\varepsilon$-tól függő)
\[
	N := N_\varepsilon \in \N
\]
is megadható, amelyik az előbbi becslésben egy $\emptyset \neq A \subset \DD_0$ halmaz mellett független az $x \in A$ elemtől. Ekkor azt mondjuk, hogy az $(f_n)$ függvénysorozat az $A$ halmazon \textit{egyenletesen konvergens} az $f$ függvényhez, azaz: minden $\varepsilon > 0$ számhoz létezik olyan $N \in \N$, amellyel
\[
	|f_n(x) - f(x)| < \varepsilon \quad (x \in A, \, N < n \in \N).
\]
Világos, hogy ekkor minden $\emptyset \neq B \subset A$ halmaz esetén is az $(f_n)$ sorozat egyenletesen konvergál a $B$-n az $f$-hez. Ha az egyenletes konvergencia definíciójában $A = \DD_0$ írható, akkor egyszerűen azt mondjuk, hogy az $(f_n)$ függvénysorozat \textit{egyenletesen konvergens}.\\

A $\sum(f_n)$ függvénysor egyenletesen konvergens az $\emptyset \neq A \subset \DD_0$ halmazon, ha a részletösszegek $\series$ sorozata egyenletesen konvergens az $A$-n.\\

Ez tehát azt jelenti, hogy létezik olyan
\[
	F : A \to \K
\]
függvény és tetszőleges $\varepsilon > 0$ számhoz van olyan $N \in \N$, amellyel
\[
	\left| F(x) - \sumk f_k(x) \right| < \varepsilon \quad (x \in A, \, N < n \in \N).
\]
A Cauchy-kritérium miatt ez azzal ekvivalens, hogy bármilyen $\varepsilon > 0$ esetén egy alkalmas $N \in \N$ természetes számmal
\[
	\left|  \sum_{k=n+1}^m f_k(x) \right| < \varepsilon \quad (x \in A, \, N < n, \, m \in \N, \, n < m)
\]
(\textit{egyenletes Cauchy-kritérium}).\\

\subsection{Weierstrass-kritérium}

\tikz \node[theorem]
{
	\textbf{Tétel.} Tegyük fel, hogy valamilyen $\emptyset \neq X$ mellett $\emptyset \neq \mathcal{D} \subset X$, és adott az
	\[
		f_n : \mathcal{D} \to \K \quad (n \in \N)
	\]
	függvények által meghatározott $\sum(f_n)$ függvénysor. Legyen továbbá egy $\emptyset \neq A \subset \mathcal{D}$ halmazzal és egy $(a_n)$ számsorozattal
	\[
		\sup \{ |f_n(x)| : x \in A \} \leq a_n \quad (n \in \N),
	\]
	ahol $\sum_{n=0}^\infty a_n < + \infty$. Ekkor a $\sum(f_n)$ függvénysor az $A$ halmazon egyenletesen konvergens. 
};\\

\textbf{Bizonyítás.} Az alábbi becslés a tétel feltételei alapján nyilvánvaló:
\[
	\left|  \sum_{k=n+1}^m f_k(x) \right| \leq \sum_{k=n+1}^m |f_k(x)| \leq \sum_{k=n+1}^m a_k \quad (x \in A, \, n, \, m \in \N, \, n < m).
\]
Ha az $\varepsilon > 0$ egy pozitív szám, akkor a $\sum_{n=0}^\infty a_n < + \infty$ feltételezés miatt van olyan $N \in \N$, amellyel
\[
	\sum_{k=n+1}^m a_k < \varepsilon \quad (n, \, m \in \N, \, N < n < m).
\]
Ezért
\[
	\left|  \sum_{k=n+1}^m f_k(x) \right| < \varepsilon \quad (x \in A, \, n, \, m \in \N, \, N < n < m).
\]
$\hfill \blacksquare$