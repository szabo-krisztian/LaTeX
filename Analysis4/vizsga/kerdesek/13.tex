\newpage
\section{Vizsgakérdés}
\begin{quote}
	\textit{A határfüggvény differenciálhatósága, ill. a deriválás és a határátmenet felcserélhetőségére vonatkozó tétel.}
\end{quote}

\subsection{Differenciálhatóság és a határátmenet}

\tikz \node[theorem]
{
	\textbf{Tétel.} Legyen az $I \subset \R$ korlátos, nyílt intervallum, az
	\[
		f_n : I \to \R \quad (n \in \N)
	\]
	függvényekről pedig tegyük fel, hogy valamennyien differenciálhatók, a deriváltakból álló $(f'_n)$ függvénysorozat pedig egyenletesen konvergens. Ha van olyan $a \in I$, hogy az $(f_n(a))$ (helyettesítési értékekből álló) sorozat konvergens, akkor
	\begin{itemize}
		\item az $(f_n)$ függvénysorozat egyenletesen konvergens
		\item ha $f$ jelöli az $(f_n)$ sorozat határfüggvényét, akkor $f \in D$ és
		\[
			f' = \lim_{n \to \infty} f'_n.
		\]
	\end{itemize}
};\\

\textbf{Bizonyítás.} Legyen $b \in I$, és vezessük be a következő jelöléseket:
\[
	\Phi_{bn}(x) := \begin{cases}
		\displaystyle \frac{f_n(x) - f_n(b)}{x - b} & (x \neq b)\\
		f_n'(b) & (x = b) 
	\end{cases} \quad (x \in I, \, n \in \N).
\]
A $\Phi_{bn}$-ek mind folytonosak. Ez ui. a $b$-től különböző helyeken az $f_n$-ek folytonossága alapján a műveleti szabályok és a folytonosság kapcsolatából következik. A $b$ helyen pedig a
\[
	\lim_{x \to b} \Phi_{bn}(x) = \lim_{x \to b} \frac{f_n(x) - f_n(b)}{x- b} = f_n'(b) = \Phi_{bn}(b) \quad (n \in \N)
\]
egyenlőségből. Azt sem nehéz belátni, hogy a $(\Phi_{bn})$ függvénysorozat egyenletesen konvergens. Ti. legyen ehhez
\[
	n, \, m \in \N, \, b \neq x \in I,
\]
ekkor a Lagrange-középértéktétel szerint alkalmasan $b$ és $x$ közötti $\xi_{nm}$-mel
\[
	\frac{ \BB{f_n(x) - f_m(x)} - \BB{ f_n(b) - f_m(b) } }{x- b} = (f_n - f_m)'(\xi_{nm}) =
\]
\[
	f'_n(\xi_{nm}) -f'_m(\xi_{nm}).
\]
Ezt felhasználva azt kapjuk, hogy
\[
	|\Phi_{bn}(x) - \Phi_{bm}(x)| = \begin{cases}
		|f'_n(\xi_{nm}) -f'_m(\xi_{nm})| & (b \neq x \in I) \\
		|f'_n(b) - f_m'(b)| & (x = b) 
	\end{cases} \quad (n, \, m \in \N).
\]
A feltételeink szerint az $(f'_n)$ sorozat egyenletesen konvergens, ezért minden $\varepsilon > 0$ mellett egy $N \in \N$ indexszel
\[
	|f'_n(t) - f'_m(t)| < \varepsilon \quad (t \in I, \, N < n, \, m \in \N).
\]
Innen világos, hogy
\[
	|\Phi_{bn}(x) - \Phi_{bm}(x)| < \varepsilon \quad (x \in I, \, N < n, \, m \in \N)
\]
azaz a $(\Phi_{bn})$ függvénysorozatra teljesül az egyenletesen Cauchy-kritérium: a $(\Phi_{bn})$ sorozat egyenletesen konvergens. Legyen
\[
	\Phi_b := \lim_{n \to \infty} \Phi_{bn},
\]
ami (ld. előző tétel) folytonos.\\

Vegyük észre, hogy
\[
	f_n(x) = (x-a) \cdot \Phi_{an}(x) + f_n(a) \quad (x \in I, \, n \in \N).
\]
Tehát a
\[
	g(x) := x-a, \, h_n(x) := f_n(a) \quad (x \in I, \, n \in \N)
\]
függvényekkel
\[
	(f_n) = (g \Phi_{an}) + (h_n).
\]
Az $I$ intervallum korlátossága miatt a $g$ függvény korlátos, ezért a $(g \Phi_{an})$ sorozat egyenletesen konvergens. Nyilván ugyanez igaz a $(h_n)$ (konstans)sorozatra is, amiből az $(f_n)$ függvénysorozat egyenletesen konvergenciája már következik, legyen
\[
	f := \lim_{n \to \infty} f_n.
\]
A $(\Phi_{bn}) \quad (b \in I)$ sorozat definíciója alapján most már világos, hogy
\[
	\Phi_b(x) = \lim_{n \to \infty} \Phi_{bn}(x) = \begin{cases}
		\displaystyle \frac{f(x) - f(b)}{x-b} & (x \neq b) \\
		\displaystyle \lim_{n \to \infty} f'_n(b) \quad (x = b)
	\end{cases} \quad (x \in I, \, n \in \N).
\]
Láttuk, hogy a $\Phi_b$ függvény folytonos, így létezik a $\displaystyle \lim_{x \to b} \Phi_b(x)$ határérték és
\[
	\lim_{x \to b} \Phi_b(x) = \Phi_b(b).
\]
Tehát
\[
	\Phi_b(b) = \lim_{n \to \infty}\Phi_{bn}(b) = \lim_{x \to b} \Phi_b(x) = \lim_{x \to b} \frac{f(x) - f(b)}{x -b }.
\]
Mindez azt jelenti, hogy $f \in D\{b\}$ és
\[
	f'(b) = \lim_{x \to b} \frac{f(x) - f(b)}{x-b} = \lim_{n \to \infty} f'_n(b) \quad (b \in I).
\]
$\hfill \blacksquare$