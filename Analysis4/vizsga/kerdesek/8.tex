\newpage
\section{Vizsgakérdés}
\begin{quote}
	\textit{Magasabb rendű lineáris differenciálegyenlet. Az átviteli elv. A megoldáshalmaz szerkezete. Az állandók variálásának a módszere.}
\end{quote}

\subsection{"Új" feladat megfogalmazása}
Legyen $1 \leq n \in \N, \, I \subset \R$ nyílt intervallum, az
\[
	a_k : I \to \R \quad (k = 0, \, \dots, \, n-1), \, c: I \to \R
\]
függvényekről tegyük fel, hogy folytonosak. Olyan $\p \in I \to \K$ függvényt keresünk, amelyikre
\begin{enumerate}
	\item $\Dp \subset I$ nyílt intervallum;
	\item $\p \in D^n$;
	\item $\displaystyle \p^{(n)}(x) + \sum_{k=0}^{n-1} a_k(x) \cdot \p^{k}(x) = c(x) \quad (x \in \Dp)$.
\end{enumerate}
Ezt a feladatot röviden \textit{$n$-edrendű lineáris differenciálegyenletnek} nevezzük. Minden olyan $\p$ függvény amelyik eleget tesz az előbbi kívánalmaknak, az illető differenciálegyenlet (egy) \textit{megoldása}.\\

Tegyük fel, hogy a fentieken túl adottak még a
\[
	\tau \in I, \, \xi_0, \, \dots, \, \xi_{n-1} \in \K
\]
számok. Ha az előbbi $\p$ megoldástól azt is elvárjuk, hogy
\begin{enumerate}[start=4]
	\item $\tau \in \Dp, \, \p^{(k)}(\tau) = \xi_k \quad (k = 0, \, \dots, \, n-1)$,
\end{enumerate}
akkor a szóban forgó $n$-edrendű lineáris differenciálegyenletre vonatkozó \textit{kezdetiérték-problémáról} beszélünk.\\

Ha $n=1$, akkor egy lineáris differenciálegyenletről van szó, ezért a továbbiakban nyugodtan feltehetjük már, hogy $n \geq 2$.\\

Az \textit{átviteli elv} segítségével a most megfogalmazott feladat visszavezethető a lineáris differenciálegyenlet-rendszerek vizsgálatára. (A későbbiekben szereplő állítások is részben ennek az elvnek a segítségével láthatók majd be.) Vezessük be ui. az alábbi jelöléseket: legyen $2 \leq n \in \N$ és
\[
	b := (b_1, \, \dots, \, b_n) : I \to \R^n, \, b(x) := \begin{pmatrix}
		0 \\
		0 \\
		\vdots \\
		c(x)
	\end{pmatrix} \quad (x \in I),
\]
\[
	A := (a_{ik})_{i, \, k = 1}^n = \begin{bmatrix}
		0 & 1 & 0 & 0 & \cdots & 0 \\
		0 & 0 & 1 & 0 & \cdots & 0 \\
		\vdots & \vdots & \vdots & \vdots & \cdots & \vdots \\
		\vdots & \vdots & \vdots & \vdots & \cdots & \vdots \\
		\vdots & \vdots & \vdots & \vdots & \cdots & \vdots \\
		0 & 0 & 0 & 0 & \cdots & 1 \\
		-a_0 & -a_1 & -a_2 & -a_3 & \cdots & -a_{n-1}
	\end{bmatrix} : I \to \R^{n \times n}.
\]
Ekkor
\[
	f(x, y) := A(x) \cdot y + b(x) \quad (x \in I, \, y \in \K^n).
\]
Ha tehát a
\[
	\psi = (\psi_1, \, \dots, \, \psi_n) \in I \to \K^n
\]
differenciálható függvény ez utóbbi lineáris differenciálegyenlet-rendszernek (egy) megoldása, akkor $\mathcal{D}_\psi \subset I$ nyílt intervallum, és bármely $x \in \mathcal{D}_\psi$ esetén
\[
	\psi'(x) = A(x) \cdot \psi(x) + b(x) \quad (x \in \mathcal{D}_\psi).
\]
Azaz
\[
	\begin{pmatrix}
		\psi_1' \\
		\psi_2' \\
		\vdots \\
		\vdots \\
		\psi_n' \\
	\end{pmatrix} =
	\psi_1 \cdot \begin{pmatrix}
		0 \\
		0 \\
		\vdots \\
		0 \\
		-a_0
	\end{pmatrix} + 
	\psi_2 \cdot \begin{pmatrix}
		1 \\
		0 \\
		\vdots \\
		0 \\
		-a_1
	\end{pmatrix} + \cdots + 
	\psi_n \cdot \begin{pmatrix}
		0 \\
		0 \\
		\vdots \\
		1 \\
		-a_{n-1}
	\end{pmatrix} +
	\begin{pmatrix}
		0 \\
		0 \\
		\vdots \\
		0 \\
		c
	\end{pmatrix},
\]
tehát
\[
	\begin{pmatrix}
		\psi_1' \\
		\psi_2' \\
		\vdots \\
		\vdots \\
		\psi_n' \\
	\end{pmatrix} =
	\begin{pmatrix}
		\psi_2 \\
		\psi_3 \\
		\vdots \\
		\vdots \\
		\sum_{k=1}^n (-a_{k-1}) \cdot \psi_k + c
	\end{pmatrix}.
\]
Ez azt jelenti, hogy
\begin{equation}
	\begin{cases}
		\psi_i'(x) = \psi_{i+1}(x) \quad (i = 1, \, \dots, \, n-1)\\
		\\
		\displaystyle \psi'_n(x) = \sum_{k=1}^n (-a_{k-1}(x)) \cdot \psi_k(x) + c(x).
	\end{cases}
	\tag{$\star$}
\end{equation}
Ennek alapján eléggé nyilvánvaló az alábbi állítás.\\

\subsection{Átviteli elv}

\tikz \node[theorem]
{
	\textbf{Tétel.} Ha a $\p$ függvény megoldása a fenti $n$-edrendű lineáris differenciálegyenletnek, akkor az
	\[
		I \ni x \mapsto \psi(x) := \big(\p(x), \, \p'(x), \, \dots, \, \p^{(n-1)}(x)\big) \in \K^n
	\]
	függvényre igazak a $(\star)$ egyenlőségek. Fordítva, ha a $\psi = (\psi_1, \, \dots, \, \psi_n)$ függvény eleget tesz a $(\star)$-nak, akkor a $\p := \psi_1$ (első) komponensfüggvény megoldása a szóban forgó $n$-edrendű lineáris differenciálegyenletnek. Ha adottak a $\tau \in I, \, \xi_0, \, \dots, \, \xi_{n-1} \in \K$ kezdeti értékek, és a $\p$, megoldása a
	\[
		\p^{(k)}(\tau) = \xi_k \quad (k = 0, \, \dots, \, n-1)
	\]
	k.é.p.-nak, akkor a $(\star)$ lineáris differenciálegyenlet-rendszer előbbi $\psi$ megoldása kielégíti a
	\[
		\psi(\tau) = (\xi_0, \, \dots, \, \xi_{n-1}) \in \K^n
	\]
	kezdeti feltételt.
};\\

Legyen most
\[
	\mathcal{M}_h := \Big\{ \p : I \to \K : \p \in D^n, \, \p^{(n)} + \sum_{k=0}^{n-1} a_k \cdot \p^{(k)} = 0 \Big\}.
\]
Az $\mathcal{M}_h$ függvényhalmaz tehát nem más, mint a
\[
	c(x) := 0 \quad (x \in I)
\]
esetnek megfelelő \textit{homogén $n$-edrendű lineáris differenciálegyenlet} $I$ intervallumon értelmezett megoldásainak a halmaza. Legyen továbbá
\[
	\mathcal{M} := \Big\{ \p : I \to \K : \p \in D^n, \, \p^{(n)} + \sum_{k=0}^{n-1} a_k \cdot \p^{(k)} = c \Big\}
\]
a kiindulási $n$-edrendű lineáris differenciálegyenlet $I$-n értelmezett megoldásainak a halmaza. Az utóbbival kapcsolatban már nyilván feltehető, hogy valamilyen $x \in I$ helyen $c(x) \neq 0$, azaz az illető egyenlet \textit{inhomogén}. Ekkor az átviteli elv alapján a következőket mondhatjuk.\\

\subsection{Állandók variálásának módszere}

\tikz \node[theorem]
{
	\textbf{Tétel.} Az $n$-edrendű lineáris differenciálegyenletet illetően
	\begin{enumerate}
		\item az $\mathcal{M}_h$ halmaz $n$ dimenziós lineáris tér a $\K$-ra vonatkozóan;
		\item tetszőleges $\omega \in \mathcal{M}$ esetén
		\[
			\mathcal{M} = \omega + \mathcal{M}_h := \{ \omega + \chi : \chi \in \mathcal{M}_h \};
		\]
		\item ha a $\p_1, \, \dots, \, \p_n$ függvények bázist alkotnak az $\mathcal{M}_h$-ban, akkor léteznek olyan differenciálható $g_k : I \to \K \quad (k = 1, \, \dots, \, n)$ függvények, amelyekkel
		\[
			\omega := \sum_{k=1}^n g_k \p_k \in \mathcal{M}.
		\]
	\end{enumerate}
};\\

Bizonyításképpen elegendő annyit megjegyezni, hogy az $\mathcal{M}_h$-beli
\[
	\p_1, \, \dots, \p_m : I \to \K \quad (1 \leq m \in \N)
\]
függvények akkor és csak akkor függetlenek, ha a
\[
	\hat{\p}_j := \BB{ \p_j, \, \p_j', \, \dots, \, \p_j^{(n-1)} } : I \to \K^n \quad (j = 1, \, \dots, \ m)
\]
(vektor)függvények is azok.\\

Ha $\phi_1, \, \dots, \, \phi_n \in \mathcal{M}_h$ bázis, akkor minden bázist (most is) \textit{alaprendszernek}, az előző tételben szereplő $\omega$ függvényt pedig \textit{partikuláris megoldásnak} nevezünk. Egy partikuláris megoldásnak az előző tétel szerinti előállítását az \textit{állandók variálásaként} említjük, \\

Tegyük fel tehát, hogy $\p_1, \, \dots, \, \p_n \in \mathcal{M}_h$ alaprendszer, ekkor a
\[
	\hat{\p_j} := \BB{ \p_j, \, \p_j', \, \dots, \, \p_j^{(n-1)} } \quad (j = 1, \, \dots, \, n)
\]
függvények alaprendszert alkotnak az átviteli elvből adódó $(\star)$ lineáris differenciálegyenlet-rendszerre vonatkozóan. Más szóval a
\[
	\Phi := [ \hat{\p}_1  \cdots \hat{\p}_n] = \begin{bmatrix}
		\p_1   & \p_1  & \cdots & \p_n \\
		\p'_1  & \p'_1 & \cdots & \p'_n \\
		\vdots & \vdots & \cdots & \vdots \\
		\p_1^{(n-1)} & \p_2^{(n-1)} & \cdots & \p_n^{(n-1)}
	\end{bmatrix} : I \to \K^{n \times n}
\]
mátrixfüggvény alapmátrixa a $(\star)$-rendszernek. Innen tudjuk, hogy a
\[
	g = (g_1, \, \dots, \, g_n) : I \to \K^n
\]
jelöléssel a $\Phi \cdot g$ függvény pontosan akkor partikuláris megoldása a $(\star)$-nak (alkalmas differenciálható $g_1, \, \dots, \, g_n : I \to \K$ függvényekkel), ha
\[
	\Phi \cdot g' = b = (0, \, \dots, \, 0, \, c).
\]
Ez azt jelenti, hogy a $\Phi \cdot g$ függvény első komponense, azaz az
\[
	\omega := \sum_{k=1}^n \p_k \cdot g_k
\]
függvény akkor és csak akkor partikuláris megoldása az $n$-edrendű lineáris differenciálegyenletnek, ha
\[
\begin{array}{ccccccccccc}
	\varphi_1 \cdot g_1' &+& \varphi_2 \cdot g_2' &+& \cdots &+& \varphi_n \cdot g_n' &=& 0 \\
	\varphi_1' \cdot g_1' &+& \varphi_2' \cdot g_2' &+& \cdots &+& \varphi_n' \cdot g_n' &=& 0 \\
	\vdots & & \vdots & & \vdots & & \vdots & & \vdots \\
	\varphi_1^{(n-2)} \cdot g_1' &+& \varphi_2^{(n-2)} \cdot g_2' &+& \cdots &+& \varphi_n^{(n-2)} \cdot g_n' &=& 0 \\
	\varphi_1^{(n-1)} \cdot g_1' &+& \varphi_2^{(n-1)} \cdot g_2' &+& \cdots &+& \varphi_n^{(n-1)} \cdot g_n' &=& c.
\end{array}
\]
Ennek a $(g_1', \, \dots, \, g_n'$ függvényekre mint "ismeretlenekre" vonatkozó) lineáris (függvény)egyenletrendszernek a determinánsa (determináns-függvénye), azaz a
\[
	W(x) := \det \BB{\p_i^{(k-1)}(x)}_{k, \, i = 1}^n = \det \B{\Phi(x)} \quad (x \in I)
\]
leképezés (az ún. \textit{Wronski-determináns}) a $\hat{\p}_1, \, \dots, \, \hat{\p}_n$ függvények lineáris függetlensége miatt egyetlen $x \in I$ helyen sem tűnik el.



