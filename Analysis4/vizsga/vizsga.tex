\documentclass[12pt]{article}
\usepackage{tikz}
\usetikzlibrary{shadows.blur}
\usepackage{setspace}
\usepackage{hyperref}
\usepackage{amsfonts, amssymb, amsmath}
\usepackage{titlesec}
\usepackage{caption}
\usepackage{enumitem}


\setlength{\parskip}{0pt}
\setlength{\parindent}{0pt}

\title{\textcolor{purple}{\Huge\textbf{Analízis alkalmazásai vizsgatematika}}}
\author{Dr. Simon Péter jegyzetéből}
\date{}
\renewcommand{\contentsname}{Tartalom}

\newcommand{\R}{\mathbf{R}}
\newcommand{\N}{\mathbf{N}}
\newcommand{\K}{\mathbf{K}}
\newcommand{\Q}{\mathbf{Q}}


\newcommand{\p}{\varphi}

\newcommand{\mm}{\mathbf{m}}
\newcommand{\MM}{\mathbf{M}}

\newcommand{\D}{\mathcal{D}_f}
\newcommand{\Dp}{\mathcal{D}_\varphi}

\tikzset
{
    theorem/.style={
        draw,
        fill=white,
        line width=1pt,
        rounded corners,
        drop shadow={shadow blur steps=5,shadow xshift=1ex,shadow yshift=-1ex},
        text width=1\textwidth,
        inner sep=8pt,
        align=justify
    }
}

\begin{document}
    \maketitle
    \tableofcontents
    
    \newpage
\section{Vizsgakérdés}
\begin{quote}
	\textit{Az implicitfüggvény fogalma, kapcsolata a feltételes szélsőérték problémával és az inverzfüggvénnyel. Implicitfüggvény-tétel, inverzfüggvény-tétel (a bizonyítás vázlata).}
\end{quote}

Legyenek $n, \, m \in \N$ természetes számok, $1 \leq m < n$. Ha
\[
	\xi = (\xi_1, \, \dots, \, \xi_n) \in \R^n,
\]
akkor legyen
\[
	x := (\xi_1, \, \dots, \, \xi_{n-m}) \in \R^{n-m}, \, y := (\xi_{n-m+1}, \, \dots, \, \xi_n) \in \R^m,
\]
és ezt következőképpen fogjuk jelölni:
\[
	\xi = (x, \, y).
\]
Röviden:
\[
	\R^n \equiv \R^{n-m} \times \R^{m}.
\]
Ha tehát
\[
	f = (f_1, \, \dots, \, f_m) \in \R^n \to \R^m,
\]
azaz
\[
	f \in \R^{n-m} \times \R^m \to \R^m,
\]
akkor az $f$-et olyan kétváltozós vektorfüggvénynek tekintjük, ahol az $f(x, \, y)$ helyettesítési értékben az argumentum első változójára $x \in \R^{n-m}$, a második változójára pedig $y \in \R^m$ teljesül.\\

Tegyük fel, hogy ebben az értelemben valamilyen $(a, \, b) \in \D$ zérushelye az $f$-nek:
\[
	f(a, \, b) = 0.
\]
Tételezzük fel továbbá, hogy van az $a$-nak egy olyan $K(a) \subset \R^{n-m}$ környezete, a $b$-nek pedig olyan $K(b) \subset \R^m$ környezete, hogy tetszőleges $x \in K(a)$ esetén egyértelműen létezik olyan $y \in K(b)$, amellyel
\[
	f(x, \, y) = 0.
\]
Definiáljuk ekkor a $\varphi(x) := y$ hozzárendeléssel a
\[
	\varphi : K(a) \to K(b)
\]
függvényt, amikor is
\[
	f(x, \, \varphi(x)) = 0 \quad (x \in K(a)).
\]
Itt minden $x \in K(a)$ mellett az $y = \varphi(x)$ az egyetlen olyan $y \in K(b)$ hely amelyre
\[
	f(x, \, y) = 0.
\]
Az előbbi $\varphi$ függvényt az $f$ által (az $(a, \, b)$ körül) meghatározott \textit{implicitfüggvénynek} nevezzük. Tehát az

\[
	\left\{
	\begin{array}{ccc}
		f_1(x_1, \, \dots, \, x_{n-m}, \, y_1, \, \dots, \, y_m) & = & 0 \\
		\vdots & & \vdots \\
		f_m(x_1, \, \dots, \, x_{n-m}, \, y_1, \, \dots, \, y_m) & = & 0
	\end{array}
	\right.
\]

egyenletrendszernek minden $x = (x_1, \, \dots, \, x_{n-m}) \in K(a)$ mellett egyértelműen létezik
\[
	y = (y_1, \, \dots, \, y_m) = \varphi(x) \in K(b)
\]
megoldása. Nyilván $\varphi(a) = b$.\\

A $\p : K(a) \to K(b)$ implicitfüggvényre a következő igaz:
\[
	\big(K(a) \times K(b)\big) \cap \{ f=0\} = \{ (x, \, \p(x)) \in \R^n : x \in K(a) \}.
\]
Geometriai szóhasználattal élve
\[
	\text{graf} \, \p := \{ (x, \, \p(x)) \in \R^n : x \in K(a) \}
\]
(a $\p$ függvény "grafikonja", ami a függvény definíciója miatt persze maga a $\p$ függvény), tehát az előbbi egyenlőség így néz ki:
\[
	\big( K(a) \times K(b) \big) \cap \{f=0\} = \text{graf} \, \p = \p.
\]

\subsection{Implicitfüggvény-tétel}

\tikz \node[theorem]
{
	\textbf{Tétel.} Adott $n, \, m \in \N$, valamint $1 \leq m < n$ mellett az
	\[
	f \in \R^{n-m} \times \R^m \to \R^m
	\]
	függvényről tételezzük fel az alábbiakat: $f \in C^1$, és az $(a, \, b) \in \text{int} \, \D$ helyen
	\[
	f(a, \, b) = 0, \, \det \partial_2f(a, \, b) \neq 0.
	\]
	Ekkor alkalmas $K(a), \, K(b)$ környezetekkel létezik az $f$ által az $(a, \, b)$ körül meghatározott
	\[
	\varphi : K(a) \to K(b)
	\]
	implicitfüggvény, ami folytonosan differenciálható, és
	\[
	\varphi'(x) = -\partial_2 f(x, \, \varphi(x))^{-1} \cdot \partial_1 f(x, \, \varphi(x)) \quad (x \in K(a)).
	\]
};\\

A tételben $f \in C^1, \, \partial_2f(a, \, b) \neq 0$ feltételekből következően a $K(a), \, K(b)$ környezetekről az is feltehető, hogy
\[
	\det \partial_2 f(x, \, y) \neq 0 \quad (x \in K(a), \, y \in K(b)),
\]
egyúttal
\[
	\det \partial_2f(x, \, \p(x)) \neq 0 \quad (x \in K(a)).
\]
Ezért az $x \in K(a)$ helyeken a $\partial_2f(x, \, \p(x))$ mátrix valóban invertálható.\\

\subsection{Inverzfüggvény-tétel}

Elöljáróban idézzük fel az egyváltozós valós függvényekkel kapcsolatban tanultakat. Ha pl.
\[
h \in \R \to \R, \, h \in C^1\{a\}
\]
és $h'(a) \neq 0$, akkor egy alkalmas $r>0$ mellett
\[
I := (a-r, \, a+r) \subset \mathcal{D}_h,
\]
létezik a $(h_{|_I})^{-1}$ inverzfüggvény, a $g := (h_{|_I})^{-1}$ függvény differenciálható és 
\[
g'(x) = \frac{1}{h'(g(x))} \quad (x \in \mathcal{D}_g).
\]
A továbbiakban a most megfogalmazott "egyváltozós" állítás megfelelőjét fogjuk vizsgálni többáltozós vektorfüggvényekre.\\

Legyen ehhez valamilyen $1 \leq n \in \N$ mellett adott az
\[
f \in \R^n \to \R^n
\]
függvény és az $a \in \text{int} \, \D$ pont. Azt mondjuk, hogy az $f$ függvény \textit{lokálisan invertálható} az $a$-ban, ha létezik olyan $K(a) \subset \D$ környezet, hogy a $g := f_{|_{K(a)}}$ leszűkítés invertálható függvény. Minden ilyen esetben a $g^{-1}$ inverzfüggvényt az $f$ $a$-beli \textit{lokális inverzének} nevezzük.\\

\tikz \node[theorem]
{
	\textbf{Tétel.} Legyen $1 \leq n \in \N$, és $f \in \R^n \to \R^n$. Tegyük fel, hogy egy $a \in \text{int} \, \D$ pontban $f \in C^1\{a\}$, $\det f'(a) \neq 0$. Ekkor alkalmas $K(a) \subset \D$ környezettel az $f_{|_{K(a)}}$ leszűkítés invertálható, a $h := (f_{|_{K(a)}})^{-1}$ lokális inverzfüggvény folytonosan differenciálható, és
	\[
	h'(x) = \big( f'(h(x)) \big)^{-1} \quad (x \in \mathcal{D}_h).
	\]
};

\subsection{Hiperkoordinátás parciális deriváltak}
Legyen adott $n, \, m \in \N, \, 1 \leq m < n \in \N$ mellett
\[
	f : \R^n \to \R^m.
\]
Az előbbiek alapján most adott $k \in \N, \, 1 \leq k < n$ esetén legyen $\R^n \equiv \R^{n-k} \times \R^k$. Ha $\xi \in \D$, akkor legyen $(a, \, b) = \xi$, ahogy eddig. Azaz $f$-et fel lehet fogni egy kétváltozós függvénynek. Tekintsük az alábbi definíciót:
\[
	\mathcal{D}_1^{(a, \, b)} := \{ x \in \R^{n-k} : (x, \, b) \in \D \},
\]
\[
	\mathcal{D}_2^{(a, \, b)} := \{ y \in \R^{k} : (a, \, y) \in \D \}.
\]
Ekkor analóg módon a \textit{szokásos} parciális deriváltakhoz
\[
	f_{(a, \, b), \, 1}  \in \R^{n-k} \to \R^m, 
\]
\[
	f_{(a, \, b), \, 2}  \in \R^{k} \to \R^m,
\]
ahol
\[
	f_{(a, \, b), \, 1}(x) := f(x, \, b) \quad (x \in \mathcal{D}_1^{(a, \, b)}),
\]
\[
	f_{(a, \, b), \, 2}(y) := f(a, \, y) \quad (y \in \mathcal{D}_2^{(a, \, b)}).
\]
Ebben az esetben a \textit{hiperkoordinátás} alakja a parciális deriváltaknak (amennyiben értelmes a derivált):
\[
	\partial_{\mathbf{1}}f(a, \, b) := \partial_1f(a, \, b) := f'_{(a, \, b), \, 1}(a),
\]
\[
	\partial_{\mathbf{2}}f(a, \, b) := \partial_2f(a, \, b) := f'_{(a, \, b), \, 2}(b).	
\]
Azaz egy $(a, \, b)$ helyen lerögzítjük az első vagy második változók és az így kapott függvénynek vesszük a deriváltját. Ha $f$ egy differenciálható függvény az $(a, \, b) \in \D$ helyen, akkor
\[
	f'(a, \, b) = \begin{bmatrix}
		\partial_{\mathbf{1}}f(a, \, b) & \partial_{\mathbf{2}}f(a, \, b)
	\end{bmatrix} = \begin{bmatrix}
		\partial_1f_1(a, \, b) & \partial_2f_1(a, \, b) & \cdots & \partial_nf_1(a, \, b) \\
		\partial_1f_2(a, \, b) & \partial_2f_2(a, \, b) & \cdots & \partial_nf_2(a, \, b) \\
		\vdots & \vdots &\cdots & \vdots \\
		\partial_1f_n(a, \, b) & \partial_2f_n(a, \, b) & \cdots & \partial_nf_n(a, \, b)
	\end{bmatrix},
\]
ahol a $\partial_{\mathbf{1}}f(a, \, b) \in \R^{m \times (n-k)}, \, \partial_{\mathbf{2}}f(a, \, b) \in \R^{m \times k}$ mátrixok rendre az $f'(a, \, b) \in \R^{m \times n}$ mátrix első $n-k$-adik és utolsó $k$-adik oszlopvektorai. Pl. legyen $f \in \R^3 \to \R^4$ $(a, \, b) \in \text{int} \, \D$-ben differenciálható függvény, $k := 2$. Ekkor
\[
	f \in \R \times \R^2 \to \R^4
\]
és 
\[
	f'(a, \, b) = \begin{bmatrix}
		\partial_1f_1(a, \, b) & \partial_2f_1(a, \, b) & \partial_3f_1(a, \, b) \\
		\partial_1f_2(a, \, b) & \partial_2f_2(a, \, b) & \partial_3f_2(a, \, b) \\
		\partial_1f_3(a, \, b) & \partial_2f_3(a, \, b) & \partial_3f_3(a, \, b) \\
		\partial_1f_4(a, \, b) & \partial_2f_4(a, \, b) & 	\partial_3f_4(a, \, b) \\
	\end{bmatrix},
\]
\[
	\partial_{\mathbf{1}}f(a, \, b) = \begin{bmatrix}
		\partial_1f_1(a, \, b) \\
		\partial_1f_2(a, \, b) \\
		\partial_1f_3(a, \, b) \\
		\partial_1f_4(a, \, b) \\
	\end{bmatrix},
	\partial_{\mathbf{2}}f(a, \, b) = \begin{bmatrix}
		\partial_2f_1(a, \, b) & \partial_3f_1(a, \, b) \\
		\partial_2f_2(a, \, b) & \partial_3f_2(a, \, b) \\
		\partial_2f_3(a, \, b) & \partial_3f_3(a, \, b) \\
		\partial_2f_4(a, \, b) & \partial_3f_4(a, \, b) \\
	\end{bmatrix}.
\]
	\intro{A nullamértékű halmaz fogalma, a majdnem mindenütt terminológia. A Riemann-integrálhatóság Lebesgue-kritériuma.}

Túl szűk azoknak a függvényeknek az összessége, amelyek Riemann-integrálhatók, nevezetesen, bizonyos értelemben a folytonosság "majdnem" szükséges az integrálhatósághoz. Másrészt pl. olyan, az analízis szempontjából alapvető művelet, mint a határátmenet eredményére nem "öröklődik" az integrálhatóság, ill. ha ez utóbbi teljesül is, akkor is csak erős feltételek mellett cserélhető fel a határátmenet és az integrálás. Az sem mellékes, hogy pl. a valós vagy a komplex számok körében alapvető fontosságú teljesség (azaz a sorozatok konvergenciájának és a Cauchy-tulajdonságának az
ekvivalenciája) nem igaz az $R[a, \, b]$-ben természetes módon értelmezhető távolságfogalom tekintetében. Többek között ezek a szempontok is tették szükségessé egy olyan integrálfogalom megalkotását, amelyik pl. a most felsorolt hiányosságokat kiküszöböli.

\subsection{Nullamértékű halmaz fogalma}

Megmutatjuk, hogy "lényegében" csak a folytonos függvények Riemann-integrálhatók. Vezessük be ehhez először is a (Lebesgue szerint) nullamértékű halmaz fogalmát: azt mondjuk, hogy az $A \subset \R$ halmaz \textit{nullamértékű}, ha tetszőleges $\varepsilon > 0$ számhoz megadható $I_k \subset \R \, \, (k \in \N)$ intervallumoknak egy olyan sorozata, hogy

\[
	A \subset \bigcup_{k=0}^\infty I_k \quad \text{ és } \quad \sum_{k=0}^\infty |I_k| < \varepsilon.
\]

Egyszerűen belátható, hogy az $\R$ minden, legfeljebb megszámlálható részhalmaza nullamértékű. Sőt, ha $X_k \subset \R \, \, (k \in \N)$ nullamértékű, akkor az $\cup_{k=0}^\infty$ halmaz is nullamértékű. Az is egyszerűen adódik, hogy a nullamértékűség előbbi definíciójában (ha adott esetben szükség van rá) nyugodtan feltehető, hogy a szóban forgó $I_k \, \, (k \in \N)$ intervallumok mindegyike nyílt. Világos, hogy egy nullamértékű halmaz minden részhalmaza is nullamértékű.

Könnyű meggondolni azt is, hogy egy $[a, \, b] \, \, (a, \, b \in \R, \, a < b)$ intervallum nem nullamértékű.

\subsection{Lebesgue-kritérium}

\bblock{Tétel}{
	Tegyük fel, hogy az $[a, \, b] \subset \R \, \, (a, \, b \in \R, \, a < b)$ kompakt intervallumon értelmezett $f : [a, \, b] \to \R$ függvény korlátos, és legyen az $f$ szakadási helyeinek a halmaza
	\[
		\mathcal{A}_f := \big\{ x \in [a, \, b] : f \not\in C\{x\} \big\}.
	\]
	Ekkor
	\[
		f \in R[a, \, b] \quad \Longleftrightarrow \quad \mathcal{A}_f \text{ nullamértékű halmaz.}
	\]
}

\textbf{Bizonyítás.} Induljunk ki először abból, hogy $f \in 
R[a, \, b]$. Legyen $\alpha \in [a, \, b]$, és valamilyen $J \subset \R$ intervallum esetén $\alpha \in \text{int} \, J$, amikor is
\[
	O_Jf := \sup \big\{ |f(x) - f(y)| : x, \, y \in J \cap [a, \, b] \big\}
\]
az $f$ \textit{oszcillációja} a $J$ intervallumon. Az $f$ függvény $\alpha$-beli lokális oszcillációját a következőképpen értelmezzük:
\[
	\Delta_\alpha f := \inf \{ O_Jf : J \subset \R \text{ intervallum}, \, \alpha \in \text{int} \, J\}.
\]
Mutassuk meg először is azt, hogy
\[
	f \in C\{\alpha\} \quad \Longleftrightarrow \quad \Delta_\alpha f = 0.
\]
Valóban, ha $f \in C\{\alpha\}$, akkor minden $\varepsilon > 0$ számhoz van olyan $\delta > 0$, hogy
\[
	|f(x) - f(\alpha)| < \varepsilon \quad (x \in [a, \, b], \, |x - \alpha| < \delta).
\]
Ezért
\[
	|f(x) - f(y)| \leq
\]
\[
	|f(x) - f(\alpha)| + |f(\alpha) - f(y)| < 2\varepsilon \quad (x, \, y \in [a, \, b], \, |x - \alpha|, \, |y - \alpha| < \delta).
\]
Így minden olyan $J \subset \R$ intervallumra, amelyre $\alpha \in \text{int} \, J$ és $d_J < \delta$, igaz, hogy
\[
	|f(x) - f(y)| < 2\varepsilon,
\]
amiből $O_Jf \leq 2\varepsilon$ következik. Ez azt jelenti, hogy $(0 \leq) \, \, \Delta_\alpha f \leq 2\varepsilon$. Mindez csak úgy lehetséges, ha $\Delta_\alpha f = 0$.

Ha most azt tesszük fel, hogy $\Delta_\alpha f = 0$, akkor az infimum tulajdonságait figyelembe véve bármilyen $\varepsilon > 0$ számhoz találunk olyan $J \subset R$ intervallumot, amellyel $\alpha \in \text{int} \, J$, és $O_J f < \varepsilon$. Tehát
\[
	|f(x) - f(y)| < \varepsilon \quad (x, \, y \in J \cap [a, \, b]),
\]
speciálisan
\[
	|f(x) - f(\alpha)| < \varepsilon \quad (x \in J \cap [a, \, b]).
\]
Mivel $\alpha \in \text{int} \, J$, ezért van olyan $\delta > 0$, hogy $x \in J \, \, (x \in [a, \, b], \, |x - \alpha| < \delta)$. Így
\[
	|f(x) - f(\alpha)| < \varepsilon \quad (x \in [a, \, b], \, |x - \alpha| < \delta),
\]
azaz $f \in C\{\alpha\}$.

A lokális oszcilláció és a pontbeli folytonosság kapcsolatáról most belátott ekvivalencia alapján
\[
	\mathcal{A}_f = \{ x \in [a, \, b] : \Delta_x f > 0\} = \bigcup_{k = 1}^\infty \left\{ x \in [a, \, b] : \Delta_x f > 1/k\right\} =: \bigcup_{k=1}^\infty A_k.
\]
Az $\mathcal{A}_f$ halmaz nullamértékűségéhez elegendő azt megmutatni, hogy az
\[
	A_\delta := \{ x \in [a, \, b] : \Delta_x f > \delta\} \quad (\delta > 0)
\]
halmazok nullamértékűek. Legyen $\sigma > 0$, amikor is a Riemann-integrálhatóságnak az oszcillációs összegekkel való jellemzése folytán az $[a, \, b]$ intervallum egy alkalmas $\tau$ felosztásával
\[
	\omega(f, \, \tau) = \sum_{J \in \mathcal{\tau}} o_J(f) \cdot |J| < \sigma,
\]
$\mathcal{F}(\tau)$ jelöli a $\tau$ felosztás által meghatározott osztásintervallumok halmazát. Ekkor tetszőleges $\delta > 0$ mellett
\[
	\sigma > \omega(f, \, \tau) = \sum_{J \in \mathcal{F}(\tau)} o_J(f) \cdot |J| \geq \sum_{J \in \mathcal{F}(\tau), \, A_\delta \cap \text{int} \, J \neq \emptyset} o_J(f) \cdot |J|.
\]
Világos, hogy minden $J \in \mathcal{F}(\tau), \, A_\delta \cap \text{int} \, J \neq \emptyset$ osztásintervallum esetén $o_J(f) \geq \delta$, ezért
\[
	\sigma > \delta \cdot \sum_{J \in \mathcal{F}(\tau), \, A_\delta \cap \text{int} \, J \neq \emptyset} |J|.
\]
Más szóval
\[
\sum_{J \in \mathcal{F}(\tau), \, A_\delta \cap \text{int} \, J \neq \emptyset} |J| < \frac{\sigma}{\delta}.
\]
Legyen itt valamilyen $\varepsilon > 0$ mellett a $\sigma > 0$ olyan, hogy $\sigma / \delta < \varepsilon / 2$. Nyilván
\[
	A_\delta \subset \left( \bigcup_{J \in \mathcal{F}(\tau), \, A_\delta \cap \text{int} \, J \neq \emptyset} J \right) \bigcup \left( \bigcup_{J \in \mathcal{F}(\tau)} (J \, \backslash \, \text{int} \, J) \right),
\]
ahol minden $J \, \backslash \, \text{int} \, J \, \, (J \in \mathcal{F}(\tau))$ nullamértékű halmaz, és így az 
\[
	\bigcup_{J \in \mathcal{F}(\tau)} (J \, \backslash \, \text{int} \, J)
\]
halmaz is nullamértékű. Ezért alkalmas $K_j \subset \R \, \, (j \in \N)$ intervallumsorozattal
\[
	\bigcup_{J \in \mathcal{F}(\tau)} (J \, \backslash \, \text{int} \, J) \subset \bigcup_{j=0}^\infty K_j,
\]
és
\[
	\sum_{j=0}^\infty |K_j| < \frac{\varepsilon}{2}.
\]
Mindezeket egybevetve
\[
	A_\delta \subset \left( \bigcup_{J \in \mathcal{F}(\tau), \, A_\delta \cap \text{int} \, J \neq \emptyset} J \right) \bigcup \left(\bigcup_{j=0}^\infty K_j\right),
\]
és
\[
	\sum_{J \in \mathcal{F}(\tau), \, A_\delta \cap \text{int} \, J \neq \emptyset} |J| + \sum_{j=0}^\infty |K_j| < \varepsilon.
\]
Ez pontosan azt jelenti, hogy az $A_\delta$ halmaz nullamértékű.

Most tegyük fel azt, hogy az $\mathcal{A}_f$ halmaz nullamértékű. Legyen adott az $\varepsilon > 0$ szám, ekkor egy alkalmas, kompakt intervallumokból álló $L_k \subset \R \, \, (k \in \N)$ intervallumsorozattal
\[
	\mathcal{A}_f \subset \bigcup_{k=0}^\infty \text{int} \, L_k, \, \sum_{k=0}^\infty |L_k| <\frac{\varepsilon}{4C},
\]
ahol $C >0$, és $|f(x)| \leq C \, \, (x \in [a, \, b])$. Ha $x \in [a, \, b] \, \backslash \, \mathcal{A}_f$, azaz $f \in C\{x\}$, akkor van olyan $I_x \subset \R$ intervallum, amelyre $x \in \text{int} \, I_x$, és
\[
	O_{I_x}f = \sup\{ |f(t) - f(y)| \in \R : t, \, y \in I_x \cap [a, \, b]\} < \frac{\varepsilon}{2(b-a)}.
\]
Világos, hogy
\[
	[a, \, b] \subset \left(\bigcup_{k=0}^\infty \text{int} \, L_k\right) \bigcup \left(\bigcup_{x \in [a, \, b] \, \backslash \, \mathcal{A}_f} \text{int} \, I_x\right).
\]
Az $[a, \, b]$ kompaktsága miatt az előbbi nyílt lefedést figyelembe véve kapunk olyan véges $A \subset \N, \, B \subset [a, \, b] \, \backslash \, \mathcal{A}_f$ halmazokat, amelyekkel
\[
	[a, \, b] \subset \left(\bigcup_{k \in A}^\infty \text{int} \, L_k\right) \bigcup \left(\bigcup_{x \in B} \text{int} \, I_x\right)
\]
Legyen $\tau \subset [a, \, b]$ az a felosztás, amit az $a, \, b$ és az $L_k \, \, (k \in A), \, I_x \, \, (x \in B)$ intervallumok $[a, \, b]$-be eső végpontjai alkotnak. Világos, hogy bármelyik $J \in \mathcal{F}(\tau)$ osztásintervallumra egy-egy alkalmas $k \in A$, vagy $x \in B$ mellett $J \subset L_k$, vagy $J \subset I_x$ (esetleg mindkét tartalmazás igaz). Ha $k \in A$ és $J \subset L_k$, akkor $o_J(f) \leq 2C$. Ha pedig $x \in B$ és $J \subset I_x$, akkor $o_J(f) \leq \varepsilon / (2(b-a))$. Ezért a $\tau$-hoz tartozó $\omega(f, \, \tau)$ oszcillációs összegről az alábbiakat mondhatjuk:
\[
	\omega(f, \, \tau) = \sum_{J \in \mathcal{F}(\tau) } o_J(f) \cdot |J| \leq
\]
\[
	\sum_{J \in \mathcal{F}(\tau), \, \exists k \in A : J \subset L_k} o_J(f) \cdot |J| + \sum_{J \in \mathcal{F}(\tau), \, \exists x \in B : J \subset I_x} o_J(f) \cdot |J| \leq
\]
\[
	2C \cdot \sum_{J \in \mathcal{F}(\tau), \, \exists k \in A : J \subset L_k} |J| + \frac{\varepsilon}{2(b-a)} \cdot \sum_{J \in \mathcal{F}(\tau), \, \exists x \in B : J \subset I_x} |J| \leq
\]
\[
	2C \cdot \sum_{k=0}^\infty |L_k| + \frac{\varepsilon}{2(b-a)} \cdot \sum_{J \in \mathcal{F}(\tau)} |J| \leq 2C \cdot \frac{\varepsilon}{4C} + \frac{\varepsilon}{2(b-a)} \cdot (b-a) = \varepsilon.
\]
Tehát $f \in R[a, \, b]$. $\blacksquare$
	


    

\end{document}