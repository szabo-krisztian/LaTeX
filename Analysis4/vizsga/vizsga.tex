\documentclass[12pt]{article}
\usepackage{tikz}
\usetikzlibrary{shadows.blur}
\usepackage{setspace}
\usepackage{hyperref}
\usepackage{amsfonts, amssymb, amsmath}
\usepackage{titlesec}
\usepackage{caption}
\usepackage{enumitem}

\setlength{\parskip}{0pt}
\setlength{\parindent}{0pt}

\title{\textcolor{purple}{\Huge\textbf{Analízis alkalmazásai vizsgatematika}}}
\author{Dr. Simon Péter jegyzetéből}
\date{}
\renewcommand{\contentsname}{Tartalom}

\newcommand{\R}{\mathbf{R}}
\newcommand{\N}{\mathbf{N}}
\newcommand{\K}{\mathbf{K}}
\newcommand{\Q}{\mathbf{Q}}

\newcommand{\trigseries}{\sum\big( \alpha_n \cdot \cos(nx) + \beta_n \cdot \sin(nx) \big)}
\newcommand{\trigserieslatin}{\sum\big( a_n \cdot \cos(nx) + b_n \cdot \sin(nx) \big)}
\newcommand{\series}{\displaystyle \Big( \sum_{k=0}^n f_k \Big)}


\newcommand{\p}{\varphi}

\newcommand{\mm}{\mathbf{m}}
\newcommand{\MM}{\mathbf{M}}

\newcommand{\sumk}{\sum_{k=0}^n}
\newcommand{\limsum}{\sum_{k=0}^\infty}
\newcommand{\limn}{\lim_{n \to \infty}}

\newcommand{\DD}{\mathcal{D}}

\newcommand{\B}[1]{\big( #1 \big)}
\newcommand{\BB}[1]{\Big( #1 \ig)}

\newcommand{\D}{\mathcal{D}_f}
\newcommand{\Dp}{\mathcal{D}_\varphi}
\newcommand{\Dpps}{\mathcal{D}_\varphi \cap \mathcal{D}_\psi}

\tikzset
{
    theorem/.style={
        draw,
        fill=white,
        line width=1pt,
        rounded corners,
        drop shadow={shadow blur steps=5,shadow xshift=1ex,shadow yshift=-1ex},
        text width=1\textwidth,
        inner sep=8pt,
        align=justify
    }
}

\begin{document}
	\emergencystretch 3em
    \maketitle
    \tableofcontents
    
    \newpage
\section{Vizsgakérdés}
\begin{quote}
	\textit{Az implicitfüggvény fogalma, kapcsolata a feltételes szélsőérték problémával és az inverzfüggvénnyel. Implicitfüggvény-tétel, inverzfüggvény-tétel (a bizonyítás vázlata).}
\end{quote}

Legyenek $n, \, m \in \N$ természetes számok, $1 \leq m < n$. Ha
\[
	\xi = (\xi_1, \, \dots, \, \xi_n) \in \R^n,
\]
akkor legyen
\[
	x := (\xi_1, \, \dots, \, \xi_{n-m}) \in \R^{n-m}, \, y := (\xi_{n-m+1}, \, \dots, \, \xi_n) \in \R^m,
\]
és ezt következőképpen fogjuk jelölni:
\[
	\xi = (x, \, y).
\]
Röviden:
\[
	\R^n \equiv \R^{n-m} \times \R^{m}.
\]
Ha tehát
\[
	f = (f_1, \, \dots, \, f_m) \in \R^n \to \R^m,
\]
azaz
\[
	f \in \R^{n-m} \times \R^m \to \R^m,
\]
akkor az $f$-et olyan kétváltozós vektorfüggvénynek tekintjük, ahol az $f(x, \, y)$ helyettesítési értékben az argumentum első változójára $x \in \R^{n-m}$, a második változójára pedig $y \in \R^m$ teljesül.\\

Tegyük fel, hogy ebben az értelemben valamilyen $(a, \, b) \in \D$ zérushelye az $f$-nek:
\[
	f(a, \, b) = 0.
\]
Tételezzük fel továbbá, hogy van az $a$-nak egy olyan $K(a) \subset \R^{n-m}$ környezete, a $b$-nek pedig olyan $K(b) \subset \R^m$ környezete, hogy tetszőleges $x \in K(a)$ esetén egyértelműen létezik olyan $y \in K(b)$, amellyel
\[
	f(x, \, y) = 0.
\]
Definiáljuk ekkor a $\varphi(x) := y$ hozzárendeléssel a
\[
	\varphi : K(a) \to K(b)
\]
függvényt, amikor is
\[
	f(x, \, \varphi(x)) = 0 \quad (x \in K(a)).
\]
Itt minden $x \in K(a)$ mellett az $y = \varphi(x)$ az egyetlen olyan $y \in K(b)$ hely amelyre
\[
	f(x, \, y) = 0.
\]
Az előbbi $\varphi$ függvényt az $f$ által (az $(a, \, b)$ körül) meghatározott \textit{implicitfüggvénynek} nevezzük. Tehát az

\[
	\left\{
	\begin{array}{ccc}
		f_1(x_1, \, \dots, \, x_{n-m}, \, y_1, \, \dots, \, y_m) & = & 0 \\
		\vdots & & \vdots \\
		f_m(x_1, \, \dots, \, x_{n-m}, \, y_1, \, \dots, \, y_m) & = & 0
	\end{array}
	\right.
\]

egyenletrendszernek minden $x = (x_1, \, \dots, \, x_{n-m}) \in K(a)$ mellett egyértelműen létezik
\[
	y = (y_1, \, \dots, \, y_m) = \varphi(x) \in K(b)
\]
megoldása. Nyilván $\varphi(a) = b$.\\

\subsection{Implicitfüggvény-tétel}

\tikz \node[theorem]
{
	\textbf{Tétel.} Adott $n, \, m \in \N$, valamint $1 \leq m < n$ mellett az
	\[
	f \in \R^{n-m} \times \R^m \to \R^m
	\]
	függvényről tételezzük fel az alábbiakat: $f \in C^1$, és az $(a, \, b) \in \text{int} \, \D$ helyen
	\[
	f(a, \, b) = 0, \, \det \partial_2f(a, \, b) \neq 0.
	\]
	Ekkor alkalmas $K(a), \, K(b)$ környezetekkel létezik az $f$ által az $(a, \, b)$ körül meghatározott
	\[
	\varphi : K(a) \to K(b)
	\]
	implicitfüggvény, ami folytonosan differenciálható, és
	\[
	\varphi'(x) = -\partial_2 f(x, \, \varphi(x))^{-1} \cdot \partial_1 f(x, \, \varphi(x)) \quad (x \in K(a)).
	\]
};\\

Elöljáróban idézzük fel az egyváltozós valós függvényekkel kapcsolatban tanultakat. Ha pl.
\[
h \in \R \to \R, \, h \in C^1\{a\}
\]
és $h'(a) \neq 0$, akkor egy alkalmas $r>0$ mellett
\[
I := (a-r, \, a+r) \subset \mathcal{D}_h,
\]
létezik a $(h_{|_I})^{-1}$ inverzfüggvény, a $g := (h_{|_I})^{-1}$ függvény differenciálható és 
\[
g'(x) = \frac{1}{h'(g(x))} \quad (x \in \mathcal{D}_g).
\]
A továbbiakban a most megfogalmazott "egyváltozós" állítás megfelelőjét fogjuk vizsgálni többáltozós vektorfüggvényekre.\\

Legyen ehhez valamilyen $1 \leq n \in \N$ mellett adott az
\[
f \in \R^n \to \R^n
\]
függvény és az $a \in \text{int} \, \D$ pont. Azt mondjuk, hogy az $f$ függvény \textit{lokálisan invertálható} az $a$-ban, ha létezik olyan $K(a) \subset \D$ környezet, hogy a $g := f_{|_{K(a)}}$ leszűkítés invertálható függvény. Minden ilyen esetben a $g^{-1}$ inverzfüggvényt az $f$ $a$-beli \textit{lokális inverzének} nevezzük.

\subsection{Inverzfüggvény-tétel}

\tikz \node[theorem]
{
	\textbf{Tétel.} Legyen $1 \leq n \in \N$, és $f \in \R^n \to \R^n$. Tegyük fel, hogy egy $a \in \text{int} \, \D$ pontban $f \in C^1\{a\}$, $\det f'(a) \neq 0$. Ekkor alkalmas $K(a) \subset \D$ környezettel az $f_{|_{K(a)}}$ leszűkítés invertálható, a $h := (f_{|_{K(a)}})^{-1}$ lokális inverzfüggvény folytonosan differenciálható, és
	\[
	h'(x) = \big( f'(h(x)) \big)^{-1} \quad (x \in \mathcal{D}_h).
	\]
};
	\newpage
\section{Vizsgakérdés}
\begin{quote}
	\textit{Feltételes szélsőérték, szükséges, ill. elégséges feltétel (a szükséges feltétel bizonyítása).}
\end{quote}

Legyen $1 \leq n, \, m \in \N, \, \emptyset \neq U \subset \R^n,$ és
\[
	f : U \to \R, \, g = (g_1, \, \dots, \, g_m) : U \to \R^m.
\]
Azt mondjuk, hogy az $f$ függvénynek a $g = 0$ \textit{feltételre vonatkozóan feltételes lokális maximuma (minimuma) van} a
\[
	c \in \{ g = 0 \} := \{ \xi \in U : g(\xi) = 0 \}
\]
pontban, ha az
\[
	\tilde{f}(\xi) := f(\xi) \quad (\xi \in \{g = 0\})
\]
függvénynek a $c$-ben lokális maximuma (minimuma) van. Feltesszük, hogy
\[
	\{ g = 0 \} \neq \emptyset.
\]
Használjuk az $f(c)$-re a \textit{feltételes lokális maximum (minimum)}, ill. \textit{szélsőérték}, továbbá $c$-re a \textit{feltételes lokális maximumhely (minimumhely)}, ill. \textit{szélsőértékhely} elnevezést is.\\

Ha tehát az $f$-nek a $c \in \{g=0\}$ helyen feltételes lokális szélsőértéke van a $g=0$ feltételre nézve, akkor egy alkalmas $K(c)$ környezettel
\[
	f(\xi) \leq f(c) \quad \big(\xi \in \{g=0\} \cap K(c)\big)
\]
(ha maximumról van szó), ill.
\[
	f(\xi) \geq f(c) \quad \big(\xi \in \{g=0\} \cap K(c)\big)
\]
(ha minimumról van szó) teljesül.

\subsection{Elsőrendű szükséges feltétel}

\tikz \node[theorem]
{
	\textbf{Tétel.} Tegyük fel, hogy $1 \leq n, \, m \in \N, \, m < n, \, \emptyset \neq U \subset \R^n$ nyílt halmaz, és $f : U \to \R, \, g : U \to \R^m$. Ha $f \in D, \, g \in C^1$, az $f$-nek a $c \in \{g=0\}$ helyen feltételes lokális szélsőértéke van a $g=0$ feltételre vonatkozóan, továbbá a $g'(c)$ Jacobi-mátrix rangja megegyezik $m$-mel, akkor létezik olyan $\lambda \in \R^m$ vektor, hogy
	\[
		\text{grad} \, (f + \lambda g)(c)=0.
	\]
};\\

A tételben szereplő $\lambda g$ függvényen a következőt értjük:
\[
(\lambda g)(\xi) := \langle \lambda, \, g(\xi) \rangle \quad (\xi \in U).
\]
Ez tehát ugyanolyan jellegű, mint a feltétel nélküli esetben, csak a szóban forgó $f$ függvény helyett (egy alkalmas $\lambda \in \R^m$ vektorral) az $F := f + \lambda g$ függvényre vonatkozóan.\\

Ez az analógia megmarad a másodrendű feltételeket illetően is.\\

\textbf{Bizonyítás.} A rangfeltétel szerint a $g'(c) \in \R^{m\times n}$ Jacobi-mátrixnak van olyan $A \in \R^{m \times m}$ részmátrixa, amelyre $\det A \neq 0$. Feltehető, hogy az $A$-t a $g'(c)$ mátrix utolsó $m$ oszlopa határozza meg, amikor is az $\R^n \equiv \R^{n-m} \times \R^{m}$ felbontást úgy képzeljük el, hogy a
\[
	\xi = (\xi_1, \, \dots, q, \xi_n) = (x, \, y) \in \R^n
\]
vektorokra
\[
	x := (\xi_1, \, \dots, \, \xi_{n-m}) \in \R^{n-m}, \, y := (\xi_{n-m+1}, \, \dots, \, \xi_n) \in \R^m.
\]
Legyen ennek megfelelően $c = (a, \, b)$. Ekkor tehát
\[
	\det \partial_2g(a, \, b) = \det A \neq 0.
\]
Mivel $g(a, \, b) = 0$, ezért alkalmazható az implicitfüggvény tétel: alkalmas
\[
	K(a) \subset \R^{n-m}, \, K(b) \subset \R^m
\]
környezettel létezik a $g$ függvény által az $(a, \, b)$ körül meghatározott
\[
	h : K(a) \to K(b)
\]
$h \in C^1$ implicitfüggvény:
\[
	\big( K(a) \times K(b) \big) \cap \{ g = 0\} = \{ (x, \, h(x)) \in U : x \in K(a)\},
\]
és
\[
	h'(x) = - \big( \partial_2g(x, \, h(x)) \big)^{-1} \cdot \partial_1g(x, \, h(x)) \quad \big( x \in K(a) \big).
\]
A feltételeink szerint egy alkalmas $K(c) \subset U$ környezettel (pl.)
\[
	f(\xi) \leq f(c) \quad \big(\xi = (x, \, y) \in K(c) \cap \{g=0\}\big).
\]
Nyilván feltehető, hogy
\[
	K(a) \times K(b) \subset K(c),
\]
ezért a
\[
	\Phi(x) := f(x, \, h(x)) \quad \big( x \in K(a) \big)
\]
függvényre $\Phi \in \R^{n-m} \to \R$ és
\[
	\Phi(x) \leq f(c) = \Phi(a) \quad \big( x \in K(a) \big).
\]
Más szóval a $\Phi$ függvénynek az $a$-ban lokális maximuma van. A $\Phi$ differenciálható, ezért
\[
	\Phi'(a) = \text{grad} \, \Phi(a) = 0.
\]
A
\[
	\p(x) := (x, \, h(x)) \quad \big(x \in K(a)\big)
\]
függvénnyel $\Phi = f \circ \p$ és $\p \in D$, valamint $I$-vel jelölve az $\R^{(n-m) \times (n-m)}$-beli egységmátrixot
\[
	\p'(x) = \begin{bmatrix}
		I\\
		h'(x)
	\end{bmatrix} \in \R^{n \times (n-m)} \quad \big(x \in K(a)\big).
\]
Következésképpen
\[
	0 = \Phi'(a) = f'(\p(a)) \cdot \p'(a) = f'(a, \, h(a)) \cdot \begin{bmatrix}
		I\\
		h'(a)
	\end{bmatrix} =
\]
\[
	f'(c) \cdot \begin{bmatrix}
		I\\
		h'(a)
	\end{bmatrix} = \partial_1f(c) + \partial_2f(c) \cdot h'(a) =
\]
\[
	\partial_1f(c) - \partial_2f(c) \cdot \big( \partial_2 f(c) \big)^{-1} \cdot \partial_1g(c) = \partial_1f(c) + \lambda \cdot \partial_1g(c),
\]
ahol
\[
	\lambda := -\partial_2f(c) \cdot \big( \partial_2g(c) \big)^{-1} \in \R^m.
\]
Tehát (a $\partial_1$ értelmezéséből)
\begin{equation}
	\partial_k f(c) + \sum_{l=1}^{m} \lambda_l \cdot \partial_k g_l(c) = 0 \quad (k = 1, \dots, n - m).
	\label{eq:eq1}
\end{equation}
A $\lambda$ vektor definíciójából "átszorzással" azt kapjuk, hogy
\[
	\partial_2f(c) + \lambda \cdot \partial_2 g(c) = 0,
\]
azaz (a $\partial_2$ definíciójából)
\begin{equation}
	\partial_jf(c) + \sum_{l=1}^m \lambda_l \cdot \partial_j g_l(c) = 0 \quad (j = n-m+1, \, \dots, \, n).
	\label{eq:eq2}
\end{equation}
A (1), (2) formulák együtt nyilván azt jelentik, hogy
\[
	\partial_kf(c) + \sum_{l=1}^m \lambda_l \cdot \partial_k g_l(c) = 0 \quad (k = 1, \, \dots, \, n),
\]
más szóval
\[
	\text{grad} \, (f + \lambda g)(c) = 0 =
\]
\[
	\Big( \partial_1f(c) + \sum_{l=1}^m \lambda_l \cdot \partial_1 g_l(c), \, \dots, \, \partial_nf(c) + \sum_{l=1}^m \lambda_l \cdot \partial_n g_l(c) \Big) = 0. 
\]
$\hfill \blacksquare$

A fentiekben az $m < n$ feltételezéssel éltünk. Ha $m = n$, akkor pl. az $g'(c) \in \R^{n \times n}$, és a $\text{rang} \, g'(c) = m = n$ \textit{rangfeltétel} jelentése az, hogy a
\[
g'(c) = \begin{bmatrix}
	\text{grad} \, g_1(a) \\
	\vdots \\
	\text{grad} \, g_n(a) \\
\end{bmatrix} \in \R^{n \times n}
\]
Jacobi-mátrix invertálható. Tehát a $\text{grad} \, g_k(a) \in \R^n \quad (k = 1, \, \dots, \, n)$ vektorok lineárisan függetlenek, más szóval bázist alkotnak az $\R^n$-ben. Így (egyértelműen) léteznek olyan $\lambda_j \in \R \quad (j = 1, \, \dots, \, n)$ számok, amelyekkel
\[
- \text{grad} \, f(c) = \sum_{j=1}^n \lambda_j \, \text{grad} \, g_j(c),
\]
azaz a $\lambda := (\lambda_1, \, \dots, \, \lambda_n) \in \R^n$ vektorral $\text{grad} \, (f + \lambda g)(c) = 0$. Röviden: ekkor is igaz a tétel, de az állítása triviális.\\

Legyen adott a
\[
	Q : \R^n \to \R
\]
kvadratikus alak, a $B \in \R^{m \times n}$ mátrix, és tekintsük az alábbi halmazt:
\[
	\mathcal{A}_B := \{ x \in \R^n : Bx = 0 \}.
\]
Feltesszük, hogy $m < n$, és a $B$ mátrix rangja $m$. Ekkor azt mondjuk, hogy a $Q$ kvadratikus alak a $B$-re nézve
\begin{enumerate}
	\item \textit{feltételesen pozitív definit}, ha $Q(x) > 0$ $(0 \neq x \in \mathcal{A}_B)$;
	\item \textit{feltételesen negatív definit}, ha $Q(x) > 0$ $(0 \neq x \in \mathcal{A}_B)$;
	\item \textit{feltételesen pozitív szemidefinit}, ha $Q(x) \geq 0$ $(x \in \mathcal{A}_B)$;
	\item \textit{feltételesen negatív szemidefinit}, ha $Q(x) \leq 0$ $(x \in \mathcal{A}_B)$.
\end{enumerate}

\subsection{Másodrendű elégséges feltétel}

\tikz \node[theorem]
{
	\textbf{Tétel.} Az $n, \, m \in \N, 1 \leq m < n$ paraméterek mellett legyen adott az $\emptyset \neq U \subset \R^n$ nyílt halmaz, és tekintsük az $f : U \to \R, \, g : U \to \R^m$ függvényeket. Feltesszük, hogy $f, \, g \in D^2, \, c \in \{g=0\}$, a $g'(c)$ Jacobi-mátrix rangja $m$, továbbá valamilyen $\lambda \in \R^m$ vektorral az $F := f + \lambda g$ függvényre
	\begin{enumerate}
		\item $\text{grad} \, F(c) = 0$;
		\item A $Q_c^F$ kvadratikus alak a $g'(c)$ mátrixra nézve feltételesen pozitív (negatív) definit.
	\end{enumerate}
	Ekkor az $f$-nek a $c$-ben a $g=0$ feltételre vonatkozóan feltételes lokális minimuma (maximuma) van.
};\\

Idézzük fel, hogy
\[
	Q_c^F(x) := \langle F''(c) \cdot x, \, x \rangle \quad (x \in \R^n).
\]

\subsection{Másodrendű szükséges feltétel}

\tikz \node[theorem]
{
	\textbf{Tétel.} Az $n, \, m \in \N, \, 1 \leq m < n$ paraméterek mellett legyen adott az $\emptyset \neq U \subset \R^n$ nyílt halmaz, és $f : U \to \R, \, g : U \to \R^m$, $f, \, g \in D^2$ függvények. Tegyük fel, hogy valamilyen $c \in \{ g=0 \}$ helyen $f$-nek lokális minimuma (maximuma) van a $\{g=0\}$ feltételre vonatkozóan, a $g'(c)$ Jacobi-mátrix rangja megegyezik $m$-mel. Ekkor létezik olyan $\lambda \in \R^m$, hogy az $F := f + \lambda g$ függvényre az alábbiak teljesülnek:
	\begin{enumerate}
		\item $\text{grad} \, F(c) = 0$;
		\item a $Q_c^F$ kvadratikus alak a $g'(c)$ mátrixra nézve feltételesen pozitív (negatív) szemidefinit.
	\end{enumerate}
};
	\newpage
\section{Vizsgakérdés}
\begin{quote}
	\textit{A differenciálegyenlet (rendszer) fogalma. Kezdetiérték-probléma (Cauchy-feladat). Egzakt egyenlet, szeparábilis egyenlet, a rakéta emelkedési idejének a kiszámítása.}
\end{quote}

\subsection{Közönséges differenciálegyenletek}
Legyen $0 < n \in \N, \, I \subset \R, \, \Omega \subset \R^n$ egy-egy nyílt intervallum. Tegyük fel, hogy az
\[
f : I \times \Omega \to \R^n
\]    
függvény folytonos, és tűzzük ki az alábbi feladat megoldását:

\begin{quote}
	határozzunk meg olyan $\varphi \in I \to \Omega$ függvényt, amelyre igazak a következő állítások:
	\begin{enumerate}
		\item $\Dp$ nyílt intervallum;
		\item $\varphi \in D$;
		\item $\varphi'(x) = f(x, \, \varphi(x)) \quad (x \in \Dp)$.
	\end{enumerate}
\end{quote}

A most megfogalmazott feladatot \textit{explicit elsőrendű közönséges differenciálegyenletnek} (röviden \textit{differenciálegyenletnek}) fogjuk nevezni, és a \textit{d.e.} rövidítéssel idézni.\\

Ha adottak a $\tau \in I, \, \xi \in \Omega$ elemek, akkor a fenti $\varphi$ függvény $1. \, 2.$ és 3. mellett tegyen eleget a
\begin{enumerate}[start=4]
	\item $\tau \in \Dp$ és $\varphi(\tau) = \xi$
\end{enumerate}
kikötésnek is. Az így "kibővített" feladatot \textit{kezdetiérték-problémának} (vagy röviden \textit{Cauchy-feladatnak}) nevezzük, és a továbbiakban minderre a \textit{k.é.p.} rövidítést fogjuk használni. Az 1., 2., 3. feltételeknek (ill. az 1., 2., 3., 4. feltételeknek) eleget tevő bármelyik $\varphi$ függvényt a \textit{d.e.} (ill. a \textit{k.é.p.}) \textit{megoldásának} nevezzük. A fenti definícióban szereplő $f$ függvény az illető \textit{d.e.} ún. \textit{jobb oldala}.

\subsection{Teljes megoldás}
Azt mondjuk, hogy a szóban forgó \textit{k.é.p. egyértelműen oldható meg}, ha tetszőleges $\varphi, \, \psi$ megoldásai esetén
\[
\varphi(x) = \psi(x) \quad (x \in \Dp \cap \mathcal{D}_\psi).
\]
(Mivel $\tau \in \Dp \cap \mathcal{D}_\phi$, ezért $\Dp \cap \mathcal{D}_\phi$ egy ($\tau$-t tartalmazó) nyílt intervallum.) Legyen ekkor $\mathcal{M}$ a szóban forgó \textit{k.é.p.} megoldásainak a halmaza és
\[
J := \bigcup_{\varphi \in \mathcal{M}} \Dp.
\]
Ez egy $\tau$-t tartalmazó nyílt intervallum és $J \subset I$. Az egyértelmű megoldhatóság értelmezése miatt definiálhatjuk a
\[
\Phi : J \to \Omega
\]
függvényt az alábbiak szerint:
\[
\Phi(x) := \varphi(x) \quad (\varphi \in \mathcal{M}, \, x \in \Dp).
\]
Nyilvánvaló, hogy $\Phi(\tau) = \xi, \, \Phi \in D$ és
\[
\Phi'(x) = f(x, \, \Phi(x)) \quad (x \in J).
\]
Ez azt jelenti, hogy $\Phi \in \mathcal{M}$, és (ld. a $\mathcal{D}_\Phi = J$ definícióját) bármelyik $\varphi \in \mathcal{M}$ esetén
\[
\varphi(x) = \Phi(x) \quad (x \in \Dp),
\]
röviden $\varphi = \Phi_{|_{\Dp}}$.\\

A $\Phi$ függvényt a kezdetiérték-probléma \textit{teljes megoldásának} nevezzük.

\subsection{Szeparábilis differenciálegyenlet}
Legyen $n := 1$, továbbá az $I, \, J \subset \R$ nyílt intervallumokkal és a
\[
g : I \to \R, \, h : J \to \R \, \backslash \, \{0\}
\]
folytonos függvényekkel
\[
f(x, \, y) := g(x) \cdot h(y) \quad ((x, \, y) \in I \times J).
\]
A $\varphi \in I \to J$ megoldásra tehát
\[
\varphi'(t) = g(t) \cdot h(\varphi(t)) \quad (t \in \Dp).
\]
Legyenek még adottak a $\tau \in I, \, \xi \in J$ számok, amikor is
\[
\tau \in \Dp, \, \varphi(\tau) = \xi
\]
(kezdetiérték-probléma).\\

\tikz \node[theorem]
{
	\textbf{Tétel.} Tetszőleges szeparábilis differenciálegyenletre vonatkozó kezdetiérték-probléma megoldható, és bármilyen $\varphi, \, \psi$ megoldásaira
	\[
	\varphi(t) = \psi(t) \quad (t \in \D \cap \mathcal{D}_\psi).
	\]
};\\

\textbf{Bizonyítás.} Mivel a $h$ függvény sehol sem nulla, ezért egy $\varphi$ megoldásra
\[
\frac{\varphi'(t)}{h(\varphi(t))} = g(t) \quad (t \in \Dp).
\]
A $g : I \to \R$ is, és az $1 / h : J \to \R$ is egy-egy nyílt intervallumon értelmezett folytonos függvény, így léteznek a
\[
G : I \to \R, \, H : J \to \R
\]
primitív függvényeik: $G' = g$ és $H' = 1/h$. Az összetett függvény deriválásával kapcsolatos tétel szerint
\[
\frac{\varphi'(t)}{h(\varphi(t))} = (H \circ \varphi)'(t) = g(t) = G'(t) \quad (t \in \Dp),
\]
azaz
\[
(H \circ \varphi - G)'(t) = 0 \quad (t \in \Dp).
\]
Tehát (mivel a $\Dp$ is egy nyílt intervallum) van olyan $c \in \R$, hogy
\[
H(\varphi(t)) - G(t) = c \quad (t \in \Dp).
\]
Az $1/h$ függvény nyilván nem vesz fel $0$-t a $J$ intervallum egyetlen pontjában sem, így ugyanez igaz a $H'$ függvényre is. A deriváltfüggvény Darboux-tulajdonsága miatt tehát a $H'$ állandó előjelű. Következésképpen a $H$ függvény szigorúan monoton függvény, amiért invertálható. A $H^{-1}$ inverz függvény segítségével ezért azt kapjuk, hogy
\[
\varphi(t) = H^{-1}(G(t) + c) \quad (t \in \Dp).
\]
Ha $\tau \in I, \, \xi \in J$, és a $\varphi$ megoldás eleget tesz a $\varphi(\tau) = \xi$ kezdeti feltételnek is, akkor
\[
\xi = H^{-1}(G(\tau) + c),
\]
azaz
\[
c = H(\xi) - G(\tau).
\]
Így
\[
\varphi(t) = H^{-1}\big(G(t) + H(\xi) - G(\tau)\big)  \quad (t \in \Dp).
\]
Ha a $G, \, H$ helyett más primitív függvényeket választunk (legyenek ezek $\tilde{G}, \, \tilde{H}$), akkor alkalmas $\alpha, \, \beta \in \R$ konstansokkal
\[
\tilde{G} = G + \alpha, \, \tilde{H} = H + \beta,
\]
és
\[
\tilde{H}(\varphi(t)) - \tilde{G}(t) = H(\varphi(t)) - G(t) + \beta - \alpha = \tilde{c} \quad (t \in \Dp)
\]
adódik valamilyen $\tilde{c} \in \R$ konstanssal. Ezért
\[
\varphi(t) = H^{-1}(G(t) + \tilde{c} - \beta + \alpha) \quad (t \in \Dp),
\]
ahol (a $t := \tau$ helyettesítés után)
\[
H(\xi) - G(\tau) = \tilde{c} - \beta + \alpha,
\]
amiből megint csak
\[
\varphi(t) = H^{-1}(G(t) + H(\xi) - G(\tau)) \quad (t \in \Dp)
\]
következik. Ez azt jelenti, hogy a fentiekben mindegy, hogy melyik $G, \, H$ primitív függvényekből indulunk ki. Más szóval, ha a $\psi$ függvény is megoldása a vizsgált kezdetiérték-problémának, akkor
\[
\psi(t) = H^{-1}(G(t) + H(\xi) - G(\tau)) \quad (t \in \mathcal{D}_\psi).
\]
Mivel a $\Dp, \, \mathcal{D}_\psi$ értelmezési tartományok mindegyike egy-egy $\tau$-t tartalmazó nyílt intervallum, ezért $\Dp \cap \mathcal{D}_\psi$ is ilyen intervallum, és
\[
\psi(t) = \varphi(t) \quad (t \in \Dp \cap \mathcal{D}_\psi).
\]
Elegendő már csak azt belátnunk, hogy van megoldás. Tekintsük ehhez azokat a $G, \, H$ primitív függvényeket, amelyekre
\[
H(\xi) = G(\tau) = 0,
\]
és legyen
\[
F(x, \, y) := H(y) - G(x) \quad (x \in I, \, y \in J).
\]
Ekkor az
\[
F : I \times J \to \R
\]
függvényre léteznek és folytonosak a
\[
\partial_1 F(x, \, y) = -G'(x) = -g(x) \quad (x \in I, \, y \in J),
\]
\[
\partial_2 F(x, \, y) = H'(y) = \frac{1}{h(y)} \quad (x \in I, \, y \in J)
\]
parciális deriváltfüggvények. Ez azt jelenti, hogy az $F$ függvény folytonosan differenciálható,
\[
F(\tau, \, \xi) = H(\xi) - G(\tau) = 0,
\]
továbbá
\[
\partial_2 F(\tau, \, \xi) = H'(\xi) = \frac{1}{h(\xi)} \neq 0.
\]
Ezért az $F$-re alkalmazható az implicitfüggvény-tétel, miszerint alkalmas $K(\tau) \subset I, \, K(\xi) \subset J$ környezetekkel létezik az $F$ által a $(\tau, \, \xi)$ körül meghatározott
\[
\varphi : K(\tau) \to K(\xi)
\]
folytonosan differenciálható implicitfüggvény, amire $\varphi(\tau) = \xi$ és
\[
\varphi'(t) = -\frac{\partial_1 F(t, \, \varphi(t))}{\partial_2 F(t, \, \varphi(t))} = g(t) \cdot h(\varphi(t)) \quad (t \in K(\tau)).
\]
Röviden: a $\varphi$ implicitfüggvény megoldása a szóban forgó kezdetiérték-problémának.
$\hfill \blacksquare$

\subsection{Rakéta emelkedési ideje}
BEFEJEZNI


\subsection{Egzakt differenciálegyenlet}
Speciálisan legyen $n := 1$, és az $I, \, J \subset \R$ nyílt intervallumok, valamint a
\[
	g : I \times J \to \R \text{ és } h : I \times J \to \R \, \backslash \, \{0\}
\]
folytonos függvényekkel
\[
	f(x, \, y) := - \frac{g(x, \, y)}{h(x, \, y)} \quad \big( (x, \, y) \in I \times J \big).
\]
Ekkor a fenti minden $\p$ megoldásra
\[
	\p'(x) = - \frac{g(x, \, \p(x))}{h(x, \, \p(x))} \quad (x \in \Dp).
\]
Azt mondjuk, hogy az így kapott d.e. \textit{egzakt differenciálegyenlet}, ha az
\[
	I \times J \ni (x, \, y) \mapsto \big( g(x, \, y), \, h(x, \, y) \big) \in \R^2
\]
leképezésnek van primitív függvénye. Ez utóbbi követelmény azt jelenti, hogy egy alkalmas differenciálható
\[
	G : I \times J \to \R
\]
függvénnyel 
\[
	\text{grad} \, G = (\partial_1 G, \, \partial_2 G) = (g, \, h).
\]
Ha $\tau \in I, \, \xi \in J$ és a $\p$ függvénytől azt is elvárjuk, hogy
\[
	\tau \in \Dp, \, \p(\tau) = \xi,
\]
akkor igaz az\\

\tikz \node[theorem]
{
	\textbf{Tétel}. Tetszőleges egzakt differenciálegyenletre vonatkozó minden
	kezdetiérték-probléma megoldható, és ennek bármilyen $\p, \, \psi$ megoldásaira
	\[
		\p(t) = \psi(t) \quad (t \in \Dpps).
	\]
};\\

\textbf{Bizonyítás.} Valóban, $0 \not \in \mathcal{R}_h$ miatt a feltételezett $\p$ megoldásra
\[
	g(x, \, \p(x)) + h(x, \, \p(x)) \cdot \p'(x) = 0 \quad (x \in \Dp).
\]
Ha van ilyen $\p$ függvény, akkor az
\[
	F(x) := G(x, \, \p(x)) \quad (x \in \Dp)
\]
egyváltozós valós függvény differenciálható, és tetszőleges $x \in \Dp$ helyen
\[
	F'(x) = \big\langle \text{grad} \, G(x, \, \p(x)), \, (1, \, \p'(x)) \big\rangle =
\]
\[
	g(x, \, \p(x)) + h(x, \, \p(x)) \cdot \p'(x) = 0.
\]
A $\mathcal{D}_F = \Dp$ halmaz nyílt intervallum, ezért az $F$ konstans függvény, azaz létezik olyan $c \in \R$, amellyel
\[
	G(x, \, \p(x)) = c \quad (x \in \Dp).
\]
Mivel $\p(\tau) = \xi$, ezért
\[
	c = G(\tau, \, \xi).
\]
A $G$-ről feltehetjük, hogy $G(\tau, \, \xi) = 0$, ezért a szóban forgó k.é.p. $\p$ megoldása eleget tesz a 
\[
	G(x, \, \p(x)) \quad (x \in \Dp)
\]
egyenletnek.\\

Világos, hogy a $\p$ nem más, mint egy, a $G$ által meghatározott implicitfüggvény. Más szóval a szóban forgó k.é.p. minden megoldása (ha létezik) a fenti implicitfüggvény-egyenletből határozható meg.\\

Ugyanakkor a feltételek alapján $G \in C^1, \, G(\tau, \, \xi) = 0$, továbbá
\[
	\partial_2G(\tau, \, \xi) = h(\tau, \, \xi) \neq 0,
\]
ezért a $G$-re (a $(\tau, \, \xi)$ helyen) teljesülnek az implicitfüggvény-tétel feltételei. Következésképpen van olyan differenciálható
\[
	\psi \in I \to J
\]
(implicit)függvény, amelyre $\mathcal{D}_\psi \subset I$ nyílt intervallum,
\[
	\tau \in \mathcal{D}_\psi, \, G(x, \, \psi(x)) = 0 \quad (x \in \mathcal{D}_\psi), \, \psi(\tau) = \xi,
\]
és
\[
	\psi'(x) = - \frac{\partial_1G(x, \, \psi(x))}{\partial_2G(x, \, \psi(x))} = - \frac{g(x, \, \psi(x))}{h(x, \, \psi(x))} \quad (x \in \mathcal{D}_\psi).
\]
$\hfill \blacksquare$
	\newpage
\section{Vizsgakérdés}
\begin{quote}
	\textit{Lineáris differenciálegyenlet. Az állandók variálásának módszere. A radioaktív bomlás felezési idejének meghatározása.}
\end{quote}

\subsection{Lineáris differenciálegyenlet}
Legyen most $n := 1$ és az $I \subset \R$ egy nyílt intervallum, valamint a
\[
g, \, h : I \to \R
\]
folytonos függvények segítségével
\[
f(x, \, y) := g(x) \cdot y + h(x) \quad \B{(x, \, y) \in I \times \R}.
\]
Ekkor
\[
\varphi'(t) = g(t) \cdot \varphi(t) + h(t) \quad (t \in \Dp).
\]
Ezt a feladatot \textit{lineáris differenciálegyenletnek} nevezzük. \\

Ha valamilyen $\tau \in I, \, \xi \in \R$ mellett
\[
\tau \in \Dp, \, \varphi(\tau) = \xi,
\]
akkor az illető lineáris differenciálegyenletre vonatkozó kezdetiérték-problémáról beszélünk.\\

Tegyük fel, hogy a $\theta$ függvény is és a $\psi$ függvény is megoldása a lineáris d.e.-nek és $\mathcal{D}_\theta \cap \mathcal{D}_\psi \neq \emptyset$. Ekkor
\[
(\theta - \psi)'(t) = g(t) \cdot \big( \theta(t) - \psi(t) \big) \quad (t \in \mathcal{D}_\theta \cap \mathcal{D}_\psi).
\]
Így a $\theta - \psi$ függvény megoldása annak a lineáris d.e.-nek, amelyben $h \equiv 0$:
\[
\varphi'(t) = g(t) \cdot \varphi(t) \quad (t \in \Dp).
\]
Ez utóbbi feladatot \textit{homogén lineáris differenciálegyenletnek} fogjuk nevezni. (Ennek megfelelően a szóban forgó lineáris differenciálegyenlet \textit{inhomogén}, ha a benne szereplő $h$ függvény vesz fel 0-tól különböző értéket is.)\\

\tikz \node[theorem]
{
	\textbf{Tétel.} Minden lineáris differenciálegyenletre vonatkozó kezdetiérték-probléma megoldható, és tetszőleges $\varphi, \, \psi$ megoldásaira
	\[
	\varphi(t) = \psi(t) \quad (t \in \mathcal{D}_\varphi \cap \mathcal{D}_\psi).
	\]
};\\

\textbf{Bizonyítás.} Legyen a
\[
G : I \to \R
\]
olyan függvény, amelyik differenciálható és $G' = g$ (a $g$-re tett  feltételek miatt ilyen $G$ primitív függvény van). Ekkor a
\[
\varphi_0(t) := e^{G(t)} \quad (t \in I)
\]
(csak pozitív értékeket felvevő) függvény megoldása az előbb említett homogén lineáris differenciálegyenletnek:
\[
\varphi_0'(t) = G'(t) \cdot e^{G(t)} = g(t) \cdot \varphi_0(t) \quad (t \in I).
\]
Tegyük fel most azt, hogy a
\[
\chi \in I \to \R
\]
függvény is megoldása a szóban forgó homogén lineáris differenciálegyenletnek:
\[
\chi'(t) = g(t) \cdot \chi(t) \quad (t \in \mathcal{D}_\chi).
\]
Ekkor a differenciálható
\[
\frac{\chi}{\varphi_0} : \mathcal{D}_\chi \to \R
\]
függvényre azt kapjuk, hogy bármelyik $t \in \mathcal{D}_\chi$ helyen
\[
\left( \frac{\chi}{\varphi_0} \right)'(t) = \frac{\chi'(t) \cdot \varphi_0(t) - \chi(t) \cdot \varphi_0'(t)}{\varphi_0^2(t)} =
\]
\[
\frac{g(t) \cdot \chi(t) \cdot \varphi_0(t) - \chi(t) \cdot g(t) \cdot \varphi_0(t)}{\varphi_0^2(t)} = 0,
\]
azaz (lévén a $\mathcal{D}_\chi$ nyílt intervallum) egy alkalmas $c \in \R$ számmal
\[
\frac{\chi(t)}{\varphi_0(t)} = c \quad (t \in \mathcal{D}_\chi).
\]
Más szóval, az illető homogén lineáris differenciálegyenlet bármelyik
\[
\chi \in I \to \R
\]
megoldása a következő alakú:
\[
\chi(t) = c \cdot \varphi_0(t) \quad (t \in \mathcal{D}_\chi),
\]
ahol $c \in \R$. Nyilván minden ilyen $\chi$ függvény -- könnyen ellenőrizhető módon -- megoldása a mondott homogén lineáris differenciálegyenletnek.\\

Ha tehát a fenti (inhomogén) lineáris differenciálegyenletnek a $\theta$ függvény is és a $\psi$ függvény is megoldása és $\mathcal{D}_\theta \cap \mathcal{D}_\psi \neq \emptyset$, akkor egy alkalmas $c \in \R$ együtthatóval
\[
\theta(t) - \psi(t) = c \cdot \varphi_0(t) \quad (t \in \mathcal{D}_\theta \cap \mathcal{D}_\psi).
\]
Mutassuk meg, hogy van olyan differenciálható
\[
m : I \to \R
\]
függvény, hogy az $m \cdot \varphi_0$ függvény megoldása a most vizsgált (inhomogén) lineáris differenciálegyenletnek (\textit{az állandók variálásának módszere}). Ehhez azt kell "biztosítani", hogy
\[
(m \cdot \varphi_0)' = g \cdot m \cdot \varphi_0 + h,
\]
azaz
\[
m' \cdot \varphi_0 + m \cdot \varphi_0' = m' \cdot \varphi_0 + m \cdot g \cdot \varphi_0 = g \cdot m \cdot \varphi_0 + h.
\]
Innen szükséges feltételként az adódik az $m$-re, hogy
\[
m' = \frac{h}{\varphi_0}.
\]
Ilyen $m$ függvény valóban létezik, mivel a
\[
\frac{h}{\varphi_0} : I \to \R
\]
folytonos leképezésnek van primitív függvénye. Továbbá -- az előbbi rövid számolás "megfordításából" -- azt is beláthatjuk, hogy a $h / \varphi_0$ függvény bármelyik $m$ primitív függvényét is véve, az $m \cdot \varphi_0$ függvény megoldása a lineáris differenciálegyenletünknek.\\

Összefoglalva az eddigieket azt mondhatjuk, hogy a fenti lineáris differenciálegyenletnek van megoldása, és tetszőleges $\varphi \in I \to \R$ megoldása
\[
\varphi(t) = m(t) \cdot \varphi_0(t) + c \cdot \varphi_0(t) \quad (t \in \Dp)
\]
alakú, ahol $m$ egy tetszőleges primitív függvénye a $h / \varphi_0$ függvénynek. Sőt, az is kiderül, hogy akármilyen $c \in \R$ és $J \subset I$ nyílt intervallum esetén a
\[
\varphi(t) := m(t) \cdot \varphi_0(t) + c \cdot \varphi_0(t) \quad (t \in J)
\]
függvény megoldás. Ez  megint csak egyszerű behelyettesítéssel ellenőrizhető:
\[
\varphi'(t) = m'(t) \cdot \varphi_0(t) + (c + m(t)) \cdot \varphi_0'(t) =
\]
\[
\frac{h(t)}{\varphi_0(t)} \cdot \varphi_0(t) + (c + m(t)) \cdot g(t) \cdot \varphi_0(t) = g(t) \cdot \varphi(t) + h(t) \quad (t \in J).
\]
Speciálisan az "egész" $I$ intervallumon értelmezett
\[
\psi_c(t) := m(t) \cdot \varphi_0(t) + c \cdot \varphi_0(t) \quad (c \in \R, \, t \in I)
\]
megoldások olyanok, hogy bármelyik $\varphi$ megoldásra egy alkalmas $c \in \R$ mellett
\[
\varphi(t) = \psi_c(t) \quad (t \in \Dp),
\]
azaz a $J := \Dp$ jelöléssel $\varphi = \psi_{c_{|_J}}$.\\

Ha $\tau \in I, \, \xi \in \R$, és a $\varphi(\tau) = \xi$ kezdetiérték-feladatot kell megoldanunk, akkor a
\[
c := \frac{\xi - m(\tau) \cdot \varphi_0(\tau)}{\varphi_0(\tau)}
\]
választással a szóban forgó kezdetiérték-probléma
\[
\psi_c : I \to \R
\]
megoldását kapjuk. Mivel a fentiek alapján a szóban forgó k.é.p. minden $\varphi, \, \psi$ megoldására $\varphi = \psi_{c_{|_{\Dp}}}$ és $\psi = \psi_{c_{|_{\mathcal{D}_\psi}}}$, ezért egyúttal az is teljesül, hogy
\[
\varphi(t) = \psi(t) \quad (t \in \Dp \cap \mathcal{D}_\psi).
\]
$\hfill \blacksquare$

A tétel bizonyításából a következők is kiderültek: legyen
\[
	\mathcal{M} := \{ \p : I \to \R : \p \in D, \, \p'(t) = g(t) \cdot \p(t) + h(t) \quad (t \in I) \},
\]
\[
	\mathcal{M}_h := \{ \p : I \to \R : \p \in D, \, \p'(t) = g(t) \cdot \p(t) \quad (t \in I) \}.
\]
Ekkor
\[
	\mathcal{M}_h = \{ c \cdot \p_0 : c \in \R \}
\]
(azaz algebrai nyelven mondva az $\mathcal{M}_h$ egy 1 dimenziós vektortér), és
\[
	\mathcal{M} = m \cdot \p_0 + \mathcal{M}_h := \{ \p + m \cdot \p_0 : \p \in \mathcal{M}_h \}.
\]
Itt $m \cdot \p_0$ helyébe bármelyik $\psi \in \mathcal{M}$ (ún. \textit{partikuláris megoldás}) írható, így
\[
	\mathcal{M} = \psi + \mathcal{M}_h = \{ \p + \psi : \p \in \mathcal{M}_h \}.
\]

\subsection{Radioaktív bomlás}
\textit{Radioaktív anyag bomlik, a bomlási sebesség egyenesen arányos a még fel nem bomlott anyag mennyiségével. A bomlás kezdetétől számítva mennyi idő alatt bomlik el az anyag fele?}\\

Legyen $m_0$ az anyag eredeti, $\p(t)$ pedig a $t \quad (t \in \R)$ időpontban még el nem bomlott anyag mennyisége. A feladatban szereplő arányossági tényező $0 < \alpha \in \R$. Ekkor
\[
	\p'(t) = -\alpha \p(t) \quad (t \in \R),
\]
ahol $\p(0) = m_0.$ A $T$ \textit{(felezési időt)} keressük, amikor is $\p(T) = m_0 / 2$.\\

Ez egy homogén lineáris differenciálegyenlet, ahol $g \equiv -\alpha$. Ezért (pl.)
\[
	G(t) = -\alpha t \quad (t \in I).
\]
valamint
\[
	\p_0(t) = e^{- \alpha t} \quad (t \in I),
\]
ill.
\[
	\p(t)  = c e ^{- \alpha t } \quad (t \in I, \, c \in \R).
\]
Mivel
\[
	m_0 = \p(0) = c,
\]
ezért
\[
	\p(t) = m_0 e^{- \alpha t} \quad (t \in I).
\]
A $T$ definíciója alapján
\[
	\p(T) = m_0 e^{- \alpha T} = \frac{m_0}{2},
\]
azaz $e^{- \alpha T} = 1 /2$. Innen
\[
	T = \frac{\ln 2}{\alpha}.
\]
	\newpage
\section{Vizsgakérdés}
\begin{quote}
	\textit{Lipschitz-feltétel. A Picard-Lindelöf-féle egzisztencia-tétel (a fixpont-tétel alkalmazása). A k.é.p. megoldásának az egyértelműsége, unicitási tétel (bizonyítás nélkül).}
\end{quote}

\subsection{Lipschitz-feltétel}
Az előzőekben definiáltuk a \textit{k.é.p.} fogalmát: határozzunk meg olyan $\varphi \in I \to \Omega$ függvényt, amelyre (a korábban bevezetett jelölésekkel) igazak a következő állítások:
\begin{enumerate}
	\item $\mathcal{D}_\varphi$ nyílt intervallum;
	\item $\varphi \in D$;
	\item $\varphi'(x) = f(x, \, \varphi(x)) \quad (x \in \Dp)$;
	\item adott $\tau \in I, \, \xi \in \Omega$ mellett $\tau \in \Dp$ és $\varphi(\tau) = \xi$.
\end{enumerate}
Értelmeztünk a megoldást, az egyértelműen való megoldhatóságot, a teljes megoldást. Speciális esetekben meg is oldottuk a gyakorlat számára is fontos kezdetiérték-problémákat. A továbbiakban megmutatjuk, hogy bizonyos feltételek mellett egy \textit{k.é.p.} mindig megoldható (egzisztenciatétel).\\

Legyenek tehát $0 < n \in \N$ mellett az $I \subset \R, \, \Omega \subset \R^n$ nyílt intervallumok, az
\[
f : I \times \Omega \to \R^n
\]
függvény pedig legyen folytonos. A $\tau \in I, \, \xi \in \Omega$ esetén keressük a fenti differenciálható $\varphi \in I \to \Omega$ függvényt. Az $f$ függvényről feltesszük, hogy minden kompakt $\emptyset \neq Q \subset \Omega$ halmazhoz létezik olyan $L_Q \geq 0$ konstans, amellyel
\[
\| f(t, \, y) - f(t, \, z) \|_\infty \leq L_Q \cdot \| y - z \|_\infty \quad (t \in I, \, y, \, z \in Q).
\]
Ekkor azt mondjuk, hogy az $f$ (a \textit{d.e.} jobb oldala) eleget tesz a \textit{Lipschitz-feltételnek}.

\subsection{Egzisztenciatétel}
\begin{center}
	\tikz \node[theorem]
	{
		\textbf{Tétel (Picard-Lindelöf).} Tegyük fel, hogy egy differenciálegyenlet jobb oldala eleget tesz a Lipschitz-feltételnek. Ekkor a szóban forgó differenciálegyenletre vonatkozó tetszőleges kezdetiérték-probléma megoldható.
	};    
\end{center}
\textbf{Bizonyítás (vázlat).} Legyenek a $\delta_1, \, \delta_2 > 0$ olyan számok, hogy
\[
I_* := [\tau - \delta_1, \, \tau + \delta_2] \subset I,
\]
és tekintsük az alábbi függvényhalmazt:
\[
\mathcal{F} := \{ \psi : I_* \to \Omega : \psi \in C \}.
\]
Az $\mathcal{F}$ halmaz a
\[
\rho(\phi, \, \psi) := \max \big\{ \| \phi(x) - \psi(x)\|_\infty : x \in I_* \big\} \quad (\phi, \, \psi \in \mathcal{F})
\]
távolságfüggvénnyel teljes metrikus tér. Ha $\mathcal{X}$ jelöli a
\[
g : I_* \to \R^n
\]
függvények összességét, akkor definiáljuk a
\[
T : \mathcal{F} \to \mathcal{X}
\]
leképezést a következőképpen:
\[
T\psi(x) := \xi + \int_\tau^x f(t, \, \psi(t)) \, dt \in \R^n \quad (\psi \in \mathcal{F}, \, x \in I_*).
\]
Tehát az $f$ függvény koordinátafüggvényeit a "szokásos" $f_1, \, \dots, \, f_n$ szimbólumokkal jelölve, a $\psi$, $f$ függvények (és egyúttal az $f_i$-k) folytonossága miatt
\[
I_* \ni t \mapsto f_i(t, \, \psi(t)) \in \R \quad (i = 1, \, \dots, \, n)
\]
függvények folytonosak. Következőképpen (minden $x \in I_*$ esetén) van értelme a
\[
d_i := \int_\tau^x f_i(t, \, \psi(t)) \, dt \quad (i = 1, \, \dots, \, n)
\]
integráloknak, és így a
\[
\xi + \int_\tau^x f(t, \, \psi(t)) \, dt := (\xi_1 + d_1, \, \dots, \, \xi_n + d_n) \in \R^n
\]
"integrálvektoroknak". Továbbá az integrálfüggvények tulajdonságai miatt a $T\psi$ függvény folytonos, minden $x \in (\tau - \delta_1, \, \tau + \delta_2)$ helyen differenciálható, és
\[
(T\psi)'(x) = f(x, \, \psi(x)).
\]
Belátjuk, hogy az $I_*$ alkalmas megválasztásával minden $\psi \in \mathcal{F}$ függvényre $T\psi \in \mathcal{F}$, azaz ekkor
\[
T : \mathcal{F} \to \mathcal{F}.
\]
Ehhez azt kell biztosítani, hogy
\[
\xi + \int_\tau^x f(t, \, \psi(t)) \, dt \in \Omega \quad (x \in I_*)
\]
teljesüljön. Válasszuk ehhez először is a $\mu > 0$ számot úgy, hogy a
\[
K_\mu := \{ y \in \R^n : \| y - \xi\|_\infty \leq \mu \} \subset \Omega
\]
tartalmazás fennáljon (ilyen $\mu$ az $\Omega$ nyíltsága miatt létezik), és legyen
\[
M := \max\{ \| f(x, \, y)\|_\infty : x\in I_*, \, y \in K_\mu \}
\]
(ami meg az $f$ folytonossága és a Weierstrass-tétel miatt létezik, ti. az $I_* \times K_\mu$ halmaz kompakt). A jelzett $T\psi \in \mathcal{F}$ tartalmazás nyilván teljesül, ha
\[
\max \left\{  \left| \int_\tau^x f_i(t, \, \psi(t)) \, dt \right| : i = 1, \, \dots, \, n \right\} \leq \mu \quad (x \in I_*).
\]
Módosítsuk most már az $\mathcal{F}$ definícióját úgy, hogy
\[
\mathcal{F} := \{ \psi : I_* \to K_\mu : \psi \in C \}.
\]
Ekkor az előbbi maximum becsülhető $M \cdot \delta$-val, ahol
\[
\delta := \max\{\delta_1, \, \delta_2\}.
\]
Így $M \cdot \delta \leq \mu$ esetén a fenti $T\psi$ is $\mathcal{F}$-beli. (Ha a kiindulásul választott $\delta_1, \, \delta_2$-re $M \cdot \delta > \mu$, akkor írjunk a $\delta_1, \, \delta_2$ helyébe olyan "új" $0 < \tilde{\delta_1}, \, \tilde{\delta_2}$-t, hogy
\[
[\tau - \tilde{\delta_1}, \, \tau + \tilde{\delta_2}] \subset [\tau - \delta_1, \, \tau + \delta_2]
\]
és
\[
M \cdot \max \{ \tilde{\delta_1}, \, \tilde{\delta_2} \} \leq \mu
\]
legyen. Az $I_*$ helyett az $\tilde{I_*} := [\tau - \tilde{\delta_1}, \, \tau + \tilde{\delta_2}]$ intervallummal az "új" $M$ az előzőnél legfeljebb kisebb lesz, így az
\[
M \cdot \max \{ \tilde{\delta_1}, \, \tilde{\delta_2}\} \leq \mu
\]
becslés nem "romlik" el.)
Ezzel értelmeztünk egy $T : \mathcal{F} \to \mathcal{F}$ leképezést, amelyre tetszőleges $\phi, \, \psi \in \mathcal{F}$ mellett
\[
\rho(T\psi, \, T\phi) = \max \big\{ \| T\psi(x) - T\phi(x)\|_\infty : x \in I_* \big\} =
\]
\[
\max \left\{  \max \left\{  \left| \int_\tau^x (f_i(t, \, \psi(t)) - f_i(t, \, \phi(t)) \, dt) \right| : i = 1, \, \dots, \, n \right\} : x \in I_* \right\} \leq
\]
\[
\max \left\{  \left| \int_\tau^x \max \{ |f_i(t, \, \psi(t)) - f_i(t, \, \phi(t))| : i = 1, \, \dots, \, n \} \, dt \right| : x \in I_* \right\} =
\]
\[
\max \left\{ \left| \int_\tau^x \| f(t, \, \psi(t)) - f(t, \, \phi(t))\|_\infty \, dt \right| : x \in I_* \right\}.
\]
A Lipschitz-feltétel miatt a $Q := K_\mu$ (nyilván kompakt) halmazhoz van olyan $L_Q \geq 0$ konstans, amellyel
\[
\| f(t, \, y) - f(t, \, z)\|_\infty \leq L_Q \cdot \| y-z \|_\infty \quad (t \in I, \, y, \, z \in Q),
\]
speciálisan
\[
\| f(t, \, \psi(t)) - f(t, \, \phi(t))\|_\infty \leq
\]
\[
L_Q \cdot \| \psi(t) - \phi(t) \|_\infty \leq L_Q \cdot \rho(\psi, \, \phi) \quad (t \in I_*).
\]
Ezért
\[
\rho(T\psi, \, T\phi) \leq L_Q \cdot \delta \cdot \rho(\psi, \, \phi).
\]
Tehát a $T$ leképezés
\[
L_Q \cdot \max \{\delta_1, \, \delta_2\} < 1
\]
esetén kontrakció. Válasszuk így a $\delta_1, \, \delta_2$-t, (ezt - az "eddigi" $I_*$-ot legfeljebb újra leszűkítve - megtehetjük), és alkalmazzuk a fixpont-tételt, miszerint van olyan $\phi \in \mathcal{F}$, amelyre
\[
T\phi = \phi.
\]
Legyen
\[
\varphi(x) := \phi(x) \quad \big( x \in (\tau - \delta_1, \, \tau + \delta_2) \big).
\]
A $T$ definíciója szerint
\[
\varphi(x) = \xi + \int_\tau^x f(t, \, \varphi(t)) \, dt \quad \big( x \in (\tau - \delta_1, \, \tau + \delta_2) \big).
\]
Ez azt jelenti, hogy a $\varphi$ függvény egy folytonos függvény integrálfüggvénye, ezért $\varphi \in D$ és
\[
\varphi'(x) = f(x, \, \varphi(x)) \quad \big( x \in (\tau - \delta_1, \, \tau + \delta_2) \big).
\]
Világos, hogy a $\varphi(\tau) = \xi$, más szóval a $\varphi$ megoldása a szóban forgó kezdetiérték-problémának. $\blacksquare$\\

A fenti Picard-Lindelöf-egzisztenciatételben szereplő Lipschitz-feltétel nem csupán a kezdetiérték-problémák megoldhatóságát, hanem azok egyértelmű megoldhatóságát is biztosítja.\\

\tikz \node[theorem]
{
	\textbf{Tétel.} Az előző tétel feltételei mellett az abban szereplő tetszőleges kezdetiérték-probléma egyértelműen oldható meg, azaz bármely $\varphi, \, \psi$ megoldásaira
	\[
	\varphi(t) = \psi(t) \quad (t \in \Dp \cap \mathcal{D}_\psi).
	\]
};\\

Legyen az $f : I \times \Omega \to \R^n$ jobb oldal olyan, hogy $\Omega := \R^n$, és (az előző tétel feltételein kívül) valamilyen $\alpha, \, \beta$ pozitív együtthatókkal
\[
	\| f(x, \, y)\|_\infty \leq \alpha \cdot \|y\|_\infty + \beta \quad (x \in I, \, y \in \R^n).
\] 
Ekkor belátható, hogy az $f$ által meghatározott differenciálegyenletre vonatkozó bármelyik k.é.p. teljes megoldása az $I$-n van értelmezve. 

	\newpage
\section{Vizsgakérdés}
\begin{quote}
	\textit{A lineáris differenciálegyenlet-rendszer vizsgálata: homogén, inhomogén rendszerek. A megoldáshalmaz szerkezete.}
\end{quote}

\subsection{Lineáris differenciálegyenlet-rendszer}
Valamilyen $1 \leq n \in N$ és egy nyílt $I \subset \R$ intervallum esetén adottak a folytonos
\[
a_{ik} : I \to \R \quad (i, \, k = 1, \, \dots, \, n), \, b = (b_1, \, \dots, \, b_n) : I \to \R^n
\]
függvények, és tekintsük az
\[
I \ni x \mapsto A(x) := \big( a_{ik}(x) \big)_{i, \, k =1}^n \in \R^{n \times n}
\]
\textit{mátrixfüggvényt}. Ha
\[
f(x, \, y) := A(x) \cdot y + b(x) \quad ((x, \, y) \in I \times \K^n),
\]
akkor az $f$ függvény, mint jobb oldal által meghatározott
\[
\varphi'(x) = A(x) \cdot \varphi(x) + b(x) \quad (x \in \Dp)
\]
differenciálegyenletet \textit{lineáris differenciálegyenletnek} ($n > 1$ esetén \textit{lineáris} \textit{differenciálegyenlet-rendszernek}) nevezzük.\\

Legyenek a fentieken túl adottak még a $\tau \in I, \, \xi \in \K^n$ értékek, és vizsgáljuk a $\varphi(\tau) = \xi$ k.é.p.-t. Ha $I_* \subset I$, $\tau \in \text{int} \, I_*$, kompakt intervallum, akkor
\[
\sup \{ |a_{ik}(x)| : x \in I_* \} \in \R \quad (i, \, k = 1, \, \dots, \, n),
\]
ezért
\[
q := \sup \{ \|A(x)\|_{(\infty)} : x \in I_* \} \in \R.
\]
Következésképpen
\[
\| f(x, \, y) - f(x, \, z) \|_\infty = \| A(x) \cdot (y- z) \|_\infty \leq
\]
\[
\|A(x)\|_{(\infty)} \cdot \|y-z\|_\infty \leq q \cdot \|y-z\|_\infty \quad (x \in I_*, \, y, \, z \in \K^n).
\]
Továbbá a
\[
\beta := \sup\{\|b(x)\| : x \in I_*\} (\in \R)
\]
jelöléssel
\[
\|f(x, \, y)\|_\infty = \| A(x) \cdot y + b(x) \|_\infty \leq \|A(x) \cdot y\|_\infty + \|b(x)\|_\infty \leq
\]
\[
\|A(x)\|_{(\infty)} \cdot \|y\|_\infty + \|b(x)\|_\infty \leq q \cdot \|y\|_\infty + \beta \quad (x \in I_*, \, y \in \K^n),
\]
ezért minden k.é.p. teljes megoldása az $I$ intervallumon van értelmezve. Azt mondjuk, hogy a szóban forgó d.e. \textit{homogén}, ha $b \equiv 0$, \textit{inhomogén}, ha létezik $x \in I$, hogy $b(x) \neq 0$. Legyenek
\[
\mathcal{M}_h := \{ \psi : I \to \K^n : \psi \in D, \, \psi' = A \cdot \psi \},
\]
\[
\mathcal{M} := \{ \psi : I \to \K^n : \psi \in D, \, \psi' = A \cdot \psi + b\}.
\]
A lineáris d.e.-ek "alaptétele" a következő\\

\tikz \node[theorem]
{
	\textbf{Tétel.} A bevezetésben mondott feltételek mellett
	\begin{enumerate}
		\item az $\mathcal{M}_h$ halmaz $n$ dimenziós lineáris tér a $\K$-ra vonatkozóan;
		\item tetszőleges $\psi \in \mathcal{M}$ esetén
		\[
		\mathcal{M} = \psi + \mathcal{M}_h := \{\psi + \chi : \chi \in \mathcal{M}_h \};
		\]
		\item ha a $\phi_k = (\phi_{k1}, \, \dots, \, \phi_{kn})$ $(k = 1, \, \dots, \, n)$ függvények bázist alkotnak az $\mathcal{M}_h$-ban, akkor léteznek olyan $g_k : I \to \K$ $(k = 1, \, \dots, \, n)$ differenciálható függvények, amelyekkel
		\[
		\psi := \sum_{k=1}^n g_k \cdot \phi_k \in \mathcal{M}.
		\]
	\end{enumerate}
};\\

\textbf{Bizonyítás.} Az 1. állítás bizonyításához mutassuk meg először is azt, hogy bármilyen $\psi, \, \p \in \mathcal{M}_h$ és $c \in \K$ esetén $\psi + c \cdot \p \in \mathcal{M}_h$:
\[
	(\psi + c \cdot \p)' = \psi' + c \cdot \p' = A \cdot \psi + c \cdot A \cdot \p = A(\psi + c \cdot \p),
\]
amiből a mondott állítás az $\mathcal{M}_h$ definíciója alapján nyilvánvaló. Tehát az $\mathcal{M}_h$ lineáris tér a $\K$ felett.\\

Most megmutatjuk, hogy ha $m \in \N$, és $\chi_1, \, \dots, \, \chi_m \in \mathcal{M}_h$, tetszőleges függvények, akkor az alábbi ekvivalencia igaz:

\begin{quote}
	a $\chi_1, \, \dots, \, \chi_m$ függvények akkor és csak akkor alkotnak lineárisan független rendszert az $\mathcal{M}_h$ vektortérben, ha bármilyen $\tau \in I$ esetén a $\chi_1(\tau), \, \dots, \, \chi_m(\tau)$ vektorok lineárisan függetlenek a $\K^n$-ben.
\end{quote}

Az ekvivalencia fele nyilvánvaló: ha a $\chi_1, \, \dots, \, \chi_m$-ek lineárisan összefüggnek, akkor alkalmas $c_1, \, \dots, \, c_m \in \K$, $|c_1| + \dots + |c_n| > 0$ együtthatókkal
\[
	\sum_{k=1}^m c_k \cdot \chi_k \equiv 0.
\]
Speciálisan minden $\tau \in I$ helyen is
\[
	\sum_{k=1}^m c_k \cdot \chi_k(\tau) = 0 \quad (\in \K^n).
\]
Így a $\chi_1(\tau), \, \dots, \, \chi_m(\tau)$ vektorok összefüggő rendszert alkotnak a $\K^n$-ben.\\

Fordítva, legyen $\tau \in I$, és tegyük fel, hogy a $\chi_1(\tau), \, \dots, \, \chi_m(\tau)$ vektorok összefüggnek. Ekkor az előbbi (nem csupa nulla) $c_1, \, \dots, \, c_m \in \K$ együtthatókkal
\[
	\sum_{k=1}^m c_k \cdot \chi_k(\tau) = 0.
\]
Már tudjuk, hogy
\[
	\phi := \sum_{k=1}^m c_k \cdot \chi_k \in \mathcal{M}_h,
\]
ezért az így definiált $\phi : I \to \K^n$ függvény megoldása a
\[
	\p' = A \cdot \p, \, \p(\tau) = 0
\]
homogén lineáris differenciálegyenletre vonatkozó kezdetiérték-problémának. Világos ugyanakkor, hogy a $\Psi \equiv 0$ is a most mondott k.é.p. megoldása az $I$-n. Azt is tudjuk azonban, hogy (ld. fent) ez a k.é.p. (is) egyértelműen oldható meg, ezért $\phi \equiv \Psi \equiv 0$. Tehát a $\chi_1, \, \dots, \, \chi_m$ függvények is összefüggnek.\\

Ezzel egyúttal azt is beláttuk, hogy az $\mathcal{M}_h$ vektortér véges dimenziós és a $\dim \mathcal{M}_h$ dimenziója legfeljebb $n$.\\

Tekintsük most a
\[
	\p' = A \cdot \p, \, \p(\tau) = e_i  \quad (i = 1, \, \dots, \, n)
\]
kezdetiérték-problémákat, ahol az $e_i \in \K^n \quad (i = 1, \, \dots, \, n)$ vektorok a $\K^n$ tér "szokásos" (kanonikus) bázisvektorait jelölik. Ha
\[
	\chi_i : I \to \K^n
\]
jelöli az említett k.é.p. teljes megoldását, akkor a
\[
	\chi_i(\tau) = e_i \quad (i = 1, \, \dots, \, n)
\]
vektorok lineárisan függetlenek. Így az előbbiek alapján a $\chi_1, \, \dots, \, \chi_n$ függvények is azok. Tehát az $\mathcal{M}_h$ dimenziója legalább $n$, azaz a fentiekre tekintettel $\dim \mathcal{M}_h = n$.\\

A 2. állítás igazolásához legyen $\chi \in \mathcal{M}_h$. Ekkor $\psi + \chi \in D$, és
\[
	(\psi + \chi)' = \psi' + \chi' = A \cdot \psi + b + A \cdot \chi = A \cdot (\psi + \chi) + b,
\]
amiből $\psi + \chi \in \mathcal{M}$ következik. Ha most egy $\p \in \mathcal{M}$ függvényből indulunk ki és $\chi := \p - \psi$, akkor $\chi \in D$, és 
\[
	\chi' = \p' - \psi' = A \cdot \p + b - (A \cdot \psi + b) = A \cdot (\p - \psi) = A \cdot \chi,
\]
amiből $\chi \in \mathcal{M}_h$ adódik. Tehát $\p = \psi + \chi$ a 2.-ben mondott előállítása a $\p$ függvénynek.\\

A tétel 3. részének a bizonyítása érdekében vezessük be az alábbi jelöléseket, ill. fogalmakat. A
\[
	\phi_k = (\phi_{k1}, \, \dots, \, \phi_{kn}) \quad (k = 1, \, \dots, \, n)
\]
bázisfüggvények mint oszlopvektor-függvények segítségével tekintsük a
\[
	\Phi : I \to \K^{n \times n}
\]
mátrixfüggvényt:
\[
	\Phi := \begin{bmatrix}
		\phi_1 & \cdots & \phi_n
	\end{bmatrix} = \begin{bmatrix}
	\phi_{11} & \phi_{21} & \cdots & \phi_{n1} \\
	\phi_{12} & \phi_{22} & \cdots & \phi_{n2} \\
	\vdots & \vdots & \cdots & \vdots \\
	\phi_{1n} & \phi_{2n} & \cdots & \phi_{nn} \\
	\end{bmatrix}.
\]
Legyen
\[
	\Phi' := \begin{bmatrix}
		\phi'_1 & \cdots & \phi'_n
	\end{bmatrix} = \begin{bmatrix}
		\phi'_{11} & \phi'_{21} & \cdots & \phi'_{n1} \\
		\phi'_{12} & \phi'_{22} & \cdots & \phi'_{n2} \\
		\vdots & \vdots & \cdots & \vdots \\
		\phi'_{1n} & \phi'_{2n} & \cdots & \phi'_{nn} \\
	\end{bmatrix}
\]
a $\Phi$ deriváltja. Ekkor könnyen belátható, hogy
\[
	\Phi' = A \cdot \Phi.
\]
Továbbá tetszőleges $g_1, \, \dots, \, g_n : I \to \K$ differenciálgató függvényekkel a
\[
	g := (g_1, \, \dots, \, g_n) : I \to \K^n
\]
vektorfüggvény differenciálható,
\[
	\psi := \sum_{k=1}^n g_k \cdot \phi_k = \Phi \cdot g,
\]
és
\[
	\psi' = \Phi' \cdot g + \Phi \cdot g' = (A \cdot \Phi) \cdot g + \Phi \cdot g'.
\]
A $\psi \in \mathcal{M}$ tartalmazás nyilván azzal ekvivalens, hogy
\[
	\psi' = (A \cdot \Phi) \cdot g + \Phi \cdot g' = A \cdot \psi + b = A \cdot (\Phi \cdot g) + b = (A \cdot \Phi) \cdot g + b,
\] 
következésképpen azzal, hogy
\[
	\Phi \cdot g' = b.
\]

A 2. pont alapján tetszőleges $x \in I$ helyen a $\phi_1(x), \, \dots, \, \phi_n(x)$ vektorok lineárisan függetlenek, azaz a $\Phi(x)$ mátrix nem szinguláris. A mátrixok inverzének a kiszámítása alapján egyszerűen adódik, hogy a
\[
	\Phi^{-1}(x) := \big(\Phi(x)\big)^{-1} \quad (x \in I)
\]
definícióval értelmezett
\[
	\Phi^{-1} : I \to \K^{n \times n}
\]
mátrixfüggvény komponens-függvényei is folytonosak. Ezért a
\[
	(h_1, \, \dots, \, h_n) := \Phi^{-1} \cdot b : I \to \K^n
\]
függvény is folytonos. Olyan folytonosan differenciálható
\[
	g : I \to \K^n
\]
függvényt keresünk tehát amelyikre $g' = \Phi^{-1} \cdot b$, azaz
\[
	g_i' = h_i \quad (i = 1, \, \dots, \, n).
\]
Ilyen $g_i$ létezik, nevezetesen a (folytonos) $h_i \quad (i = 1, \, \dots, \, n)$ függvények bármelyik primitív függvénye ilyen. $\hfill \blacksquare$
	\newpage
\section{Vizsgakérdés}
\begin{quote}
	\textit{Alaprendszer, alapmátrix. Az állandók variálásának a módszere. Alapmátrix előállítása állandó együtthatós diagonalizálható mátrix esetén. Az $n=2$ eset vizsgálata tetszőleges, állandó együtthatós mátrixra.}
\end{quote}

\subsection{Alaprendszer, alapmátrix}

Az $\mathcal{M}_h$ vektortérben minden bázist az illető egyenlet \textit{alaprendszerének} nevezünk. Ha $\phi_1, \, \dots, \, \phi_n \in \mathcal{M}_h$ egy alaprendszer, akkor a
\[
	\Phi := \begin{bmatrix}
		\phi_1 & \cdots & \phi_n
	\end{bmatrix} : I \to \K^{n \times n}
\]
mátrixfüggvény a szóban forgó lineáris differenciálegyenlet (egy) ún. \textit{alapmátrixa}. Tehát
\[
	\mathcal{M}_h = \left\{ \sum_{k=1}^n c_k \cdot \phi_k : c_1, \, \dots, \, c_n \in \K \right\} = \{ \Phi \cdot c : c \in \K^n \}.
\]
Az $\mathcal{M} = \psi + \mathcal{M}_h$ előállításban minden $\psi \in \mathcal{M}$ függvényt \textit{partikuláris megoldásként} említünk.

\subsection{Állandók variálásának módszere}
Ld. ~\ref{subsec:diff_systems_core} alcímben tárgyalt tétel 3. pontjának bizonyítása. A partikuláris megoldás
\[
	\psi = \Phi \cdot g
\]
alakban való előállítása (alkalmas $g : I \to \K^n$ differenciálható függvénnyel) az ~\ref{subsec:diff_systems_core} alcímben tárgyalt tétel bizonyításában bemutatott módszer az \textit{állandók variálása}. Tetszőleges $\phi_1, \, \dots, \, \phi_n$ alaprendszerrel és $\psi \in \mathcal{M}$ partikuláris megoldással
\[
	\mathcal{M} = \Big\{ \psi + \sum_{k=1}^n c_k \cdot \phi_k : c_1, \, \dots, \, c_n \in \K  \Big\}.
\]
Ha $\Phi$ egy alapmátrix, akkor ugyanez a következőképpen írható:
\[
	\mathcal{M} = \{ \psi + \Phi \cdot c : c \in \K^n \} = \{ \Phi \cdot (g + c) : c \in \K^n \}.
\]

\subsection{Állandó együtthatós diagonalizálható eset}

Legyen most
\[
f(x, \, y) := A \cdot y + b(x) \quad \big( (x, \, y) \in I \times \K^n \big),
\]
ahol $1 \leq n \in \N, \, I \subset \R$ nyílt intervallum mellett
\[
A \in \R^{n \times n}, \, b : I \to \R^n, \, b \in C.
\]
Tegyük fel, hogy $A$ diagonalizálható, azaz létezik $T \in \K^{n \times n}, \, \det T \neq 0$, hogy $T^{-1}AT$ mátrix diagonális: alkalmas $\lambda_1, \, \dots, \, \lambda_n \in \K$ számokkal
\[
\Lambda := T^{-1}AT = \begin{bmatrix}
	\lambda_1 & 0 & \cdots & 0 \\
	0 & \lambda_2 & \cdots & 0 \\
	\vdots & \vdots & \cdots & \vdots \\
	0 & 0 & \cdots & \lambda_n
\end{bmatrix}.
\]
$T$ invertálhatósága miatt a
\[
T = [t_1 \cdots t_n]
\]
$t_i$ $(i=1, \, \dots, \, n)$ oszlopvektorok lineárisan függetlenek, azaz
\[
AT = [At_1 \cdots At_n] = T\Lambda = [\lambda_1 \cdot t_1 \cdots \lambda_n \cdots t_n]
\]
miatt
\[
A \cdot t_i = \lambda_i \cdot t_i \quad (i = 1, \, \dots, \, n).
\]
Mivel
\[
t_i \neq 0 \quad (i = 1, \, \dots, \, n),
\]
ezért mindez röviden azt jelenti, hogy a $\lambda_1, \, \dots, \, \lambda_n$ számok az $A$ mátrix sajátértékei, a $t_1, \, \dots, \, t_n$ vektorok pedig rendre a megfelelő sajátvektorok. Lévén, a $t_i$-k lineárisan függetlenek, az $A$-ra vonatkozó feltételünk úgy fogalmazható, hogy van a $\K^n$-ben (az $A$ sajátvektoraiból álló) sajátvektorbázis.\\

A homogén egyenlet tehát a következőképpen írható fel:
\[
\varphi' = A \cdot \varphi = T \Lambda T^{-1} \cdot \varphi,
\]
amiből
\[
(T^{-1}\varphi)' = \Lambda \cdot (T^{-1} \varphi)
\]
következik. Vegyük észre, hogy ha $\varphi \in \mathcal{M}_h$, akkor a $\psi := T^{-1}\varphi$ függvény megoldása a $\Lambda$ diagonális mátrix által meghatározott állandó együtthatós homogén lineáris egyenletnek. Ez utóbbit ez előző tétel alapján nem nehéz megoldani. Legyenek ui. a
\[
\psi_i : I \to \K^n \quad (i = 1, \, \dots, \, n)
\]
függvények a következők:
\[ 
\psi_i(x) := e^{\lambda_i \cdot x} \cdot e_i \quad (x \in I, \, i = 1, \, \dots, \, n).
\]
Világos, hogy $\psi_i \in D$ és
\[
\psi_i'(x) = \lambda_i \cdot e^{\lambda_i \cdot x} \cdot e_i = e^{\lambda_i \cdot x} \cdot (\Lambda \cdot e_i) =
\]
\[
\Lambda \cdot (e^{\lambda_i \cdot x} \cdot e_i) = \Lambda \cdot \psi_i(x) \quad (x \in I, \, i = 1, \, \dots, \, n).
\]
Más szóval a $\psi_i$-k valóban megoldásai a $\Lambda$ által meghatározott homogén lineáris differenciálegyenletnek. Mivel bármely $\tau \in I$ esetén a
\[
\psi_i(\tau) = e^{\lambda_i \cdot \tau} \cdot e_i \quad (i = 1, \, \dots, \, n)
\]
vektorok nyilván lineárisan függetlenek, ezért az előző tétel bizonyításában mondottak szerint a $\psi_i$ $(i = 1, \, \dots, \, n)$ függvények lineárisan függetlenek. Ha
\[
\phi_i := T \cdot \psi_i \quad (i = 1, \, \dots, \, n),
\]
akkor nyilván a $\phi_i$-k is lineárisan függetlenek,
\[
\phi_i(x) = e^{\lambda_i \cdot x} \cdot t_i \quad (x \in I, \, i = 1, \, \dots, \, n),
\]
és minden $i = 1, \, \dots, \, n$ indexre
\[
\phi_i' = A \cdot \phi_i.
\]
Tehát $\phi_i \in \mathcal{M}_h$ $(i = 1, \, \dots, \, n)$ egy bázis. Ezzel beláttuk az alábbi tételt:\\

\tikz \node[theorem]
{
	\textbf{Tétel.} Tegyük fel, hogy az $A \in \R^{n \times n}$ mátrix diagonalizálható. Legyenek a sajátértékei $\lambda_1, \, \dots, \, \lambda_n \in \K$, egy-egy megfelelő sajátvektora pedig $t_1, \, \dots, \, t_n \in \K^n$. Ekkor a
	\[
	\varphi' = A \cdot \varphi
	\]
	homogén lineáris differenciálegyenletnek a
	\[
	\phi_i(x) := e^{\lambda_i \cdot x} \cdot t_i \quad (x \in \R, \, i = 1, \, \dots, \, n)
	\]
	függvények lineárisan független megoldásai.
};

\subsection{Tetszőleges állandó együtthatós mátrix}

Tekintsük az $n=2$ esetet, amikor is valamilyen $a, \, b, \, c, \, d \in \R$ számokkal
\[
	A = \begin{bmatrix}
		a & b \\
		c & d
	\end{bmatrix} \in \R^{2 \times 2}.
\]
Könnyű meggyőződni arról, hogy ez a mátrix pontosan akkor nem diagonalizálható, ha
\[
	(a-d)^2 + 4bc = 0 \text{ és } |b| + |c| > 0.
\]
Ekkor egyetlen sajátértéke van az $A$-nak nevezetesen
\[
	\lambda := \frac{a+d}{2},
\]

legyen a $t_1$ egy hozzá tartozó sajátvektor:
\[
	0 \neq t_1 \in \R^2, \, At_1 = \lambda t_1.
\]
Egyszerű számolással igazolható olyan $t_2 \in \R^2$ vektor létezése, amelyik lineárisan független a $t_1$-től és
\[
	At_2 = t_1 + \lambda t_2.
\]
Ha mármost a $T \in \R^{2 \times 2}$ mátrix oszlopvektorai rendre a $t_1, \, t_2$ vektorok: $T := [t_1 \, t_2]$, akkor
\[
	T^{-1}AT = \begin{bmatrix}
		\lambda & 1 \\
		0 & \lambda
	\end{bmatrix}.
\]
Ezt felhasználva könnyen belátható, hogy a
\[
	\phi_1(x) := e^{\lambda x} \cdot t_1, \, \phi_2(x) := e^{\lambda x} \cdot (t_2 + xt_1) \quad (x \in \R)
\]
függvénypár egy alaprendszer. Valóban, $\phi_i \in \mathcal{M}_h \quad (i = 1, \, 2)$, mert egyrészt
\[
	\phi_1'(x) = \lambda e^{\lambda x} \cdot t_1 = e^{\lambda x} A t_1 = A(e^{\lambda x} \cdot t_1) = A\phi_1(x),
\]
másrészt
\[
	\phi_2'(x) = \lambda e^{\lambda x} \cdot t_2 + e^{\lambda x} \cdot t_1 + \lambda e^{\lambda x}x \cdot t_1 = e^{\lambda x} \big( (t_1 + \lambda \cdot t_2) + \lambda x \cdot t_1 \big) =
\]
\[
	e^{\lambda x} (At_2 + xAt_1) = A\big( e^{\lambda x}(t_2 + x \cdot t_1) \big) = A \phi_2(x) \quad (x \in \R).
\]
Mivel a
\[
	\phi_1(0) = t_1, \, \phi_2(0) = t_2
\]
vektorok lineárisan függetlenek, ezért a $\phi_1, \, \phi_2$ függvények is lineárisan függetlenek, azaz a $\phi_1, \, \phi_2$ egy alaprendszer.

\subsection{Valós értékű megoldások}
Tegyük fel, hogy a $\lambda \in \K$ szám az $A \in \R^{n \times n}$ (diagonalizálható) együtthatómátrixnak egy sajátértéke, a $t_\lambda \in \K^n$ vektor pedig egy $\lambda$-hoz tartozó sajátvektor. Ha a $\lambda$ valós, akkor nyilván a $t_\lambda$ sajátvektor is választható "valósnak", azaz feltehető, hogy $t_\lambda \in \R^n$. Ebben az esetben az $\mathcal{M}_h$-beli
\[
	\phi_\lambda(x) := e^{\lambda x} \cdot t_\lambda \quad (x \in \R)
\]
bázisfüggvény is "valós" tehát $\phi_\lambda : \R \to \R^n$.\\

Ha viszont a $\lambda$ (nem valós) komplex szám, azaz
\[
	\lambda = u + \imath v \in \mathbf{C} \, \backslash \, \R
\]
(alkalmas $u \in \R$ és $0 \neq v \in \R$ számokkal), akkor -- lévén az $A$ karakterisztikus polinomja valós együtthatós -- az $A$-nak egyúttal a
\[
	\overline{\lambda} = u + \imath v
\]
(komplex konjugált) is (ugyanannyiszoros) sajátértéke. Hasonlóan, ha a
\[
	t_\lambda = S_\lambda + \imath Y_\lambda \in \K^n
\]
vektor (alkalmas $S_\lambda, \, Y_\lambda \in \R^n$ vektorokkal) az $A$-nak a $\lambda$-hoz tartozó sajátvektora, akkor a
\[
	\overline{t}_\lambda = S_\lambda - \imath Y_\lambda
\]
vektor a $\overline{\lambda}$-hoz tartozó sajátvektor. Továbbá a megfelelő bázisfüggvények a következők:
\[
	\phi_\lambda(x) := e^{\lambda x} \cdot t_\lambda, \, \phi_{\overline{\lambda}}(x) := e^{\overline{\lambda}x} \cdot \overline{t}_\lambda \quad (x \in \R).
\]
Rövid számolással ellenőrizhető, hogy
\[
	\phi_{\overline{\lambda}}(x) = \overline{\phi_\lambda(x)} \quad (x \in \R).
\]
Mivel a homogén egyenlet (teljes) megoldásainak az $\mathcal{M}_h$ halmaza a $\K$ felett vektortér, ezért a
\[
	\frac{\phi_\lambda + \overline{\phi_\lambda}}{2} = \text{Re}\, \phi_\lambda, \, \frac{\phi_\lambda - \overline{\phi_\lambda}}{2} = \text{Im} \, \phi_\lambda
\]
függvények is $\mathcal{M}_h$-beliek. Világos, hogy
\[
	e^{\lambda x} = e^{u x} \cdot \B{ \cos(vx) + \imath \sin(vx) } \quad (x \in \R)
\]
miatt
\[
	\phi_{\lambda, \, r}(x) := \text{Re} \, \phi_\lambda(x) = e^{ux} \cdot \B{ \cos(vx) \cdot S_\lambda - \sin(vx) \cdot Y_\lambda } \quad (x \in \R),
\]
\[
	\phi_{\lambda, \, i}(x) := \text{Im} \, \phi_\lambda(x) = e^{ux} \cdot \B{ \sin(vx) \cdot S_\lambda + \cos(vx) \cdot Y_\lambda } \quad (x \in \R).
\]
Továbbá
\[
	\phi_{\lambda, \, r}(0) = S_\lambda = \frac{t_\lambda + \overline{t_\lambda}}{2},
\]
\[
	\phi_{\lambda, \, i}(0) = Y_\lambda = \frac{t_\lambda - \overline{t_\lambda}}{2 \imath}.
\]
A $t_\lambda, \, \overline{t}_\lambda$ sajátvektorok lineárisan függetlenek, ezért az $S_\lambda, \, Y_\lambda$ vektorok is azok, következésképpen a $\phi_{\lambda, \, r}, \, \phi_{\lambda, \, i}$ függvények is lineárisan függetlenek. Így $\phi_\lambda, \, \phi_{\overline{\lambda}}$ függvényeket kicserélve az előző függvényekre, továbbra is alaprendszert kapunk.

    \newpage
\section{Vizsgakérdés}
\begin{quote}
	\textit{Magasabb rendű lineáris differenciálegyenlet. Az átviteli elv. A megoldáshalmaz szerkezete. Az állandók variálásának a módszere.}
\end{quote}

\subsection{"Új" feladat megfogalmazása}
Legyen $1 \leq n \in \N, \, I \subset \R$ nyílt intervallum, az
\[
	a_k : I \to \R \quad (k = 0, \, \dots, \, n-1), \, c: I \to \R
\]
függvényekről tegyük fel, hogy folytonosak. Olyan $\p \in I \to \K$ függvényt keresünk, amelyikre
\begin{enumerate}
	\item $\Dp \subset I$ nyílt intervallum;
	\item $\p \in D^n$;
	\item $\displaystyle \p^{(n)}(x) + \sum_{k=0}^{n-1} a_k(x) \cdot \p^{k}(x) = c(x) \quad (x \in \Dp)$.
\end{enumerate}
Ezt a feladatot röviden \textit{$n$-edrendű lineáris differenciálegyenletnek} nevezzük. Minden olyan $\p$ függvény amelyik eleget tesz az előbbi kívánalmaknak, az illető differenciálegyenlet (egy) \textit{megoldása}.\\

Tegyük fel, hogy a fentieken túl adottak még a
\[
	\tau \in I, \, \xi_0, \, \dots, \, \xi_{n-1} \in \K
\]
számok. Ha az előbbi $\p$ megoldástól azt is elvárjuk, hogy
\begin{enumerate}[start=4]
	\item $\tau \in \Dp, \, \p^{(k)} = \xi_k \quad (k = 0, \, \dots, \, n-1)$,
\end{enumerate}
akkor a szóban forgó $n$-edrendű lineáris differenciálegyenletre vonatkozó \textit{kezdetiérték-problémáról} beszélünk.\\

Ha $n=1$, akkor egy lineáris differenciálegyenletről van szó, ezért a továbbiakban nyugodtan feltehetjük már, hogy $n \geq 2$.\\

Az \textit{átviteli elv} segítségével a most megfogalmazott feladat visszavezethető a lineáris differenciálegyenlet-rendszerek vizsgálatára. (A későbbiekben szereplő állítások is részben ennek az elvnek a segítségével láthatók majd be.) Vezessük be ui. az alábbi jelöléseket: legyen $2 \leq n \in N$ és
\[
	b := (b_1, \, \dots, \, b_n) : I \to \R^n, \, b(x) := \begin{pmatrix}
		0 \\
		0 \\
		\vdots \\
		c(x)
	\end{pmatrix} \quad (x \in I),
\]
\[
	A := (a_{ik})_{i, \, k = 1}^n = \begin{bmatrix}
		0 & 1 & 0 & 0 & \cdots & 0 \\
		0 & 0 & 1 & 0 & \cdots & 0 \\
		\vdots & \vdots & \vdots & \vdots & \cdots & \vdots \\
		\vdots & \vdots & \vdots & \vdots & \cdots & \vdots \\
		\vdots & \vdots & \vdots & \vdots & \cdots & \vdots \\
		0 & 0 & 0 & 0 & \cdots & 1 \\
		-a_0 & -a_1 & -a_2 & -a_3 & \cdots & -a_{n-1}
	\end{bmatrix} : I \to \R^{n \times n}
\]
Ekkor
\[
	f(x, y) := A(x) \cdot y + b(x) \quad (x \in I, \, y \in \K^n).
\]
Ha tehát a
\[
	\psi = (\psi_1, \, \dots, \, \psi_n) \in I \to \K^n
\]
differenciálható függvény ez utóbbi lineáris differenciálegyenlet-rendszernek (egy) megoldása, akkor $\mathcal{D}_\psi \subset I$ nyílt intervallum, és bármely $x \in \mathcal{D}_\psi$ esetén
\[
	\psi'(x) = A(x) \cdot \psi(x) + b(x) \quad (x \in \mathcal{D}_\psi),
\]
azaz

\begin{equation}
	\begin{cases}
		\psi_i'(x) = \psi_{i+1}(x) \quad (i = 1, \, \dots, \, n-1)\\
		\\
		\displaystyle \psi'_n(x) = \sum_{k=1}^n a_{k-1}(x) \cdot \psi_k(x) + c(x).
	\end{cases}
	\tag{$\star$}
\end{equation}
Ennek alapján eléggé nyilvánvaló az alábbi állítás.\\

\subsection{Átviteli elv}

\tikz \node[theorem]
{
	\textbf{Tétel.} Ha a $\p$ függvény megoldása a fenti $n$-edrendű lineáris differenciálegyenletnek, akkor az
	\[
		I \ni x \mapsto \psi(x) := \big(\p(x), \, \p'(x), \, \dots, \, \p^{(n-1)}(x)\big) \in \K^n
	\]
	függvényre igazak a $(\star)$ egyenlőségek. Fordítva, ha a $\psi = (\psi_1, \, \dots, \, \psi_n)$ függvény eleget tesz a $(\star)$-nak, akkor a $\p := \psi_1$ (első) komponensfüggvény megoldása a szóban forgó $n$-edrendű lineáris differenciálegyenletnek. Ha adottak a $\tau \in I, \, \xi_0, \, \dots, \, \xi_{n-1} \in \K$ kezdeti értékek, és a $\p$, megoldása a
	\[
		\p^{(k)}(\tau) = \xi_k \quad (k = 0, \, \dots, \, n-1)
	\]
	k.é.p.-nak, akkor a $(\star)$ lineáris differenciálegyenlet-rendszer előbbi $\psi$ megoldása kielégíti a
	\[
		\psi(\tau) = (\xi_0, \, \dots, \, \xi_{n-1}) \in \K^n
	\]
	kezdeti feltételt.
};\\

Legyen most
\[
	\mathcal{M}_h := \Big\{ \p : I \to \K : \p \in D^n, \, \p^{(n)} + \sum_{k=0}^{n-1} a_k \cdot \p^{(k)} = 0 \Big\}.
\]
Az $\mathcal{M}_h$ függvényhalmaz tehát nem más, mint a
\[
	c(x) := 0 \quad (x \in I)
\]
esetnek megfelelő \textit{homogén $n$-edrendű lineáris differenciálegyenlet} $I$ intervallumon értelmezett megoldásainak a halmaza. Legyen továbbá
\[
	\mathcal{M} := \Big\{ \p : I \to \K : \p \in D^n, \, \p^{(n)} + \sum_{k=0}^{n-1} a_k \cdot \p^{(k)} = c \Big\}
\]
a kiindulási $n$-edrendű lineáris differenciálegyenlet $I$-n értelmezett megoldásainak a halmaza. Az utóbbival kapcsolatban már nyilván feltehető, hogy valamilyen $x \in I$ helyen $c(x) \neq 0$, azaz az illető egyenlet \textit{inhomogén}. Ekkor az átviteli elv alapján a következőket mondhatjuk.\\

\subsection{Állandók variálásának módszere}

\tikz \node[theorem]
{
	\textbf{Tétel.} Az $n$-edrendű lineáris differenciálegyenletet illetően
	\begin{enumerate}
		\item az $\mathcal{M}_h$ halmaz $n$ dimenziós lineáris tér a $\K$-ra vonatkozóan;
		\item tetszőleges $\omega \in \mathcal{M}$ esetén
		\[
			\mathcal{M} = \omega + \mathcal{M}_h := \{ \omega + \chi : \chi \in \mathcal{M}_h \};
		\]
		\item ha a $\p_1, \, \dots, \, \p_n$ függvények bázist alkotnak az $\mathcal{M}_h$-ban, akkor léteznek olyan differenciálható $g_k : I \to \K \quad (k = 1, \, \dots, \, n)$ függvények, amelyekkel
		\[
			\omega := \sum_{k=1}^n g_k \p_k \in \mathcal{M}.
		\]
	\end{enumerate}
};

    \newpage
\section{Vizsgakérdés}
\begin{quote}
	\textit{Állandó együtthatós magasabb rendű homogén lineáris differenciálegyenlet egy alaprendszerének az előállítása, a karakterisztikus polinom szerepe (a bizonyítás vázlata).}
\end{quote}

Az előzőekben vizsgált
\[
	\p^{(n)} + \sum_{k=1}^{n-1} a_k(x) \cdot \p^{(k)}(x) = c(x) \quad (x \in \Dp)
\]
$n$-edrendű lineáris differenciálegyenlet megoldásához tehát elegendő az $\mathcal{M}_h$ egy bázisát meghatározni. Ez általában "reménytelen" feladat, általános módszer nem is adható.\\

Ezért csak abban az esetben tesszük ezt meg, ha az $a_k \quad (k = 0, \, \dots, \, n-1)$ együtthatófüggvények mindegyike konstansfüggvény. Ez az ún. \textit{állandó együtthatós} eset:
\[
	\p^{(n)}(x) = \sum_{k=1}^{n-1} a_k \cdot \p^{(k)}(x) = c(x) \quad (x \in \Dp),
\]
A $c=0$ (\textit{homogén egyenlet}) választással
\[
	\p^{(n)} + \sum_{k=0}^{n-1} a_k \cdot \p^{(k)} \equiv 0.
\]
Tekintsük ehhez a
\[
	P(x) := x^n + \sum_{i=0}^{n-1} a_i x^i \quad (x \in \K)
\]
$n$-edfokú polinomot, a differenciálegyenlet \textit{karakterisztikus polinomját}.\\

Ha a $\lambda \in \K$ szám a $P$-nek gyöke, akkor az
\[
	e_\lambda(x) := e^{\lambda x} \quad (x \in \R)
\]
és az $a_n := 1$ jelöléssel
\[
	\sum_{k=0}^n a_k e_\lambda^{(k)}(x) = \sum_{k=0}^n a_k \lambda^k e^{\lambda x} = e^{\lambda x} \cdot P(\lambda) = 0 \quad (x \in \R)
\]
miatt az $e_\lambda$ függvény megoldása a szóban forgó homogén differenciálegyenletnek.\\

Legyen az előbbi $\lambda$ gyök multiplicitása $\nu \geq 2$. Belátjuk, hogy az
\[
	e_{\lambda, \, j}(x) := x^j e^{\lambda x} \quad (x \in \R, \, j = 0, \, \dots, \, \nu - 1)
\]
függvények is megoldásai a homogén differenciálegyenletnek:
\[
	\sum_{k=0}^n a_k e_{\lambda, \, j}^{(k)} \equiv 0.
\]
Tegyük fel ui., hogy ezt valamilyen $j \in \N$ mellett minden olyan esetben már tudjuk, amikor (az aktuális differenciálegyenletre) $\nu - 1 \geq j$. Például a $j=0$ ilyen, hiszen ezt az $e_{\lambda, \, 0} \equiv e_\lambda$ függvényre az előbb láttuk. Vegyük észre, hogy
\[
	e_{\lambda, \, j+1}(x) = x e_{\lambda, \, j}(x) \quad (x \in \R),
\]
ezért (ami teljes indukcióval rögtön adódik)
\[
	e_{\lambda, \, j+1}^{(k)}(x) = x e_{\lambda, \, j}^{(k)}(x) + k e_{\lambda, \, j}^{(k-1)}(x) \quad (x \in \R, \, 1 \leq k \in \N).
\]
Így -- feltételezve most azt, hogy $\nu -1 \geq j + 1$ -- az alábbiakat kapjuk:
\[
	\sum_{k=0}^n a_k e_{\lambda, \, j+1}^{(k)}(x)  = x \cdot \sum_{k=0}^n e_{\lambda, \, j}^{(k)}(x) + \sum_{k=1}^n k a_k e_{\lambda, \, j}^{(k-1)}(x) =
\]
\[
	\sum_{k=0}^{n-1} (k+1) a_{k+1} e_{\lambda, \, j}^{(k)}(x) \quad (x \in \R).
\]
A
\[
	\sum_{k=0}^{n-1} (k+1)a_{k+1} \p^{(k)} \equiv 0
\]
homogén lineáris differenciálegyenletnek a karakterisztikus polinomja:
\[
	\sum_{k=0}^{n-1} (k+1)a_{k+1} t^k = P'(t) \quad (t \in \R),
\]
ahol a $\lambda$ a $P'$ (derivált)polinomnak $\mu := \nu - 1$-szeres gyöke. Mivel $\mu - 1 \geq j$, ezért az indukciós feltevés szerint
\[
	\sum_{k=0}^{n-1}(k+1)a_{k+1}e_{\lambda, \, j}^{(k)} \equiv 0,
\]
következésképpen (a fentiekre tekintettel)
\[
	\sum_{k=0}^{n} a_k e_{\lambda, \, j+1}^{(k)} \equiv 0.
\]
Tehát az $e_{\lambda, \, j+1}$ függvény is megoldása a homogén egyenletnek.\\

\subsection{Alaprendszer}

Tegyük fel, hogy a $P$ gyöktényezős előállítása a következő:
\[
	P(x) = \prod_{l = 1}^k (x - \lambda_l)^{\nu_l} \quad (x \in \K),
\]
ahol $1 \leq k \in \N$ és $\lambda_1, \, \dots, \, \lambda_k \in \K$ jelöli a $P$ összes, páronként különböző gyökét, $1 \leq \nu_l \in \N$ pedig a $\lambda_l$ gyök multiplicitását $(l = 1, \, \dots, \, k)$. Ekkor tehát a
\[
	\p_{lj}(x) := x^j \cdot e^{\lambda_l x} \quad (x \in I, \, l = 1, \, \dots, \, k \text{ és } j = 0, \, \dots, \, \nu_l - 1)
\]
függvények valamennyien $\mathcal{M}_h$-beliek.\\

Ennél még több is igaz, nevezetesen:\\

\tikz \node[theorem]
{
	\textbf{Tétel.} A fentiekben definiált
	\[
		\p_{lj} \quad (l = 1, \, \dots, \, k \text{ és } j = 0, \, \dots, \, \nu_l - 1)
	\]
	függvények a szóban forgó állandó együtthatós $n$-edrendű lineáris differenciálegyenlet egy alaprendszerét alkotják.
};

\subsection{Valós értékű megoldások}

Az előző tétel alapján kapott
\[
	\p_{lj}(x) := x^j e^{\lambda_l x} \quad (x \in I, \, l = 1, \, \dots, \, k; \, j = 0, \, \dots, \, \nu_l)
\]
alaprendszerben a $\p_{lj} \quad (l = 1, \, \dots, \, k; \, j = 0, \, \dots, \, \nu_l)$ függvények valós értékűek, ha a szóban forgó $n$-edrendű lineáris differenciálegyenlet $P$ karakterisztikus polinomjában a $\lambda_l$ gyök valós szám. Ha viszont valamilyen $l = 1, \, \dots, \, k$ esetén a $\lambda_l$ gyök nem valós komplex szám, akkor a következőket mondhatjuk. Legyen ekkor
\[
	\lambda_k = u_k + \imath v_l,
\]
ahol $u_l, \, v_l \in \R$ és $v_l \neq 0$. Mivel a $P$ polinom valós együtthatós, ezért a
\[
	\overline{\lambda_l} = u_l + \imath v_l
\]
komplex konjugált is $v_l$-szeres gyöke a $P$-nek. Ez azt jelenti, hogy a fenti alaprendszerben a
\[
	\hat{\p}_{lj}(x) := x^j \cdot e^{\overline{\lambda_l} x} = \overline{\p_{lj}(x)} \quad (x \in I, \, j = 0, \, \dots, \, \nu_l - 1)
\]
függvények is szerepelnek. Tudjuk, hogy az $\mathcal{M}_h$ halmaz vektortér a $\K$-ra nézve, ezért
\[
	\phi_{lj} := \frac{\p_{lj} + \hat{\p}_{lj}}{2} = \text{Re} \, \p_{lj} \in \mathcal{M}_h \text{ és } \hat{\phi}_{lj} := \frac{\p_{lj} - \hat{\p}_{lj}}{2i} = \text{Im} \, \p_{lj} \in \mathcal{M}_h.
\]
Itt tetszőleges $j = 0, \, \dots, \, \nu_l - 1$ mellett
\[
	\phi_{lj}(x) = \text{Re} \, \p_{lj}(x) = \text{Re} \, (x^j \cdot e^{\lambda_l x}) = \text{Re} \, (x^j \cdot e^{u_lx + \imath v_l x}) =
\]
\[
	\text{Re} \, \B{ x^j \cdot e^{u_l x} (\cos(v_lx) + \imath \sin(v_l x)) = x^j \cdot e^{u_l x} \cdot \cos(v_l x)} \quad (x \in I),
\]
és (analóg számolás után)
\[
	\hat{\phi}_{lj}(x) = x^j \cdot e^{u_l x} \cdot \sin(v_l x) \quad (x \in I).
\]
Könnyen belátható, hogy ha a fenti $\p_{lj}, \, \hat{\p}_{lj}$ (összesen $2\nu_l$ darab) függvényt kicseréljük a $\phi_{lj}, \, \hat{\phi}_{lj}$ (ugyancsak $2\nu_l$ darab) függvényre, akkor továbbra is lineárisan független függvényrendszert kapunk. Ha ezt a cserét a $P$ polinom minden nem valós gyökével kapcsolatban megtesszük, akkor az $\mathcal{M}_h$ egy valós értékű függvényekből álló bázisát kapjuk, azaz egy valós függvényekből álló alaprendszert. 
    \newpage
\section{Vizsgakérdés}
\begin{quote}
	\textit{Partikuláris megoldás kvázi-polinom jobb oldal esetén (a bizonyítás vázlata). A csillapítás nélküli kényszerrezgés vizsgálata, rezonancia.}
\end{quote}
    \newpage
\section{Vizsgakérdés}
\begin{quote}
	\textit{A függvénysorozat, függvénysor fogalma. Hatványsorok, trigonometrikus sorok, Fourier-sorok. A Dirichlet-féle magfüggvény. Konvergencia, határfüggvény (összegfüggvény), egyenletes konvergencia. A Weierstrass-féle majoráns kritérium.}
\end{quote}

\subsection{Függvénysorozatok, függvénysorok}

A függvénysorozatok fogalmával részben találkoztunk már korábban is: az $(f_n)$ sorozatot \textit{függvénysorozatnak} nevezzük, ha minden $n \in \N$ esetén az $f_n$ függvény. A továbbiakban mindig azzal a feltételezéssel élünk, hogy valamilyen $\emptyset \neq X$ halmazzal
\[
	f_n \in X \to \K \quad (n \in \N),
\]
és egy $\emptyset \neq \DD \subset X$ halmazzal
\[
	\DD_{f_n} = \DD \quad (n \in \N).
\]
Pl. a
\[
	h_n(t) := t^n \quad (t \in \DD := \R, \, n \in \N)
\]
függvények egy $(h_n)$ függvénysorozatot határoznak meg.\\

A fenti $(f_n)$ függvénysorozat által meghatározott $\sum(f_n)$ \textit{függvénysor}:
\[
	\sum(f_n) := \left( \sumk f_k \right).
\]
A $\sum(f_n)$ függvénysor tehát nem más, mint az
\[
	F_n := \sumk f_k \quad (n \in \N)
\]
\textit{részletösszegfüggvények} által meghatározott $(F_n)$ függvénysorozat:
\[
	\sum(f_n) := (F_n).
\]
Így pl. az előbbi
\[
	h_n(t) := t^n \quad (t \in \R, \, n \in \N)
\]
függvények esetén $\sum(h_n) = (H_n)$, ahol az $n \in \N$ indexekre
\[
	H_n(t) = \sumk h_k(t) = \sumk t^k = \begin{cases}
		n + 1 & (t = 1) \\
		\displaystyle \frac{1-t^{n+1}}{1-t} & (t \neq 1)
	\end{cases} \quad (t \in \R).
\]

\subsection{Konvergencia, határfüggvény}

Tekintsük a fenti $(f_n)$ függvénysorozatot. Ha egy $x \in \DD$ elem esetén konvergens a helyettesítési értékeknek az $\big(f_n(x)\big)$ sorozata, akkor azt mondjuk, hogy az $(f_n)$ függvénysorozat \textit{konvergens} az $x$ helyen. A
\[
	\DD_0 := \big\{ x \in \DD : \big(f_n(x)\big) \text{ konvergens} \big\}
\]
halmaz az $(f_n)$ függvénysorozat \textit{konvergenciatartománya}. Ha $\DD_0 \neq \emptyset$, akkor az
\[
	f(x) := \limn f_n(x) \quad (x \in \DD_0)
\]
definícióval értelmezett
\[
	f : \DD_0 \to \K
\]
függvény az $(f_n)$ függvénysorozat \textit{határfüggvénye}. A $\DD_0 = \D$ esetben röviden azt mondjuk, hogy az $(f_n)$ függvénysorozat \textit{pontonként konvergens}.\\

Pl. az előbbi $(h_n)$ függvénysorozattal $\DD_0 = (-1, \, 1]$, és
\[
	h(x) := \begin{cases}
		0 & (-1 < x < 1) \\
		1 & (x = 1)
	\end{cases}
\]
a $(h_n)$ sorozat határfüggvénye.\\

A függvénysorok "nyelvén" a pontonkénti konvergencia a következőképpen fogalmazható meg: legyen $X \neq \emptyset$, és a $\emptyset \neq \DD \subset X$ halmazzal adott az
\[
	f_n : \DD \to \K \quad (n \in \N)
\]
függvénysorozat. Ekkor a $\sum(f_n)$ függvénysor $x$-beli konvergenciája azt jelenti, hogy a részletösszegek $\displaystyle \Big(\sumk f_k \Big)$ sorozata konvergens az $x$ helyen, azaz a $\displaystyle \Big(\sumk f_k(x) \Big)$ sorozat konvergens. Nem fog félreértést okozni, ha az ilyen $x \in \DD$ elemek összegét fogjuk most $\DD_0$-val jelölni. Tehát $\DD_0$ most nem más, mint a $\displaystyle \Big(\sumk f_k \Big)$ függvénysorozat konvergenciatartománya. Ha $\DD_0 \neq \emptyset$, akkor legyen
\[
	F(x) := \limsum f_k(x) = \limn \sumk f_k(x) \quad (x \in \DD_0)
\]
A szóban forgó függvénysor \textit{összegfüggvénye}. Pl. a
\[
	h_n(x) := x^n \quad (x \in \R, \, n \in \N)
\]
függvényekkel
\[
	\sumk h_k(x) = \sumk x^k = \begin{cases}
		n + 1 & (x = 1) \\
		\displaystyle \frac{1 - x^{n+1}}{1-x} & (x \neq 1)
	\end{cases} \quad (x \in \R, \, n \in \N)
\]
miatt a $\sum(h_n)$ függvénysor konvergenciatartománya a $(-1, \, 1)$ intervallum, a $H$ összegfüggvénye pedig a
\[
	H(x) := \frac{1}{1-x} \quad (-1 < x < 1)
\]
függvény.\\

Emlékeztetünk a hatványsor fogalmára: legyen valamilyen $a \in \K$ \textit{középpont} és egy
\[
	(a_n) : \N \to \K
\]
\textit{együttható-sorozat} esetén
\[
	f_n(x) := a_n(x-a)^n \quad (x \in \R, \, n \in \N).
\]
Ekkor a
\[
	\sum\Big( a_n(x-a)^n \Big) := \sum(f_n)
\]
függvénysort neveztük \textit{hatványsornak}. A Cauchy-Hadamard-tétel szerint egyértelműen létezik olyan
\[
	0 \leq r \leq + \infty
\]
(\textit{konvergenciasugár}) amellyel a hatványsor $\DD_0$ konvergenciatartományára a $0 < r < + \infty$ esetben
\[
	K_r(a) \subset \DD_0 \subset \overline{K_r(a)}.
\]
Nyilvánvaló, hogy $a \in \DD_0$ mindig igaz, és az $a$ helyen a fenti hatványsor összege $0$.

\subsection{Trigonometrikus sorok, Fourier-sorok}
A $\sum(f_n)$ függvénysort \textit{trigonometrikus sornak} nevezzük, ha
\[
	f_0(x) := \alpha_0, \, f_n(x) := \alpha_n \cdot \cos(nx) + \beta_n \cdot \sin(nx) \quad (1 \leq n \in \N, \, x \in \R),
\]
ahol adottak az $\alpha_k \in \R \quad (k \in \N)$ és a $\beta_j \quad (1 \leq j \in \N)$ \textit{együtthatók}. Használni fogjuk minderre a
\[
	\trigseries
\]
szimbólumot is. Tehát egy adott trigonometrikus sor $n$-edik részletösszege egy $x \in \R$ helyen az alábbi módon néz ki:
\[
	\alpha_0 + \alpha_1 \cdot \cos(x) + \beta_1 \cdot \sin(x) + \cdots + \alpha_n \cdot \cos(nx) + \beta_n \cdot \sin(nx).
\]
A szóban forgó $\trigseries$ trigonometrikus sor
\[
	S_n(x) := \alpha_0 + \sum_{k=1}^n \Big( \alpha_k \cdot \cos(kx) + \beta_k \cdot \sin(kx) \Big) \quad (x \in \R, \, n \in \N)
\]
részletösszegfüggvényei \textit{trigonometrikus polinomok}.\\

Legyen $R_{2\pi}$ az összes olyan $2\pi$ szerint periodikus
\[
	f : \R \to \R
\]
függvény halmaza, amelyre
\[
	f \in R[0, \, 2\pi]
\]
teljesül. A periodicitás miatt nyilvánvaló, hogy ekkor tetszőleges $2\pi$-hosszúságú kompakt $I \subset \R$ intervallumra is (az előbbi értelemben) $f \in R(I)$.\\

Legyen továbbá $C_{2\pi}$ az olyan $2\pi$ szerint periodikus
\[
	f : \R \to \R
\]
függvények halmaza, amelyekre $f \in C$. Ekkor
\[
	C_{2\pi} \subset R_{2\pi},
\]
továbbá $C_{2\pi}, \, R_{2\pi}$ lineáris terek az $\R$-re vonatkozóan, a $C_{2\pi}$ altere az $R_{2\pi}$-nek. Továbbá bármely $f \in R_{2\pi}$ függvény az $f \in R[0, \, 2\pi]$ integrálhatóság miatt korlátos, azaz
\[
	\sup\{|f(x)| : x \in \R\} = \sup\{|f(x)| : x \in [0, \, 2\pi]\} < + \infty.
\]
Vezessük be az alábbi fogalmakat: $f \in R_{2\pi}$ esetén legyen
\[
	a_0(f) := a_0 := \frac{1}{2\pi} \int\limits_0^{2\pi} f(x) \, dx,
\]
\[
	a_n(f) := a_n := \frac{1}{\pi} \int\limits_0^{2\pi} f(x) \cdot \cos(nx) \, dx \quad (1 \leq n \in \N),
\]
\[
	b_n(f) := b_n := \frac{1}{\pi} \int\limits_0^{2\pi} f(x) \cdot \sin(nx) \, dx \quad (1 \leq n \in \N),
\]
\[
	Sf := \trigserieslatin \quad (n \in \N, \, x \in \R).
\]
Ekkor az $Sf$ trigonometrikus sor az $f$ \textit{Fourier-sora}, az együtthatói az $f$ \textit{Fourier-együtthatói}, az $S_nf \quad (n \in \N)$ trigonometrikus polinom pedig az $f$ függvény $n$-edik \textit{Fourier-részletösszege}.\\

Ha $f \in R_{2\pi}, \, n \in \N$, akkor a fenti $f$ \textit{Fourier-részletösszegei} a következők:
\[
	S_0f(x) = a_0 \quad (x \in \R),
\]
ill. $1 \leq n \in \N, \, x \in \R$ esetén
\[
	S_nf(x) = \frac{1}{2\pi} \int\limits_0^{2\pi} f(t) \, dt +
\]
\[
	\sum_{k=1}^n \left( \frac{1}{\pi} \int\limits_0^{2\pi} f(t) \cdot \cos(kt) \, dt \cdot \cos(kx) + \frac{1}{\pi} \int\limits_0^{2\pi} f(t) \cdot \sin(kt) \, dt \cdot \sin(kx) \right) =
\]
\[
	\frac{1}{2\pi} \int\limits_0^{2\pi} f(t) \, dt + \frac{1}{\pi} \int\limits_0^{2\pi} f(t) \cdot \sum_{k=1}^n \Big( \cos(kt) \cdot \cos(kx) + \sin(kt) \cdot \sin(kx) \Big) \, dt =
\]
\[
	\frac{1}{\pi} \int\limits_0^{2\pi} f(t) \cdot \left( \frac{1}{2} + \sum_{k=1}^n \cos\Big(k(x-t)\Big) \right) \, dt.
\]
Ha tehát
\[
	D_0(z) := \frac{1}{2}, \, D_n(z) := \frac{1}{2} + \sum_{k=1}^n \cos(kz) \quad (1 \leq n \in \N, \, z \in \R),
\]
akkor
\[
	S_nf(x) = \frac{1}{\pi} \int\limits_0^{2\pi} f(t) \cdot D_n(x-t) \, dt \quad (n \in \N, \, x \in \R).
\]
A most definiált $D_n \quad (n \in \N)$ függvény az $n$-edik \textit{Dirichlet-magfüggvény}. Világos, hogy minden $D_n$ páros függvény, periodikus $2\pi$ szerint, és bármilyen $2\pi$ hosszúságú kompakt $I \subset \R$ intervallumra
\[
	\int_I D_n = \int_I \frac{1}{2} \, dz + \sum_{k=1}^n \int_I \cos(kz) \, dz = \int_I \frac{1}{2} \, dz = \pi \quad (n \in \N).
\]
Nem nehéz "zárt" alakra hozni a szóvan forgó magfüggvényeket. Ha ui. $0 < u < 2\pi$ és $n \in \N$, akkor
\[
	\sin(z/2) \cdot D_n(z) = \frac{\sin(z/2)}{2} + \sum_{k=1}^n \sin(z/2) \cdot \cos(kz) =
\]
\[
	\frac{\sin(z/2)}{2} + \frac{1}{2} \cdot \sum_{k=1}^n \Bigg( \sin\Big( (k+ 1/2)z \Big) - \sin\Big( (k-1/2)z \Big) \Bigg) =
\]
\[
	\frac{\sin(z/2)}{2} + \frac{\sin\Big( (n+1 / 2)z  - \sin(z/2) \Big)}{2} = \frac{\sin\Big( (n+1/2)z \Big)}{2}.
\]
Innen az következik, hogy
\[
	D_n(z) = \frac{\sin\Big( (n+1/2)z \Big)}{2 \cdot \sin(z/2)} \quad (0 < z < 2\pi).
\]
Tehát a $D_n$ definíciójából adódóan a
\[
	\frac{\sin\Big( (n+1/2)0 \Big)}{2 \cdot \sin(0/2)} := D_n(0) = \frac{1}{2} + n
\]
megállapodással tetszőleges $f \in R_{2\pi}$ függvényre az alábbi integrál-előállítást kapjuk a Fourier-részletösszegekre:
\[
	S_nf(x) = \frac{1}{\pi} \int\limits_0^{2\pi} f(t) \cdot D_n(x-t) \, dt \quad (n \in \N, \, x \in \R).
\]

\subsection{Egyenletes konvergencia}
Tekintsük az $(f_n)$ függvénysorozatot, ahol
\[
	f_n \in X \to \K \quad (n \in \N)
\]
és
\[
	\DD_{f_n} =: \DD \quad (n \in \N).
\]
Legyen
\[
	\DD_0 := \Big\{ t \in \DD : \big( f_n(x)\big) \text { konvergens} \Big\} \neq \emptyset
\]
az $(f_n)$ konvergenciatartománya, és 
\[
	f(x) := \limn f_n(x) \quad (x \in \DD_0)
\]
az $(f_n)$ függvénysorozat határfüggvénye. Tehát $f : \DD_0 \to \K$ és tetszőleges $x \in \DD_0$, valamint $\varepsilon > 0$ esetén van olyan $N_{x, \, \varepsilon} \in \N$, hogy
\[
	|f_n(x) - f(x)| < \varepsilon \quad (N_{x, \, \varepsilon} < n \in \N).
\]
Hangsúlyozni kell, hogy az itt szereplő $N_{x, \, \varepsilon}$ küszöbindex általában függ az $x$-től is, és az $\varepsilon$-tól is. Elképzelhető ugyanakkor, hogy bizonyos esetekben bármilyen $\varepsilon > 0$ mellett olyan (csak az $\varepsilon$-tól függő)
\[
	N := N_\varepsilon \in \N
\]
is megadható, amelyik az előbbi becslésben egy $\emptyset \neq A \subset \DD_0$ halmaz mellett független az $x \in A$ elemtől. Ekkor azt mondjuk, hogy az $(f_n)$ függvénysorozat az $A$ halmazon \textit{egyenletesen konvergens} az $f$ függvényhez, azaz: minden $\varepsilon > 0$ számhoz létezik olyan $N \in \N$, amellyel
\[
	|f_n(x) - f(x)| < \varepsilon \quad (x \in A, \, N < n \in \N).
\]
Világos, hogy ekkor minden $\emptyset \neq B \subset A$ halmaz esetén is az $(f_n)$ sorozat egyenletesen konvergál a $B$-n az $f$-hez. Ha az egyenletes konvergencia definíciójában $A = \DD_0$ írható, akkor egyszerűen azt mondjuk, hogy az $(f_n)$ függvénysorozat \textit{egyenletesen konvergens}.\\

A $\sum(f_n)$ függvénysor egyenletesen konvergens az $\emptyset \neq A \subset \DD_0$ halmazon, ha a részletösszegek $\series$ sorozata egyenletesen konvergens az $A$-n.\\

Ez tehát azt jelenti, hogy létezik olyan
\[
	F : A \to \K
\]
függvény és tetszőleges $\varepsilon > 0$ számhoz van olyan $N \in \N$, amellyel
\[
	\left| F(x) - \sumk f_k(x) \right| < \varepsilon \quad (x \in A, \, N < n \in \N).
\]
A Cauchy-kritérium miatt ez azzal ekvivalens, hogy bármilyen $\varepsilon > 0$ esetén egy alkalmas $N \in \N$ természetes számmal
\[
	\left|  \sum_{k=n+1}^m f_k(x) \right| < \varepsilon \quad (x \in A, \, N < n, \, m \in \N, \, n < m)
\]
(\textit{egyenletes Cauchy-kritérium}).\\

\subsection{Weierstrass-kritérium}

\tikz \node[theorem]
{
	\textbf{Tétel.} Tegyük fel, hogy valamilyen $\emptyset \neq X$ mellett $\emptyset \neq \mathcal{D} \subset X$, és adott az
	\[
		f_n : \mathcal{D} \to \K \quad (n \in \N)
	\]
	függvények által meghatározott $\sum(f_n)$ függvénysor. Legyen továbbá egy $\emptyset \neq A \subset \mathcal{D}$ halmazzal és egy $(a_n)$ számsorozattal
	\[
		\sup \{ |f_n(x)| : x \in A \} \leq a_n \quad (n \in \N),
	\]
	ahol $\sum_{n=0}^\infty a_n < + \infty$. Ekkor a $\sum(f_n)$ függvénysor az $A$ halmazon egyenletesen konvergens. 
};\\

\textbf{Bizonyítás.} Az alábbi becslés a tétel feltételei alapján nyilvánvaló:
\[
	\left|  \sum_{k=n+1}^m f_k(x) \right| \leq \sum_{k=n+1}^m |f_k(x)| \leq \sum_{k=n+1}^m a_k \quad (x \in A, \, n, \, m \in \N, \, n < m).
\]
Ha az $\varepsilon > 0$ egy pozitív szám, akkor a $\sum_{n=0}^\infty a_n < + \infty$ feltételezés miatt van olyan $N \in \N$, amellyel
\[
	\sum_{k=n+1}^m a_k < \varepsilon \quad (n, \, m \in \N, \, N < n < m).
\]
Ezért
\[
	\left|  \sum_{k=n+1}^m f_k(x) \right| < \varepsilon \quad (x \in A, \, n, \, m \in \N, \, N < n < m).
\]
$\hfill \blacksquare$

\end{document}