\documentclass[12pt]{article}
\usepackage[left=0.9in, right=0.9in, top=0.7in, bottom=0.7in]{geometry}
\usepackage{tikz}
\usepackage{setspace}
\usepackage{hyperref}
\usepackage{amsfonts, amssymb, amsmath} 
\usepackage{titlesec}
\usepackage{pgfplots}
\pgfplotsset{compat=newest}
\usepackage{graphicx}
\usepackage{wrapfig}
\usepackage{caption}
\usepackage{enumitem}
\usetikzlibrary{shadows.blur}
\usepackage{lmodern}
\setlength{\parskip}{0pt}
\setlength{\parindent}{0pt}

\title{\textcolor{purple}{\Huge\textbf{Analízis IV}}}
\date{Szabó Krisztián}

\renewcommand{\contentsname}{Tartalom}
\newcommand{\R}{\mathbb{R}}
\newcommand{\N}{\mathbb{N}}
\newcommand{\E}{\exists}
\newcommand{\mm}{\mathbf{m}}
\newcommand{\MM}{\mathbf{M}}
\newcommand{\K}{\mathbb{K}}
\newcommand{\D}{\mathcal{D}_f}

\definecolor{modernyellow}{HTML}{F4E4BC}
\definecolor{moderngreen}{HTML}{BDDCBD}

\tikzset
{
	definition/.style={
		draw,
		fill=modernyellow,
		line width=1pt,
		rounded corners,
		drop shadow={shadow blur steps=5,shadow xshift=1ex,shadow yshift=-1ex},
		text width=0.9\textwidth,
		inner sep=10pt
	},
	theorem/.style={
		draw,
		fill=white,
		line width=1pt,
		rounded corners,
		drop shadow={shadow blur steps=5,shadow xshift=1ex,shadow yshift=-1ex},
		text width=0.9\textwidth,
		inner sep=10pt
	},
	proof/.style={
		fill=white,
		rectangle,
		drop shadow={shadow blur steps=5,shadow xshift=1ex,shadow yshift=-1ex, moderngreen},
		text width=0.9\textwidth,
		inner sep=6pt,
	},
	proof1/.style={
		fill=white,
		rectangle,
		drop shadow={shadow blur steps=5,shadow xshift=1ex,shadow yshift=0, moderngreen},
		text width=0.9\textwidth,
		inner sep=6pt,
	}
}

\begin{document}
    \section*{Taylor-formula}
    \tikz \node[theorem]
    {
        \textbf{Tétel.} Tegyük fel, hogy az $f \in \R^n \to \R$ függvényre valamilyen $s \in \N$ mellett $f \in D^{s+1}$ teljesül. Ekkor bármely $[a, b] \subset \D$ esetén van olyan $c \in [a, \, b]$, hogy
        \[
            f(b) = T_{a, \, s}f(b) + \sum_{i \in \N^n, \, |i| = s+1} \frac{\partial^i f(c)}{i!} \cdot (b - a)^i.
        \]
    };\newline

    \textbf{Bizonyítás:} Lásd Simon P. - Analízis III. 123. oldal.\newline

    Legyen
    \[
        1 \leq n, \, k \in \N, \, f \in \R^n \to \R, \, a \in \text{int} \, \D, \, j = 1, \, \dots, \, n,
    \]
    és
    \[
        \partial^k_j f(a) := \partial_{j\dots j}f(a).
    \]
    Ha $i := (i_1, \, \dots, \, i_n) \in \N^n$, akkor legyen
    \[
        \partial^i f(a):= \partial_1^{i_1} \dots \partial_n^{i_n} f(a) = \partial_{1 \dots 1 \dots n \dots n} f(a).
    \]
    Az $i \in \N^n$ \textit{multiindex} esetén $i$ \textit{hosszát} a következőképpen definiáljuk:
    \[
        |i| := ||i||_1 = \sum_{j=1}^n i_j.
    \]
    Legyen az $i = (i_1, \, \dots, \, i_n) \in \N^n$ multiindex és az $x = (x_1, \, \dots, \, x_n) \in \R^n$ vektor esetén
    \[
        i! := \prod_{j=1}^n i_j! = \prod_{j=1}^n \prod_{k=1}^{i_j} k
    \]
    és
    \[
        x^i := \prod_{j=1}^n x_j^{i_j}
    \]
    az $i$ \textit{faktoriálisa}, ill. az $x$ vektor $i$-kitevős \textit{hatványa.}\\

    Tekintsük ezek után az
    \[
        f \in \R^n \to \R
    \]
    függvényt, és tegyük fel, hogy valamilyen $a \in \text{int} \, \D, \, s \in \N$ mellett $f \in D^s\{a\}$. Ekkor a
    \[
        T_{a, \, s} f(x) := \sum_{k=0}^s \sum_{i \in \N^n, \, |i| = k} \frac{\partial^i f(a)}{i!} \cdot (x - a)^i \quad (x \in \R^n)
    \]
    előírással definiált $T_{a, \, s} f : \R^n \to \R$ függvényt az $f$ függvény $a$-hoz tartozó $s$-edrendű \textit{Taylor-polinomjának} nevezzük.

    \newpage
    \section*{Lagrange-féle középértéktétel}
    \tikz \node[theorem]
    {
        \textbf{Tétel.} Legyen adott a differenciálható $f \in \R^n \to \R$ függvény, és valamilyen $a, \, b \in \R^n, \, a \neq b$ végpontokkal $[a, \, b] \subset \D$. Ekkor egy alkalmas $c \in (a, \, b)$ mellett
        \[
            f(b)-f(a) = \langle \text{grad} \, f(c), \, b-a \rangle.
        \]
    };\newline

    \textbf{Bizonyítás:} Lásd Simon P. - Analízis III. 125. oldal.\newline

    Valamilyen $1 \leq n, \, m \in \N$ mellett legyen $f = (f_1, \, \dots, \, f_m) \in \R^n \to \R^m$, $f \in D$. Tegyük fel, hogy az $a, \, b \in \R^n$, $a \neq b$ végpontokkal meghatározott szakaszra $[a, \, b] \subset \D$. Ekkor minden $i = 1, \, \dots, \, m$ mellett egy alkalmas $\xi^{(i)} \in (a, \, b)$ helyen a
    \[
        h := (h_1, \, \dots, \, h_n) := b-a
    \]
    jelöléssel
    \[
        f_i(b)-f_i(a) = \langle \text{grad} \, f_i(\xi^{(i)}), \, h \rangle = \sum_{j=1}^n \partial_j f_i(\xi^{(i)}) \cdot h_j,
    \]
    ezért
    \[
        |f_i(b)-f_i(a)| \leq \sum_{j=1}^n |\partial_j f_i(\xi^{(i)})| \cdot |h_j| \leq \sum_{j=1}^n |\partial_j f_i(\xi^{(i)})| \cdot ||h||_\infty \leq
    \]
    \[
        \leq \text{sup} \left\{ \sum_{j=1}^n |\partial_j f_i(x)| : x \in (a, \, b)\right\} \cdot ||h||_\infty.
    \]
    Következésképpen
    \[
        ||f(b)-f(a)||_\infty = \text{max} \, \{ |f_i(b)-f_i(a)| : i = 1, \, \dots, \, m \} = 
    \]
    \[
        \text{max} \{ |\langle \text{grad} \, f_i(\xi^{(i)}), \, h \rangle| : i = 1, \, \dots, \, m\} \leq
    \]
    \[
        \leq \text{max} \, \left\{ \text{sup} \, \left\{ \sum_{j=1}^n |\partial_j f_i(x)| : x \in (a, \, b) \right\} : i = 1, \, \dots, \, m\right\} \cdot ||h||_\infty =
    \]
    \[
        \leq \text{sup} \, \left\{ \text{max} \, \left\{ \sum_{j=1}^n |\partial_j f_i(x)| : i = 1, \, \dots, \, m \right\} : x \in (a, \, b) \right\} \cdot ||h||_\infty =
    \]
    \[
        \text{sup} \, \{ ||f'(x)||_{(\infty)}  : x \in (a, \, b)\} \cdot ||h||_\infty.
    \]
    Ha tehát az $\R^n$-en és az $\R^m$-en is a $||.||_\infty$ vektornormát vezetjük be, akkor az $f'(x)$ $(x \in \D)$ Jacobi-mátrix által generált $||f'(x)||_\infty$ (sor)normáját tekintve a
    \[
        q := \text{sup} \, \{ ||f'(x)||_{(\infty)} :x \in (a, \, b) \}
    \]
    szimbólummal
    \[
        \boxed{||f(b)-f(a)||_\infty \leq q \cdot ||b-a||_\infty}.
    \]
    \[
    A :=
    \begin{bmatrix}
        \text{grad} \, f_1(\xi^{(1)}) \\
        \text{grad} \, f_2(\xi^{(2)}) \\
        \vdots \\
        \text{grad} \, f_m(\xi^{(m)})
    \end{bmatrix}
    \in \R^{m \times n} \text{ mátrixszal } f(b)- f(a) = A(b-a).
    \]

    \newpage
    \section*{Elsőrendű szükséges feltétel}
    \tikz \node[theorem]
    {
        \textbf{Tétel.} Tegyük fel, hogy $1 \leq n, \, m \in \N, \, m < n, \, \emptyset \neq U \subset \R^n$ nyílt halmaz, és $f : U \to \R, \, g : U \to \R^m$. Ha $f \in D, \, g \in C^1$, az $f$-nek az a $c \in \{ g = 0 \}$ helyen feltételes lokákis szélsőértéke van a $g = 0$ feltételre vonatkozóan, továbbá a $g'(c)$ Jacobi-mátrix rangja megegyezik $m$-mel, akkor létezik olyan $\lambda \in \R^m$ vektor, hogy
        \[
            \text{grad} \, (f + \lambda g)(c) = 0.
        \]
    };\newline

    \textbf{Bizonyítás:} Lásd Simon P. - Analízis IV. 19. oldal.

    \section*{Másodrendű elégséges feltétel}
    \tikz \node[theorem]
    {
        \textbf{Tétel.} Az $1 \leq n, \, m \in \N, \, m < n$ paraméterek mellett legyen adott az $\emptyset \neq U \subset \R^n$ nyílt halmaz, és tekintsük az $f : U \to \R, \, g : U \to \R^m$ függvényeket. Feltesszük, hogy $f, \, g \in D^2, \, c \in \{g=0\}$, a $g'(c)$ mátrix rangja $m$, továbbá valamilyen $\lambda \in \R^m$ vektorral az $F := f + \lambda g$ Lagrange-függvényre
        \begin{enumerate}
            \item $\text{grad} \, F(c) = 0$;
            \item a $Q_c^F$ kvadratikus alak a $g'(c)$ mátrixra nézve feltételesen pozitív (negatív) definit.
        \end{enumerate}
        Ekkor az $f$-nek a $c$-ben a $g=0$ feltételre vonatkozóan feltételes lokális minimuma (maximuma) van.
    };\newline

    \textbf{Bizonyítás:} Nincs - nehéz.

    \section*{Másodrendű szükséges feltétel}
    \tikz \node[theorem]
    {
        \textbf{Tétel.} Legyen $1 \leq n, \, m \in \N, \, m < n$, és az $\emptyset \neq U \subset \R^n$ nyílt halmazon legyenek adottak az $f : U \to \R, \, g : U \to \R^m$ függvények. Feltesszük, hogy $f, \, g \in D^2, \, c \in \{g=0\}$, a $g'(c)$ mátrix rangja $m$, és az $f$-nek a $c$-ben a $g=0$ feltételre vonatkozóan feltételes lokális minimuma (maximuma) van. Ekkor létezik olyan $\lambda \in \R^m$ vektor, amellyel az $F := f + \lambda g$ Lagrange-függvényre az alábbiak teljesülnek:
        \begin{enumerate}
            \item $\text{grad} \, F(c) = 0$;
            \item a $Q_c^F$ kvadratikus alak a $g'(c)$ mátrixra nézve feltételesen pozitív (negatív) szemidefinit.
        \end{enumerate}
    };\newline

    \textbf{Bizonyítás:} Nincs - nehéz.

    \newpage
    \section*{Implicitfüggvény-tétel}
    \tikz \node[theorem]
    {
        \textbf{Tétel.} Adott $n, \, m \in \N, \, 2 \leq n$, valamint $1 \leq m < n$ mellett az
        \[
            f \in \R^{n-m} \times \R^{m} \to \R^m
        \]
        függvényről tételezzük fel az alábbiakat: $f \in C^1$, és az $(a, \, b) \in \text{int} \, \D$ helyen
        \[
            f(a, \, b) = 0, \, \text{det} \, \partial_2 f(a, \, b) \neq 0.
        \]
        Ekkor alkalmas $K(a), \, K(b)$ környezetekkel létezik az $f$ által az $(a, \, b)$ körül meghatározott
        \[
            \varphi : K(a) \to K(b)
        \]
        implicitfüggvény, ami folytonosan differenciálható, és
        \[
            \varphi'(x) = -\partial_2 f(x, \, \varphi(x))^{-1} \cdot \partial_1 f(x, \, \varphi(x)) \quad (x \in K(a)).
        \]
    };\newline

    \textbf{Bizonyítás:} Nincs - nehéz.

    \section*{Inverzfüggvény-tétel I.}
    \tikz \node[theorem]
    {
        \textbf{Tétel.} Tegyük fel, hogy az $f \in \R^n \to \R^n$ függvény folytonosan differenciálható az $a \in \text{int} \, \D$ pontban, és az $a$-beli Jacobi-mátrixa invertálható. Ekkor az $f$ függvény $a$-ban lokálisan invertálható, és az $a$-beli lokális inverze folytonos.
    };\newline

    \textbf{Bizonyítás:} Lásd Simon P. - Analízis IV. 16. oldal.

    \section*{Inverzfüggvény-tétel II.}
    \tikz \node[theorem]
    {
        \textbf{Tétel.} Tegyük fel, hogy $1 \leq n \in \N$, az $f \in \R^n \to \R^n$ függvény folytonosan differenciálható, az $a \in \text{int} \, \D$ pontban az $f'(a)$  Jacobi-mátrixra $\text{det} \, f'(a) \neq 0$ teljesül. Ekkor alkalmas $K(a) \subset \D$ környezettel az $f_{|K(a)}$ leszűkítés invertálható, a $h := (f_{|K(a)})^{-1}$ lokális inverfüggvény folytonosan differenciálható, és
        \[
            h'(x) = \big( f'(h(x)) \big)^{-1} \quad (x \in \mathcal{D}_h).
        \]
    };\newline

    \textbf{Bizonyítás:} Nincs - nehéz.

    \newpage
    \section{Közönséges differenciálegyenletek}
    Legyen $0 < n \in \N, \, I \subset \R, \, \Omega \subset \R^n$ egy-egy nyílt intervallum. Tegyük fel, hogy az
    \[
        f : I \times \Omega \to \R^n
    \]
    függvény folytonos, és tűzzük ki az alábbi feladat megoldását: határozzunk meg olyan $\varphi \in I \to \Omega$ függvényt, amelyre igazak a következő állítások:

    \begin{enumerate}
        \item $\mathcal{D}_\varphi$ nyílt intervallum;
        \item $\varphi \in \D$;
        \item $\varphi'(x) = f(x, \, \varphi(x)) \quad (x \in \mathcal{D}_\varphi)$.
    \end{enumerate}

    A most megfogalmazott feladatot \textit{explicit elsőrendű közönséges differenciálegyenletnek} (röviden \textit{differenciálegyenletnek}) fogjuk nevezni, és a \textit{d.e.} rövidítéssel idézni.\newline

    Ha adottak a $\tau \in I, \, \xi \in \Omega$ elemek, akkor a fenti $\phi$ függvény $1., \, 2.$ és $3.$ mellett tegyen eleget a
    \begin{enumerate}[start=4]
        \item $\tau \in \mathcal{D}_\varphi$ és $\varphi(\tau) = \xi$
    \end{enumerate}
    kikötésnek is. Az így "kibővített" feladatot \textit{kezdetiérték-problémának} (vagy röviden \textit{Cauchy-feladatnak}) nevezzük, és a továbbiakban minderre a \textit{k.é.p.} rövidítést fogjuk használni. 

    \section{Szeparábilis differenciálegyenlet}


\end{document}