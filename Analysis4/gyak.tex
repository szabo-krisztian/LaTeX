\documentclass{article}
\usepackage{tikz}

\usepackage{hyperref}
\usepackage{amsfonts, amssymb, amsmath}
\usepackage{titlesec}
\usepackage{pgfplots}
\pgfplotsset{compat=newest}



\usepackage{enumitem}
\usetikzlibrary{shadows.blur}
\usepackage{lmodern}

\usepackage[utf8]{inputenc}    % For UTF-8 encoding
\usepackage[T1]{fontenc}       % For proper character output
\usepackage[hungarian]{babel}  % For Hungarian language support


\setlength{\parskip}{0pt}
\setlength{\parindent}{0pt}

\title{\textcolor{purple}{\Huge\textbf{Analízis 4}}}
\author{Gyakorlati feladatok}
\date{}

\renewcommand{\contentsname}{Tartalom}
\newcommand{\R}{\mathbb{R}}
\newcommand{\N}{\mathbb{N}}
\newcommand{\E}{\exists}
\newcommand{\mm}{\mathbf{m}}
\newcommand{\MM}{\mathbf{M}}
\newcommand{\K}{\mathbb{K}}
\newcommand{\D}{\mathcal{D}_f}
\newcommand{\Dp}{\mathcal{D}_\varphi}
\newcommand{\p}{\varphi}

\definecolor{modernyellow}{HTML}{F4E4BC}
\definecolor{moderngreen}{HTML}{BDDCBD}

\titleformat{\section}[block]{\large\bfseries}{\thesection}{1em}{}
\titlespacing*{\section}{0pt}{1.5ex plus 1ex minus .2ex}{1.3ex plus .2ex}

\tikzset
{
    definition/.style={
        draw,
        fill=white,
        line width=1pt,
        rounded corners,
        drop shadow={shadow blur steps=5,shadow xshift=1ex,shadow yshift=-1ex},
        text width=1\textwidth,
        inner sep=8pt,
        align=justify
    },
    theorem/.style={
        draw,
        fill=white,
        line width=1pt,
        rounded corners,
        drop shadow={shadow blur steps=5,shadow xshift=1ex,shadow yshift=-1ex},
        text width=1\textwidth,
        inner sep=8pt,
        align=justify
    },
    proof/.style={
            fill=white,
            rectangle,
            drop shadow={shadow blur steps=5,shadow xshift=1ex,shadow yshift=-1ex, moderngreen},
            text width=1\textwidth,
            inner sep=8pt,
            align=justify
    },
    proof1/.style={
            fill=white,
            rectangle,
            drop shadow={shadow blur steps=5,shadow xshift=1ex,shadow yshift=0, moderngreen},
            text width=1\textwidth,
            inner sep=8pt,
            align=justify
    }
}

\begin{document}
    \maketitle
    \tableofcontents
    \newpage
	
	\section{Gyakorlat}
	\subsection{Emlékeztető}

	\tikz \node[definition]
	{
		\textbf{Definíció.} Legyen $1 \leq s \in \N, \, A \subset \R^s$. Azt mondjuk, hogy az $A$ halmaz \textit{nullmértékű}, ha tetszőleges $\varepsilon > 0$ számhoz megadható $I_k \subset \R^s \, (k \in \N)$ intervallumoknak egy olyan sorozata, hogy
		\[
			A \subset \bigcup_{k=0}^\infty I_k \text{ és } \sum_{k=0}^\infty |I_k| < \varepsilon.
		\]
	};\\

	Tekintsük pl. az $I \subset \R^n \, (1 \leq n \in \N)$ intervallum esetén az $f \in R(I)$ függvényt, és legyen
	\[
		\text{graf} \,  f := \{ (x, \, f(x)) \in \R^{n+1} : x \in I \}.
	\]
	Ekkor a $\text{graf} \, f \subset \R^{n+1}$ halmaz nullmértékű.\\

	Ha az $f \in \R^n \to \R$ függvény $\D \subset \R^n$ értelmezési tartománya korlátos, akkor van olyan $I \subset \R^n$ intervallum, amellyel $\D \subset I.$ Legyen ekkor
	\[
		F_I(x) := \begin{cases}
			f(x) & (x \in \D) \\
			0 & (x \in I \, \backslash \, \D).
		\end{cases}
	\]
	Nem nehéz meggondolni, hogy ha $F_I \in R(I)$, akkor $F_J \in R(J)$ minden olyan $J \subset \R^n$ intervallumra, amelyre $\D \subset J$ és
	\[
		\int_I F_I = \int_J F_J.
	\]
	Ez ad értelmet a következő definíciónak:\\

	\tikz \node[definition]
	{
		\textbf{Definíció.} A fenti $f$ függvény \textit{Riemann-integrálható}, ha egy alkalmas $I \subset \R^n$ intervallummal $\D \subset I$ és $F_I \in R(I)$. Az utóbbi esetben
		\[
			\int_{\R^n} f := \int_{\D} f := \int_I F_I.
		\]
	};\\

	Az előző definició olyan függvényekre is kiterjeszthető, amik értelmezési tartománya nem korlátos, viszont a
	\[
		\text{supp}\, f := \overline{\{ x \in \D : f(x) \neq 0 \}}
	\]
	\textit{tartója} igen.
	
	\tikz \node[definition]
	{
		\textbf{Definíció.} Legyen az $A \subset \R^n$ halmaz korlátos,
		\[
			\chi_A(x) := \begin{cases}
				1 & (x \in A) \\
				0 & (x \in \R^n \, \backslash \, A)
			\end{cases}
		\]
		az $A$ \textit{karakterisztikus függvénye}. Azt mondjuk, hogy az $A$ halmaz \textit{Jordan-mérhető}, ha a $\chi_A$ függvény integrálható, amikor is a
		\[
			\mu(A) := \int_{\R^n} \chi_A
		\]
		nemnegatív szám az $A$ halmaz \textit{Jordan-mértéke}. 
	};\\

	\tikz \node[theorem]
	{
		\textbf{Tétel.} Tekintsük a nyílt halmazon értelmezett és folytonosan differenciálható
		\[
			g \in \R^n \to \R^n
		\]
		függvényt. Tegyük fel, hogy az $I \subset \mathcal{D}_g$ halmaz kompakt intervallum, továbbá az $I$ belsejére való $g_{|_{\text{int} \, I}}$ leszűkítés injektív függvény. Ekkor az
		\[
			f : g(I) \to \R
		\]
		korlátos függvény akkor és csak akkor integrálható, ha az
		\[
			I \ni x \mapsto f(g(x)) \cdot |\det g'(x)|
		\]
		függvény is integrálható. Az utóbbi esetben
		\[
			\int_I f(g(x)) \cdot |\det g'(x)| \, dx = \int_{g[I]} f.
		\]
	};\\

	\textbf{Speciális esetek.}

	\begin{enumerate}
		\item \textbf{Síkbeli polárkoordináta-transzformáció.} Legyen $g : \R^2 \to \R^2$ leképezés, amire
		\[
			g(r, \, \varphi) := \begin{pmatrix}
				r \cdot \cos \varphi, \\
				r \cdot \sin \varphi
			\end{pmatrix} \quad \big( (r, \, \varphi) \in \R^2 \big).
		\]
		Ekkor $g \in C^1$, és
		\[
			\det g'(r, \, \varphi) = r \quad \big( (r, \, \varphi) \in \R^2 \big).
		\]
		\item \textbf{Módosított síkbeli polárkoordináta-transzformáció.} Legyen $a, \, b \in \R, \, g : \R^2 \to \R^2$ leképezés, amire
		\[
			g(r, \, \varphi) := \begin{pmatrix}
				ar \cdot \cos \varphi, \\
				br \cdot \sin \varphi
			\end{pmatrix} \quad \big( (r, \, \varphi) \in \R^2 \big).
		\]
		Ekkor $g \in C^1$, és
		\[
			\det g'(r, \, \varphi) = abr \quad \big( (r, \, \varphi) \in \R^2 \big).
		\]
		\item \textbf{Térbeli polárkoordináta-transzformáció.} Legyen $g : \R^3 \to \R^3$ leképezés, amire
		\[
			g(r, \, \varphi, \, \psi) := \begin{pmatrix}
				r \cdot \cos \varphi \sin \psi \\
				r \cdot \sin \varphi \sin \psi \\
				r \cdot \cos \psi
			\end{pmatrix} \quad \big( (r, \, \varphi, \, \psi) \in \R^3\big).
		\]
		Ekkor $g \in C^1$, és
		\[
			\det g'(r, \, \varphi, \, \psi) = -r^2 \cdot \sin \psi \quad \big( (r, \, \varphi, \, \psi) \in \R^3\big).
		\]
		\item \textbf{Módosított térbeli polárkoordináta-transzformáció (elliptikus koordináták).} Legyen $a, \, b, \, c \in \R, \, g : \R^3 \to \R^3$ leképezés, amire
		\[
			g(r, \, \varphi, \, \psi) := \begin{pmatrix}
				ar \cdot \cos \varphi \sin \psi \\
				br \cdot \sin \varphi \sin \psi \\
				cr \cdot \cos \psi
			\end{pmatrix} \quad \big( (r, \, \varphi, \, \psi) \in \R^3\big).
		\]
		Ekkor $g \in C^1$, és
		\[
			\det g'(r, \, \varphi, \, \psi) = -abcr^2 \cdot \sin \psi \quad \big( (r, \, \varphi, \, \psi) \in \R^3\big).
		\]
		\item \textbf{Hengerkoordináta-transzformáció.} Legyen $g : \R^3 \to \R^3$ leképezés, amire
		\[
			g(r, \, \varphi, \, z) := \begin{pmatrix}
				r \cdot \cos \varphi \\
				r \cdot \sin \varphi \\
				z
			\end{pmatrix} \quad \big( (r, \, \varphi, \, z) \in \R^3 \big).
		\]
		Ekkor $g \in C^1$, és
		\[
			\det g'(r, \, \varphi, \, z) = r \quad \big( (r, \, \varphi, \, z) \in \R^3 \big).
		\]
		\item \textbf{Módosított hengerkoordináta-transzformáció.} Legyen $a, \, b, \, c \in \R, \, g : \R^3 \to \R^3$ leképezés, amire
		\[
			g(r, \, \varphi, \, z) := \begin{pmatrix}
				ar \cdot \cos \varphi \\
				br \cdot \sin \varphi \\
				cz
			\end{pmatrix} \quad \big( (r, \, \varphi, \, z) \in \R^3 \big).
		\]
		Ekkor $g \in C^1$, és
		\[
			\det g'(r, \, \varphi, \, z) = abcr \quad \big( (r, \, \varphi, \, z) \in \R^3 \big).
		\]
	\end{enumerate}

	\tikz \node[theorem]
	{
		\textbf{Tétel.} Tegyük fel, hogy valamely $\emptyset \neq \Omega \subset \R^3$ Jordan-mérhető ponthalmazzal jellemzett test sűrűsége a Riemann-integrálható $\rho : \Omega \to [0, \, +\infty)$ függvénnyel írható le. Ekkor $\Omega$-nak valamely $0 \neq e \in \R^3$ irányvektorú, ill. $a \in \R^3$ pontra illeszkedő tengelyére vonatkozó tehetetlenségi nyomatéka a
		\[
			\mathbf{\Theta}_t = \int_\Omega \ell^2 (r) \rho(r) \, dr
		\]
		valós szám, ahol $\ell(r)$ jelöli az $r \in \Omega$ pontnak a $t$ tengelytől mért távolságát:
		\[
			\ell(r) := \text{inf} \, \{ \|r - (a + \tau e)\| \in \R : \tau \in \R \}.
		\]
	};\\

	Világos, hogy ha $a = 0$ és $e \in \{e_x, \, e_y, \, e_z\}$, akkor $\Omega$-nak a koordináta-tengelyekre vonatkoztatott tehetetlenségi nyomatéka:
	\[
		\mathbf{\Theta}_{t_x} = \int_\Omega (y^2 + z^2) \rho(x, \, y, \, z) \, dx \, dy \, dz
	\]
	\[
		\mathbf{\Theta}_{t_y} = \int_\Omega (x^2 + z^2) \rho(x, \, y, \, z) \, dx \, dy \, dz
	\]
	\[
		\mathbf{\Theta}_{t_z} = \int_\Omega (x^2 + y^2) \rho(x, \, y, \, z) \, dx \, dy \, dz
	\]

	\tikz \node[definition]
	{
		\textbf{Definíció.} Tegyük fel, hogy valamely $\emptyset \neq \Omega \subset \R^3$ Jordan-mérhető ponthalmazzal jellemzett test (tömeg)sűrűsége a Riemann-integrálható $\rho : \Omega \to [0, \, + \infty)$ függvénnyel írható le. (Ha $\rho$ állandó, akkor a testet homogénnek nevezzük.) Ekkor az
		\[
			m(\Omega) := \int_\Omega \rho
		\]
		számot az $\Omega$ test tömegének nevezzük. Az $\Omega$ tömegközéppontjának koordinátái:
		\[
			\overline{x} := \frac{1}{m(\Omega)} \int_\Omega x \cdot \rho(x, \, y, \, z) \, dx \, dy \, dz,
		\]
		\[
			\overline{y} := \frac{1}{m(\Omega)} \int_\Omega y \cdot \rho(x, \, y, \, z) \, dx \, dy \, dz,
		\]
		\[
			\overline{z} := \frac{1}{m(\Omega)} \int_\Omega z \cdot \rho(x, \, y, \, z) \, dx \, dy \, dz.
		\]
	};\\

	\subsection{Feladatok}
	\subsubsection{Feladat}
	Számítsuk ki annak a korlátos és zárt térbeli testnek a \textit{térfogatát}, amelyet az alábbi egyenletű felületek fognak közre:
	\[
		z = x^2 + y^2 -1 \text{ és } z = 2 - x^2 - y^2.
	\]
	Az egyenletekből kapunk két paraboloidot. Ezek metszetét kell \textit{paraméterezni}. Válasszuk két részre a ponthalmazt. Nézzük meg az $y = 0$ síkban lévő pontokat:
	\[
		z = x^2 - 1 \text{ és } z = 2 - x^2.
	\]
	Ez két ellentétes irányba néző parabola, amik metszete
	\[
		x^2 - 1 = 2 - x^2 \quad \Longleftrightarrow \quad x_1 := - \sqrt{\frac{3}{2}}, \, x_2 := \sqrt{\frac{3}{2}}.
	\]
	Azaz az alábbi térbeli pontokban metszi egymást a két paraboloid:
	\[
		p_1 := \left( - \sqrt{\frac{3}{2}}, 0, \frac{1}{2} \right), \, p_2 := \left( \sqrt{\frac{3}{2}}, 0, \frac{1}{2} \right).
	\]
	Szimmetriai okok miatt, be tudjuk vezetni a következő hengerkoordináta-transzformációt:
	\[
		\Phi(r, \, \varphi, \, z) := \begin{pmatrix}
			r \cdot \cos \varphi \\
			r \cdot \sin \varphi \\
			z
		\end{pmatrix} \quad \big( (r, \, \varphi, \, z) \in \R^3 \big).
	\]
	Legyen
	\[
		H_1 := [0, \, \sqrt{3 / 2}] \times [0, \, 2 \pi] \times [1/2, \, 2-r^2] \subset \R^3,
	\]
	\[
		H_2 := [0, \, \sqrt{3 / 2}] \times [0, \, 2 \pi] \times [r^2-1, \, 1/2] \subset \R^3,
	\]
	Mivel ezek kompakt intervallumok, igaz lesz, hogy
	\[
		\int_{\mathcal{H}} 1 = \int_{H_1} 1 \cdot |r| \, dr \, d\varphi \, dz + \int_{H_2} 1 \cdot |r| \, dr \, d\varphi \, dz, 
	\]
	ahol
	\[
		\mathcal{H} := \Phi[H_1] \cup \Phi[H_2].
	\]
	
	\subsubsection{Feladat}
	Számítsuk ki annak a korlátos és zárt térbeli $T$ testnek a \textit{térfogatát, tömegét} és a $z$-tengelyre vonatkozó \textit{tehetetlenségi nyomatékát}, amelyet az alábbi egyenlőtlenségek definiálnak:
	\[
		\sqrt{x^2 + y^2} \leq z \text{ és } x^2 + y^2 + z^2 \leq 1,
	\]
	illetve a kitöltő anyag pontonkénti sűrűsége egyenesen arányos az origótól mért távolsággal.

	Tehát a sűrűségfüggvény adott $\alpha > 0$ mellett
	\[
		\rho(x, \, y, \, z) := \alpha \cdot \sqrt{x^2 + y^2 + z^2} \quad \big( (x, \, y, \, z) \in \R^3\big).
	\]

	Alkalmazzuk az alábbi térbeli polárkoordináta-transzformációt:
	\[
		\Phi(r, \, \varphi, \, \psi) := \begin{pmatrix}
			r \cdot \cos \varphi \sin \psi \\
			r \cdot \sin \varphi \sin \psi \\
			r \cdot \cos \psi
		\end{pmatrix} \quad \big( (r, \, \varphi, \, \psi) \in \R^3\big).
	\]
	Legyen
	\[
		H := [0, \, 1] \times [0, \, 2 \pi] \times [0, \, \pi / 4] \subset \R^3,
	\]
	ekkor $H$ kompakt intervallum és
	\[
		\Phi(H) = T.
	\]
	Térfogat:
	\[
		\int_H 1 \cdot |-r^2 \sin \psi| \, dr \, d\varphi \, d\psi,
	\]
	tömeg:
	\[
		\int_H \rho(r \cos \varphi \sin \psi, \, r \sin \varphi \sin \psi, \, r \cos \psi) \cdot |-r^2 \sin \psi| \, dr \, d\varphi \, d\psi,
	\]
	tehetetlenségi nyomaték:
	\[
		\int_H (r^2 \cos^2\varphi + r^2 \sin^2 \psi) \cdot \rho(r \cos \varphi \sin \psi, \, r \sin \varphi \sin \psi, \, r \cos \psi) \cdot |-r^2 \sin \psi| \, dr \, d\varphi \, d\psi.
	\]
	Számoljuk ki a tehetetlenségi nyomatékot. Először írjuk fel tetszőleges $(r, \, \varphi, \, \psi) \in H$ esetén az integrandust ($\alpha$-val történő osztás után)
	\[
		r^2 \big( \cos^2 \varphi \sin^2 \psi + \sin^2 \varphi \sin^2 \psi \big) \cdot r \sqrt{\cos^2 \varphi \sin^2 \psi + \sin^2 \varphi \sin^2 \psi + \cos^2 \psi} \cdot r^2 \sin \psi =
	\]
	\[
		r^5 \sin^2 \psi \cdot \sqrt{\cos^2 \varphi \sin^2 \psi + \sin^2 \varphi \sin^2 \psi + 1 - \sin^2 \psi} \cdot \sin \psi =
	\]
	\[
		r^5 \sin^3 \psi \cdot \sqrt{ \sin^2 \psi \big( \cos^2 \varphi + \sin^2 \varphi - 1 \big) + 1} =
	\]
	\[
		r^5 \sin^3 \psi.
	\]
	Tehát
	\[
		\frac{\mathbf{\Theta}_{t_x}}{\alpha} = \int_H r^5 \sin^3 \psi \, dr \, d\varphi \, d \psi =
	\]
	\[
		\int\limits_0^{\pi / 4} \int\limits_0^{2 \pi} \int \limits_0^{1} r^5 \sin^3 \psi \, dr \, d\varphi \, d\psi = \int\limits_0^{\pi / 4} \int\limits_0^{2 \pi} \left[ \frac{r^6}{6} \sin^3 \psi \right]_{r=0}^{r=1} \, d\varphi \, d\psi = \frac{1}{6} \int\limits_0^{\pi / 4} \int\limits_0^{2 \pi} \sin^3 \psi \, d\varphi \, d\psi =
	\]
	\[
		\frac{1}{6} \int\limits_0^{\pi / 4} [\varphi \sin^3\psi]_{\varphi=0}^{\varphi=2\pi} \, d\psi = \frac{2 \pi}{6} \int\limits_0^{\pi / 4} \sin^3\psi \, d\psi
	\]

	\newpage
	\section{Gyakorlat}
	\subsection{Emlékeztető}
	Megjegyezzük, hogy ha a
	\[
		\Psi = (\Psi_1, \, \Psi_2, \, \Psi_3) \in \R^2 \to \R^3
	\]
	függvény differenciálható, akkor a
	\[
		\partial_k \Psi := (\partial_k \Psi_1, \, \partial_k \Psi_2, \, \partial_k \Psi_3) \quad (k \in \{1, \, 2\})
	\]
	jelöléssel
	\[
		\Psi' = \begin{bmatrix}
			\partial_1 \Psi & \partial_2 \Psi
		\end{bmatrix} =
		\begin{bmatrix}
			\partial_1 \Psi_1 & \partial_2 \Psi_1 \\
			\partial_1 \Psi_2 & \partial_2 \Psi_2 \\
			\partial_1 \Psi_3 & \partial_2 \Psi_3
		\end{bmatrix}.
	\]

	\tikz \node[definition]
	{
		\textbf{Definíció.} Valamely $\mathcal{F} \subset \R^3$ halmazt \textit{felületdarabnak} nevezünk, ha alkalmas $\emptyset \neq K \subset \R^2$ kompakt és Jordan-mérhető halmaz, ill.
		\[
			\Psi = (\Psi_1, \, \Psi_2, \, \Psi_3) : K \to \R^3, \, \Psi \in C^1, \, \text{rang} (\Psi') = 2
		\]
		függvény esetén
		\[
			\mathcal{F} = \mathcal{R}_\Psi = \{ \Psi(u, \, v) \in \R^3 : (u, \, v) \in K\}.
		\]
		A $\Psi$ leképezést az adott $\mathcal{F}$ felületdarab \textit{paraméterezésének} nevezzük.
	};\\

	\begin{enumerate}
		\item A fenti definícióban a $\Psi$ leképezés folytonos differenciálhatóságán azt értjük, hogy van olyan $K \subset U \subset \R^2$ nyílt halmaz, ill.
		\[
			\hat{\Psi} : U \to \R^3, \, \hat{\Psi} \in C^1
		\]
		függvény, amelyre $\hat{\Psi}_{|_K} = \Psi$.
		\item Ha a $\Psi = (\Psi_1, \, \Psi_2, \, \Psi_3) : K \to \R^3$ függvényre $\Psi \in C^1$, akkor a $\text{rang}(\Psi')=2$ feltétel azt jelenti, hogy a $\partial_1 \Psi, \, \partial_2 \Psi$ vektorok lineárisan függetlenek, azaz
		\[
			\partial_1 \Psi \times \partial_2 \Psi \neq 0.
		\]
		Ha a $\Psi : K \to \R^3$ leképezés valamely felületdarab paraméterezése, akkor az
		\[
			n_\Psi(u, \, v) := \partial_1 \Psi (u, \, v) \times \partial_2 \Psi(u, \, v) \quad \big((u, \, v) \in K\big)
		\]
		vektor az adott felület $\Psi(u, \, v)$ pontjához tartozó \textit{érintősíkjának} normálvektora.
	\end{enumerate}
	
	\tikz \node[definition]
	{
		\textbf{Definíció.} Tegyük fel, hogy az $\mathcal{F} \subset \R^3$ felületdarab paraméterezése az
		\[
			\Psi = (\Psi_1, \, \Psi_2, \, \Psi_3) : K \to \R^3
		\]
		függvény: $\mathcal{R}_\Psi = \mathcal{F}$. Ekkor az $\mathcal{F}$ felület \textit{felszínén} az
		\[
			\mathcal{A}(\mathcal{F}) := \int_K \|n_\Psi\|
		\]
		valós számot értjük.
	};\\

	\begin{enumerate}
		\item Ha $\mathcal{F}_1 := \mathcal{R}_{\Psi_1}$, ill. $\mathcal{F}_2 := \mathcal{R}_{\Psi_2}$ nullmértékű halmazban (pl. élekben) csatlakozó felületdarabok, akkor
		\[
			\mathcal{A}(\mathcal{F}_1 \cup \mathcal{F}_2) = \mathcal{A}(\mathcal{F}_1) + \mathcal{A}(\mathcal{F}_2) = \int_{K_1} \|n_{\Psi_1}\| + \int_{K_2} \|n_{\Psi_2}\|.
		\]
	\end{enumerate}

	\subsection{Feladatok}
	\subsubsection{Feladat}
	Legyen $0 < R \in \R$. Számítsuk ki az
	\[
		x^2 + y^2 + z^2 = R^2, \, z \geq 0
	\]
	félgömbfelületnek az
	\[
		x^2 + y^2 = Rx
	\]
	egyenletű hengerfelület által kimetszett részének (Viviani-féle levél) a felszínét!\\

	A $z \geq 0$ féltérben lévő félgömbfelület paraméterezése:
	\[
		\Psi(u, \, v) := (u, \, v, \, \sqrt{R^2 - u^2 - v^2}) \quad \big( (u, \, v) \in \R^2 : u^2 + v^2 \leq R^2 \big).
	\]
	Ekkor $\Psi \in C^1, \, \text{rang}(\Psi') =2$, továbbá ($u^2+v^2 < R$ esetén)
	\[
		\partial_1 \Psi(u, \, v) = \left(1, \, 0, \, -\frac{u}{\sqrt{R^2-u^2-v^2}}\right),
	\]
	\[
		\partial_2 \Psi(u, \, v) = \left(1, \, 0, \, -\frac{v}{\sqrt{R^2-u^2-v^2}}\right).
	\]
	\[
		n_\Psi(u, \, v) = \begin{pmatrix}
			- \partial_1 g(u, \, v) \\
			- \partial_2 g(u, \, v) \\
			1
		\end{pmatrix} =
		\begin{pmatrix}
			\frac{u}{\sqrt{R^2-u^2-v^2}} \\
			\frac{v}{\sqrt{R^2-u^2-v^2}} \\
			1
		\end{pmatrix},
	\]
	Azaz
	\[
		\|n_\Psi(u, \, v)\| = \sqrt{ \frac{u^2}{R^2-u^2-v^2} + \frac{v^2}{R^2-u^2-v^2} + 1} =
	\]
	\[
		\frac{R}{\sqrt{R^2-u^2-v^2}}.
	\]
	Ha most
	\[
		\Phi(r, \, \varphi) := \begin{pmatrix}
			r \cos\varphi \\
			r \sin \varphi
		\end{pmatrix} \quad \big( \varphi \in [0, \, \pi / 2], \, r \in [0, \, R \cos(\varphi)]\big)
	\]
	Akkor a felszínt így kapjuk meg:
	\[
		2R \cdot \lim_{\varepsilon \to 0} \int\limits_\varepsilon^{\pi/2} \int\limits_0^{R\cos(\varphi)} \Big\|n_\Psi\big(\Phi(r, \, \varphi)\big)\Big\| \cdot r \, dr \, d\varphi.
	\]
	
	\subsubsection{Feladat}
	Határozzuk meg az alábbi feltételekkel megadott felület felszínét:
	\[
		z = \sqrt{x^2 + y^2} \text{ és } x^2 + y^2 \leq 2ax \quad (a > 0).
	\]
	Legyen
	\[
		f(x, \, y) := (x, \, y, \, \sqrt{x^2 + y^2}) \quad \big( (x, \, y) \in \Omega \big),
	\]
	ahol $\Omega := \{ (x, \, y) \in \R^2 : x^2 + y^2 \leq 2 ax \}$, $f$ egy Euler-Monge módon megadott paraméterezése a $z$-tengely irányába felfelé néző körkúp palástjának, \textit{leszűkítve} annak a hengernek a belsejére, aminek alapját az $\Omega$ halmaz adja. Egyből írjuk is át az $f$ függvényt a megfelelő polár-transzformációval
	\[
		\Phi(r, \, \varphi) := (r \cos(\varphi), \, r \sin(\varphi)) \quad \big( \varphi \in [- \pi / 2, \, \pi / 2], \, r \in [0, \, 2a \cos(\varphi)]\big).
	\]
	A $\Phi$ transzformáció-függvény értelmezési tartományát a Thálész-tétel alkalmazásával könnyedén megkaptuk. Tehát ha $\mathcal{F} \subset \R^3$ a feladatban definiált felület, akkor
	\[
		\Psi(r, \, \varphi) := f \circ \Phi = (r \cos(\varphi), \, r \sin(\varphi), \, r) \quad \big( (r, \, \varphi) \in \mathcal{D}_\Phi \big).
	\]
	A felület kiszámításához szükségünk lesz a következőkre:
	\[
		\partial_1 \Psi(r, \, \varphi) = (\cos(\varphi), \, \sin(\varphi), \, 1) \quad \big( (r, \, \varphi) \in \mathcal{D}_\Phi \big),
	\]
	\[
		\partial_1 \Psi(r, \, \varphi) = (-r \sin(\varphi), \, r \cos(\varphi), \, 0) \quad \big( (r, \, \varphi) \in \mathcal{D}_\Phi \big),
	\]
	\[
		n_\Psi(r, \, \varphi) = \begin{vmatrix}
			i & j & k \\
			\cos(\varphi) & \sin(\varphi) & 1 \\
			-r\sin(\varphi) & r\cos(\varphi) & 0
		\end{vmatrix} =
		\begin{pmatrix}
			r\cos(\varphi) \\
			r\sin(\varphi) \\
			r
		\end{pmatrix} \quad \big( (r, \, \varphi) \in \mathcal{D}_\Phi \big),
	\]
	\[
		\|n_\Psi(r, \, \varphi)\| = \sqrt{2r^2} = \sqrt{2} r \quad \big( (r, \, \varphi) \in \mathcal{D}_\Phi \big).
	\]
	Tehát a keresett felszín
	\[
		\mathcal{A}(\mathcal{F}) = \int_{\mathcal{D}_\Phi} \|n_\Psi\| = \sqrt{2} \int\limits_{-\pi/2}^{\pi/2} \int\limits_0^{2a\cos(\varphi)} r \, dr \, d\varphi.
	\]

	\newpage
	\section{Gyakorlat}
	\subsection{Emlékeztető}
	\tikz \node[definition]
	{
		\textbf{Definíció.} Legyen $K \subset \R^2$ kompakt, Jordan-mérhető halmaz,
		\[
			\Psi = (\Psi_1, \, \Psi_2, \, \Psi_3) : K \to \R^3, \, \Psi \in C^1, \, \text{rang}(\Psi') = 2,
		\]
		továbbá tegyük fel, hogy $\Psi_{|_{\text{int}(K)}}$ injektív. Ekkor
		\begin{enumerate}
			\item ha $f : \Psi[K] \to \R, \, f \in C$, akkor az $f$ függvény $\Psi$-re vonatkozó \textit{elsőfajú felületi integráljának} (vagy \textit{felszíni integráljának}) nevezzük az
			\[
				\int_\Psi f := \int_K (f \circ \Psi) \cdot \|n_\Psi\|
			\]
			valós számot,
			\item ha $f : \Psi[K] \to \R^3, \, f \in C$, akkor az $f$ függvény $\Psi$-re vonatkozó (\textit{másodfajú}) \textit{felületi integráljának} nevezzük az
			\[
				\int_\Psi f := \langle  f \circ \Psi, \, n_\Psi \rangle
			\]
			valós számot.
		\end{enumerate}
	};

	\newpage
	\section{Gyakorlat}
	\subsection{Emlékeztető}
	Valamilyen kompakt $[a, \, b]$ intervallum $(a, \, b \in \R, \, a < b)$ és $(\emptyset \neq U \subset \R^n)$ $(1 \leq n \in N)$ nyílt halmaz esetén tekintsük az
	\[
		f : U \times [a, \, b] \to \R
	\]
	függvényt. Ha $x \in U$, akkor legyen $f_x : [a, \, b] \to \R$ az a függvény, amire
	\[
		f_x(t) := f(x, \, t) \quad (t \in [a, \, b]).
	\]
	Tegyük fel, hogy minden $x \in U$ esetén az $f_x$ függvény Riemann-integrálható: $f_x \in R[a, \, b]$, legyen ekkor
	\[
		F(x) := \int_a^b f(x, \, t) \, dt := \int_a^b f_x \quad (x \in U).
	\]
	Az így definiált $F : U \to \R$ függvényt az $f$ által meghatározott \textit{paraméteres integrálnak} nevezzük.\\

	Világos, hogy
	\[
		\D = U \times [a, \, b] \subset \R^{n+1}.
	\]
	Vezessük be $\R^n$-en is és $\R^{n+1}$-en is a $\|.\| := \|.\|_\infty$ normát. Nem fog félreértést okozni, ha $\xi \in \R^n$ és $\eta \in \R^{n+1}$ esetén egyaránt a $\|\xi\|, \, \|\eta\|$ jelölést használjuk. Így pl. nyilvánvaló, hogy az
	\[
		(x, \, t) \in U \times [a, \, b]
	\]
	vektorra
	\[
		\| (x, \, t)\| = \max \{ \|x\|, |t| \}.
	\]
	Továbbá, ha az
	\[
		f : U \times [a, \, b] \to \R
	\]
	függvény folytonos, akkor tetszőleges $x \in U$ mellett az $f_x : [a, \, b] \to \R$ függvény is folytonos. Ui., ha $s \in [a, \, b]$ és $\xi := (x, \, s)$, akkor $f \in \mathcal{C}\{\xi\}$ miatt bármilyen $\varepsilon > 0$ esetén van olyan $\delta > 0$, amellyel
	\[
		|f(\omega) - f(\xi)| < \varepsilon \quad (\omega \in U \times [a, \, b], \, \|\omega - \xi\| < \delta),
	\]
	azaz
	\[
		|f(y, \, t) - f(x, \, s)| < \varepsilon \quad ((y, \, t) \in U \times [a, \, b], \, \|(y, \, t) - (x, \, s)\| < \delta).
	\]
	Ha itt $y := x$, akkor
	\[
		\|(x, \, t) - (x, \, s)\| = \|(0, \, t-s)\| = |t-s|,
	\]
	ezért
	\[
		|f(x, \, t) - f(x, \, s)| = |f_x(t) - f_x(s)| < \varepsilon (t \in [a, \, b], \, |t-s| < \delta),
	\]
	más szóval $f_x \in \mathcal{C}\{s\}$. Következésképpen $f_x \in C[a, \, b]$, így $f_x \in R[a, \, b]$, létezik tehát a fenti $F$ paraméteres integrál.\\

	\tikz \node[theorem]
	{
		\textbf{Tétel.} Tegyük fel, hogy adott az $[a, \, b] \, (a, \, b \in \R, \, a < b)$ kompakt intervallum, $1 \leq n \in \N$, és $\emptyset \neq U \subset \R^n$ nyílt halmaz. Ekkor tetszőleges folytonos
		\[
			f : U \times [a, \, b] \to \R
		\]
		függvény esetén az
		\[
			F(x) := \int_a^b f(x, \, t) \, dt \quad (x \in U)
		\]
		paraméteres integrálra az alábbiak igazak:
		\begin{enumerate}
			\item az $F$ függvény folytonos;
			\item ha valamilyen $i = 1, \, \dots, \, n$ indexre létezik és folytonos a $\partial_i f$ parciális deriváltfüggvény, akkor létezik a $\partial_i F$ parciális deriváltfüggvény is, és
			\[
				\partial_i F(x) = \int_a^b \partial_i f(x, \, t) \, dt \quad (x \in U);
			\]
			\item amennyiben az $f$ folytonosan differenciálható, akkor $F$ is.
		\end{enumerate}
	};\\

	\subsection{Feladatok}
	\subsubsection{Feladat}
	Igazoljuk, hogy
	\[
		\int_0^{\pi / 2} \frac{\text{arctg} (y \cdot \text{tg}(x))}{\text{tg}(x)} \, dx = \frac{\pi}{2} \ln(y + 1) \quad (y \geq 0).
	\]
	Legyen
	\[
		f(y, \, x) := \begin{cases}
			y & x = 0, \, y \in [0, \, + \infty) \\
			0 & x = \pi / 2, \, y \in [0, \, + \infty) \\
			\displaystyle \frac{\text{arctg} (y \cdot \text{tg}(x))}{\text{tg}(x)} & ((y, \, x) \in [0, \, +\infty) \times (0, \, \pi / 2)).	
		\end{cases}
	\]
	Ekkor $f \in C$, ezért a paraméteres integrál is folytonos. Mivel
	\[
		\partial_1 f(y, \, x) := \begin{cases}
			0 & x = \pi / 2, \, y \in [0, \, + \infty) \\
			\displaystyle \frac{1}{1 + y^2 \text{tg}^2(x)} & ((y, \, x) \in [0, \, +\infty) \times [0, \, \pi / 2))
		\end{cases}
	\]
	$\partial_1f \in C$, tehát a paraméteres integrál deriválható.


	\subsection{Feladat}
	Indokoljuk meg, hogy tetszőleges $0 < \alpha \in \R$ esetén
	\[
		F(\alpha) := \int\limits_0^1 \frac{\text{arctg}(\alpha \sqrt{x})}{\sqrt{x}} \, dx \in \R,
	\]
	majd mutassuk meg, hogy fennáll az
	\[
		F(\alpha) = 2 \text{arctg}(\alpha) - \frac{\ln(1+\alpha^2)}{\alpha} \quad (0 < \alpha \in \R)
	\]
	egyenlőség!\\

	Legyen
	\[
		f(\alpha, \, x) := \begin{cases}
			\alpha & \alpha \in (0, \, + \infty), \, x = 0; \\
			\frac{\text{arctg}(\alpha \sqrt{x})}{\sqrt{x}} & \alpha \in (0, \, + \infty), \, x \in (0, \, 1];
		\end{cases}
	\]
	akkor $f \in C$, hiszen (Bernoulli-L'Hospital-szabály felhasználásával)
	\[
		\lim_{x \to 0 + 0} \frac{\text{arctg}(\alpha\sqrt{x})}{\sqrt{x}} = \lim_{x \to 0 + 0} \frac{\frac{1}{1 + \alpha^2 x} \cdot \frac{1}{2\sqrt{x}} \cdot \alpha}{\frac{1}{2\sqrt{x}}} = \alpha.
	\]
	Tehát $F \in C$ is teljesül. Mivel
	\[
		\partial_1 f(\alpha, \, x) = \frac{1}{1+ \alpha^2x} \quad \big( (\alpha, \, x) \in (0, \, + \infty) \times [0, \, 1] \big),
	\]
	és $\partial_1 f \in C$, ezért $F \in D$.
	\[
		F'(\alpha)= \int\limits_0^1 \partial_\alpha \left( \frac{\text{arctg}(\alpha \sqrt{x})}{\sqrt{x}} \, dx \right) = \int\limits_0^1 \frac{1}{1+\alpha^2 x} \, dx = \left[ \frac{\ln(1+\alpha^2x)}{\alpha^2} \right]_0^1 =
	\]
	\[
		\frac{\ln(1+\alpha^2)}{\alpha^2}.
	\]
	Tehát
	\[
		F'(\alpha) = \frac{\ln(1+\alpha^2)}{\alpha^2}.
	\]
	Számoljuk ki $F'$ határozatlan integrálját:
	\[
		\int F' = \int \frac{\ln(1+\alpha^2)}{\alpha^2} \, d\alpha = -\frac{\ln(1+\alpha^2)}{\alpha} - \int \left(-\frac{1}{\alpha}\right) \cdot \frac{2\alpha}{1 + \alpha^2} \, d\alpha =
	\]
	\[
		2 \text{arctg}(\alpha) - \frac{\ln(1+\alpha^2)}{\alpha} + c \quad (c \in \R).
	\]
	Azaz, mivel
	\[
		F \in \int F',
	\]
	ezért létezik egy $c \in \R$ konstans, amivel
	\[
		F(\alpha) = 2 \text{arctg}(\alpha) - \frac{\ln(1+\alpha^2)}{\alpha} + c.
	\]
	Számoljuk ki $F(1)$-et:
	\[
		F(1) = \int\limits_0^1 \frac{\text{arctg}(\sqrt{x})}{\sqrt{x}} \, dx = \frac{\pi}{2} - \ln(2).
	\]
	Ebből
	\[
		F(1) = 2 \text{arctg}(1) - \ln(2) = \frac{\pi}{2} - \ln(2),
	\]
	ahonnan $c=0$, tehát
	\[
		F(\alpha) = 2 \text{arctg}(\alpha) - \frac{\ln(1+\alpha^2)}{\alpha}
	\]
	következik.



	\newpage
	\section{Gyakorlat}
	\subsection{Emlékeztető}

	Legyenek $n, \, m \in \N$ természetes számok, ahol $2 \leq n$ és $1 \leq m < n$. Ha
	\[
		\xi = (\xi_1, \, \dots, \, \xi_n) \in \R^n,
	\]
	akkor legyen
	\[
		x := (\xi_1, \, \dots, \, \xi_{n-m}) \in \R^{n-m}, \, y := (\xi_{n-m+1}, \, \dots, \, \xi_n) \in \R^m,
	\]
	és ezt következésképpen fogjuk jelölni:
	\[
		\xi =(x, \, y).
	\]
	Röviden:
	\[
		\R^n \equiv \R^{n-m} \times \R^m.
	\]
	Ha tehát
	\[
		f = (f_1, \, \dots, \, f_m) \in \R^n \to \R^m,
	\]
	azaz
	\[
		f \in \R^{n-m} \times \R^m \to \R^m,
	\]
	akkor $f$-et olyan kétváltozós vetkorfüggvényként tekintjük, ahol az $f(x, \, y)$ helyettesítési értékekben az argumentum első változójára $x \in \R^{n-m}$, a második változójára pedig $y \in \R^m$ teljesül.\\

	Tegyük fel, hogy ebben az értelmemben valamilyen $(a, \, b) \in \text{int} \, \D$ zérushelye az $f$-nek:
	\[
		f(a, \, b) = 0.
	\]
	Tételezzük fel továbbá, hogy van az $a$-nak olyan $K(a) \subset \R^{n-m}$ környezete, a $b$-nek pedig olyan $K(b) \subset \R^m$ környezete, hogy tetszőleges $x \in K(a)$ esetén egyértelműen létezik olyan $y \in K(b)$, amellyel
	\[
		f(x, \, y) = 0.
	\]
	Definiáljuk ekkor a $\varphi(x) := y$ hozzárendeléssel a
	\[
		\varphi : K(a) \to K(b)
	\]
	függvényt, amikor is
	\[
		f(x, \, \varphi(x)) = 0 \quad \big(x \in K(a) \big).
	\]
	Az előbbi $\varphi$ függvényt az $f$ által (az $(a, \, b)$ körül) meghatározott \textit{implicitfüggvénynek} nevezzük. Nyilván $\varphi(a) = b$.\\

	\tikz \node[theorem]
	{
		\textbf{Tétel (implicitfüggvény-tétel).} Adott $n, \, m \in \N, \, 2 \leq n$, valamint $1 \leq m < n$ mellett az
		\[
			f \in \R^{n-m} \times \R^m \to \R^m
		\]
		függvényről tételezzük fel az alábbiakat: $f \in C^1$ és az $(a, \, b) \in \text{int} \, \D$ helyen
		\[
			f(a, \, b) = 0, \, \det \partial_2 f(a, \, b) \neq 0.
		\]
		Ekkor alkalmas $K(a), \, K(b)$ környezetekkel létezik az $f$ által az $(a, \, b)$ körül meghatározott
		\[
			\varphi : K(a) \to K(b)
		\]
		implicitfüggvény, ami folytonosan differenciálható, és
		\[
			\varphi'(x) = -\partial_2f(x, \, \varphi(x))^{-1} \cdot \partial_1f(x, \, \varphi(x)) \quad \big(x \in K(a)\big).
		\]
	};

	\subsection{Feladatok}
	\subsection{Feladat}
	Bizonyítsuk be, hogy az
	\[
		x^2 +2xy - y^2 = 7
	\]
	egyenletből $y$ kifejezhető $x$ implicitfüggvényeként a $2$ egy környezetében, majd határozzuk meg így adódó implicitfüggvénynek deriváltját az $x=2$ helyen, továbbá írjuk fel az érintőegyenesek egyenleteit!\\

	Legyen
	\[
		f(x, \, y) := x^2 + 2xy - y^2 - 7 \big( (x, \, y) \in \R^2 \big).
	\]
	Ekkor
	\[
		f(2, \, y) = 0 \quad \Longleftrightarrow \quad y \in \{1, \, 3\}.
	\]
	Mivel $f \in C^1$, és
	\[
		\partial_2 f(2, \, 1) = 2 \neq 0,
	\]
	\[
		\partial_2 f(2, \, 3) = -2 \neq 0,
	\]
	így teljesülnek az implicitfüggvényre vonatkozó tétel feltételei. Legyen
	\[
		H := \{ (x, \, y) \in \R^2 : x^2 + 2xy - y^2 = 7 \}.
	\]
	Így valóban léteznek olyan $U =: K(2), \, V_1 := K(1), \, V_2 := K(3)$ környezetek, hogy $\varphi_1 : U \to V_1$ és $\varphi_2 : U \to V_2$ folytonosan differenciálható függvények grafikonjai rendre megegyeznek a $H$ halmaz $(2, \, 1), \, (2, \, 3)$ pontban vett környezetével.
	\[
		f(x, \, \varphi_i(x)) = 0 \quad (x \in U, \, i = 1, \, 2).
	\]
	Mivel
	\[
		\partial_1 f(2, \, 1) = 6, \text{ ill. } \partial_1f(2, \, 3) = 10,
	\]
	ezért
	\[
		\varphi_1(2) = 1, \text{ill. a } \varphi_2(2) = 3
	\]
	egyenlőségek felhasználásával azt kapjuk, hogy
	\[
		\varphi_1'(2) = -3, \, \varphi_2(2) = 5.
	\]
	A megfelelő érintők egyenlete:
	\[
		y = \varphi_1(2) + \varphi_1'(2)(x-2) = 7-3x \quad (x \in \R),
	\]
	\[
		y = \varphi_2(2) + \varphi_2'(2)(x-2) = 5x-7 \quad (x \in \R).
	\]
	
	\newpage
	\section{Gyakorlat}
	\subsection{Emlékeztető}
	\subsubsection{Közönséges differenciálegyenletek}
	
	Legyen $0 < n \in \N, \, I \subset \R, \, \Omega \subset \R^n$ egy-egy nyílt intervallum. Tegyük fel, hogy az
	\[
		f : I \times \Omega \to \R^n
	\]
	függvény folytonos, és tűzzük ki az alábbi feladat megoldását:
	\begin{quote}
		határozzunk meg olyan $\varphi \in I \to \Omega$ függvény, amelyre igazak a következő állítások:
		\begin{enumerate}
			\item $\mathcal{D}_\varphi$ nyílt intervallum;
			\item $\varphi \in D$;
			\item $\varphi'(x) = f(x, \, \varphi(x)) \quad (x \in \mathcal{D}_\varphi)$.
		\end{enumerate}
	\end{quote}
	A most megfogalmazott feladatot \textit{explicit elsőrendű közönséges differenciálegyenletnek} (röviden \textit{differenciálegyenletnek}) fogjuk nevezni, és a \textit{d.e.} rövidítéssel idézni.\\
	
	Ha adottak a $\tau \in I, \, \xi \in \Omega$ elemek, akkor a fenti $\varphi$ függvény $1., \, 2.$ és $3.$ mellett tegyen eleget a
	\begin{enumerate}[start=4]
		\item $\tau \in \mathcal{D}_\varphi$ és $\varphi(\tau) = \xi$
	\end{enumerate}
	kikötésnek is. Az így "kibővített" feladatot \textit{kezdetiérték-problémának} (vagy \textit{Cauchy-feladatnak}) nevezzük, és a továbbiakban minderre a \textit{k.é.p.} rövidítést fogjuk használni.\\

	Azt mondjuk, hogy a szóban forgó k.é.p. \textit{egyértelműen oldható meg}, ha tetszőleges $\varphi, \, \psi$ megoldásai esetén
	\[
		\varphi(x) = \psi(x) \quad (x \in \mathcal{D}_\varphi \cap \mathcal{D}_\psi).
	\]
	Legyen ekkor $\mathcal{M}$ a szóban forgó k.é.p. megoldásainak a halmaza és
	\[
		J := \bigcap_{\varphi \in \mathcal{M}} \Dp.
	\]
	Ez egy $\tau$-t tartalmazó nyílt intervallum és $J \subset I$. Az egyértelmű megoldhatóság értelmezése miatt definiálhatjuk a
	\[
		\Phi : J \to \Omega
	\]
	függvényt az alábbiak szerint:
	\[
		\Phi(x) := \varphi(x) \quad (\varphi \in \mathcal{M}, \, x \in \Dp).
	\]
	Nyilván, $\Phi(\tau) = \xi, \, \Phi \in D$ és
	\[
		\Phi'(x) = f(x, \, Phi(x)) \quad (x \in J).
	\]
	Ez azt jelenti, hogy $\Phi \in \mathcal{M}$, és bármelyik $\varphi \in \mathcal{M}$ esetén $\varphi = \Phi_{|_{\Dp}}$.\\
	
	A $\Phi$ függvényt a k.é.p. \textit{teljes megoldásának} nevezzük.
	
	\subsubsection{Szeparábilis differenciálegyenlet}
	Legyen $n := 1$, továbbá az $I, \, J \subset \R$ nyílt intervallumokkal és a
	\[
		g : I \to \R, \, h : J \to \R \, \backslash \, \{0\}
	\]
	folytonos függvényekkel
	\[
		f(x, \, y) := g(x) \cdot h(y) \quad \big( (x, \, y) \in I \times J \big).
	\]
	A $\varphi \in I \to J$ megoldásra tehát
	\[
		\varphi'(t) = g(t) \cdot h(\varphi(t)) \quad (t \in \mathcal{D}_\varphi).
	\]
	(\textit{Szeparábilis} (vagy más szóval \textit{szétválasztható változójú}) differenciálegyenlet.) Legyenek még adottak a $\tau \in I, \, \xi \in J$ számok, amikor is
	\[
		\tau \in \Dp, \, \varphi(\tau) = \xi.
	\]
	\tikz \node[theorem]
	{
		\textbf{Tétel.} Tetszőleges szeparábilis differenciálegyenletre vonatkozó kezdetiérték-probléma megoldható, és bármilyen $\varphi, \, \psi$ megoldásaira
		\[
			\varphi(t) = \psi(t) \quad (t \in \Dp \cap \mathcal{D}_\psi).
		\]
	};\\
	
	\textbf{Bizonyítás.} Mivel a $h$ függvény sehol sem nulla, ezért egy $\varphi$ megoldásra (ha az létezik)
	\[
		\frac{\varphi'(t)}{h(\varphi(t))} = g(t) \quad (t \in \Dp).
	\]
	A $g : I \to \R$ is, és az $1 / h : J \to \R$ is egy-egy nyílt intervallumon értelmezett folytonos függvények, így léteznek a
	\[
		G \in \int g, \, H \in \int \frac{1}{h},
	\]
	\[
		G : I \to \R, \, H : J \to \R
	\]
	primitív függvényeik. Az összetett függvény deriválásával kapcsolatos tétel szerint
	\[
		\frac{\varphi'(t)}{h(\varphi(t))} = (H \circ \varphi - G)'(t) = 0 \quad (t \in \Dp).
	\]
	Tehát (mivel a $\Dp$ is nyílt intervallum) van olyan $c \in \R$, hogy
	\[
		H(\varphi(t)) - G(t) = c \quad (t \in \Dp).
	\]
	Az $1/h$ függvény nyilván nem vesz fel 0-t a $J$ intervallum egyetlen pontjában sem, így ugyanez igaz a $H'$ függvényre is. A deriváltfüggvény Darboux-tulajdonsága miatt tehát a $H'$ állandó előjelű. Következésképpen $H$ szigorúan monoton, amiért invertálható. A $H^{-1}$ inverzzel
	\[
		\varphi(t) = H^{-1}(G(t) + c) \quad (t \in \Dp).
	\]
	Ha $\tau \in I, \, \xi \in J$, és a $\varphi$ megoldás eleget tesz a $\varphi(\tau) = \xi$ kezdeti feltételnek is, akkor
	\[
		\xi = H^{-1}(G(\tau) + c),
	\]
	azaz $H(\xi) = G(\tau) + c$, ill.
	\[
		c = H(\xi) - G(\tau).
	\]
	Így
	\[
		\varphi(t) = H^{-1}\big( G(t) + H(\xi) - G(\tau) \big) \quad (t \in \Dp).
	\]
	Ha $G, \, H$ helyett más primitív függvényeket választunk, akkor ugyanezt az egyenlőséget kapjuk.\\
	
	Elegendő már csak azt belátni, hogy létezik megoldás. Ld. előadás (implicitfüggvény-tétel megfelelő alkalmazása). $\blacksquare$\\
	
	A tétel bizonyításával egyúttal egy módszert ("képletet") is adtunk a
	\[
		\varphi'(t) = g(t) \cdot h(\varphi(t)) \quad (t \in \Dp)
	\]
	szeparábilis egyenletre vonatkozó $\varphi(\tau) = \xi$ k.é.p. megoldására. Ennek a kulcsmozzanata a $G, \, H$ primitív függvények meghatározása:
	\[
		G(x) = \int\limits_\tau^x g(t) \, dt, \, H(y) = \int\limits_\xi^y \frac{dt}{h(t)} \quad (x \in I, \, y \in J),
	\]
	amiből aztán a $\varphi$-re vonatkozó
	\[
		H(\varphi(x)) = \int\limits_\xi^{\varphi(x)} \frac{dt}{h(t)} = G(x) = \int\limits_\tau^x g(t) \, dt \quad (x \in \Dp),
	\]
	azaz az
	\[
		\int\limits_\xi^{\varphi(x)} \frac{dt}{h(t)} = \int\limits_\tau^x g(t) \, dt \quad (x \in \Dp)
	\]
	implicit egyenlet adódott.
	
	
	
	
	\subsubsection{Egzakt differenciálegyenlet}
	Speciálisan legyen $n := 1$, és az $I, \, J \subset$ nyílt intervallumok, valamint a
	\[
		g : I \times J \to \R \text{ és } h : I \times J \to \R \, \backslash \, \{0\}
	\]
	folytonos függvényekkel
	\[
		f(x, \, y) := - \frac{g(x, \, y)}{h(x, \, y)} \quad ((x, \, y) \in I \times J).
	\]
	Ekkor a fenti minden $\varphi$ megoldásra
	\[
		\varphi'(x) = -\frac{g(x, \, \varphi(x))}{h(x, \, \varphi(x))} \quad (x \in \Dp).
	\]
	Azt mondjuk, hogy az így kapott \textit{d.e. egzakt differenciálegyenlet}, ha az
	\[
		I \times J \ni (x, \, y) \mapsto (g(x, \, y), \, h(x, \, y)) \in \R^2
	\]
	leképezésnek van primitív függvénye. Ez utóbbi követelmény azt jelenti, hogy egy alkalmas differenciálható
	\[
		G : I \times J \to \R	
	\]
	függvénnyel
	\[
		\text{grad} \, G = (\partial_1 G, \, \partial_2 G) = (g, \, h).
	\]
	Ha $\tau \in I, \, \xi \in J$ és a $\varphi$ függvénytől azt is elvárjuk, hogy
	\[
		\tau \in \Dp, \, \varphi(\tau) = \xi,
	\]
	akkor igaz az\\
	
	\tikz \node[theorem]
	{
		\textbf{Tétel.} Tetszőleges egzakt differenciálegyenletre vonatkozó minden kezdetiérték-probléma megoldható, és ennek bármilyen $\varphi, \, \psi$ megoldásaira
		\[
			\varphi(t) = \psi(t) \quad (t \in \Dp \cap \mathcal{D}_\psi).
		\]
	};\\
	
	\textbf{Bizonyítás.} Valóban, $0 \not \in \mathcal{R}_h$ miatt a feltételezett $\varphi$ megoldásra
	\[
		g(x, \, \varphi(x)) + h(x, \, \varphi(x)) \cdot \varphi'(x) = 0 \quad (x \in \Dp).
	\]
	Ha van ilyen $\varphi$ függvény, akkor az
	\[
		F(x) := G(x, \, \varphi(x)) \quad (x \in \Dp)
	\]
	egyváltozós valós függvény differenciálható, és tetszőleges $x \in \Dp$ helyen
	\[
		F'(x) = \langle \text{grad} \, G(x, \, \varphi(x)), \, (1, \, \varphi'(x)) \rangle = 
	\]
	\[
		g(x, \, \varphi(x)) + h(x, \, \varphi(x)) \cdot \varphi'(x) = 0.
	\]
	A $\mathcal{D}_F = \mathcal{D}_\varphi$ halmaz nyílt intervallum, ezért az $F$ konstans függvény, azaz létezik olyan $c \in \R$, amellyel
	\[
		G(x, \, \varphi(x)) = c \quad (x \in \Dp).
	\]
	Mivel $\varphi(\tau) = \xi$, ezért
	\[
		c = G(\tau, \, \xi).
	\]
	A $G$-ről feltehetjük, hogy $G(\tau, \, \xi) = 0$, ezért a szóban forgó kezdetiérték-probléma $\varphi$ megoldása eleget tesz a
	\[
		G(x, \, \varphi(x)) = 0 \quad (x \in \Dp)
	\]
	egyenletnek.\\
	
	Világos, hogy a $\varphi$ nem más, mint egy, a $G$ által meghatározott implicitfüggvény. Más szóval a szóban forgó kezdetiérték-probléma minden megoldása (ha létezik) a fenti implicitfüggvény-egyenletből határozható meg. \\
	
	Ugyanakkor a feltételek alapján $G \in C^1, \, G(\tau, \, \xi) = 0$, továbbá
	\[
		\partial_2 G(\tau, \, \xi) = h(\tau, \, \xi) \neq 0,
	\]
	ezért $G$-re (a $(\tau, \, \xi)$ helyen) teljesülnek az implicitfüggvény-tétel feltételei. Következésképpen van olyan differenciálható
	\[
		\psi \in I \to J
	\]
	(implicit)függvény, amelyre $\mathcal{D}_\psi \subset I$ nyílt intervallum,
	\[
		\tau \in \mathcal{D}_\psi, \, G(x, \, \psi(x)) = 0 \, (x \in \mathcal{D}_\psi), \, \psi(\tau) = \xi,
	\]
	és
	\[
		\psi'(x) = - \frac{\partial_1 G(x, \, \psi(x))}{\partial_2 G(x, \, \psi(x))} = - \frac{g(x, \, \psi(x))}{h(x, \, \psi(x))} \quad (x \in \mathcal{D}_\psi).
	\]
	$\hfill\blacksquare$
	
	
	
	
	\subsubsection{Lineáris differenciálegyenlet}
	Legyen most $n =: 1$ és az $I \subset \R$ egy nyílt intervallum, valamint a
	\[
		g, \, h : I \to \R
	\]
	folytonos függvények segítségével
	\[
		f(x, \, y) := g(x) \cdot y + h(x) \quad ((x, \, y) \in I \times \R).
	\]
	Ekkor
	\[
		\varphi'(t) = g(t) \cdot \varphi(t) + h(t) \quad (t \in \Dp).
	\]
	Ezt a feladatot \textit{lineáris differenciálegyenletnek} nevezzük. \\
	
	Ha valamilyen $\tau \in I, \, \xi \in \R$ mellett
	\[
		\tau \in \Dp, \, \varphi(\tau) = \xi,
	\]
	akkor az illető lineáris differenciálegyenletre vonatkozó kezdetiérték-problémáról beszélünk.\\
	
	Tegyük fel, hogy a $\theta$ függvény is és a $\psi$ függvény is megoldása a lineáris d.e.-nek és $\mathcal{D}_\theta \cap \mathcal{D}_\psi \neq \emptyset$. Ekkor
	\[
		(\theta - \psi)'(t) = g(t) \cdot \big( \theta(t) - \psi(t) \big) \quad (t \in \mathcal{D}_\theta \cap \mathcal{D}_\psi).
	\]
	Így a $\theta - \psi$ függvény megoldása annak a lineáris d.e.-nek, amelyben $h \equiv 0$:
	\[
		\varphi'(t) = g(t) \cdot \varphi(t) \quad (t \in \Dp).
	\]
	Ez utóbbi feladatot \textit{homogén lineáris differenciálegyenletnek} fogjuk nevezni. (Ennek megfelelően a szóban forgó lineáris differenciálegyenlet \textit{inhomogén}, ha a benne szereplő $h$ függvény vesz fel 0-tól különböző értéket is.)\\
	
	\tikz \node[theorem]
	{
		\textbf{Tétel.} Minden lineáris differenciálegyenletre vonatkozó kezdetiérték-probléma megoldható, és tetszőleges $\varphi, \, \psi$ megoldásaira
		\[
			\varphi(t) = \psi(t) \quad (t \in \mathcal{D}_\varphi \cap \mathcal{D}_\psi).
		\]
	};\\
	
	\textbf{Bizonyítás.} Legyen a
	\[
		G : I \to \R
	\]
	olyan függvény, amelyik differenciálható és $G' = g$ (a $g$-re tett  feltételek miatt ilyen $G$ primitív függvény van). Ekkor a
	\[
		\varphi_0(t) := e^{G(t)} \quad (t \in I)
	\]
	(csak pozitív értékeket felvevő) függvény megoldása az előbb említett homogén lineáris differenciálegyenletnek:
	\[
		\varphi_0'(t) = G'(t) \cdot e^{G(t)} = g(t) \cdot \varphi_0(t) \quad (t \in I).
	\]
	Tegyük fel most azt, hogy a
	\[
		\chi \in I \to \R
	\]
	függvény is megoldása a szóban forgó homogén lineáris differenciálegyenletnek:
	\[
		\chi'(t) = g(t) \cdot \chi(t) \quad (t \in \mathcal{D}_\chi).
	\]
	Ekkor a differenciálható
	\[
		\frac{\chi}{\varphi_0} : \mathcal{D}_\chi \to \R
	\]
	függvényre azt kapjuk, hogy bármelyik $t \in \mathcal{D}_\chi$ helyen
	\[
		\left( \frac{\chi}{\varphi_0} \right)'(t) = \frac{\chi'(t) \cdot \varphi_0(t) - \chi(t) \cdot \varphi_0'(t)}{\varphi_0^2(t)} =
	\]
	\[
		\frac{g(t) \cdot \chi(t) \cdot \varphi_0(t) - \chi(t) \cdot g(t) \cdot \varphi_0(t)}{\varphi_0^2(t)} = 0,
	\]
	azaz (lévén a $\mathcal{D}_\chi$ nyílt intervallum) egy alkalmas $c \in \R$ számmal
	\[
		\frac{\chi(t)}{\varphi_0(t)} = c \quad (t \in \mathcal{D}_\chi).
	\]
	Más szóval, az illető homogén lineáris differenciálegyenlet bármelyik
	\[
		\chi \in I \to \R
	\]
	megoldása a következő alakú:
	\[
		\chi(t) = c \cdot \varphi_0(t) \quad (t \in \mathcal{D}_\chi),
	\]
	ahol $c \in \R$. Nyilván minden ilyen $\chi$ függvény -- könnyen ellenőrizhető módon -- megoldása a mondott homogén lineáris differenciálegyenletnek.\\
	
	Ha tehát a fenti (inhomogén) lineáris differenciálegyenletnek a $\theta$ függvény is és a $\psi$ függvény is megoldása és $\mathcal{D}_\theta \cap \mathcal{D}_\psi \neq \emptyset$, akkor egy alkalmas $c \in \R$ együtthatóval
	\[
		\theta(t) - \psi(t) = c \cdot \varphi_0(t) \quad (t \in \mathcal{D}_\theta \cap \mathcal{D}_\psi).
	\]
	Mutassuk meg, hogy van olyan differenciálható
	\[
		m : I \to \R
	\]
	függvény, hogy az $m \cdot \varphi_0$ függvény megoldása a most vizsgált (inhomogén) lineáris differenciálegyenletnek (\textit{az állandók variálásának módszere}). Ehhez azt kell "biztosítani", hogy
	\[
		(m \cdot \varphi_0)' = g \cdot m \cdot \varphi_0 + h,
	\]
	azaz
	\[
		m' \cdot \varphi_0 + m \cdot \varphi_0' = m' \cdot \varphi_0 + m \cdot g \cdot \varphi_0 = g \cdot m \cdot \varphi_0 + h.
	\]
	Innen szükséges feltételként az adódik az $m$-re, hogy
	\[
		m' = \frac{h}{\varphi_0}.
	\]
	Ilyen $m$ függvény valóban létezik, mivel a
	\[
		\frac{h}{\varphi_0} : I \to \R
	\]
	folytonos leképezésnek van primitív függvénye. Továbbá -- az előbbi rövid számolás "megfordításából" -- azt is beláthatjuk, hogy a $h / \varphi_0$ függvény bármelyik $m$ primitív függvényét is véve, az $m \cdot \varphi_0$ függvény megoldása a lineáris differenciálegyenletünknek.\\
	
	Összefoglalva az eddigieket azt mondhatjuk, hogy a fenti lineáris differenciálegyenletnek van megoldása, és tetszőleges $\varphi \in I \to \R$ megoldása
	\[
		\varphi(t) = m(t) \cdot \varphi_0(t) + c \cdot \varphi_0(t) \quad (t \in \Dp)
	\]
	alakú, ahol $m$ egy tetszőleges primitív függvénye a $h / \varphi_0$ függvénynek. Sőt, az is kiderül, hogy akármilyen $c \in \R$ és $J \subset I$ nyílt intervallum esetén a
	\[
		\varphi(t) := m(t) \cdot \varphi_0(t) + c \cdot \varphi_0(t) \quad (t \in J)
	\]
	függvény megoldás. Ezt megint csak egyszerű behelyettesítéssel ellenőrizhető:
	\[
		\varphi'(t) = m'(t) \cdot \varphi_0(t) + (c + m(t)) \cdot \varphi_0'(t) =
	\]
	\[
		\frac{h(t)}{\varphi_0(t)} \cdot \varphi_0(t) + (c + m(t)) \cdot g(t) \cdot \varphi_0(t) = g(t) \cdot \varphi(t) + h(t) \quad (t \in J).
	\]
	Speciálisan az "egész" $I$ intervallumon értelmezett
	\[
		\psi_c(t) := m(t) \cdot \varphi_0(t) + c \cdot \varphi_0(t) \quad (c \in \R, \, t \in I)
	\]
	megoldások olyanok, hogy bármelyik $\varphi$ megoldásra egy alkalmas $c \in \R$ mellett
	\[
		\varphi(t) = \psi_c(t) \quad (t \in \Dp),
	\]
	azaz a $J := \Dp$ jelöléssel $\varphi = \psi_{c_{|_J}}$.\\
	
	Ha $\tau \in I, \, \xi \in \R$, és a $\varphi(\tau) = \xi$ kezdetiérték-feladatot kell megoldanunk, akkor a
	\[
		c := \frac{\xi - m(\tau) \cdot \varphi_0(\tau)}{\varphi_0(\tau)}
	\]
	választással a szóban forgó kezdetiérték-probléma
	\[
		\psi_c : I \to \R
	\]
	megoldását kapjuk. Mivel a fentiek alapján a szóban forgó k.é.p. minden $\varphi, \, \psi$ megoldására $\varphi = \psi_{c_{|_{\Dp}}}$ és $\psi = \psi_{c_{|_{\mathcal{D}_\psi}}}$, ezért egyúttal az is teljesül, hogy
	\[
		\varphi(t) = \psi(t) \quad (t \in \Dp \cap \mathcal{D}_\psi).
	\]
	$\hfill \blacksquare$\\
	
	A
	\[
		\varphi'(t) = g(t) \cdot \varphi(t) + h(t) \quad (t \in \Dp)
	\]
	lineáris d.e.-re vonatkozó k.é.p. megoldására "megoldóképletet" is felírhatunk. Legyen ui.
	\[
		G(x) := \int\limits_\tau^x g(t) \, dt, \, m(x) := \int\limits_\tau^x \frac{h(t)}{\varphi_0(t)} \, dt \quad (x \in I),
	\]
	ekkor
	\[
		\varphi_0(\tau) = e^{G(\tau)} = 1, \, m(\tau) = 0,
	\]
	és a szóban forgó k.é.p. "teljes" megoldását jelentő
	\[
		\psi(x) := m(x) \cdot \varphi_0(x) + c \cdot \varphi_0(x) \quad (x \in I)
	\]
	függvényben
	\[
		c := \frac{\xi - m(\tau) \cdot \varphi_0(\tau)}{\varphi_0(\tau)} = \xi.
	\]
	
	\newpage
	\section{Gyakorlat}
	\subsection{Emlékeztető}
	\subsubsection{Lineáris differenciálegyenlet-rendszer}
	Valamilyen $1 \leq n \in N$ és egy nyílt $I \subset \R$ intervallum esetén adottak a folytonos
	\[
		a_{ik} : I \to \R \quad (i, \, k = 1, \, \dots, \, n), \, b = (b_1, \, \dots, \, b_n) : I \to \R^n
	\]
	függvények, és tekintsük az
	\[
		I \ni x \mapsto A(x) := \big( a_{ik}(x) \big)_{i, \, k =1}^n \in \R^{n \times n}
	\]
	\textit{mátrixfüggvényt}. Ha
	\[
		f(x, \, y) := A(x) \cdot y + b(x) \quad ((x, \, y) \in I \times \K^n),
	\]
	akkor az $f$ függvény, mint jobb oldal által meghatározott
	\[
		\varphi'(x) = A(x) \cdot \varphi(x) + b(x) \quad (x \in \Dp)
	\]
	differenciálegyenletet \textit{lineáris differenciálegyenletnek} ($n > 1$ esetén \textit{lineáris} \textit{differenciálegyenlet-rendszernek}) nevezzük.\\
	
	Legyenek a fentieken túl adottak még a $\tau \in I, \, \xi \in \K^n$ értékek, és vizsgáljuk a $\varphi(\tau) = \xi$ k.é.p.-t. Ha $I_* \subset I$, $\tau \in \text{int} \, I_*$, kompakt intervallum, akkor
	\[
		\sup \{ |a_{ik}(x)| : x \in I_* \} \in \R \quad (i, \, k = 1, \, \dots, \, n),
	\]
	ezért
	\[
		q := \sup \{ \|A(x)\|_{(\infty)} : x \in I_* \} \in \R.
	\]
	Következésképpen
	\[
		\| f(x, \, y) - f(x, \, z) \|_\infty = \| A(x) \cdot (y- z) \|_\infty \leq
	\]
	\[
		\|A(x)\|_{(\infty)} \cdot \|y-z\|_\infty \leq q \cdot \|y-z\|_\infty \quad (x \in I_*, \, y, \, z \in \K^n).
	\]
	Továbbá a
	\[
		\beta := \sup\{\|b(x)\| : x \in I_*\} (\in \R)
	\]
	jelöléssel
	\[
		\|f(x, \, y)\|_\infty = \| A(x) \cdot y + b(x) \|_\infty \leq \|A(x) \cdot y\|_\infty + \|b(x)\|_\infty \leq
	\]
	\[
		\|A(x)\|_{(\infty)} \cdot \|y\|_\infty + \|b(x)\|_\infty \leq q \cdot \|y\|_\infty + \beta \quad (x \in I_*, \, y \in \K^n),
	\]
	ezért minden k.é.p. teljes megoldása az $I$ intervallumon van értelmezve. Azt mondjuk, hogy a szóban forgó d.e. \textit{homogén}, ha $b \equiv 0$, \textit{inhomogén}, ha létezik $x \in I$, hogy $b(x) \neq 0$. Legyenek
	\[
		\mathcal{M}_h := \{ \psi : I \to \K^n : \psi \in D, \, \psi' = A \cdot \psi \},
	\]
	\[
		\mathcal{M} := \{ \psi : I \to \K^n : \psi \in D, \, \psi' = A \cdot \psi + b\}.
	\]
	A lineáris d.e.-ek "alaptétele" a következő\\
	
	\tikz \node[theorem]
	{
		\textbf{Tétel.} A bevezetésben mondott feltételek mellett
		\begin{enumerate}
			\item az $\mathcal{M}_h$ halmaz $n$ dimenziós lineáris tér a $\K$-ra vonatkozóan;
			\item tetszőleges $\psi \in \mathcal{M}$ esetén
			\[
				\mathcal{M} = \psi + \mathcal{M}_h := \{\psi + \chi : \chi \in \mathcal{M}_h \};
			\]
			\item ha a $\phi_k = (\phi_{k1}, \, \dots, \, \phi_{kn})$ $(k = 1, \, \dots, \, n)$ függvények bázist alkotnak az $\mathcal{M}_h$-ban, akkor léteznek olyan $g_k : I \to \K$ $(k = 1, \, \dots, \, n)$ differenciálható függvények, amelyekkel
			\[
				\psi := \sum_{k=1}^n g_k \cdot \phi_k \in \mathcal{M}.
			\]
		\end{enumerate}
	};\\
	
	\textbf{Bizonyítás.} Az $1.$ állítás bizonyításához mutassuk meg először is azt, hogy bármilyen $\psi, \, \varphi \in \mathcal{M}_h$ és $c \in \K$ esetén $\psi + c \cdot \varphi \in \mathcal{M}_h$:
	\[
		(\psi + c \cdot \varphi)' = \psi' + c \cdot \varphi' = A \cdot \psi + c \cdot A \cdot \varphi = A \cdot (\psi + c \cdot \varphi),
	\]
	amiből a mondott állítás az $\mathcal{M}_h$ definíciója alapján nyilvánvaló. Tehát az $\mathcal{M}_h$ lineáris tér a $\K$ felett.\\
	
	Most megmutatjuk, hogy ha $m \in \N$, és $\chi_1, \, \dots, \, \chi_m \in \mathcal{M}_h$ tetszőleges függvények, akkor az alábbi ekvivalencia igaz:
	\begin{quote}
		a $\chi_1, \, \dots, \, \chi_m$ függvények akkor és csak akkor alkotnak lineárisan független rendszert az $\mathcal{M}_h$ vektortérben, ha bármilyen $\tau \in I$ esetén a $\chi_1(\tau), \, \dots, \, \chi_m(\tau)$ vektorok lineárisan függetlenek a $\K^n$-ben.
	\end{quote}
	
	Az ekvivalencia egyik fele nyilvánvaló: ha a $\chi_1, \, \dots, \, \chi_m$-ek lineárisan összefüggnek, akkor alkalmas $c_1, \, \dots, \, c_m \in \K$, $|c_1| + \cdots + |c_m| > 0$ együtthatókkal
	\[
		\sum_{k=1}^m c_k \cdot \chi_k \equiv 0.
	\]
	Speciálisan minden $\tau \in I$ helyen is
	\[
		\sum_{k=1}^m c_k \cdot \chi_k(\tau) = 0.
	\]
	Így a $\chi_1(\tau), \, \dots, \, \chi_m(\tau)$ vektorok összefüggő rendszert alkotnak $\K^n$-ben.\\
	
	Fordítva, legyen $\tau \in I$, és tegyük fel, hogy a $\chi_1(\tau), \, \dots, \, \chi_m(\tau)$ vektorok összefüggnek. Ekkor az előbbi (nem csupa nulla) $c_1, \, \dots, \, c_m \in K$ együtthatókkal
	\[
		\sum_{k=1}^m c_k \cdot \chi_k(\tau) = 0.
	\]
	Már tudjuk. hogy
	\[
		\phi := \sum_{k=1}^m c_k \cdot \chi_k \in \mathcal{M}_h,
	\]
	ezért az így definiált $\phi : I \to \K^n$ függvény megoldása a
	\[
		\varphi' = A \cdot \varphi, \, \varphi(\tau) = 0
	\]
	homogén lineáris differenciálegyenletre vonatkozó kezdetiérték-problémának. Világos ugyanakkor, hogy a $\Psi \equiv 0$ függvény is a most mondott k.é.p. megoldása az $I$-n. Azt is tudjuk azonban, hogy ez a k.é.p. (is) egyértelműen oldható meg, ezért $\phi \equiv \Psi \equiv 0$. Tehát a $\chi_1, \, \dots, \, \chi_m$ függvények is összefüggnek.\\
	
	Ezzel egyúttal azt is beláttuk, hogy az $\mathcal{M}_h$ vektortér véges dimenziós, és a $\dim \mathcal{M}_h$ dimenziója legfeljebb $n$.\\
	
	Tekintsük most a 
	\[
		\varphi' = A \cdot \varphi, \, \varphi(\tau) = e_i \quad (i = 1, \, \dots, \, n)
	\]
	kezdetiérték-problémákat, ahol az $e_i \in \K^n$ $(i = 1, \, \dots, \, n)$ vektorok az $\K^n$ tér "szokásos" (kanonikus) bázisvektorait jelölik. Ha
	\[
		\chi_i : I \to \K^n
	\]
	jelöli az említett k.é.p. teljes megoldását, akkor a
	\[
		\chi_i(\tau) = e_i \quad (i = 1, \, \dots, \, n)
	\]
	vektorok lineárisan függetlenek. Így az előbbiek alapján $\chi_1, \, \dots, \, \chi_m$ függvények is azok. Tehát az $\mathcal{M}_h$ dimenziója legalább $n$, azaz a fentiekre tekintettel $\dim \mathcal{M}_h = n$.\\
	
	A 2. állítás igazolásához legyen $\chi \in \mathcal{M}_h$. Ekkor $\psi + \chi \in D$, és
	\[
		(\psi + \chi)' = \psi' + \chi' = A \cdot \psi + b + A \cdot \chi = A \cdot (\psi + \chi) + b,
	\]
	amiből $\psi + \chi \in \mathcal{M}$ következik. Ha most egy $\varphi \in \mathcal{M}$ függvényből indulunk ki és $\chi := \varphi - \psi$, akkor $\chi \in D$, és
	\[
		\chi' = \varphi' - \psi' = A \cdot \varphi + b - (A \cdot \psi + b) = A \cdot \chi,
	\]
	amiből $\chi \in \mathcal{M}_h$ adódik. Tehát $\varphi = \psi + \chi$.\\
	
	A tétel. 3. részének a bizonyítása érdekében vezessük be az alábbi jelöléseket, ill, fogalmakat. A
	\[
		\phi_k = (\phi_{k1}, \, \dots, \, \phi_{kn}) \quad (k=1, \, \dots, \, n)
	\]
	bázisfüggvények mint oszlopvektor-függvények segítségével tekintsük a
	\[
		\Phi : I \to \K^{n \times n}
	\]
	mátrixfüggvényt:
	\[
		\Phi := [\phi_1 \cdots \phi_n] = \begin{bmatrix}
			\phi_{11} & \phi_{21} & \cdots & \phi_{n1} \\
			\phi_{12} & \phi_{22} & \cdots & \phi_{n2} \\
			\vdots & \vdots & \cdots & \vdots \\
			\phi_{1n} & \phi_{2n} & \cdots & \phi_{nn}
		\end{bmatrix}.
	\]
	Legyen
	\[
		\Phi' := [\phi'_1 \cdots \phi'_n] = \begin{bmatrix}
			\phi'_{11} & \phi'_{21} & \cdots & \phi'_{n1} \\
			\phi'_{12} & \phi'_{22} & \cdots & \phi'_{n2} \\
			\vdots & \vdots & \cdots & \vdots \\
			\phi'_{1n} & \phi'_{2n} & \cdots & \phi'_{nn}
		\end{bmatrix}
	\]
	a $\Phi$ deriváltja. Ekkor könnyen belátható, hogy
	\[
		\Phi' = A \cdot \Phi.
	\]
	Továbbá tetszőleges $g_1, \, \dots, \, g_n : I \to \K$ differenciálható függvényekkel a
	\[
		g := (g_1, \, \dots, \, g_n) : I \to \K^n
	\]
	vektorfüggvény differenciálható,
	\[
		\psi := \sum_{k=1}^n g_k \cdot \phi_k = \Phi \cdot g,
	\]
	és
	\[
		\psi' = \Phi' \cdot g + \Phi \cdot g' = (A \cdot \Phi) \cdot g + \Phi \cdot g'.
	\]
	A $\psi \in \mathcal{M}$ tartalmazás nyilván azzal ekvivalens, hogy
	\[
		\psi' = (A \cdot \Phi) \cdot g + \Phi \cdot g' = A \cdot \psi + b = A \cdot (\Phi \cdot g) + b = (A \cdot \Phi) \cdot g + b,
	\]
	következésképpen azzal, hogy
	\[
		\Phi \cdot g' = b.
	\]
	A 2. pont alapján tetszőleges $x \in I$ helyen a $\phi_1(x), \, \dots, \, \phi_n(x)$ vektorok lineárisan függetlenek, azaz a $\Phi(x)$ mátrix nem szinguláris. A mátrixok inverzének a kiszámítása alapján egyszerűen adódik, hogy a
	\[
		\Phi^{-1}(x) := \big( \Phi(x) \big)^{-1} \quad (x \in I)
	\]
	definícióval értelmezett
	\[
		\Phi^{-1} : I \to \K^{n\times n}
	\]
	mátrixfüggvény komponens-függvényei folytonosak. Ezért a
	\[
		(h_1, \, \dots, \, h_b) := \Phi^{-1} \cdot b : I \to \K^n
	\]
	függvény is folytonos. Olyan folytonosan differenciálható
	\[
		g : I \to \K^n
	\]
	függvényt keresünk tehát, amelyikre $g' = \Phi^{-1} \cdot b$, azaz
	\[
		g_i' = h_i \quad (i = 1, \, \dots, \, n).
	\]
	Ilyen $g_i$ létezik, nevezetesen a (folytonos) $h_i$ $(i = 1, \, \dots, \, n)$ függvény bármely primitív függvénye ilyen. $\blacksquare$\\
	
	Ezt összefoglalva, egy lineáris differenciálegyenlet-rendszer megoldása a következőképpen történik:
	\begin{enumerate}
		\item Meghatározzuk a $\phi_1, \, \dots, \, \phi_n$ bázist (\textit{alaprendszert}). (Ez általában \textit{lehetetlen}, nincs rá "képlet", viszont bizonyos feltételek mellett működik.) Ezzel a bázissal definiáljuk a $\Phi$ mátrixot (\textit{alapmátrix}).
		\item A tétel 3. pontja szerint keresnünk kell egy $g : I \to \K^n$ differenciálható függvényt, amellyel
		\[
			\psi := \Phi \cdot g \in \mathcal{M}.
		\]
		(\textit{Az állandók variálásának módszere.} $\psi$ az ún. \textit{partikuláris megoldás}). Azaz olyan $g$-t, ami eleget tesz a
		\[
			g' = \Phi^{-1} \cdot b
		\]
		egyenlőségnek. Mivel $\Phi^{-1} \cdot b$ folytonos (és $I$ nyílt intervallum), ilyen $g$ létezik:
		\[	
			(h_1, \, \dots, \, h_n) := \Phi^{-1} \cdot b : I \to \K^n,
		\]
		\[
			g_i \in \int h_i.
		\]	
	\end{enumerate}
	
	\subsubsection{Állandó együtthatós lineáris differenciálegyenlet-rendszer, diagonalizálható mátrix}
	Legyen most
	\[
		f(x, \, y) := A \cdot y + b(x) \quad \big( (x, \, y) \in I \times \K^n \big),
	\]
	ahol $1 \leq n \in \N, \, I \subset \R$ nyílt intervallum mellett
	\[
		A \in \R^{n \times n}, \, b : I \to \R^n, \, b \in C.
	\]
	Tegyük fel, hogy $A$ diagonalizálható, azaz létezik $T \in \K^{n \times n}, \, \det T \neq 0$, hogy $T^{-1}AT$ mátrix diagonális: alkalmas $\lambda_1, \, \dots, \, \lambda_n \in \K$ számokkal
	\[
		\Lambda := T^{-1}AT = \begin{bmatrix}
			\lambda_1 & 0 & \cdots & 0 \\
			0 & \lambda_2 & \cdots & 0 \\
			\vdots & \vdots & \cdots & \vdots \\
			0 & 0 & \cdots & \lambda_n
		\end{bmatrix}.
	\]
	$T$ invertálhatósága miatt a
	\[
		 T = [t_1 \cdots t_n]
	\]
	$t_i$ $(i=1, \, \dots, \, n)$ oszlopvektorok lineárisan függetlenek, azaz
	\[
		AT = [At_1 \cdots At_n] = T\Lambda = [\lambda_1 \cdot t_1 \cdots \lambda_n \cdots t_n]
	\]
	miatt
	\[
		A \cdot t_i = \lambda_i \cdot t_i \quad (i = 1, \, \dots, \, n).
	\]
	Mivel
	\[
		t_i \neq 0 \quad (i = 1, \, \dots, \, n),
	\]
	ezért mindez röviden azt jelenti, hogy a $\lambda_1, \, \dots, \, \lambda_n$ számok az $A$ mátrix sajátértékei, a $t_1, \, \dots, \, t_n$ vektorok pedig rendre a megfelelő sajátvektorok. Lévén, a $t_i$-k lineárisan függetlenek, az $A$-ra vonatkozó feltételünk úgy fogalmazható, hogy van a $\K^n$-ben (az $A$ sajátvektoraiból álló) sajátvektorbázis.\\
	
	A homogén egyenlet tehát a következőképpen írható fel:
	\[
		\varphi' = A \cdot \varphi = T \Lambda T^{-1} \cdot \varphi,
	\]
	amiből
	\[
		(T^{-1}\varphi)' = \Lambda \cdot (T^{-1} \varphi)
	\]
	következik. Vegyük észre, hogy ha $\varphi \in \mathcal{M}_h$, akkor a $\psi := T^{-1}\varphi$ függvény megoldása a $\Lambda$ diagonális mátrix által meghatározott állandó együtthatós homogén lineáris egyenletnek. Ez utóbbit ez előző tétel alapján nem nehéz megoldani. Legyenek iu. a
	\[
		\psi_i : I \to \K^n \quad (i = 1, \, \dots, \, n)
	\]
	függvények a következők:
	\[
		\psi_i(x) := e^{\lambda_i \cdot x} \cdot e_i \quad (x \in I, \, i = 1, \, \dots, \, n).
	\]
	Világos, hogy $\psi_i \in D$ és
	\[
		\psi_i'(x) = \lambda_i \cdot e^{\lambda_i \cdot x} \cdot e_i = e^{\lambda_i \cdot x} \cdot (\Lambda \cdot e_i) =
	\]
	\[
		\Lambda \cdot (e^{\lambda_i \cdot x} \cdot e_i) = \Lambda \cdot \psi_i(x) \quad (x \in I, \, i = 1, \, \dots, \, n).
	\]
	Más szóval a $\psi_i$-k valóban megoldásai a $\Lambda$ által meghatározott homogén lineáris differenciálegyenletnek. Mivel bármely $\tau \in I$ esetén a
	\[
		\psi_i(\tau) = e^{\lambda_i \cdot \tau} \cdot e_i \quad (i = 1, \, \dots, \, n)
	\]
	vektorok nyilván lineárisan függetlenek, ezért az előző tétel bizonyításában mondottak szerint a $\psi_i$ $(i = 1, \, \dots, \, n)$ függvények lineárisan függetlenek. Ha
	\[
		\phi_i := T \cdot \psi_i \quad (i = 1, \, \dots, \, n),
	\]
	akkor nyilván a $\phi_i$-k is lineárisan függetlenek,
	\[
		\phi_i(x) = e^{\lambda_i \cdot x} \cdot t_i \quad (x \in I, \, i = 1, \, \dots, \, n),
	\]
	és minden $i = 1, \, \dots, \, n$ indexre
	\[
		\phi_i' = A \cdot \phi_i.
	\]
	Tehát $\phi_i \in \mathcal{M}_h$ $(i = 1, \, \dots, \, n)$ egy bázis. Ezzel beláttuk az alábbi tételt:\\
	
	\tikz \node[theorem]
	{
		\textbf{Tétel.} Tegyük fel, hogy az $A \in \R^{n \times n}$ mátrix diagonalizálható. Legyenek a sajátértékei $\lambda_1, \, \dots, \, \lambda_n \in \K$, egy-egy megfelelő sajátvektora pedig $t_1, \, \dots, \, t_n \in \K^n$. Ekkor a
		\[
			\varphi' = A \cdot \varphi
		\]
		homogén lineáris differenciálegyenletnek a
		\[
			\phi_i(x) := e^{\lambda_i \cdot x} \cdot t_i \quad (x \in \R, \, i = 1, \, \dots, \, n)
		\]
		függvények lineárisan független megoldásai.
	};

	\subsubsection{Állandó együtthatós lineáris differenciálegyenlet-rendszer, általános eset}
	Csak az $n=2$ esettel foglalkozunk. Legyen $I \subset \R$ nyílt intervallum, $b : I \to \R^2$ folytonos függvény, $A \in \R^{2 \times 2}$ és vizsgáljuk a
	\[
		\p'(x) = A \cdot \p(x) + b(x) \quad (x \in \Dp)
	\]
	lineáris differenciálegyenletet. Valamilyen $a, \, b, \, c, \, d \in \R$ számokkal
	\[
		A = \begin{bmatrix}
			a & b \\ 
			c & d
		\end{bmatrix} \in \R^{2 \times 2}.
	\]
	Könnyű meggyőződni arról, hogy ez a mátrix pontosan akkor nem diagonalizálható, ha
	\[
		(a - d)^2 + 4bc = 0 \text{ és } |b| + |c| > 0.
	\]
	Ekkor egyetlen sajátértéke van az $A$-nak, nevezetesen
	\[
		\lambda := \frac{a+d}{2},
	\]
	legyen a $t_1$ egy hozzá tartozó sajátvektor:
	\[
		0 \neq t_1 \in \R^2, \, At_1 = \lambda t_1.
	\]
	Egyszerű számolással igazolható, hogy létezik $t_2 \in \R^2$ vektor, amelyik lineárisan független a $t_1$-től és
	\[
		At_2 = t_1 + \lambda t_2.
	\]
	Ekkor
	\[
	\boxed{\phi_1(x) := e^{\lambda x} \cdot t_1, \, \phi_2(x) := e^{\lambda x} \cdot (t_2 + x t_1)} \quad (x \in \R)
	\]
	függvénypár egy alaprendszer. 

\end{document}
