\documentclass{article}
\usepackage{tikz}
\usepackage{setspace}
\usepackage{hyperref}
\usepackage{amsfonts, amssymb, amsmath}
\usepackage{titlesec}
\usepackage{pgfplots}
\pgfplotsset{compat=newest}
\usepackage{graphicx}
\usepackage{wrapfig}
\usepackage{caption}
\usepackage{enumitem}
\usetikzlibrary{shadows.blur}
\usepackage{lmodern}
\usepackage{mathrsfs}
\setlength{\parskip}{0pt}
\setlength{\parindent}{0pt}

\title{\textcolor{purple}{\Huge\textbf{Analízis 4}}}
\author{Gyakorlati feladatok}
\date{}

\renewcommand{\contentsname}{Tartalom}
\newcommand{\R}{\mathbb{R}}
\newcommand{\N}{\mathbb{N}}
\newcommand{\E}{\exists}
\newcommand{\mm}{\mathbf{m}}
\newcommand{\MM}{\mathbf{M}}
\newcommand{\K}{\mathbb{K}}
\newcommand{\D}{\mathcal{D}_f}
\newcommand{\Dp}{\mathcal{D}_\varphi}

\definecolor{modernyellow}{HTML}{F4E4BC}
\definecolor{moderngreen}{HTML}{BDDCBD}

\titleformat{\section}[block]{\large\bfseries}{\thesection}{1em}{}
\titlespacing*{\section}{0pt}{1.5ex plus 1ex minus .2ex}{1.3ex plus .2ex}

\tikzset
{
    definition/.style={
        draw,
        fill=white,
        line width=1pt,
        rounded corners,
        drop shadow={shadow blur steps=5,shadow xshift=1ex,shadow yshift=-1ex},
        text width=1\textwidth,
        inner sep=8pt,
        align=justify
    },
    theorem/.style={
        draw,
        fill=white,
        line width=1pt,
        rounded corners,
        drop shadow={shadow blur steps=5,shadow xshift=1ex,shadow yshift=-1ex},
        text width=1\textwidth,
        inner sep=8pt,
        align=justify
    },
    proof/.style={
            fill=white,
            rectangle,
            drop shadow={shadow blur steps=5,shadow xshift=1ex,shadow yshift=-1ex, moderngreen},
            text width=1\textwidth,
            inner sep=8pt,
            align=justify
    },
    proof1/.style={
            fill=white,
            rectangle,
            drop shadow={shadow blur steps=5,shadow xshift=1ex,shadow yshift=0, moderngreen},
            text width=1\textwidth,
            inner sep=8pt,
            align=justify
    }
}

\begin{document}
    \maketitle
    \tableofcontents
    \newpage
	
	\section{Gyakorlat}
	\subsection{Emlékeztető}

	\tikz \node[definition]
	{
		\textbf{Definíció.} Legyen $1 \leq s \in \N, \, A \subset \R^s$. Azt mondjuk, hogy az $A$ halmaz \textit{nullmértékű}, ha tetszőleges $\varepsilon > 0$ számhoz megadható $I_k \subset \R^s \, (k \in \N)$ intervallumoknak egy olyan sorozata, hogy
		\[
			A \subset \bigcup_{k=0}^\infty I_k \text{ és } \sum_{k=0}^\infty |I_k| < \varepsilon.
		\]
	};\\

	Tekintsük pl. az $I \subset \R^n \, (1 \leq n \in \N)$ intervallum esetén az $f \in R(I)$ függvényt, és legyen
	\[
		\text{graf} \,  f := \{ (x, \, f(x)) \in \R^{n+1} : x \in I \}.
	\]
	Ekkor a $\text{graf} \, f \subset \R^{n+1}$ halmaz nullmértékű.\\

	Ha az $f \in \R^n \to \R$ függvény $\D \subset \R^n$ értelmezési tartománya korlátos, akkor van olyan $I \subset \R^n$ intervallum, amellyel $\D \subset I.$ Legyen ekkor
	\[
		F_I(x) := \begin{cases}
			f(x) & (x \in \D) \\
			0 & (x \in I \, \backslash \, \D).
		\end{cases}
	\]
	Nem nehéz meggondolni, hogy ha $F_I \in R(I)$, akkor $F_J \in R(J)$ minden olyan $J \subset \R^n$ intervallumra, amelyre $\D \subset J$ és
	\[
		\int_I F_I = \int_J F_J.
	\]
	Ez ad értelmet a következő definíciónak:\\

	\tikz \node[definition]
	{
		\textbf{Definíció.} A fenti $f$ függvény \textit{Riemann-integrálható}, ha egy alkalmas $I \subset \R^n$ intervallummal $\D \subset I$ és $F_I \in R(I)$. Az utóbbi esetben
		\[
			\int_{\R^n} f := \int_{\D} f := \int_I F_I.
		\]
	};\\

	Az előző definició olyan függvényekre is kiterjeszthető, amik értelmezési tartománya nem korlátos, viszont a
	\[
		\text{supp}\, f := \overline{\{ x \in \D : f(x) \neq 0 \}}
	\]
	\textit{tartója} igen.
	
	\tikz \node[definition]
	{
		\textbf{Definíció.} Legyen az $A \subset \R^n$ halmaz korlátos,
		\[
			\chi_A(x) := \begin{cases}
				1 & (x \in A) \\
				0 & (x \in \R^n \, \backslash \, A)
			\end{cases}
		\]
		az $A$ \textit{karakterisztikus függvénye}. Azt mondjuk, hogy az $A$ halmaz \textit{Jordan-mérhető}, ha a $\chi_A$ függvény integrálható, amikor is a
		\[
			\mu(A) := \int_{\R^n} \chi_A
		\]
		nemnegatív szám az $A$ halmaz \textit{Jordan-mértéke}. 
	};\\

	\tikz \node[theorem]
	{
		\textbf{Tétel.} Tekintsük a nyílt halmazon értelmezett és folytonosan differenciálható
		\[
			g \in \R^n \to \R^n
		\]
		függvényt. Tegyük fel, hogy az $I \subset \mathcal{D}_g$ halmaz kompakt intervallum, továbbá az $I$ belsejére való $g_{|_{\text{int} \, I}}$ leszűkítés injektív függvény. Ekkor az
		\[
			f : g(I) \to \R
		\]
		korlátos függvény akkor és csak akkor integrálható, ha az
		\[
			I \ni x \mapsto f(g(x)) \cdot |\det g'(x)|
		\]
		függvény is integrálható. Az utóbbi esetben
		\[
			\int_I f(g(x)) \cdot |\det g'(x)| \, dx = \int_{g[I]} f.
		\]
	};\\

	\textbf{Speciális esetek.}

	\begin{enumerate}
		\item \textbf{Síkbeli polárkoordináta-transzformáció.} Legyen $g : \R^2 \to \R^2$ leképezés, amire
		\[
			g(r, \, \varphi) := \begin{pmatrix}
				r \cdot \cos \varphi, \\
				r \cdot \sin \varphi
			\end{pmatrix} \quad \big( (r, \, \varphi) \in \R^2 \big).
		\]
		Ekkor $g \in C^1$, és
		\[
			\det g'(r, \, \varphi) = r \quad \big( (r, \, \varphi) \in \R^2 \big).
		\]
		\item \textbf{Módosított síkbeli polárkoordináta-transzformáció.} Legyen $a, \, b \in \R, \, g : \R^2 \to \R^2$ leképezés, amire
		\[
			g(r, \, \varphi) := \begin{pmatrix}
				ar \cdot \cos \varphi, \\
				br \cdot \sin \varphi
			\end{pmatrix} \quad \big( (r, \, \varphi) \in \R^2 \big).
		\]
		Ekkor $g \in C^1$, és
		\[
			\det g'(r, \, \varphi) = abr \quad \big( (r, \, \varphi) \in \R^2 \big).
		\]
		\item \textbf{Térbeli polárkoordináta-transzformáció.} Legyen $g : \R^3 \to \R^3$ leképezés, amire
		\[
			g(r, \, \varphi, \, \psi) := \begin{pmatrix}
				r \cdot \cos \varphi \sin \psi \\
				r \cdot \sin \varphi \sin \psi \\
				r \cdot \cos \psi
			\end{pmatrix} \quad \big( (r, \, \varphi, \, \psi) \in \R^3\big).
		\]
		Ekkor $g \in C^1$, és
		\[
			\det g'(r, \, \varphi, \, \psi) = -r^2 \cdot \sin \psi \quad \big( (r, \, \varphi, \, \psi) \in \R^3\big).
		\]
		\item \textbf{Módosított térbeli polárkoordináta-transzformáció (elliptikus koordináták).} Legyen $a, \, b, \, c \in \R, \, g : \R^3 \to \R^3$ leképezés, amire
		\[
			g(r, \, \varphi, \, \psi) := \begin{pmatrix}
				ar \cdot \cos \varphi \sin \psi \\
				br \cdot \sin \varphi \sin \psi \\
				cr \cdot \cos \psi
			\end{pmatrix} \quad \big( (r, \, \varphi, \, \psi) \in \R^3\big).
		\]
		Ekkor $g \in C^1$, és
		\[
			\det g'(r, \, \varphi, \, \psi) = -abcr^2 \cdot \sin \psi \quad \big( (r, \, \varphi, \, \psi) \in \R^3\big).
		\]
		\item \textbf{Hengerkoordináta-transzformáció.} Legyen $g : \R^3 \to \R^3$ leképezés, amire
		\[
			g(r, \, \varphi, \, z) := \begin{pmatrix}
				r \cdot \cos \varphi \\
				r \cdot \sin \varphi \\
				z
			\end{pmatrix} \quad \big( (r, \, \varphi, \, z) \in \R^3 \big).
		\]
		Ekkor $g \in C^1$, és
		\[
			\det g'(r, \, \varphi, \, z) = r \quad \big( (r, \, \varphi, \, z) \in \R^3 \big).
		\]
		\item \textbf{Módosított hengerkoordináta-transzformáció.} Legyen $a, \, b, \, c \in \R, \, g : \R^3 \to \R^3$ leképezés, amire
		\[
			g(r, \, \varphi, \, z) := \begin{pmatrix}
				ar \cdot \cos \varphi \\
				br \cdot \sin \varphi \\
				cz
			\end{pmatrix} \quad \big( (r, \, \varphi, \, z) \in \R^3 \big).
		\]
		Ekkor $g \in C^1$, és
		\[
			\det g'(r, \, \varphi, \, z) = abcr \quad \big( (r, \, \varphi, \, z) \in \R^3 \big).
		\]
	\end{enumerate}

	\tikz \node[theorem]
	{
		\textbf{Tétel.} Tegyük fel, hogy valamely $\emptyset \neq \Omega \subset \R^3$ Jordan-mérhető ponthalmazzal jellemzett test sűrűsége a Riemann-integrálható $\rho : \Omega \to [0, \, +\infty)$ függvénnyel írható le. Ekkor $\Omega$-nak valamely $0 \neq e \in \R^3$ irányvektorú, ill. $a \in \R^3$ pontra illeszkedő tengelyére vonatkozó tehetetlenségi nyomatéka a
		\[
			\mathbf{\Theta}_t = \int_\Omega \ell^2 (r) \rho(r) \, dr
		\]
		valós szám, ahol $\ell(r)$ jelöli az $r \in \Omega$ pontnak a $t$ tengelytől mért távolságát:
		\[
			\ell(r) := \text{inf} \, \{ \|r - (a + \tau e)\| \in \R : \tau \in \R \}.
		\]
	};\\

	Világos, hogy ha $a = 0$ és $e \in \{e_x, \, e_y, \, e_z\}$, akkor $\Omega$-nak a koordináta-tengelyekre vonatkoztatott tehetetlenségi nyomatéka:
	\[
		\mathbf{\Theta}_{t_x} = \int_\Omega (y^2 + z^2) \rho(x, \, y, \, z) \, dx \, dy \, dz
	\]
	\[
		\mathbf{\Theta}_{t_y} = \int_\Omega (x^2 + z^2) \rho(x, \, y, \, z) \, dx \, dy \, dz
	\]
	\[
		\mathbf{\Theta}_{t_z} = \int_\Omega (x^2 + y^2) \rho(x, \, y, \, z) \, dx \, dy \, dz
	\]

	\tikz \node[definition]
	{
		\textbf{Definíció.} Tegyük fel, hogy valamely $\emptyset \neq \Omega \subset \R^3$ Jordan-mérhető ponthalmazzal jellemzett test (tömeg)sűrűsége a Riemann-integrálható $\rho : \Omega \to [0, \, + \infty)$ függvénnyel írható le. (Ha $\rho$ állandó, akkor a testet homogénnek nevezzük.) Ekkor az
		\[
			m(\Omega) := \int_\Omega \rho
		\]
		számot az $\Omega$ test tömegének nevezzük. Az $\Omega$ tömegközéppontjának koordinátái:
		\[
			\overline{x} := \frac{1}{m(\Omega)} \int_\Omega x \cdot \rho(x, \, y, \, z) \, dx \, dy \, dz,
		\]
		\[
			\overline{y} := \frac{1}{m(\Omega)} \int_\Omega y \cdot \rho(x, \, y, \, z) \, dx \, dy \, dz,
		\]
		\[
			\overline{z} := \frac{1}{m(\Omega)} \int_\Omega z \cdot \rho(x, \, y, \, z) \, dx \, dy \, dz.
		\]
	};\\

	\subsection{Feladatok}
	\subsubsection{Feladat}
	Számítsuk ki annak a korlátos és zárt térbeli testnek a \textit{térfogatát}, amelyet az alábbi egyenletű felületek fognak közre:
	\[
		z = x^2 + y^2 -1 \text{ és } z = 2 - x^2 - y^2.
	\]
	Az egyenletekből kapunk két paraboloidot. Ezek metszetét kell \textit{paraméterezni}. Válasszuk két részre a ponthalmazt. Nézzük meg az $y = 0$ síkban lévő pontokat:
	\[
		z = x^2 - 1 \text{ és } z = 2 - x^2.
	\]
	Ez két ellentétes irányba néző parabola, amik metszete
	\[
		x^2 - 1 = 2 - x^2 \quad \Longleftrightarrow \quad x_1 := - \sqrt{\frac{3}{2}}, \, x_2 := \sqrt{\frac{3}{2}}.
	\]
	Azaz az alábbi térbeli pontokban metszi egymást a két paraboloid:
	\[
		p_1 := \left( - \sqrt{\frac{3}{2}}, 0, \frac{1}{2} \right), \, p_2 := \left( \sqrt{\frac{3}{2}}, 0, \frac{1}{2} \right).
	\]
	Szimmetriai okok miatt, be tudjuk vezetni a következő hengerkoordináta-transzformációt:
	\[
		\Phi(r, \, \varphi, \, z) := \begin{pmatrix}
			r \cdot \cos \varphi \\
			r \cdot \sin \varphi \\
			z
		\end{pmatrix} \quad \big( (r, \, \varphi, \, z) \in \R^3 \big).
	\]
	Legyen
	\[
		H_1 := [0, \, \sqrt{3 / 2}] \times [0, \, 2 \pi] \times [1/2, \, 2-r^2] \subset \R^3,
	\]
	\[
		H_2 := [0, \, \sqrt{3 / 2}] \times [0, \, 2 \pi] \times [r^2-1, \, 1/2] \subset \R^3,
	\]
	Mivel ezek kompakt intervallumok, igaz lesz, hogy
	\[
		\int_{\mathcal{H}} 1 = \int_{H_1} 1 \cdot |r| \, dr \, d\varphi \, dz + \int_{H_2} 1 \cdot |r| \, dr \, d\varphi \, dz, 
	\]
	ahol
	\[
		\mathcal{H} := \Phi[H_1] \cup \Phi[H_2].
	\]
	
	\subsubsection{Feladat}
	Számítsuk ki annak a korlátos és zárt térbeli $T$ testnek a \textit{térfogatát, tömegét} és a $z$-tengelyre vonatkozó \textit{tehetetlenségi nyomatékát}, amelyet az alábbi egyenlőtlenségek definiálnak:
	\[
		\sqrt{x^2 + y^2} \leq z \text{ és } x^2 + y^2 + z^2 \leq 1,
	\]
	illetve a kitöltő anyag pontonkénti sűrűsége egyenesen arányos az origótól mért távolsággal.

	Tehát a sűrűségfüggvény adott $\alpha > 0$ mellett
	\[
		\rho(x, \, y, \, z) := \alpha \cdot \sqrt{x^2 + y^2 + z^2} \quad \big( (x, \, y, \, z) \in \R^3\big).
	\]

	Alkalmazzuk az alábbi térbeli polárkoordináta-transzformációt:
	\[
		\Phi(r, \, \varphi, \, \psi) := \begin{pmatrix}
			r \cdot \cos \varphi \sin \psi \\
			r \cdot \sin \varphi \sin \psi \\
			r \cdot \cos \psi
		\end{pmatrix} \quad \big( (r, \, \varphi, \, \psi) \in \R^3\big).
	\]
	Legyen
	\[
		H := [0, \, 1] \times [0, \, 2 \pi] \times [0, \, \pi / 4] \subset \R^3,
	\]
	ekkor $H$ kompakt intervallum és
	\[
		\Phi(H) = T.
	\]
	Térfogat:
	\[
		\int_H 1 \cdot |-r^2 \sin \psi| \, dr \, d\varphi \, d\psi,
	\]
	tömeg:
	\[
		\int_H \rho(r \cos \varphi \sin \psi, \, r \sin \varphi \sin \psi, \, r \cos \psi) \cdot |-r^2 \sin \psi| \, dr \, d\varphi \, d\psi,
	\]
	tehetetlenségi nyomaték:
	\[
		\int_H (r^2 \cos^2\varphi + r^2 \sin^2 \psi) \cdot \rho(r \cos \varphi \sin \psi, \, r \sin \varphi \sin \psi, \, r \cos \psi) \cdot |-r^2 \sin \psi| \, dr \, d\varphi \, d\psi.
	\]
	Számoljuk ki a tehetetlenségi nyomatékot. Először írjuk fel tetszőleges $(r, \, \varphi, \, \psi) \in H$ esetén az integrandust ($\alpha$-val történő osztás után)
	\[
		r^2 \big( \cos^2 \varphi \sin^2 \psi + \sin^2 \varphi \sin^2 \psi \big) \cdot r \sqrt{\cos^2 \varphi \sin^2 \psi + \sin^2 \varphi \sin^2 \psi + \cos^2 \psi} \cdot r^2 \sin \psi =
	\]
	\[
		r^5 \sin^2 \psi \cdot \sqrt{\cos^2 \varphi \sin^2 \psi + \sin^2 \varphi \sin^2 \psi + 1 - \sin^2 \psi} \cdot \sin \psi =
	\]
	\[
		r^5 \sin^3 \psi \cdot \sqrt{ \sin^2 \psi \big( \cos^2 \varphi + \sin^2 \varphi - 1 \big) + 1} =
	\]
	\[
		r^5 \sin^3 \psi.
	\]
	Tehát
	\[
		\frac{\mathbf{\Theta}_{t_x}}{\alpha} = \int_H r^5 \sin^3 \psi \, dr \, d\varphi \, d \psi =
	\]
	\[
		\int\limits_0^{\pi / 4} \int\limits_0^{2 \pi} \int \limits_0^{1} r^5 \sin^3 \psi \, dr \, d\varphi \, d\psi = \int\limits_0^{\pi / 4} \int\limits_0^{2 \pi} \left[ \frac{r^6}{6} \sin^3 \psi \right]_{r=0}^{r=1} \, d\varphi \, d\psi = \frac{1}{6} \int\limits_0^{\pi / 4} \int\limits_0^{2 \pi} \sin^3 \psi \, d\varphi \, d\psi =
	\]
	\[
		\frac{1}{6} \int\limits_0^{\pi / 4} [\varphi \sin^3\psi]_{\varphi=0}^{\varphi=2\pi} \, d\psi = \frac{2 \pi}{6} \int\limits_0^{\pi / 4} \sin^3\psi \, d\psi
	\]

	\newpage
	\section{Gyakorlat}
	\subsection{Emlékeztető}
	Megjegyezzük, hogy ha a
	\[
		\Psi = (\Psi_1, \, \Psi_2, \, \Psi_3) \in \R^2 \to \R^3
	\]
	függvény differenciálható, akkor a
	\[
		\partial_k \Psi := (\partial_k \Psi_1, \, \partial_k \Psi_2, \, \partial_k \Psi_3) \quad (k \in \{1, \, 2\})
	\]
	jelöléssel
	\[
		\Psi' = \begin{bmatrix}
			\partial_1 \Psi & \partial_2 \Psi
		\end{bmatrix} =
		\begin{bmatrix}
			\partial_1 \Psi_1 & \partial_2 \Psi_1 \\
			\partial_1 \Psi_2 & \partial_2 \Psi_2 \\
			\partial_1 \Psi_3 & \partial_2 \Psi_3
		\end{bmatrix}.
	\]

	\tikz \node[definition]
	{
		\textbf{Definíció.} Valamely $\mathcal{F} \subset \R^3$ halmazt \textit{felületdarabnak} nevezünk, ha alkalmas $\emptyset \neq K \subset \R^2$ kompakt és Jordan-mérhető halmaz, ill.
		\[
			\Psi = (\Psi_1, \, \Psi_2, \, \Psi_3) : K \to \R^3, \, \Psi \in C^1, \, \text{rang} (\Psi') = 2
		\]
		függvény esetén
		\[
			\mathcal{F} = \mathcal{R}_\Psi = \{ \Psi(u, \, v) \in \R^3 : (u, \, v) \in K\}.
		\]
		A $\Psi$ leképezést az adott $\mathcal{F}$ felületdarab \textit{paraméterezésének} nevezzük.
	};\\

	\begin{enumerate}
		\item A fenti definícióban a $\Psi$ leképezés folytonos differenciálhatóságán azt értjük, hogy van olyan $K \subset U \subset \R^2$ nyílt halmaz, ill.
		\[
			\hat{\Psi} : U \to \R^3, \, \hat{\Psi} \in C^1
		\]
		függvény, amelyre $\hat{\Psi}_{|_K} = \Psi$.
		\item Ha a $\Psi = (\Psi_1, \, \Psi_2, \, \Psi_3) : K \to \R^3$ függvényre $\Psi \in C^1$, akkor a $\text{rang}(\Psi')=2$ feltétel azt jelenti, hogy a $\partial_1 \Psi, \, \partial_2 \Psi$ vektorok lineárisan függetlenek, azaz
		\[
			\partial_1 \Psi \times \partial_2 \Psi \neq 0.
		\]
		Ha a $\Psi : K \to \R^3$ leképezés valamely felületdarab paraméterezése, akkor az
		\[
			n_\Psi(u, \, v) := \partial_1 \Psi (u, \, v) \times \partial_2 \Psi(u, \, v) \quad \big((u, \, v) \in K\big)
		\]
		vektor az adott felület $\Psi(u, \, v)$ pontjához tartozó \textit{érintősíkjának} normálvektora.
	\end{enumerate}
	
	\tikz \node[definition]
	{
		\textbf{Definíció.} Tegyük fel, hogy az $\mathcal{F} \subset \R^3$ felületdarab paraméterezése az
		\[
			\Psi = (\Psi_1, \, \Psi_2, \, \Psi_3) : K \to \R^3
		\]
		függvény: $\mathcal{R}_\Psi = \mathcal{F}$. Ekkor az $\mathcal{F}$ felület \textit{felszínén} az
		\[
			\mathcal{A}(\mathcal{F}) := \int_K \|n_\Psi\|
		\]
		valós számot értjük.
	};\\

	\begin{enumerate}
		\item Ha $\mathcal{F}_1 := \mathcal{R}_{\Psi_1}$, ill. $\mathcal{F}_2 := \mathcal{R}_{\Psi_2}$ nullmértékű halmazban (pl. élekben) csatlakozó felületdarabok, akkor
		\[
			\mathcal{A}(\mathcal{F}_1 \cup \mathcal{F}_2) = \mathcal{A}(\mathcal{F}_1) + \mathcal{A}(\mathcal{F}_2) = \int_{K_1} \|n_{\Psi_1}\| + \int_{K_2} \|n_{\Psi_2}\|.
		\]
	\end{enumerate}

	\subsection{Feladatok}
	\subsubsection{Feladat}
	Legyen $0 < R \in \R$. Számítsuk ki az
	\[
		x^2 + y^2 + z^2 = R^2, \, z \geq 0
	\]
	félgömbfelületnek az
	\[
		x^2 + y^2 = Rx
	\]
	egyenletű hengerfelület által kimetszett részének (Viviani-féle levél) a felszínét!\\

	A $z \geq 0$ féltérben lévő félgömbfelület paraméterezése:
	\[
		\Psi(u, \, v) := (u, \, v, \, \sqrt{R^2 - u^2 - v^2}) \quad \big( (u, \, v) \in \R^2 : u^2 + v^2 \leq R^2 \big).
	\]
	Ekkor $\Psi \in C^1, \, \text{rang}(\Psi') =2$, továbbá ($u^2+v^2 < R$ esetén)
	\[
		\partial_1 \Psi(u, \, v) = \left(1, \, 0, \, -\frac{u}{\sqrt{R^2-u^2-v^2}}\right),
	\]
	\[
		\partial_2 \Psi(u, \, v) = \left(1, \, 0, \, -\frac{v}{\sqrt{R^2-u^2-v^2}}\right).
	\]
	\[
		n_\Psi(u, \, v) = \begin{pmatrix}
			- \partial_1 g(u, \, v) \\
			- \partial_2 g(u, \, v) \\
			1
		\end{pmatrix} =
		\begin{pmatrix}
			\frac{u}{\sqrt{R^2-u^2-v^2}} \\
			\frac{v}{\sqrt{R^2-u^2-v^2}} \\
			1
		\end{pmatrix},
	\]
	Azaz
	\[
		\|n_\Psi(u, \, v)\| = \sqrt{ \frac{u^2}{R^2-u^2-v^2} + \frac{v^2}{R^2-u^2-v^2} + 1} =
	\]
	\[
		\frac{R}{\sqrt{R^2-u^2-v^2}}.
	\]
	Ha most
	\[
		\Phi(r, \, \varphi) := \begin{pmatrix}
			r \cos\varphi \\
			r \sin \varphi
		\end{pmatrix} \quad \big( \varphi \in [0, \, \pi / 2], \, r \in [0, \, R \cos(\varphi)]\big)
	\]
	Akkor a felszínt így kapjuk meg:
	\[
		2R \cdot \lim_{\varepsilon \to 0} \int\limits_\varepsilon^{\pi/2} \int\limits_0^{R\cos(\varphi)} \Big\|n_\Psi\big(\Phi(r, \, \varphi)\big)\Big\| \cdot r \, dr \, d\varphi.
	\]
	
	\subsubsection{Feladat}
	Határozzuk meg az alábbi feltételekkel megadott felület felszínét:
	\[
		z = \sqrt{x^2 + y^2} \text{ és } x^2 + y^2 \leq 2ax \quad (a > 0).
	\]
	Legyen
	\[
		f(x, \, y) := (x, \, y, \, \sqrt{x^2 + y^2}) \quad \big( (x, \, y) \in \Omega \big),
	\]
	ahol $\Omega := \{ (x, \, y) \in \R^2 : x^2 + y^2 \leq 2 ax \}$, $f$ egy Euler-Monge módon megadott paraméterezése a $z$-tengely irányába felfelé néző körkúp palástjának, \textit{leszűkítve} annak a hengernek a belsejére, aminek alapját az $\Omega$ halmaz adja. Egyből írjuk is át az $f$ függvényt a megfelelő polár-transzformációval
	\[
		\Phi(r, \, \varphi) := (r \cos(\varphi), \, r \sin(\varphi)) \quad \big( \varphi \in [- \pi / 2, \, \pi / 2], \, r \in [0, \, 2a \cos(\varphi)]\big).
	\]
	A $\Phi$ transzformáció-függvény értelmezési tartományát a Thálész-tétel alkalmazásával könnyedén megkaptuk. Tehát ha $\mathcal{F} \subset \R^3$ a feladatban definiált felület, akkor
	\[
		\Psi(r, \, \varphi) := f \circ \Phi = (r \cos(\varphi), \, r \sin(\varphi), \, r) \quad \big( (r, \, \varphi) \in \mathcal{D}_\Phi \big).
	\]
	A felület kiszámításához szükségünk lesz a következőkre:
	\[
		\partial_1 \Psi(r, \, \varphi) = (\cos(\varphi), \, \sin(\varphi), \, 1) \quad \big( (r, \, \varphi) \in \mathcal{D}_\Phi \big),
	\]
	\[
		\partial_1 \Psi(r, \, \varphi) = (-r \sin(\varphi), \, r \cos(\varphi), \, 0) \quad \big( (r, \, \varphi) \in \mathcal{D}_\Phi \big),
	\]
	\[
		n_\Psi(r, \, \varphi) = \begin{vmatrix}
			i & j & k \\
			\cos(\varphi) & \sin(\varphi) & 1 \\
			-r\sin(\varphi) & r\cos(\varphi) & 0
		\end{vmatrix} =
		\begin{pmatrix}
			r\cos(\varphi) \\
			r\sin(\varphi) \\
			r
		\end{pmatrix} \quad \big( (r, \, \varphi) \in \mathcal{D}_\Phi \big),
	\]
	\[
		\|n_\Psi(r, \, \varphi)\| = \sqrt{2r^2} = \sqrt{2} r \quad \big( (r, \, \varphi) \in \mathcal{D}_\Phi \big).
	\]
	Tehát a keresett felszín
	\[
		\mathcal{A}(\mathcal{F}) = \int_{\mathcal{D}_\Phi} \|n_\Psi\| = \sqrt{2} \int\limits_{-\pi/2}^{\pi/2} \int\limits_0^{2a\cos(\varphi)} r \, dr \, d\varphi.
	\]

	\newpage
	\section{Gyakorlat}
	\subsection{Emlékeztető}
	\tikz \node[definition]
	{
		\textbf{Definíció.} Legyen $K \subset \R^2$ kompakt, Jordan-mérhető halmaz,
		\[
			\Psi = (\Psi_1, \, \Psi_2, \, \Psi_3) : K \to \R^3, \, \Psi \in C^1, \, \text{rang}(\Psi') = 2,
		\]
		továbbá tegyük fel, hogy $\Psi_{|_{\text{int}(K)}}$ injektív. Ekkor
		\begin{enumerate}
			\item ha $f : \Psi[K] \to \R, \, f \in C$, akkor az $f$ függvény $\Psi$-re vonatkozó \textit{elsőfajú felületi integráljának} (vagy \textit{felszíni integráljának}) nevezzük az
			\[
				\int_\Psi f := \int_K (f \circ \Psi) \cdot \|n_\Psi\|
			\]
			valós számot,
			\item ha $f : \Psi[K] \to \R^3, \, f \in C$, akkor az $f$ függvény $\Psi$-re vonatkozó (\textit{másodfajú}) \textit{felületi integráljának} nevezzük az
			\[
				\int_\Psi f := \langle  f \circ \Psi, \, n_\Psi \rangle
			\]
			valós számot.
		\end{enumerate}
	};


\end{document}