\documentclass[12pt]{article}
\usepackage[left=0.9in, right=0.9in, top=1in, bottom=1in]{geometry}
\usepackage{tikz}
\usepackage{setspace}
\usepackage{hyperref}
\usepackage{amsfonts, amssymb, amsmath} 
\usepackage{titlesec}
\usepackage{pgfplots}
\pgfplotsset{compat=newest}
\usepackage{graphicx}
\usepackage{wrapfig}
\usepackage{caption}
\usepackage{enumitem}
\usetikzlibrary{shadows.blur}
\usepackage{lmodern}
\setlength{\parskip}{0pt}
\setlength{\parindent}{0pt}

\title{\textcolor{purple}{\Huge\textbf{Analízis IV}}}
\author{2. gyakorlat}
\date{Szabó Krisztián}

\renewcommand{\contentsname}{Tartalom}
\newcommand{\R}{\mathbb{R}}
\newcommand{\N}{\mathbb{N}}
\newcommand{\E}{\exists}
\newcommand{\mm}{\mathbf{m}}
\newcommand{\MM}{\mathbf{M}}
\newcommand{\K}{\mathbb{K}}
\newcommand{\D}{\mathcal{D}_f}

\definecolor{modernyellow}{HTML}{F4E4BC}
\definecolor{moderngreen}{HTML}{BDDCBD}

\tikzset
{
	definition/.style={
		draw,
		fill=modernyellow,
		line width=1pt,
		rounded corners,
		drop shadow={shadow blur steps=5,shadow xshift=1ex,shadow yshift=-1ex},
		text width=0.9\textwidth,
		inner sep=10pt
	},
	theorem/.style={
		draw,
		fill=moderngreen,
		line width=1pt,
		rounded corners,
		drop shadow={shadow blur steps=5,shadow xshift=1ex,shadow yshift=-1ex},
		text width=0.9\textwidth,
		inner sep=10pt
	},
	proof/.style={
		fill=white,
		rectangle,
		drop shadow={shadow blur steps=5,shadow xshift=1ex,shadow yshift=-1ex, moderngreen},
		text width=0.9\textwidth,
		inner sep=6pt,
	},
	proof1/.style={
		fill=white,
		rectangle,
		drop shadow={shadow blur steps=5,shadow xshift=1ex,shadow yshift=0, moderngreen},
		text width=0.9\textwidth,
		inner sep=6pt,
	}
}

\begin{document}
    \maketitle
    \tableofcontents
    \newpage
    \section{Emlékeztető}
    
    \subsection{Feltételes szélsőérték}
    \tikz \node[definition]
    {
        \textbf{Definíció.} Legyen $1 \leq n, \, m \in \N, \, \emptyset \neq U \subset \R^n,$ és
        \[
            f : U \to \R, \, g = (g_1, \, \dots, \, g_m) : U \to \R^m.
        \]
        Azt mondjuk, hogy az $f$ függvénynek a $g = 0$ \textit{feltételre vonatkozóan feltételes lokális maximuma (minimuma) van} a
        \[
            c \in \{ g = 0 \} := \{ \xi \in U : g(\xi) = 0\}
        \]
        pontban, ha az
        \[
            \cap{f}(\xi) := f(\xi) \quad (\xi \in \{ g = 0 \})
        \]
        függvénynek a $c$-ben lokális maximuma (minimuma) van. Feltesszük, hogy
        \[
            \{ g = 0\} \neq \emptyset.
        \]
        Használjuk az $f(c)$-re a \textit{feltételes lokális maximum (minimum)}, ill. \textit{szélsőérték}, továbbá $c$-re a \textit{feltételes lokális maximumhely (minimumhely)}, ill. \textit{szélsőértékhely} elnevezést is.
    };

    \subsection{Elsőrendű szükséges feltétel}
    \tikz \node[theorem]
    {
        \textbf{Tétel.} Tegyük fel, hogy $1 \leq n, \, m \in \N, \, m < n, \, \emptyset \neq U \subset \R^n$ nyílt halmaz, és $f : U \to \R, \, g : U \to \R^m$. Ha $f \in D$, $g \in C^1$, az $f$-nek a $c \in \{ g = 0 \}$ helyen feltételes lokális szélsőértéke van a $g = 0$ feltételre vonatkozóan, továbbá a $g'(c)$ Jacobi-mátrix rangja megegyezik $m$-mel, akkor létezik olyan $\lambda \in \R^m$ vektor, hogy
        \[
            \text{grad } (f + \lambda g)(c) = 0.
        \]
    };\newline

    A tételben szereplő $\lambda g$ függvényen a következőt értjük:
    \[
        (\lambda g)(\xi) := \langle \lambda, \, g(\xi) \rangle \quad (\xi \in U).
    \]
    Más szóval a $\lambda = (\lambda_1, \, \dots, \, \lambda_m), \, g = (g_1, \, \dots, \, g_m)$ koordinátázással
    \[
        (\lambda g)(\xi) = \sum_{i=1}^m \lambda_i g_i(\xi) \quad (\xi \in U).
    \]

    Ez tehát ugyanolyan jellegű, mint a feltétel nélküli esetben, csak a szóban forgó $f$ függvény helyett (egy alkalmas $\lambda \in \R^m$ vektorral) az $F := f + \lambda g$ függvényre vonatkozóan.\newline

    Ez az analógia megmarad a másodrendű feltételeket illetően is. Ezek megfogalmazásához vezessük be a következő definíciót. Legyen adott a
    \[
        Q : \R^n \to \R
    \]
    kvadratikus alak, a $B \in \R^{m \times n}$ mátrix, és tekintsük az alábbi halmazt:
    \[
        \mathcal{A}_B := \{ x \in \R^n : B \cdot x = 0 \}.
    \]
    Feltesszük, hogy $m < n$, és a $B$ mátrix rangja $m$. Ekkor azt mondjuk, hogy a $Q$ kvadratikus alak a $B$-re nézve
    \begin{enumerate}
        \item \textit{feltételesen pozitív definit}, ha $Q(x) > 0 \quad (0 \neq x \in \mathcal{A}_B)$;
        \item \textit{feltételesen negatív definit}, ha $Q(x) < 0 \quad (0 \neq x \in \mathcal{A}_B)$;
        \item \textit{feltételesen pozitív szemidefinit}, ha $Q(x) \geq 0 \quad (x \in \mathcal{A}_B)$;
        \item \textit{feltételesen negatív szemidefinit}, ha $Q(x) \leq 0 \quad (x \in \mathcal{A}_B)$;
    \end{enumerate}

    \subsection{Másodrendű elégséges feltétel}
    \tikz \node[theorem]
    {
        \textbf{Tétel.} Az $1 \leq n, \, m \in \N, \, m < n$ paraméterek mellett legyen adott az $\emptyset \neq U \subset \R^n$ nyílt halmaz, és tekintsük az $f : U \to \R, \, g : U \to \R^m$ függvényeket. Feltesszük, hogy $f, \, g \in D^2$, $c \in \{ g=0\}$, a $g'(c)$ mátrix rangja $m$, továbbá valamilyen $\lambda \in \R^m$ vektorral az $F := f + \lambda g$ függvényre
        \begin{enumerate}
            \item grad $F(c) = 0$;
            \item a $Q_c^F$ kvadratikus alak a $g'(c)$ mátrixra nézve feltételesen pozitív (negatív) definit.
        \end{enumerate}
        Ekkor az $f$-nek a $c$-ben a $g=0$ feltételre vonatkozóan feltételes lokális minimuma (maximuma) van.
    };

\end{document}