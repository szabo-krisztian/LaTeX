\newpage
\section{Vizsgakérdés}
\begin{quote}
	\textit{Állandó együtthatós magasabb rendű homogén lineáris differenciálegyenlet egy alaprendszerének az előállítása, a karakterisztikus polinom szerepe (a bizonyítás vázlata).}
\end{quote}

Az előzőekben vizsgált
\[
	\p^{(n)} + \sum_{k=0}^{n-1} a_k(x) \cdot \p^{(k)}(x) = c(x) \quad (x \in \Dp)
\]
$n$-edrendű lineáris differenciálegyenlet megoldásához tehát elegendő az $\mathcal{M}_h$ egy bázisát meghatározni. Ez általában "reménytelen" feladat, általános módszer nem is adható.\\

Ezért csak abban az esetben tesszük ezt meg, ha az $a_k \quad (k = 0, \, \dots, \, n-1)$ együtthatófüggvények mindegyike konstansfüggvény. Ez az ún. \textit{állandó együtthatós} eset:
\[
	\p^{(n)}(x) = \sum_{k=1}^{n-1} a_k \cdot \p^{(k)}(x) = c(x) \quad (x \in \Dp),
\]
A $c=0$ (\textit{homogén egyenlet}) választással
\[
	\p^{(n)} + \sum_{k=0}^{n-1} a_k \cdot \p^{(k)} \equiv 0.
\]
Tekintsük ehhez a
\[
	P(x) := x^n + \sum_{i=0}^{n-1} a_i x^i \quad (x \in \K)
\]
$n$-edfokú polinomot, a differenciálegyenlet \textit{karakterisztikus polinomját}.\\

Ha a $\lambda \in \K$ szám a $P$-nek gyöke, akkor az
\[
	e_\lambda(x) := e^{\lambda x} \quad (x \in \R)
\]
és az $a_n := 1$ jelöléssel
\[
	\sum_{k=0}^n a_k e_\lambda^{(k)}(x) = \sum_{k=0}^n a_k \lambda^k e^{\lambda x} = e^{\lambda x} \cdot P(\lambda) = 0 \quad (x \in \R)
\]
miatt az $e_\lambda$ függvény megoldása a szóban forgó homogén differenciálegyenletnek.\\

Legyen az előbbi $\lambda$ gyök multiplicitása $\nu \geq 2$. Belátjuk, hogy az
\[
	e_{\lambda, \, j}(x) := x^j e^{\lambda x} \quad (x \in \R, \, j = 0, \, \dots, \, \nu - 1)
\]
függvények is megoldásai a homogén differenciálegyenletnek:
\[
	\sum_{k=0}^n a_k e_{\lambda, \, j}^{(k)} \equiv 0.
\]
Tegyük fel ui., hogy ezt valamilyen $j \in \N$ mellett minden olyan esetben már tudjuk, amikor (az aktuális differenciálegyenletre) $\nu - 1 \geq j$. Például a $j=0$ ilyen, hiszen ezt az $e_{\lambda, \, 0} \equiv e_\lambda$ függvényre az előbb láttuk. Vegyük észre, hogy
\[
	e_{\lambda, \, j+1}(x) = x e_{\lambda, \, j}(x) \quad (x \in \R),
\]
ezért (ami teljes indukcióval rögtön adódik)
\[
	e_{\lambda, \, j+1}^{(k)}(x) = x e_{\lambda, \, j}^{(k)}(x) + k e_{\lambda, \, j}^{(k-1)}(x) \quad (x \in \R, \, 1 \leq k \in \N).
\]
Így -- feltételezve most azt, hogy $\nu -1 \geq j + 1$ -- az alábbiakat kapjuk:
\[
	\sum_{k=0}^n a_k e_{\lambda, \, j+1}^{(k)}(x)  = x \cdot \sum_{k=0}^n a_k e_{\lambda, \, j}^{(k)}(x) + \sum_{k=1}^n k a_k e_{\lambda, \, j}^{(k-1)}(x) =
\]
\[
	\sum_{k=0}^{n-1} (k+1) a_{k+1} e_{\lambda, \, j}^{(k)}(x) \quad (x \in \R).
\]
A
\[
	\sum_{k=0}^{n-1} (k+1)a_{k+1} \p^{(k)} \equiv 0
\]
homogén lineáris differenciálegyenletnek a karakterisztikus polinomja:
\[
	\sum_{k=0}^{n-1} (k+1)a_{k+1} t^k = P'(t) \quad (t \in \R),
\]
ahol a $\lambda$ a $P'$ (derivált)polinomnak $\mu := \nu - 1$-szeres gyöke. Mivel $\mu - 1 \geq j$, ezért az indukciós feltevés szerint
\[
	\sum_{k=0}^{n-1}(k+1)a_{k+1}e_{\lambda, \, j}^{(k)} \equiv 0,
\]
következésképpen (a fentiekre tekintettel)
\[
	\sum_{k=0}^{n} a_k e_{\lambda, \, j+1}^{(k)} \equiv 0.
\]
Tehát az $e_{\lambda, \, j+1}$ függvény is megoldása a homogén egyenletnek.\\

\subsection{Alaprendszer}

Tegyük fel, hogy a $P$ gyöktényezős előállítása a következő:
\[
	P(x) = \prod_{l = 1}^k (x - \lambda_l)^{\nu_l} \quad (x \in \K),
\]
ahol $1 \leq k \in \N$ és $\lambda_1, \, \dots, \, \lambda_k \in \K$ jelöli a $P$ összes, páronként különböző gyökét, $1 \leq \nu_l \in \N$ pedig a $\lambda_l$ gyök multiplicitását $(l = 1, \, \dots, \, k)$. Ekkor tehát a
\[
	\p_{lj}(x) := x^j \cdot e^{\lambda_l x} \quad (x \in I, \, l = 1, \, \dots, \, k \text{ és } j = 0, \, \dots, \, \nu_l - 1)
\]
függvények valamennyien $\mathcal{M}_h$-beliek.\\

Ennél még több is igaz, nevezetesen:\\

\tikz \node[theorem]
{
	\textbf{Tétel.} A fentiekben definiált
	\[
		\p_{lj} \quad (l = 1, \, \dots, \, k \text{ és } j = 0, \, \dots, \, \nu_l - 1)
	\]
	függvények a szóban forgó állandó együtthatós $n$-edrendű lineáris differenciálegyenlet egy alaprendszerét alkotják.
};

\subsection{Valós értékű megoldások}

Az előző tétel alapján kapott
\[
	\p_{lj}(x) := x^j e^{\lambda_l x} \quad (x \in I, \, l = 1, \, \dots, \, k; \, j = 0, \, \dots, \, \nu_l - 1)
\]
alaprendszerben a $\p_{lj} \quad (l = 1, \, \dots, \, k; \, j = 0, \, \dots, \, \nu_l)$ függvények valós értékűek, ha a szóban forgó $n$-edrendű lineáris differenciálegyenlet $P$ karakterisztikus polinomjában a $\lambda_l$ gyök valós szám. Ha viszont valamilyen $l = 1, \, \dots, \, k$ esetén a $\lambda_l$ gyök nem valós komplex szám, akkor a következőket mondhatjuk. Legyen ekkor
\[
	\lambda_l = u_l + \imath v_l,
\]
ahol $u_l, \, v_l \in \R$ és $v_l \neq 0$. Mivel a $P$ polinom valós együtthatós, ezért a
\[
	\overline{\lambda_l} = u_l - \imath v_l
\]
komplex konjugált is $v_l$-szeres gyöke a $P$-nek. Ez azt jelenti, hogy a fenti alaprendszerben a
\[
	\hat{\p}_{lj}(x) := x^j \cdot e^{\overline{\lambda_l} x} = \overline{\p_{lj}(x)} \quad (x \in I, \, j = 0, \, \dots, \, \nu_l - 1)
\]
függvények is szerepelnek. Tudjuk, hogy az $\mathcal{M}_h$ halmaz vektortér a $\K$-ra nézve, ezért
\[
	\phi_{lj} := \frac{\p_{lj} + \hat{\p}_{lj}}{2} = \text{Re} \, \p_{lj} \in \mathcal{M}_h \text{ és } \hat{\phi}_{lj} := \frac{\p_{lj} - \hat{\p}_{lj}}{2i} = \text{Im} \, \p_{lj} \in \mathcal{M}_h.
\]
Itt tetszőleges $j = 0, \, \dots, \, \nu_l - 1$ mellett
\[
	\phi_{lj}(x) = \text{Re} \, \p_{lj}(x) = \text{Re} \, (x^j \cdot e^{\lambda_l x}) = \text{Re} \, (x^j \cdot e^{u_lx + \imath v_l x}) =
\]
\[
	\text{Re} \, \B{ x^j \cdot e^{u_l x} (\cos(v_lx) + \imath \sin(v_l x))} = x^j \cdot e^{u_l x} \cdot \cos(v_l x) \quad (x \in I),
\]
és (analóg számolás után)
\[
	\hat{\phi}_{lj}(x) = x^j \cdot e^{u_l x} \cdot \sin(v_l x) \quad (x \in I).
\]
Könnyen belátható, hogy ha a fenti $\p_{lj}, \, \hat{\p}_{lj}$ (összesen $2\nu_l$ darab) függvényt kicseréljük a $\phi_{lj}, \, \hat{\phi}_{lj}$ (ugyancsak $2\nu_l$ darab) függvényre, akkor továbbra is lineárisan független függvényrendszert kapunk. Ha ezt a cserét a $P$ polinom minden nem valós gyökével kapcsolatban megtesszük, akkor az $\mathcal{M}_h$ egy valós értékű függvényekből álló bázisát kapjuk, azaz egy valós függvényekből álló alaprendszert. 