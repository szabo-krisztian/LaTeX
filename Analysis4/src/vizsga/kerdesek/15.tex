\newpage
\section{Vizsgakérdés}
\begin{quote}
	\textit{Egyenletesen konvergens Fourier-sorok, a trigonometrikus rendszer teljessége $C_{2\pi}$-re.}
\end{quote}

\subsection{Teljesség}

Emlékeztető: ha a
\[
	\trigseries
\]
trigonometrikus sor egyenletesen konvergens és
\[
	F(x) := \alpha_0 + \sum_{n=1}^\infty \B{ \alpha_n \cdot \cos(nx) + \beta_n \cdot \sin(nx) } \quad (x \in \R)
\]
az összegfüggvénye, akkor $F \in C_{2\pi}$, a szóban forgó trigonometrikus sor pedig az $F$ függvény Fourier-sora.\\

Legyen most $f \in C_{2\pi}$, és tegyük fel, hogy az $Sf$ Fourier-sora egyenletesen konvergens. Ha az $Sf$ trigonometrikus sor összegfüggvényét $g$-vel jelöljük, akkor joggal vetődik fel a kérdés: mi köze van egymáshoz az $f$ és a $g$ függvénynek? Az előbbiek szerint $Sf = Sg$, azaz
\[
	a_0(f) = a_0(g), \, a_n(f) = a_n(g), \, b_n(f) = b_n(g) \quad (1 \leq n \in \N).
\]
Ha
\[
	h := f - g,
\]
akkor egyrészt $h \in C_{2\pi}$, másrészt a Fourier-együtthatók definíciója alapján rögtön adódik az, hogy
\[
	a_0(h) = a_0(f) - a_0(g) = 0
\]
és
\[
	a_n(h) = a_n(f) - a_n(g) = 0, \, b_n(h) = b_n(f) - b_n(g) = 0 \quad (1 \leq n \in \N).
\]
Meg fogjuk mutatni, hogy mindebből $h \equiv 0$ következik, azaz $f = g$.\\

Ehhez szükségünk lesz az alábbi, önmagában is fontos állításra.\\

\tikz \node[theorem]
{
	\textbf{Tétel.} Ha a $h \in C_{2\pi}$ függvény valamennyi Fourier-együtthatója nulla, akkor a $h$ az azonosan nulla függvény. 
};\\

\textbf{Bizonyítás.} Tegyük fel indirekt módon, hogy valamilyen $a \in \R$ esetén $h(a) \neq 0$. Nyugodtan feltehetjük a továbbiakban, hogy
\[
	a = 0 \text{ és } h(0) > 1.
\]
Figyelembe véve, hogy a $h$ függvény folytonos, van olyan $0 < \delta < \pi / 2$, amellyel
\[
	h(x) > 1 \quad (|x| \leq \delta).
\]
Vegyük észre, hogy bármilyen
\[
	T(x) := \alpha_0 + \sum_{k=1}^n \B{ \alpha_k \cdot \cos(kx) + \beta_k \cdot \sin(kx) } \quad (x \in \R, \, n \in \N)
\]
trigonometrikus polinomra
\[
	\int\limits_{-\pi}^\pi h(x) \cdot T(x) \, dx =
\]
\[
	\alpha_0 \cdot \int\limits_{-\pi}^\pi h(x) \, dx + \sum_{k=1}^n \left(  \alpha_k \cdot \int\limits_{-\pi}^\pi h(x) \cdot \cos(kx) \, dx + \beta_k \cdot \int\limits_{-\pi}^\pi h(x) \cdot \sin(kx) \, dx \right) =
\]
\begin{equation}
	2\pi \cdot \alpha_0 \cdot a_0(h) + \pi \cdot \sum_{k=1}^n \B{ \alpha_k \cdot a_k(h) + \beta_k \cdot b_k(h) } = 0.
	\tag{$\star$}
\end{equation}

Meg fogjuk mutatni, hogy ennek ellenére egy alkalmas $T$ trigonometrikus polinommal
\[
	\int\limits_{-\pi}^\pi h(x) \cdot T(x) \, dx \neq 0.
\]
Legyen ehhez
\[
	T_n(x) := (\cos x + 1 - \cos \delta)^n \quad (x \in \R, \, n \in \N).
\]
Teljes indukcióval könnyen ellenőrizhető, hogy minden $n \in \N$ esetén a $T_n$ függvény egy trigonometrikus polinom. Továbbá
\[
	\cos x + 1 - \cos \delta \geq 1 \quad (|x| \leq \delta),
\]
valamint bármilyen $\delta < \tau < \pi / 2$ választással
\[
	q_\tau := \max \{ |\cos x + 1 - \cos \delta| : x \in [-\pi, \, -\tau] \cup [\tau, \, \pi]\} < 1
\]
és
\[
	|\cos x + 1 - \cos \delta| \leq 1 \quad (x \in [-\tau, \, -\delta] \cup [\delta, \, \tau]).
\]

\begin{center}
	\begin{tikzpicture}
		\begin{axis}[
			axis lines = middle,
			xlabel = \(x\),
			ylabel = {\(\cos x\)},
			xmin = -1.25*pi, xmax = 1.25*pi,
			ymin = -1.5, ymax = 1.5,
			samples = 100,
			domain = -2*pi:2*pi,
			xtick = {-pi, -pi/2,-1, -0.6, 0.6, 1, pi/2, pi},
			xticklabels={$-\pi$, $-\frac{\pi}{2}$,-$\tau$, -$\delta$, $\delta$,$\tau$, $\frac{\pi}{2}$, $\pi$},
			x = 1.5cm
			]
			\addplot[color=black, thick] {cos(deg(x))};
			   
			\addplot[thin, dotted, color=black] coordinates {(-1, -1.5) (-1, 1.5)};
			\addplot[thin, dotted, color=black] coordinates {(1, -1.5) (1, 1.5)};
			\addplot[thin, dotted, color=black] coordinates {(-0.6, -1.5) (-0.6, 1.5)};
			\addplot[thin, dotted, color=black] coordinates {(0.6, -1.5) (0.6, 1.5)};
		\end{axis}
	\end{tikzpicture}	
\end{center}
Ekkor a
\[
	J_1 := [-\pi, \, -\tau], \, J_2 := [-\tau, \, -\delta], \, J_3 := [-\delta, \, \delta], \, J_4 := [\delta, \, \tau], \, J_5 := [\tau, \, \pi]
\]
intervallumokkal minden $n \in \N$ esetén
\[
	\left| \int\limits_{-\pi}^\pi h(x) \cdot T_n(x) \, dx \right| = \left| \sum_{i=1}^5 \int_{J_i} h(x) \cdot T_n(x) \, dx\right| \geq
\]
\[
	\left| \int\limits_{-\delta}^\delta h(x) \cdot T_n(x) \, dx \right| - \sum_{3 \neq i = 1}^5 \int_{J_i} |h(x) \cdot T_n(x)| \, dx.
\]
Az előzményekre tekintettel itt a
\[
	C := \max \big\{ |h(x)| : |x| \leq \pi \big\} \quad (< + \infty)
\]
jelöléssel
\[
	\left| \int\limits_{-\delta}^\delta h(x) \cdot T_n(x) \, dx \right| = \int\limits_{-\delta}^\delta h(x) \cdot T_n(x) \, dx \geq \int\limits_{-\delta}^\delta 1 \, dx = 2\delta,
\]
valamint
\[
	\int_{J_1} |h(x) \cdot T_n(x)| \, dx, \, \int_{J_5} |h(x) \cdot T_n(x)| \, dx \leq C \cdot q_\tau^n \cdot \tau
\]
és
\[
	\int_{J_2} |h(x) \cdot T_n(x)| \, dx,
	\int_{J_4} |h(x) \cdot T_n(x)| \, dx \leq C \cdot (\tau - \delta).
\]
Következésképpen
\[
	\left| \int\limits_{-\pi}^\pi h(x) \cdot T_n(x) \, dx \right| \geq 2\delta - \B{ 2 C \cdot q_\tau^n \cdot \pi + 2C \cdot (\tau - \delta) }.
\]
A $\tau$ (az eddigieken túl) nyilván megválasztható úgy is, hogy
\[
	2C \cdot (\tau - \delta) < \delta / 2
\]
teljesüljön. Ekkor
\[
	\left| \int\limits_{-\pi}^\pi h(x) \cdot T_n(x) \, dx \right| > 3\pi / 2 - 2C \cdot q_\tau^n \cdot \pi.
\]
Mivel $0 < q_\tau < 1$, ezért
\[
	q_\tau^n \to 0 \quad (n \to \infty).
\]
Így létezik olyan $N \in \N$, amellyel
\[
	2C \cdot q_\tau^N \cdot \pi < \delta/2
\]
amikor is
\[
	\left| \int\limits_{-\pi}^\pi h(x) \cdot T_N(x) \, dx \right| > \delta.
\]
Tehát
\[
	\int\limits_{-\pi}^\pi h(x) \cdot T_N(x) \, \neq 0,
\]
azaz a $T := T_N$ választás ellentmond a $(\star)$ egyenlőségnek. $\hfill \blacksquare$\\

Az előzményekre is tekintettel igaz az alábbi tétel:\\

\tikz \node[theorem]
{
	\textbf{Tétel.} Ha az $f \in C_{2\pi}$ függvény Fourier-sora egyenletesen konvergens, akkor az $Sf$ Fourier-sor összegfüggvénye megegyezik az $f$-fel, azaz
	\[
		f(x) = a_0 + \sum_{n=1}^\infty \BB{ a_n \cdot \cos(nx) + b_n \cdot \sin(nx) } \quad (x \in \R).
	\]
};
