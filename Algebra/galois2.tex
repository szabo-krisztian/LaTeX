\documentclass[12pt]{article}
\usepackage[left=1in, right=1in, top=1in, bottom=1in]{geometry}
\usepackage{tikz}
\usepackage{setspace}
\usepackage{hyperref}
\usepackage{amsfonts, amssymb, amsmath} 
\usepackage[T1]{fontenc}
\usetikzlibrary{shadows.blur}
\usepackage{titlesec}

\author{Krisztián Szabó}
\date{Source: Contemporary Abstract Algebra by Joseph A. Gallian}
\renewcommand{\contentsname}{Content}

\titleformat{\section}[block]{\color{purple}\normalfont\Large\bfseries}{\thesection}{1em}{}
\title{\textcolor{purple}{\Huge\textbf{\textsc{Abstract algebra}}}}

\newcommand{\cent}[1]{\langle #1 \rangle}
\newcommand{\bcent}[1]{\big\langle #1 \big\rangle}
\newcommand{\bp}[1]{\big( #1 \big)}

\setlength{\parindent}{0pt}

\definecolor{modernyellow}{HTML}{F4E4BC}
\definecolor{moderngreen}{HTML}{BDDCBD}
\definecolor{colo}{HTML}{FFFC9B}

\setcounter{section}{5}


\tikzset{
	definition/.style={
		draw,
		fill=modernyellow, % Sárga színű keret a definíciókhoz
		line width=1pt,
		rounded corners, % Lekerekített szélek
		drop shadow={shadow blur steps=5,shadow xshift=1ex,shadow yshift=-1ex}, % Árnyék hozzáadása
		text width=0.9\textwidth, % Szöveg szélessége
		inner sep=10pt,
		align=justify % Sorkizárt szöveg
	},
	corollary/.style={
		draw,
		fill=colo, % Zöld színű keret a tételekhez
		line width=1pt,
		rounded corners, % Lekerekített szélek
		drop shadow={shadow blur steps=5,shadow xshift=1ex,shadow yshift=-1ex}, % Árnyék hozzáadása
		text width=0.9\textwidth,
		inner sep=10pt,
		align=justify % Sorkizárt szöveg
	},
	theorem/.style={
		draw,
		fill=moderngreen, % Zöld színű keret a tételekhez
		line width=1pt,
		rounded corners, % Lekerekített szélek
		drop shadow={shadow blur steps=5,shadow xshift=1ex,shadow yshift=-1ex}, % Árnyék hozzáadása
		text width=0.9\textwidth,
		inner sep=10pt,
		align=justify % Sorkizárt szöveg
	},
	proof/.style={
		fill=white,
		rectangle,
		drop shadow={shadow blur steps=5,shadow xshift=1ex,shadow yshift=-1ex, moderngreen},
		%rounded corners=3pt,
		text width=0.9\textwidth,
		inner sep=6pt,
		align=justify % Sorkizárt szöveg
	},
	proof1/.style={
		fill=white,
		rectangle,
		drop shadow={shadow blur steps=5,shadow xshift=1ex,shadow yshift=0, moderngreen},
		text width=0.9\textwidth,
		inner sep=6pt,
		align=justify % Sorkizárt szöveg
	},
	ex/.style={
		fill=white,
		rectangle,
		drop shadow={shadow blur steps=5,shadow xshift=1ex,shadow yshift=0, black},
		text width=0.9\textwidth,
		inner sep=6pt,
		align=justify % Sorkizárt szöveg
	}
}

\begin{document}
	\maketitle
	\tableofcontents
	\newpage
	
	\section{Isomorphisms}
	
	\subsection{Definition and Examples}
	In this chapter, we give a formal method for determining whether two groups defined in different terms are the same. When this is the case, we say that there is an isomorphism between the two groups. This notion was first introduced by Galois about 175 years ago. The term \textit{isomorphism} is derived from the Greek words \textit{isos}, meaning "same" or "equal", and \textit{morphe}, meaning "form".
	
	\subsubsection{Group isomorphism}
	\begin{center}
		\tikz \node[definition]
		{
			\textbf{Definition.} An \textit{isomorphism} $\phi$ from a group $G$ to a group $\overline{G}$ is a one-to-one mapping (or function) from $G$ onto $\overline{G}$ that preserves the group operation. That is,
			\[
			\phi(ab) = \phi(a)\phi(b) \quad \text{ for all } a, \, b \in G.
			\]
			If there is an isomorphism from $G$ onto $\overline{G}$, we say that $G$ and $\overline{G}$ are \textit{isomorphic} and write $G \approx \overline{G}$.
		};
	\end{center}
	
	
	It is implicit in the definition of isomorphism that the operation on the left side of the equal sign is that of $G$, whereas the operation on the right side is that of $\overline{G}$. The four cases involving $\cdot$ and $+$ are shown in the table below.\newline
	
	\begin{center}
		\begin{tabular}{|c|c|c|}
			\hline 
			$G$ operation & $\overline{G}$ operation & Operation preservation \\
			\hline
			$\cdot$ & $\cdot$ & $\phi(a \cdot b) = \phi(a) \cdot \phi(b)$ \\
			$\cdot$ & $+$ & $\phi(a \cdot b) = \phi(a) + \phi(b)$ \\
			$+$ & $\cdot$ & $\phi(a + b) = \phi(a) \cdot \phi(b)$ \\
			$+$ & $+$ & $\phi(a + b) = \phi(a) + \phi(b)$ \\
			\hline
		\end{tabular}	
	\end{center}
	
	There are four separate steps involved in proving that a group $G$ is isomorphic to a group $\overline{G}$.\newline
	
	\textbf{Step 1.} "Mapping". Define a candidate for the isomorphism; that is, define a function $\phi$ from $G$ to $\overline{G}$.\newline
	\textbf{Step 2.} "1-1". Prove that $\phi$ is one-to-one; that is, assume that $\phi(a) = \phi(b)$ and prove that $a = b$.\newline
	\textbf{Step 3.} "Onto". Prove that $\phi$ is onto; that is, for any element $\overline{g} \in \overline{G}$, find an element $g \in G$ such that $\phi(g) = \overline{g}$.
	\textbf{Step 4.} "O.P." Prove that $\phi$ is operation-preserving; that is, show that $\phi(ab) = \phi(a)\phi(b)$ for all $a$ and $b$ in $G$.\newline
	
	None of these steps are unfamiliar to you. The only one that may appear novel is the fourth one. It requires that one be able to obtain the same result by combining two elements and then mapping, or by mapping two elements and then combining them. Roughly speaking, this says that the two processes-operating and mapping-can be done in either order without affecting the result. This same concept arises in calculus when we say
	\[
		\lim_{x \to a} \big( f(x) \cdot g(x) \big) = \lim_{x \to a} f(x) \lim_{x \to a} g(x)
	\]
	or
	\[
		\int\limits_a^b \big( f(x) + g(x) \big) \, dx = \int\limits_a^b f(x) \, dx + \int\limits_a^b g(x) \, dx.
	\]
	Let us consider some examples.\newline
	
	%%% EXAMPLES
	\begin{center}
		\tikz \node[ex]
		{
			\textbf{Example.} Let $G = \mathbb{R}$ under addition and let $\overline{G} = \mathbb{R}^+$ under multiplication. Then $G$ and $\overline{G}$ are isomorphic under the mapping $\phi(x) = 2^x$. Certinly, $\phi$ is a function from $G$ to $\overline{G}$. To prove that it is one-to-one, suppose that $2^x = 2^y$. Then $\log_2 2^x = \log_2 2^y$, and therefore $x = y$. For "onto", we must find for any positive real number $y$ some real number $x$ such that $\phi(x) = y$; that is $2^x = y$. Well, solving for $x$ gives $\log_2 y$. Finally,
			\[
			\phi(x + y) = 2^{x + y} = 2^x \cdot 2^y = \phi(x)\phi(y).
			\]	
		};\newline
		
		\tikz \node[ex]
		{
			\textbf{Example.} Any infinite cyclic group is isomorphic to $Z$. Indeed, if $a$ is a generator of the cyclic group, the mapping $a^k \to k$ is an isomorphism. Any finite cyclic group $\cent{a}$ of order $n$ is isomorphic to $Z_n$ under the mapping $a^k \to k$ mod $n$. That these correspodences are functions and are one-to-one is the essence of Theorem 4.1.
		};\newline
		
		\tikz \node[ex]
		{
			\textbf{Example.} The mapping from $\mathbb{R}$ under addition to itself given by $\phi(x) = x^3$ is not an isomorphism. Although $\phi$ is one-to-one and onto, it is not operation-preserving, since it is not true that $(x+y)^3 = x^3 + y^3$ for all $x$ and $y$ in $\mathbb{R}$. 
		};\newline
		
		\tikz \node[ex]
		{
			\textbf{Example.} $U(10) \approx Z_4$ and $U(5) \approx Z_4$. To verify this, one need only observe that both $U(10)$ and $U(5)$ are cyclic of order $4$. 
		};\newline
	\end{center}
	
	\subsubsection{Cayley's theorem (1854)}
	\begin{center}
		\tikz \node[theorem]
		{
			\textbf{Theorem.} Every group is isomorphic to a group of permutations.
		};\newline
		
		\tikz \node[proof1]
		{
			\textbf{Proof.} To prove this, let $G$ be a group. We must find a group $\overline{G}$ of permutations that we believe is isomorphic to $G$. Since $G$ is all we have to work with, we will have to use it to construct $\overline{G}$. 
		};\newline
		
		\tikz \node[proof]
		{
			For any $g$ in $G$, define a function $T_g$ from $G$ to $G$ by
			\[
			T_g(x) = gx \quad \forall \, x \in G.
			\]
			(In words, $T_g$ is just a multiplication by $g$ on the left.) We prove that $T_g$ is a permutation on the set of elements $G$. $T_g$ is one-to-one, because
			\[
			\forall \, x, \, y \in G : T_g(x) = T_g(y) \quad \Longleftrightarrow \quad gx = gy \quad \Longleftrightarrow \quad (g^{-1}g)x = (g^{-1}g)y
			\]
			$T_g$ is onto. Let $y \in G$. We have to find $x \in G$, that satisfies
			\[
			T_g(x) = y \quad \Longleftrightarrow \quad gx = y
			\] 
			$x := g^{-1}y$ shows that $T_g$ is onto, so we proved that $T_g$ is a permutation.\newline
			
			Now let $\overline{G} := \{ T_g \mid g \in G \}$. Then, $\overline{G}$ is a group under the operation of function composition. To verify this, we first observe that for any $g$ and $h$ in $G$ we have $T_gT_h(x) = T_g(T_h(x)) = T_g(hx) = g(hx) = (gh)x = T_{gh}(x)$, so that $T_gT_h(x) = T_{gh}(x)$. From this it follows that $T_e$ is the identity and $(T_g)^{-1} = T_{g^{-1}}$. Since function composition is associative, we have verified all the conditions for $\overline{G}$ to be a group. 
			The isomorphism $\phi$ between $G$ and $\overline{G}$ is now ready-made. For every $g$ in $G$, define $\phi(g) = T_g$. If $T_g = T_h$, then $T_g(e) = T_h(e)$ or $ge = he$. Thus, $g = h$ and $\phi$ is one-to-one. By the way $\overline{G}$ was constructed, we see that $\phi$ is onto. The only condition that remains to be checked is that $\phi$ is operation-preserving. To this end, let $a$ and $b$ belong to $G$. Then
			\[
			\phi(ab) = T_{ab} = T_aT_b = \phi(a)\phi(b).
			\]
		};\newline	
	\end{center}
	
	The group $\overline{G}$ constructed above is called the \textit{left regular representation of $G$}.
	
	%%% EXAMPLES
	\begin{center}
			\tikz \node[ex]
		{
			\textbf{Example.} For concreteness, let us calculate the left regular representation $\overline{U(12)}$ for $U(12) = \{ 1, \, 5, \, 7, \, 11 \}$.
			\[
			\phi : U(12) \to \overline{U(12)}, \quad \phi(x) = T_x : U(12) \to U(12)
			\]
			\[
			T_1 =
			\begin{bmatrix}
				1 & 5 & 7 & 11 \\
				1 & 5 & 7 & 11
			\end{bmatrix}, \quad
			T_5 =
			\begin{bmatrix}
				1 & 5 & 7 & 11 \\
				5 & 1 & 11 & 7
			\end{bmatrix},
			\]
			\[
			T_7 =
			\begin{bmatrix}
				1 & 5 & 7 & 11 \\
				7 & 11 & 1 & 5
			\end{bmatrix}, \quad
			T_{11} =
			\begin{bmatrix}
				1 & 5 & 7 & 11 \\
				11 & 7 & 5 & 1
			\end{bmatrix}. \quad
			\]
			It is instructive to compare the Cayley table for $U(12)$ and its left regular representation $\overline{U(12)}$.
		};
		
		\tikz \node[ex]
		{
			\begin{center}
				\begin{tabular}{c|c|c|c|c}
					
					$U(12)$ & 1 & 5 & 7 & 11 \\
					\hline
					1 & 1 & 5 & 7 & 11 \\
					\hline
					5 & 5 & 1 & 11 & 7 \\
					\hline
					7 & 7 & 11 & 1 & 5 \\
					\hline
					11 & 11 & 7 & 5 & 1 \\
					
				\end{tabular}\quad\quad\quad
				\begin{tabular}{c|c|c|c|c}
					
					$\overline{U(12)}$ & $T_1$ & $T_5$ & $T_7$ & $T_{11}$ \\
					\hline
					$T_1$ & $T_1$ & $T_5$ & $T_7$ & $T_{11}$ \\
					\hline
					$T_5$ & $T_5$ & $T_1$ & $T_{11}$ & $T_7$ \\
					\hline
					$T_7$ & $T_7$ & $T_{11}$ & $T_1$ & $T_5$ \\
					\hline
					$T_{11}$ & $T_{11}$ & $T_7$ & $T_5$ & $T_1$ \\
					
				\end{tabular}	
			\end{center}
		};		
	\end{center}
	
	\subsection{Properties of isomorphisms}
	
	Our next two theorems give a catalog of properties of isomorphisms and isomorphic groups.
	
	\subsubsection{Properties of isomorphisms acting on elements}
	\begin{center}
		\tikz \node[theorem]
		{
			\textbf{Theorem.} Suppose that $\phi$ is an isomorphism from a group $G$ onto a group $\overline{G}$. Then
			\begin{enumerate}
				\item $\phi$ carries the identity of $G$ to the identity of $\overline{G}$.
				\item For every integer $n$ and for every group element $a$ in $G$, $\phi(a^n) = \big(\phi(a)\big)^n.$
				\item For any elements $a$ and $b$ in $G$, $a$ and $b$ commute if and only if $\phi(a)$ and $\phi(b)$ commute.
				\item $G = \cent{a}$ if and only if $\overline{G} = \cent{\phi(a)}.$
				\item $|a| = |\phi(a)|$ for all $a$ in $G$ (isomorphisms preserve orders).
				\item For a fixed integer $k$ and a fixed group element $b$ in $G$, the equation $x^k = b$ has the same number of solutions in $G$ as does the equation $x^k = \phi(b)$ in $\overline{G}$.
				\item If $G$ is finite, then $G$ and $\overline{G}$ have exactly the same number of elements of every order.
			\end{enumerate}
		};\newline
		
		\tikz \node[proof1]
		{
			\textbf{Proof.} We will restrict ourselves to proving only properties 1, 2, and 4, but observe that property 5 follows from properties 1 and 2, property 6 follows from property 2, and property 7 follows from property 5. For convenience, let us denote the identity in G by $e$ and the identity in $\overline{G}$ by $\overline{e}$. Then, since $e = ee$, we have
			\[
			\phi(e) = \phi(e)\phi(e).
			\]
			Also, because $\phi(e) \in \overline{G}$, we have $\phi(e) = \overline{e}\phi(e)$, as well. Thus, by cancellation, $\overline{e} = \phi(e)$. This proves property 1.\newline
			
			For positive integers, property 2 follows from the definition of an isomorphism and mathematical induction: if $n = 1$, $\phi(a^1) = \big( \phi(a) \big)^1$.
		};\newpage
		
		\tikz \node[proof]
		{
			Suppose that $\phi(a^{n-1}) = \big( \phi(a) \big)^{n-1}$. Then $\phi(a^n) = \phi(a^{n-1}a) = \phi(a^{n-1})\phi(a) = \big( \phi(a) \big)^{n-1}\phi(a) = \big( \phi(a) \big)^n$. So it is true for positive $n$ values. If $n$ is negative, then $-n$ is positive, and we have from property 1 and the observation about the positive integer case that $e = \phi(e) = \phi(g^ng^{-n}) = \phi(g^n)\phi(g^{-n}) = \phi(g^n)\big( \phi(g) \big)^{-n}$. Thus, multiplying both sides on the right by $\big( \phi(g) \big)^n$, we have $\big(\phi(g)\big)^n = \phi(g^n)$. Property 1 takes care of the case $n = 0$.\newline
			
			To prove property 4, let $G = \cent{a}$ and note that, by closure, $\bcent{\phi(a)} \subseteq \overline{G}$. Because $\phi$ is onto, for any element $b \in \overline{G}$, there is an element $a^k \in G$ such that $\phi(a^k) = b$. Thus, $b = \bp{\phi(a)}^k$ so $b \in \bcent{\phi(a)}$. This proves that $\overline{G} = \bcent{\phi(a)}$.\newline
			
			Now suppose that $\overline{G} = \bcent{\phi(a)}$. Clearly, $\cent{a} \subseteq G$. For any element $b \in G$, we have $\phi(b) \in \bcent{\phi(a)}$. So, for some integer $k$ we have $\phi(b) = \bp{\phi(a)}^k = \phi(a^k)$. Because $\phi$ is one-to-one, $b = a^k$. This proves that $\cent{a} = G$.
		};
	\end{center}
	
	When the group operation is addition, property 2 of Theorem $6.2$ is $\phi(na) = n\phi(a)$; property 4 says that an isomorphism between two cyclic groups takes a generator to a generator.\newline
	
	Property $6$ is quite useful for showing that two groups are not isomorphic. Often $b$ is picked to be the identity. For example, consider $\mathbb{C}^*$ and $\mathbb{R}^*$. Because the equation $x^4 = 1$ has four solutions in $\mathbb{C}^*$ but only two in $\mathbb{R}^*$, no matter how one attempts to define an isomophism from $\mathbb{C}^*$ to $\mathbb{R}^*$, property 6 cannot hold.
	
	\subsubsection{Properties of isomorphisms acting on groups}
	\begin{center}
		\tikz \node[theorem]
		{
			\textbf{Theorem.} Suppose that $\phi$ is an isomorphism from a group $G$ onto a group $\overline{G}$. Then
			\begin{enumerate}
				\item $\phi^{-1}$ is an isomorphism from $\overline{G}$ onto $G$.
				\item $G$ is Abelian if and only if $\overline{G}$ is Abelian.
				\item $G$ is cyclic if and only if $\overline{G}$ is cyclic.
				\item If $K$ is a subgroup of $G$, then $\phi(K) = \{ \phi(k) \mid k \in K \}$ is a subgroup of $\overline{G}$.
			\end{enumerate}
		};\newline
		
		\tikz \node[proof]
		{
			\textbf{Theorem.} Properties 1 and 4 are left as exercises. Property 2 is a direct consequence of property 3 of Theorem $6.2$. Property 3 follows from property 4 of Theorem $6.2$ and property 1 of Theorem $6.3$.
		};	
	\end{center}
	
	
	Theorems $6.2$ nad $6.3$ show that isomorphic groups have many properties in common. Actually, the definition is precisely formulated so that isomorphic groups have all group-theoretic properties in common. By this we mean that if two groups are isomorphic, then any property that can be expressed in the language of group theory is true for one if and only if it is true for the other. This is why algebraists speak of isomorphic groups as "equal" or "the same". Admittedly, calling groups equivalent, rather than the same, might be more appropriate, but we bot to long-standing tradition.

	\subsection{Automorphisms}
	
	Certain kinds of isomorphisms are referred to so often that they have been given special names.
	
	\subsubsection{Automorphism}
	\begin{center}
		\tikz \node[definition]
		{
			\textbf{Definition.} An isomorphism from a group $G$ onto itself is called in \textit{automorphism} of $G$.
		};	
	\end{center}
	
	%%% EXAMPLES
	\begin{center}
		\tikz \node[ex]
		{
			\textbf{Example.} Let $\mathbb{R}^2 = \{ (a, \, b) \mid a, \, b \in \mathbb{R} \}$. Then $\phi(a, \, b) = (b, \, a)$ is an automorphism of the group $\mathbb{R}^2$ under componentvise addition. Geometrically, $\phi$ reflects each point in the plane across the line $y = x$. More generally, any reflection across a line passing through the origin or any rotation of the plane about the origin is an automorphism of $\mathbb{R}^2$.
		};	
	\end{center}
	
	\subsubsection{Inner automorphism induced by $a$}
	\begin{center}
		\tikz \node[definition]
		{
			\textbf{Definition.} Let $G$ be a group, and let $a \in G$. The function $\phi_a : G \to G$ defined by $\phi_a(x) = axa^{-1}$ for all $x \in G$ is called the \textit{inner automorphism of $G$ induced by $a$}.
		};
	\end{center}
	
	\subsubsection{$\phi_a$ is an automorphism}
	\begin{center}
		\tikz \node[theorem]
		{
			\textbf{Theorem.} Every inner automorphisms of $G$ are automorphisms.
		};\newline
		
		\tikz \node[proof1]
		{
			\textbf{Proof. (personal)} Let $G$ be a group and $a \in G$ an arbitrary element. The inner automorphism of $G$ induced by $a$:
			\[
				\phi_a : G \to G, \quad \phi_a(x) = axa^{-1} \quad (x \in G).
			\]
			It is certainly a \textbf{function}:
			\[
				x = y \quad \Longleftrightarrow \quad axa^{-1} = aya^{-1} \quad (x, \, y \in G)
			\]
			Let us prove that $\phi_a$ is \textbf{one-to-one}.
		};
		
		\tikz \node[proof]
		{
			For all $x, \, y \in G$:
			\[
				\phi_a(x) = \phi_a(y) \quad \Longleftrightarrow \quad axa^{-1} = aya^{-1}
			\]
			By left- and right-hand side cancellation we get $x = y$. Now proove that $\phi_a$ is \textbf{onto}; that is, for every $y \in G$, there exists $x \in G$, such that
			\[
				\phi_a(x) = y \quad \Longleftrightarrow \quad axa^{-1} = y \quad  (x := a^{-1}ya)
			\]
			The only thing left is to proove that $\phi_a$ is operation-preserving, such that for every $x, \, y \in G:$
			\[
				\phi_a(xy) = axya^{-1} = ax(aa^{-1})ya^{-1} = axa^{-1}aya^{-1} = \phi_a(x)\phi_a(y)
			\]
			
		};
	\end{center}
	
	\subsubsection{Aut($G$) and Inn($G$)}
	\begin{center}
		\tikz \node[definition]
		{
			\textbf{Definition.} Let $G$ be a group. Aut($G$) denotes the set of all automorphisms of $G$ and Inn($G$) denotes the set of all inner automorphisms of $G$, so
			\[
				\text{Aut}(G) := \{ \phi : G \to G \mid \phi \text{ is an automorphism of } G \}
			\]
			\[
				\text{Inn}(G) := \{ \phi_g : G \to G \mid g \in G \}
			\]
		};
	\end{center}
	
	\subsubsection{Aut($G$) and Inn($G$) are groups}
	\begin{center}
		\tikz \node[theorem]
		{
			\textbf{Theorem.} The set of all automorphisms of a group and the set of inner automorphisms of a group are both groups under the operation of function composition.
		};\newline
		
		\tikz \node[proof1]
		{
			\textbf{Proof. (personal)} Let $G$ be a group. We prove that Aut($G$) is also a group under function composition. We see that closure holds. Let $\alpha, \, \beta \in$ Aut($G$). $\alpha\beta \in$ Aut($G$) means that $\alpha\beta$ is also an automorphism, so let us prove that:
			\begin{enumerate}
				\item \textbf{Function}: Because $\alpha$ and $\beta$ are clearly functions, we get $x = y \Longrightarrow \beta(x) = \beta(y) \Longrightarrow \alpha\beta(x) = \alpha\beta(y)$ for all $x, \, y \in G$. 
				\item \textbf{One-to-one}: Because $\alpha$ and $\beta$ are one-to-one functions we get $\alpha\big(\beta(x)\big) = \alpha\big(\beta(y)\big) \Longrightarrow \beta(x) = \beta(y) \Longrightarrow x = y$.
				\item \textbf{Onto}: We have to proove that for all $y \in G$, there is an $x \in G$ such that $\alpha\beta(x) = y$. Because $\alpha$ is onto, there is a $z \in G$ such that $\alpha(z) = y$. Because $\beta$ is onto, there is a $w \in G$ such that $\beta(w) = z$. We get $\alpha\bp{\beta(x)} = y$ $(x := w)$.
				\item \textbf{O.P.}: Finally, for all $x, \, y \in G:$
				\[
					\alpha\beta(xy) = \alpha\bp{\beta(x)\beta(y)} = \alpha\bp{\beta(x)}\alpha\bp{\beta(y)} = \alpha\beta(x)\alpha\beta(y).
				\]
			\end{enumerate}
		};
		
		\tikz \node[proof]
		{
			We proved that closure holds. The associativity comes from the fact that function composition is associative.\newline
			
			Now let us prove that the identity $\gamma \in$ Aut($G$) exists, such that for all $\alpha \in \text{Aut}(G) : \alpha\gamma(x) = \gamma\alpha(x) = \alpha(x)$ for all $x \in G$. The function $\gamma : G \to G, \, \,  \gamma(x) = x$ is certainly an automorphism, and we get $\alpha\bp{\gamma(x)} = \gamma\bp{\alpha(x)} = \alpha(x)$.\newline
			
			We know that if $\alpha \in$ Aut($G$) is an automorphism, then $\alpha^{-1}$ is also an automorphism, but let us proove the O.P. part of it:
			\[
				\alpha^{-1}(xy) = \alpha^{-1}\Big( \alpha\bp{\alpha^{-1}(x)} \alpha\bp{\alpha^{-1}(y)} \Big) = \alpha^{-1}\Big( \alpha\bp{\alpha^{-1}(x)\alpha^{-1}(y)} \Big) = \alpha^{-1}(x)\alpha^{-1}(y).
			\]
			We proved that Aut($G$) is indeed a group. 
		};
	\end{center}
	
	%%% EXAMPLE
	\begin{center}
		\tikz \node[ex]
		{
			\textbf{Example.} To compute Aut($Z_{10}$), we try to discover enough information about an element $\alpha$ of Aut($Z_{10}$) to determine how $\alpha$ must be defined. To begin with, observe that once we know $\alpha(1)$, we know $\alpha(k)$ for any $k \in Z_{10}$, because
			\[
				\boxed{\alpha(k)} = \underbrace{\alpha(1 + 1 + \cdots + 1)}_{k \text{ terms}} = \underbrace{\alpha(1) + \alpha(1) + \cdots + \alpha(1)}_{k \text{ terms}} = \boxed{k\alpha(1)}.
			\]
			So, we need only determine the choices for $\alpha(1)$ that make $\alpha$ an automorphism of $Z_{10}$. Since property 5 of Theorem $6.2$ tells us that $|\alpha(1)| = |1| = 10$, there are four candidates for $\alpha(1)$:
			\[
				\alpha(1) = 1; \quad \alpha(1) = 3; \quad \alpha(1) = 7; \quad \alpha(1) = 9.
			\]
			To distinguish among the four possiblitities, we refine our notation by denoting the mapping that sends 1 to 1 by $\alpha_1$, and so on. Clearly $\alpha_1$ is the identity. Let us check $\alpha_3$. Since $x \text{ mod } 10 = y \text{ mod } 10 \Longrightarrow 3x \text{ mod } 10 = 3y \text{ mod } 10$, $\alpha_3$ is a function indeed. Moreover, because $\alpha_3(1) = 3$ is a generator of $Z_{10}$, for all $y \in Z_{10}$, there should be $x \in Z_{10}$ such that $\alpha_3(x) = y$, therefore $3x = 3k$ for some $k$. $3x - 3k = 0 \Longrightarrow 3(x-k) = 0$, $x:=k$, so $\alpha_3$ is onto. We know if $S$ is a finite set, then $f : S \to S$ is onto if and only if $f$ is one-to-one, so $\alpha_3$ is one-to-one. Finally, since $\alpha_3(a+b) = 3(a+b) / 3a + 3b = \alpha_3(a) + \alpha_3(b)$ for all $a, \, b \in Z_{10}$, we see that $\alpha_3$ is operation-preserving as well. Thus, $\alpha_3 \in $ Aut($Z_{10}$). The same argument shows that $\alpha_7$ and $\alpha_9$ are also automorphisms.\newline
			
			This gives us the elements of Aut($Z_{10}$) but not the structure. For instance, what is $\alpha_3\alpha_3$?. Direct calculations show that $|\alpha_3| = 4 = |\text{Aut}(Z_{10})|$, thus Aut($Z_{10}$) is cyclic. Actually, the following Cayley tables reveal that Aut($Z_{10}$) is isomorphic to $U(10)$.
		};
		
		\tikz \node[ex]
		{
			\begin{center}
				\begin{tabular}{c|cccc}
					$U(10)$ & 1 & 3 & 7 & 9 \\
					\hline
					1 & 1 & 3 & 7 & 9 \\
					7 & 7 & 1 & 9 & 3 \\
					9 & 9 & 7 & 3 & 1
				\end{tabular}\quad\quad
				\begin{tabular}{c|cccc}
					Aut($Z_{10}$) & $\alpha_1$ & $\alpha_3$ & $\alpha_7$ & $\alpha_9$ \\
					\hline
					$\alpha_1$ & $\alpha_1$ & $\alpha_3$ & $\alpha_7$ & $\alpha_9$ \\
					$\alpha_7$ & $\alpha_7$ & $\alpha_1$ & 9 & $\alpha_3$ \\
					$\alpha_9$ & $\alpha_9$ & $\alpha_7$ & $\alpha_3$ & $\alpha_1$
				\end{tabular}	
			\end{center}
		};
	\end{center}
	
	With example above as a guide, we are now ready to tackle the group Aut($Z_n$). The result is particularly nice, since it relates the two kinds of groups we have most frequently encountered thus far-the cyclic groups $Z_n$ and the $U$-groups $U(n)$.
	
	\subsubsection{Aut$(Z_n)$ $\approx U(n)$}
	\begin{center}
		\tikz \node[theorem]
		{
			\textbf{Theorem.} For every positive integer $n$, Aut$(Z_n)$ is isomorphic to $U(n)$.
		};\newline
		
		\tikz \node[proof]
		{
			\textbf{Proof.} As in the example above, any automorphism $\alpha$ is determined by the value of $\alpha(1)$, and $\alpha(1) \in U(n)$. Now consider the correspodence from Aut$(Z_n)$ to $U(n)$ given by
			\[
				T : \text{Aut}(Z_n) \to U(n), \quad T(\alpha) = \alpha(1) \quad (\alpha \in \text{Aut}(Z_n)).
			\]
			The fact that $\alpha(k) = k\alpha(1)$ implies that $T$ is a one-to-one mapping, becaues if $\alpha, \, \beta \in \text{Aut}(Z_n)$ we get:
			\[
				T(\alpha) = T(\beta) \Longleftrightarrow \alpha(1) = \beta(1)
			\]
			We have to show that $\alpha(k) = \beta(k)$ for all $k \in Z_n$.
			\[
				\alpha(k) = k\alpha(1) = k\beta(1) = \beta(k) \Longrightarrow \alpha = \beta.
			\]
			To prove that $T$ is onto, let $r \in U(n)$ and consider the mapping $\alpha : Z_n \to Z_n$ defined by $\alpha(s) = sr \, \, (\text{mod } n)$ for all $s \in Z_n$. \textit{It is an automorphism of $Z_n$}. Then, since $T(\alpha) = \alpha(1) = 1r$, $T$ is onto $U(n)$.\newline
			
			Finally, we establish the fact that $T$ is operation-preserving. Let $\alpha, \, \beta \in \text{Aut}(Z_n)$. We then have
			\[
				T(\alpha\beta) = (\alpha\beta)(1) = \alpha\bp{\beta(1)} = \underbrace{\alpha(1 + 1 + \cdots + 1)}_{\beta(1) \text{ terms}} = \underbrace{\alpha(1) + \alpha(1) + \cdots + \alpha(1)}_{\beta(1) \text{ terms}}
			\]
			\[
				= \alpha(1)\beta(1) = T(\alpha)T(\beta).
			\]
		This completes the proof.
		};
	\end{center}
	
\end{document}